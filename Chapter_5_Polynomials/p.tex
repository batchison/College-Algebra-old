\subsection{Notation and Basic Examples}
 
{\tmstrong{Objective: Identify the degree, leading coefficient, leading term, and constant term.  Classify each polynomial by type (linear, quadratic, cubic,~...) and number of terms (monomial, binomial, trinomial,~...)}.}\pp

  1)~$y=-2x^3 + 4x+1$\\
	2)~$y=32x^5+x^2+15$\\
	3)~$y=-3x^4+4x^2$\\
	4)~$y=15x^4-32x^2-x-14$\\
  5)~$y=x^5+40$\\
  6)~$y=5x^5+3x^2+x+14$\\
  7)~$y=123x^4-7x^3-5x^2-3x+1$\\
	8)~$y=-2x^3-1$\\
  9)~$y=-23x^6+x^3+x^2+x+1$\\
  10)~$y=-3x^4-15x^4+x^4-27x^3+x^2-13$

\newpage

\subsection{Factoring}
\subsubsection{Special Products} 

{\tmstrong{Objective: Classify each expression as one of the following and perform the necessary factorization:
difference of squares, perfect squares, and/or sum and difference of cubes.}}

\begin{multicols}{2}
  1)~$x^4-81$\\
  3)~$2x^4-32$\\
  5)~$x^2-2x+1$\\
  7)~$16 x^2 + 24 x y + 9 y^2$\\
  9)~$25 x^2 + 20 x y + 4 y^2$\\
  2)~$x^4-1$\\
  4)~$x^2-25$\\
  6)~$x^3-8$\\
  8)~$x^3+8$\\
  10)~$x^3-27$
\end{multicols}

\subsubsection{Quadratic Type}

{\tmstrong{Factor each quadratic type completely over the real numbers.}}

\begin{multicols}{2}
  1)~ $x^4 +13x^2+40$\\
  3)~ $x^4-5x^2+4$\\
  5)~ $x^4 -17x^2+16$\\
  7)~ $x^4 +13x^2+40$\\
  9)~ $3x^4 -32x^2+45$\\
~\\
	2)~ $x^4 +x^2-12$\\
  4)~ $x^4 -3x^2-10$\\
  6)~ $x^6 -82x^3+81$\\
  8)~ $8x^4 +2x^2-3$\\
  10)~ $2x^4 -19x^2+9$%\\
  
\end{multicols}

\subsubsection{Factoring Summary}

{\tmstrong{Factor each expression completely, making certain to factor out a GCF whenever possible.}}\pp

  1)~$24 a z - 18 a h + 60 y z - 45 y h$\\
  2)~$2 x^2 - 11 x + 15$\\
  3)~$5 u^2 - 9 u v + 4 v^2$\\
  4)~$16 x^2 + 48 x y + 36 y^2$\\
  5)~$- 2 x^3 + 128 y^3$\\
  6)~$20 u v - 60 u^3 - 5 x v + 15 x u^2$\\
  7)~$5 n^3 + 7 n^2 - 6 n$\\
  8)~$2 x^3 + 5 x^2 y + 3 y^2 x$\\
  9)~$54 u^3 - 16$\\
  10)~$54 - 128 x^3$\\
  11)~$n^2 - n$\\
  12)~$2x^4 -21x^2-11$

\newpage

\subsection{Division}
\subsubsection{Polynomial Division}

{\tmstrong{Use polynomial long division to divide and simplify the given expression.  Express each answer in the form below.}}
$$\frac{\text{dividend}}{\text{divisor}}~=~\text{quotient~}+~\frac{\text{remainder}}{\text{divisor}}$$
\\
\begin{multicols}{2}
  1) $\frac{20 x^4 + x^3 + 2 x^2}{4 x^3}$\\
  3) $\frac{20 n^4_{} + n^3_{} + 40 n^2}{10 n}$\\
  5) $\frac{12 x^4 + 24 x^3 + 3 x^2}{6 x}$\\
  7) $\frac{10 n^4 + 50 n^3 + 2 n^2}{10 n^2}$\\
  9) $\frac{x^2 - 2 x - 71}{x + 8}$\\
  11) $\frac{n^2 + 13 n + 32}{n + 5}$\\
  13) $\frac{v^2 - 2 v - 89}{v - 10}$\\
  15) $\frac{a^2 - 4 a - 38}{a - 8}$\\
  17) $\frac{45 p^2 + 56 p + 19}{9 p + 4}$\\
  19) $\frac{10 x^2 - 32 x + 9}{10 x - 2}$\\
  21) $\frac{4 r^2 - r - 1}{4 r + 3}$\\
  23) $\frac{n^2 - 4}{n - 2}$\\
  25) $\frac{27 b^2 + 87 b + 35}{3 b + 8}$\\
  27) $\frac{4 x^2 - 33 x + 28}{4 x - 5}$\\
  29) $\frac{a^3 + 15 a^2 + 49 a - 55}{a + 7}$\\
  31) $\frac{x^3 - 26 x - 41}{x + 4}$\\
  33) $\frac{3 n^3 + 9 n^2 - 64 n - 68}{n + 6}$\\
  35) $\frac{x^3 - 46 x + 22}{x + 7}$\\
  37) $\frac{9 p^3 + 45 p^2 + 27 p - 5}{9 p + 9}$\\
  39) $\frac{r^3 - r^2 - 16 r + 8}{r - 4}$\\
  41) $\frac{12 n^3 + 12 n^2 - 15 n - 4}{2 n + 3}$\\
  43) $\frac{4 v^3 - 21 v^2 + 6 v + 19}{4 v + 3}$\\  
  2) $\frac{5 x^4 + 45 x^3 + 4 x^2}{9 x}$\\
  4) $\frac{3 k^3 + 4 k^2 + 2 k}{8 k}$\\
  6) $\frac{5 p^4 + 16 p^3 + 16 p^2}{4 p}$\\
  8) $\frac{3 m^4 + 18 m^3 + 27 m^2}{9 m^2}$\\
  10) $\frac{r^2 - 3 r - 53}{r - 9}$\\
  12) $\frac{b^2 - 10 b + 16}{b - 7}$\\
  14) $\frac{x^2 + 4 x - 26}{x + 7}$\\
  16) $\frac{x^2 - 10 x + 22}{x - 4}$\\
  18) $\frac{48 k^2 - 70 k + 16}{6 k - 2}$\\
  20) $\frac{n^2 + 7 n + 15}{n + 4}$\\
  22) $\frac{3 m^2 + 9 m - 9}{3 m - 3}$\\
  24) $\frac{2 x^2 - 5 x - 8}{2 x + 3}$\\
  26) $\frac{3 v^2 - 32}{3 v - 9}$\\
  28) $\frac{4 n^2 - 23 n - 38}{4 n + 5}$\\
  30) $\frac{8 k^3 - 66 k^2 + 12 k + 37}{k - 8}$\\
  32) $\frac{x^3 - 16 x^2 + 71 x - 56}{x - 8}$\\
  34) $\frac{k^3 - 4 k^2 - 6 k + 4}{k - 1}$\\
  36) $\frac{2 n^3 + 21 n^2 + 25 n}{2 n + 3}$\\
  38) $\frac{8 m^3 - 57 m^2 + 42}{8 m + 7}$\\
  40) $\frac{2 x^3 + 12 x^2 + 4 x - 37}{2 x + 6}$\\
  42) $\frac{24 b^3 - 38 b^2 + 29 b - 60}{4 b - 7}$\\
~\\
\end{multicols}

\newpage

\subsubsection{Synthetic Division}

{\tmstrong{Use synthetic division to divide and simplify the given expression.  Express each answer in the form below.}}
$$\frac{\text{dividend}}{\text{divisor}}~=~\text{quotient~}+~\frac{\text{remainder}}{\text{divisor}}$$
\\
\begin{multicols}{2}
  1)~$\frac{x^4-4x^3+2x^2-x+1 }{x+2}$\\
  3)~$\frac{x^4-2x^3+7x^2-6x+3 }{x-2}$\\
  5)~$\frac{2x^3-2x^3+10x^2+1 }{x+2}$\\
  7)~$\frac{5x^3-2x^3+4x^2-5x }{x-5}$\\
  9)~$\frac{-x^4-x^3+x^2+x+1 }{x+5}$\\
  11)~$\frac{x^4-3x^3+2x^2-x+1 }{x-4}$\\
  2)~$\frac{12x^4-x^3+x^2-3x+1 }{x+2}$\\
  4)~$\frac{3x^4+3x^3+13x^2-4x+14 }{x+1}$\\
  6)~$\frac{1x^4-3x^3+5x^2-14x+2 }{x-2}$\\
  8)~$\frac{2x^4-2x+1 }{x+3}$\\
  10)~$\frac{x^4-3x-4 }{x-3}$\\
  12)~$\frac{x^4-4x^3+13x^2-5x+7 }{x-4}$
\end{multicols}

\newpage

\subsection{Graphing}
\subsubsection{End Behavior}

{\tmstrong{Use the degree and leading coefficient of each polynomial function below to identify the end behavior of its graph.}\pp

  1)~$f(x)=-2x^3 + 4x+1$\\
	2)~$g(x)=32x^5+x^2+15$\\
	3)~$h(x)=-3x^4+4x^2$\\
	4)~$k(x)=15x^4-32x^2-x-14$\\
  5)~$\ell(x)=x^5+40$\\
  6)~$m(x)=5x^5+3x^2+x+14$\\
  7)~$n(x)=123x^4-7x^3-5x^2-3x+1$\\
%~\\
	8)~$p(x)=x^3-1$\\
%\end{multicols}
  9)~$q(x)=-23x^6+x^3+x^2+x+1$\\
 
{\tmstrong{Identify the degree and leading coefficient of each polynomial function below.  Use the degree and leading coefficient to identify the end behavior of the graph of each function.}\pp
\label{Mults}
%\begin{multicols}{2}
  10)~$f(x)=x^3(x-2)(x+2)$\\
	11)~$g(x)=(x^2+1)(1-x)$\\
	12)~$h(x)=x(x-3)^2(x+3)$\\
	13)~$k(x)=(3x-4)(3-4x)$\\
  14)~$\ell(x)=(x^2+2)(x^2+3)$\\
  15)~$m(x)=-2(x+7)^2(1-2x)^2$\\
  16)~$f(x)=(x^2-1)(x+4)$\\
	17)~$g(x)=(x^2-1)(x^2-16)$\\
	18)~$h(x)=-2x^3(3x-1)(2-x)$\\
	19)~$k(x)=(x^2-4x+1)(x+2)^2$\\

\subsubsection{Zeros and Local Behavior}

{\tmstrong{Identify the set of real zeros and their respective multiplicities for each of functions (10)-(19) above.  Provide a rough sketch of each function, making sure to identify any $x-$ and $y-$intercepts.  Use a graphing utility to check your graph.}\pp

\newpage

\subsection{Polynomial Inequalities and Sign Diagrams}

{\tmstrong{Construct a sign diagram for each of functions (10)-(19) on page \pageref{Mults}.  Use your diagram to identify the intervals where each function is nonnegative ($y\geq 0$).}\pp
