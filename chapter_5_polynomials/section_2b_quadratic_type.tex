\subsection{Factoring Summary}

{\tmstrong{Objective: Identify and use the correct method to factor various
polynomials.}}\pp

In this subsection, we will summarize the many 
factoring methods we have seen thus far. An important part of the process for solving any polynomial equation
is the identification of the number of terms in the simplified equation. For any equation we will
always try to factor out a GCF first.

\begin{center}
  {\tmstrong{Factoring Summary}}
\end{center}
\begin{itemizedot}
	\item {\tmstrong{GCF}} - Always look for a GCF first!
  \item {\tmstrong{2 terms}} - Sum or difference of squares or cubes.
  
 $ a^2 - b^2 = (a + b) (a - b)$
  
  $a^2 + b^2$~~~\text{irreducible~over~the~reals}
  
 $ a^3 + b^3 = (a + b) (a^2 - a b + b^2)$
  
  $a^3 - b^3 = (a - b) (a^2 + ab + b^2)$
  
  \item {\tmstrong{3 terms}} - Factor; watch for a perfect square.
  
  $ax^2+bx+c~~~\tmop{Multiply} \tmop{to} a c \tmop{and} \tmop{add} \tmop{to} b$
  
	$a^2 + 2 a b + b^2 = (a + b)^2$
    
  \item {\tmstrong{4 terms}} - Grouping
	\item {\tmstrong{Special case}} - Quadratic type (used in cases with polynomials having even degree and containing 2 or 3 terms)

\end{itemizedot}
We will employ a few of the techniques summarized above in order to factor each of the following examples. Here
the emphasis will be on which strategy to use, rather than the steps that follow.
\begin{example}~~~Factor the given expression completely over the real numbers.
  \begin{eqnarray*}
    4 x^2 + 56 x y + 196 y^2 &  & \tmop{Factor~out~a~GCF~of~} 4\\
    4 (x^2 + 14 x y + 49 y^2) &  & 3 \tmop{terms}, \tmop{try~} ac\tmop{-method}
	\end{eqnarray*}
	We need two terms to multiply to $ac$, or 49, and add to 14.  Use 7 and 7.  We have a perfect square.
  \begin{eqnarray*}
		4 (x + 7 y)^2 &  & \tmop{Our} \tmop{solution}
  \end{eqnarray*}
\end{example}

\begin{example}~~~Factor the given expression completely over the real numbers.
  \begin{eqnarray*}
    5 x^2 y + 15 x y - 35 x^2 - 105 x &  & \tmop{Factor~out~a~GCF~of~} 5 x\\
    5 x (x y + 3 y - 7 x - 21) &  & 4 \tmop{~terms}, \tmop{try}
    \tmop{factor~by~grouping}\\
    5 x [y (x + 3) - 7 (x + 3)] &  & \tmop{We~have~a~common~factor~of~}(x + 3)\\
    5 x (x + 3) (y - 7) &  & \tmop{Our} \tmop{solution}
  \end{eqnarray*}
\end{example}

\begin{example}~~~Factor the given expression completely over the real numbers.
  \begin{eqnarray*}
    100 x^2 - 400 &  & \tmop{Factor~out~a~GCF~of~} 100\\
    100 (x^2 - 4) &  & \tmop{We~have~a~difference} \tmop{of}
    \tmop{squares}\\
    100 (x + 2) (x - 2) &  & \tmop{Our} \tmop{solution}
  \end{eqnarray*}
\end{example}

\begin{example}~~~Factor the given expression completely over the real numbers.
  \begin{eqnarray*}
    108 x^3 y^2 - 39 x^2 y^2 + 3 x y^2 &  & \tmop{Factor~out~a~GCF~of~} 3 x y^2\\
    3 x y^2 (36 x^2 - 13 x + 1) &  & 3\tmop{terms}, \tmop{try~} ac
    \tmop{-method}
	\end{eqnarray*}
	We need two terms to multiply to $ac$, or 36, and add to $-13$.\\
	Use $-9$ and $-4$.
  \begin{eqnarray*}
    3 x y^2 (36 x^2 - 9 x - 4 x + 1) &  & \tmop{Split}
    \tmop{middle} \tmop{term}\\
    3 x y^2 [9 x (4 x - 1) - 1 (4 x - 1)] &  & \tmop{Factor} \tmop{by}
    \tmop{grouping}\\
    3 x y^2 (4 x - 1) (9 x - 1) &  & \tmop{Our} \tmop{solution}
  \end{eqnarray*}
\end{example}

{\tmstrong{World View Note:}} Variables originated in ancient Greece where
Aristotle would use a single capital letter to represent a number.

\begin{example}~~~Factor the given expression completely over the real numbers.
  \begin{eqnarray*}
    5 + 625 y^3 &  & \tmop{Factor~out~a~GCF~of~} 5\\
    5 (1 + 125 y^3) &  & 2 \tmop{terms}, \tmop{a~sum} \tmop{of}
    \tmop{cubes}\\
    5 (1 + 5 y) (1 - 5 y + 25 y^2) &  & \tmop{Our} \tmop{solution}
  \end{eqnarray*}
\end{example}

It is important to be comfortable and confident not just with using all the
factoring methods, but also on deciding which method is best to use. This is why practice is
so important.