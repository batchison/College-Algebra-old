\begin{center}\textbf{About the Text} \end{center}\par \indent This text has been specifically structured with the MATH123 College Algebra student in mind.  It is the result of a collaboration of several full and part-time faculty members at both Framingham State University and Framingham High School and has been made possible through a grant provided by the Educational Technology Office's Teaching with Technology Program at Framingham State University during the Spring of 2015.  Specific details surrounding this Teaching with Technology grant application and anticipated outcomes may be obtained by emailing {\small batchison@framingham.edu}.
\pp \indent The majority of the content has been either taken verbatim or adapted from those \textit{open educational resources} (OER) that have been generously provided by their respective authors and institutions via a Creative Commons license.  Separate sections have been created only when no acceptable available resource could be located.  A detailed list of each utilized resource can be found later on in this introduction.  The collaborators are extremely grateful and appreciative to all OER authoring individuals and institutions for the contributions that have been utilized by this text and which are freely accessible to the greater academic community.
\par
\comment{
\begin{center}\textbf{Textbook Section Identifiers} \end{center}\par \indent Many sections of the text appearing in the table of contents include additional identifiers.
\par \indent Those sections that are considered by the collaborators as essential to the MATH123 College Algebra curriculum have been identified in bold.  Sections which are traditionally treated in a College Algebra course but may not necessarily be considered essential or are often assigned for additional reading have been italicized.
\par \comment Any section not appearing in bold or italicized should be considered as a review of MATH095 General Mathematics (or a previous course). %will be identified with a {\bf GM}.
These sections can be either quickly reviewed or referenced and skipped entirely at the instructor's discretion.  Students enrolled in MATH123 College Algebra should anticipate reviewing such sections in detail, as needed, \textit{outside} of the traditional course requirements, and should be prepared to seek out additional resources with the help of their instructor.
\par \comment Additionally, sections that have been taken verbatim or adapted from an existing OER will be identified with the initials of their author(s) or institution(s).  For example, sections taken from Carl Stitz and Jeff Zeager's \textit{College Algebra} text will be identifed by {\bf SZ}.\par}

%\pagebreak

\begin{center}\textbf{Additional Resources and Recommended Use} \end{center}\par \noindent Additional areas of focus for the Teaching with Technology grant include:
\begin{itemize}
	\item creation of concept summary and homework handouts.
	\item creation of an online repository for new and existing concept tutorial videos.
	\item utilization of the free online homework website www.myopenmath.org and graphing utility found at www.desmos.com.
\end{itemize}
\indent By including access to these additional course enhancements, instructors may utilize this text as a secondary resource to their own course lecture notes and handouts, if they so choose.  Students are strongly urged to take advantage of \textit{all} course inclusions made available to them.  This text should prove most useful as a resource, when additional explanations, examples and/or practice problems are desired.
\par
\begin{center}\textbf{A Note to the Student} \end{center}\par

\indent Like any such undertaking, especially in its initial phase, the project's collaborators apologize for any unforeseen errors or formatting issues, and greatly appreciate the user's understanding with regards to such issues.  As this text is very much a work-in-progress, any feedback on how it can be improved is very much appreciated and will be carefully considered.\pp
{\bf Students wishing to provide feedback may either do so through their course instructor or by emailing {\small batchison@framingham.edu}.}   
\par
%\pagebreak
\begin{center}\textbf{Project Collaborators and Supporters} (as of August 2015) \end{center}\par
\noindent The following individuals have contributed in a variety of ways, including: consulting, content creation and instruction and assessment.
\begin{itemize}
	\item Benjamin Atchison, Assistant Professor, Framingham State University\\{\bf Lead Instructor} - Spring 2016
	\item Jeffrey Barden, Instructor, Framingham State University and Wentworth College
	\item Karen Dawe, Teacher and Instructor (College Tomorrow Summer Program), Framingham High School and Framingham State University
	\item Mark Morabito, Teacher and Instructor, Framingham High School and Framingham State University\\{\bf Lead Instructor} - Fall 2015, Spring 2016
\end{itemize}
\pagebreak
Additional support provided through the Framingham State University Educational Technology Office (ETO) and Whittemore Library, specifically:
\begin{itemize}
	\item Stacy Cohen
	\item Millie Gonzalez
	\item Justin Lauzon
	\item Robin Robinson, ETO Director
\end{itemize}\par
\begin{center}\textbf{Copyright} \end{center}%\par Keeping within the requirements of {\bf SZ} (listed below), this work is licensed under a Creative Commons Attribution-NonCommercial-ShareAlike 4.0 International License.  To view a copy of this license, visit the following site.
%\begin{center}
%http://creativecommons.org/licenses/by-nc-sa/4.0/ 
\doclicenseThis
%\end{center}

\begin{center}\textbf{List of Utilized OERs} \end{center}\par The following is a list of open educational resources whose content has been taken verbatim or, in most instances, adapted for the purposes of the creation of this text.  Permission has been granted by the authors.  The resource location and Creative Commons licensing has been included at the end of each item.  %More information regarding Creative Commons licenses may be found at http://creativecommons.org/licenses.
\begin{itemize}
%	\item[{\bf R}] \textit{Prealgebra}.  College of the Redwoods, Department of Mathematics.  Second Edition.  2012-2013.  http://mathrev.redwoods.edu/PreAlgText/.  CC-BY-NC-SA.
	\item[{\bf SZ}] \textit{College Algebra}.  Stitz, Carl and Zeager, Jeff.  Third (corrected) Edition.  2013. http://www.stitz-zeager.com.  CC-BY-NC-SA.
	\item[{\bf W}] \textit{Beginning and Intermediate Algebra}.  Wallace, Tyler.  2010.\\
	http://www.wallace.ccfaculty.org/book/book.html. CC-BY.
\end{itemize}
A more detailed itemization of this textbook's sections and corresponding attributed works will be included in a future edition of the text, and can be made available upon request.%\pp

\newpage

\begin{center}\textbf{Text Update - ``To Do'' List} \end{center}\par
As of \today, the following chapters/sections have yet to be included in this version of the text.
\begin{itemize}
		\item Applications of Quadratic Equations (Chapter 3)
		\item Inverse Functions (Chapter 4)
		\item Applications of Functions (Chapter 4)
		\item Sign Diagrams for Polynomials (Chapter 5)
		\item Rational Functions (Chapter 6)
\end{itemize}
The sections in Chapter 4 on Domain and Range and Transformations of Functions are also only partially completed at this time, since many of the graphs in these sections have been left for the reader to create directly in the text.  Additionally, plans to further enhance existing sections of the text with more content and advanced examples are also ongoing.  Answers to selected practice problems will also be included in the Fall 2016 version of the text.