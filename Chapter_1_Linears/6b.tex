\subsection{Age Problems}\pp

 {\tmstrong{Objective: Solve age problems by creating and solving a linear
equation.}}\pp

 Age problems present another application of linear equations. When we
are solving age problems we generally will be comparing the age of two people
both now and in the future (or past). Using the clues given in the problem we
will be working to find their current age. There can be a lot of information
in these problems and we can easily get lost in all the information. To help
us organize and solve our problem we will fill out a three by three table for
each problem. An example of the basic structure of the table is shown below.
\[%\begin{table}[h]
  \begin{array}{|c|c|c|}
    \hline
    & \mbox{Age~Now} & \mbox{Change}\\
    \hline
    \mbox{Person~1} &  & \\
    \hline
    \mbox{Person~2} &  & \\
    \hline
  \end{array} \]
\begin{center}
\small{Structure of Age Table}
	%\end{table}
\end{center}

 Normally where we see ``Person 1'' and ``Person 2'' we will use the name of
the person we are talking about. We will use this table to set up the
following example.

\begin{example}\label{Lin82}~~~ Adam is 20 years younger than Brian. In two years Brian will be twice as
  old as Adam. How old are they now?
  \begin{eqnarray*}
    \begin{array}{|c|c|c|}
      \hline
      & \tmop{Age~} \tmop{Now} & + 2\\
      \hline
      \tmop{Adam} &  & \mbox{\hspace{.7in}}\\
      \hline
      \tmop{Brian} &  & \\
      \hline
    \end{array} &  & \begin{array}{p{0.425\textwidth}}
      \tmop{We} \tmop{use} \tmop{Adam} \tmop{and} \tmop{Brian} \tmop{for}
      \tmop{our} \tmop{persons}.  
      \tmop{We} \tmop{use} + 2 \tmop{for} \tmop{change} \tmop{because}
      \tmop{the} \tmop{second} \tmop{phrase}
      \tmop{is} \tmop{two} \tmop{years} \tmop{in} \tmop{the} \tmop{future}.
    \end{array}\\
  \end{eqnarray*}
  \begin{eqnarray*}
    \begin{array}{|c|c|c|}
      \hline
      & \tmop{Age~} \tmop{Now} & + 2\\
      \hline
      \tmop{Adam} & x - 20 & \mbox{\hspace{.70in}}\\
      \hline
      \tmop{Brain} & x & \\
      \hline
    \end{array} &  & \begin{array}{p{0.425\textwidth}}
      \tmop{Consider} \tmop{the} ``\tmop{Now}'' \tmop{part},
      \tmop{Adam} \tmop{is} 20 \tmop{years}
      \tmop{younger} \tmop{than} \tmop{Brian}.\\
			\tmop{We} \tmop{are}
      \tmop{given} \tmop{information} \tmop{about} \tmop{Adam}, \tmop{not} \tmop{Brian}. \tmop{So} \tmop{Brian} \tmop{is}
      $x$ \tmop{now}.\\
			\tmop{To} \tmop{show} \tmop{Adam}
      \tmop{is} 20 \tmop{years} \tmop{younger} \tmop{we} \tmop{subtract} 20,
      \tmop{Adam} \tmop{is} $x -$ 20.
    \end{array}\\
    &  & \\
    \begin{array}{|c|c|c|}
      \hline
      & \tmop{Age~} \tmop{Now} & + 2\\
      \hline
      \tmop{Adam} & x - 20 & x - 20 + 2\\
      \hline
      \tmop{Brian} & x & x + 2\\
      \hline
    \end{array} &  & \begin{array}{p{0.425\textwidth}}
      \tmop{Now} \tmop{the} + 2 \tmop{column} \tmop{is} \tmop{filled}
      \tmop{in}. \tmop{This} \tmop{is} \tmop{done} \tmop{by} \tmop{adding}
      2 \tmop{to} \tmop{both} \tmop{Adam`s} \tmop{and} \tmop{Brian`s}
      \tmop{`Age~Now'} \tmop{column~entry,} \tmop{as} \tmop{shown}
      \tmop{in} \tmop{the} \tmop{table}.
    \end{array}\\
    &  & \\
    \begin{array}{|c|c|c|}
      \hline
      & \tmop{Age~} \tmop{Now} & + 2\\
      \hline
      \tmop{Adam} & x - 20 & ~~~x - 18~~~\\
      \hline
      \tmop{Brian} & x & x + 2\\
      \hline
    \end{array} &  & \begin{array}{p{0.425\textwidth}}
      \tmop{Combine} \tmop{like} \tmop{terms} \tmop{in} \tmop{Adam`s}
      \tmop{future} \tmop{age} : - 20 + 2.\\
      \tmop{This} \tmop{table} \tmop{is} \tmop{now} \tmop{completed},
      \tmop{and} \tmop{we} \tmop{are} \tmop{ready} \tmop{to} \tmop{solve}.
    \end{array}\\
    &  & \\
    B = 2 A &  & \begin{array}{p{0.425\textwidth}}
      \tmop{Our} \tmop{equation} \tmop{comes} \tmop{from} \tmop{the}
      \tmop{future} \tmop{statement:} \tmop{Brian} \tmop{will} \tmop{be} \tmop{twice} \tmop{as} \tmop{old} \tmop{as} \tmop{Adam}.\\
			\tmop{This} \tmop{means} \tmop{the} \tmop{younger}, \tmop{Adam}, \tmop{needs} \tmop{to} \tmop{be}
      \tmop{multiplied} \tmop{by} 2.
    \end{array}\\
	    (x + 2) = 2 (x - 18) &  & \begin{array}{p{0.425\textwidth}}
      \tmop{Replace} $B$ \tmop{and} $A$ \tmop{with} \tmop{the} \tmop{information}
      \tmop{in} \tmop{their} \tmop{future}
      \tmop{cells}, \tmop{Adam} ($A$) \tmop{is} \tmop{replaced} \tmop{with} $x -$ 18 \tmop{and} \tmop{Brian} ($B$) \tmop{is} \tmop{replaced} \tmop{with} $x +$ 2.\\ \tmop{This} \tmop{is} \tmop{the} \tmop{equation} \tmop{to} \tmop{solve}!
    \end{array}\\
    x + 2 = 2 x - 36 &  & \begin{array}{p{0.425\textwidth}}
      \tmop{Distribute} \tmop{through} \tmop{parentheses}
    \end{array}\\
    \tmmathbf{\underline{- x ~~~~~- x}}~~~~~~ &  & \begin{array}{p{0.425\textwidth}}
      \tmop{Subtract} $x$ \tmop{from} \tmop{both} \tmop{sides} \tmop{to}
      \tmop{get} \tmop{variable} \tmop{on} \tmop{one} \tmop{side}
    \end{array}\\
    2 = x - 36 &  & \begin{array}{p{0.425\textwidth}}
      \tmop{Need} \tmop{to} \tmop{clear} \tmop{the} $-$ 36
    \end{array}\\
    \tmmathbf{\underline{+ 36 ~~~+ 36}} &  & \tmop{~Add} 36 \tmop{to} \tmop{both}
    \tmop{sides}\\
    38 = x &  & \tmop{~Our} \tmop{solution} \tmop{for} x%\\
  \end{eqnarray*}
  \begin{eqnarray*}
    \begin{array}{|c|c|}
      \hline
      & \tmop{Age~} \tmop{now}\\
      \hline
      \tmop{Adam} & 38 - 20 = 18\\
      \hline
      \tmop{Brian} & 38\\
      \hline
    \end{array} &  & \begin{array}{p{0.425\textwidth}}
      \tmop{The} \tmop{first} \tmop{column} \tmop{will} \tmop{help} \tmop{us}
      \tmop{answer} \tmop{the} \tmop{question}.\\
      \tmop{Replace} \tmop{the} $x$'s \tmop{with} 38 \tmop{and} \tmop{simplify}.\\
      \tmop{Adam} \tmop{is} 18 \tmop{and} \tmop{Brian} \tmop{is} 38.
    \end{array}
  \end{eqnarray*}
\end{example}


 Solving age problems can be summarized in the following five steps. These five
steps are guidelines to help organize the problem we are trying to solve.
\begin{enumerate}
  \item Fill in the `Now' column. The age of the person we know nothing about is $x$.
  
  \item Fill in the future/past column by adding/subtracting the change to
  the `Now' column.
  
  \item Make an equation for the relationship in the future. This is
  independent of the table.
  
  \item Replace variables in the equation with information in the future/past cells of table.
  
  \item Solve the equation for $x$, use the solution to answer the question.
\end{enumerate}
 These five steps are illustrated in the following example.

\begin{example}\label{Lin83}~~~ Carmen is 12 years older than David. Five years ago the sum of their ages was 28. How old are they now?
  \begin{eqnarray*}
    \begin{array}{|c|c|c|}
      \hline
      & \tmop{Age~} \tmop{Now} & - 5\\
      \hline
      \tmop{Carmen} &  & \mbox{\hspace{.7in}}\\
      \hline
      \tmop{David} &  & \\
      \hline
    \end{array} &  & \tmop{Five} \tmop{years} \tmop{ago} \tmop{is} - 5
    \tmop{in} \tmop{the} \tmop{change} \tmop{column}.\\
    &  & \\
    \begin{array}{|c|c|c|}
      \hline
      & \tmop{Age~} \tmop{Now} & - 5\\
      \hline
      \tmop{Carmen} & x + 12 & \mbox{\hspace{.7in}}\\
      \hline
      \tmop{David} & x & \\
      \hline
    \end{array} &  & \begin{array}{p{0.425\textwidth}}
      \tmop{Carmen} \tmop{is} 12 \tmop{years} \tmop{older} \tmop{than}
      \tmop{David}. \tmop{We} \tmop{don`t} \tmop{know} \tmop{about} \tmop{David} \tmop{so} \tmop{his~age} \tmop{is} $x$.\\
      \tmop{Carmen`s~age~is} \tmop{then} $x +$ 12.
    \end{array}\\
    &  & \\
	%\end{eqnarray*}
	%\begin{eqnarray*}
    \begin{array}{|c|c|c|}
      \hline
      & \tmop{Age~} \tmop{Now} & - 5\\
      \hline
      \tmop{Carmen} & x + 12 & x + 12 - 5\\
      \hline
      \tmop{David} & x & x - 5\\
      \hline
    \end{array} &  & \begin{array}{p{0.425\textwidth}}\tmop{Subtract} 5 \tmop{from} \tmop{`Now'} \tmop{column}\\
    \tmop{to} \tmop{get} \tmop{the} \tmop{change}.\\
    \end{array}
%	& & \\
	\end{eqnarray*}
	\begin{eqnarray*}
  	 \begin{array}{|c|c|c|}
      \hline
      & \tmop{Age~} \tmop{Now} & - 5\\
      \hline
      \tmop{Carmen} & x + 12 & ~~~~x + 7~~~~\\
      \hline
      \tmop{David} & x & x - 5\\
      \hline
    \end{array} &  &  \begin{array}{p{0.425\textwidth}}
      \tmop{Simplify} \tmop{by} \tmop{combining} \tmop{like} \tmop{terms}, 12 - 5.\\  \tmop{Our} \tmop{table} \tmop{is} \tmop{ready}!
    \end{array}\\
    &  & \\
    C + D = 28~ &  & \tmop{The} \tmop{sum} \tmop{of} \tmop{their} \tmop{ages}
    \tmop{will} \tmop{be~} 28.\\
		& & \tmop{So} \tmop{we} \tmop{add~} C \tmop{~and~} D.\\
    (x + 7) + (x - 5) = 28~ &  & \tmop{Replace~} C \tmop{and} D \tmop{with}
    \tmop{the} \tmop{change} \tmop{cells}.\\
    x + 7 + x - 5 = 28~ &  & \tmop{Remove} \tmop{parentheses}\\
    2 x + 2 = 28~ &  & \tmop{Combine} \tmop{like} \tmop{terms} x + x \tmop{and}
    7 - 5\\
    \tmmathbf{\underline{- 2 ~- 2}} &  & \tmop{Subtract} 2 \tmop{from}
    \tmop{both} \tmop{sides}\\
    2 x = 26~ &  & \tmop{Notice} x \tmop{is} \tmop{multiplied} \tmop{by} 2\\
    \tmmathbf{\overline{2} ~~~~~ \overline{2}}~~ &  & \tmop{Divide} \tmop{both}
    \tmop{sides} \tmop{by} 2\\
    x = 13~ &  & \tmop{Our} \tmop{solution} \tmop{for} x\\
		& & \\
    \begin{array}{|c|c|}
      \hline
      & \tmop{Age~} \tmop{Now}\\
      \hline
      \tmop{Carmen} & 13 + 12 = 25\\
      \hline
      \tmop{David} & 13\\
      \hline
    \end{array} &  & \begin{array}{p{0.425\textwidth}}
      \tmop{Replace} $x$ \tmop{with} 13 \tmop{to} \tmop{answer} \tmop{the}
      \tmop{question}. \tmop{Carmen} \tmop{is} 25 \tmop{and} \tmop{David} \tmop{is} 13.
    \end{array}
  \end{eqnarray*}
\end{example}

 Sometimes we are given the sum of two (or more) people`s ages right now. These problems can be tricky. In this case we will write the sum above the `Now' column and assign $x$ to the first person's age now. The second person`s age will then involve subtraction: (Total age) $-~x$. This is shown in the next example.

\begin{example}\label{Lin84}~~~ The sum of the ages of Nicole and Kristen is 32. In two years Nicole will be three times as old as Kristen. How old are they now?
  \begin{eqnarray*}
 %   \begin{array}{p{0.425\textwidth}}
%%      32\\
      \begin{array}{|c|c|c|}
        \hline
        & \tmop{Age~} \tmop{Now} & + 2\\
        \hline
        \tmop{Nicole} & x & \mbox{\hspace{.7in}}\\
        \hline
        \tmop{Kristen} & 32 - x & \\
        \hline
      \end{array}
  %  \end{array} 
	&  & \begin{array}{p{0.425\textwidth}}
      \\
      \tmop{The} \tmop{change} \tmop{is} + 2 \tmop{for} \tmop{two}
      \tmop{years} \tmop{in} \tmop{the} \tmop{future}.\\
%      \tmop{The} \tmop{total} \tmop{is} \tmop{placed} \tmop{above} \tmop{Age}
      %\tmop{Now}\\
      \tmop{The} \tmop{first} \tmop{person`s~age} \tmop{is} $x$.\\
			\tmop{The~second~person`s~age~becomes} 32 $- x$.
    \end{array}\\
    &  & \\
    \begin{array}{|c|c|c|}
      \hline
      & \tmop{Age~} \tmop{Now} & + 2\\
      \hline
      \tmop{Nicole} & x & x + 2\\
      \hline
      \tmop{Kristen} & 32 - x & 32 - x + 2\\
      \hline
    \end{array} &  & \begin{array}{p{0.425\textwidth}}
		\\
		\tmop{Fill~in~the~change~column}\\
		\tmop{by~adding~2~to~each~cell}.%\\
    \end{array}%&  & \\
 	\end{eqnarray*}
	\begin{eqnarray*}
			\begin{array}{|c|c|c|}
      \hline
      & \tmop{Age~} \tmop{Now} & + 2\\
      \hline
      \tmop{Nicole} & x & ~~~~x + 2~~~~\\
      \hline
      \tmop{Kristen} & 32 - x & 34 - x\\
      \hline
    \end{array} &  & \tmop{Combine} \tmop{like} \tmop{terms} 32 + 2,
    \tmop{our} \tmop{table} \tmop{is} \tmop{done}!\\
    &  & \\
   N = 3 K &  & \tmop{Nicole} \tmop{is} \tmop{three} \tmop{times} \tmop{as}
    \tmop{old} \tmop{as} \tmop{Kristen}\\
    (x + 2) = 3 (34 - x) &  & \tmop{Replace} \tmop{variables} \tmop{with}
    \tmop{information} \tmop{in} \tmop{change} \tmop{cells}\\
    x + 2 = 102 - 3 x &  & \tmop{Distribute} \tmop{through}
    \tmop{parentheses}\\
    \tmmathbf{\underline{+ 3 x ~~~~~~~~~~+ 3 x}} &  & \tmop{Add} 3 x \tmop{to}
    \tmop{both} \tmop{sides} \tmop{so} \tmop{variable} \tmop{is} \tmop{only}
    \tmop{on} \tmop{one} \tmop{side}\\
    4 x + 2 = 102 &  & \tmop{Solve} \tmop{the} \tmop{two} - \tmop{step}
    \tmop{equation}\\
    \tmmathbf{\underline{- 2 ~~ - 2}} &  & \tmop{Subtract} 2 \tmop{from}
    \tmop{both} \tmop{sides}\\
    4 x = 100 &  & \tmop{The} \tmop{variable} \tmop{is} \tmop{multiplied}
    \tmop{by} 4\\
    \tmmathbf{\overline{~4~} ~~~~ \overline{~4~}} &  & \tmop{Divide} \tmop{both}
    \tmop{sides} \tmop{by} 4\\
    x = 25 &  & \tmop{Our} \tmop{solution} \tmop{for} x\\
		& & \\
    \begin{array}{|c|c|}
      \hline
      & \tmop{Age~} \tmop{Now}\\
      \hline
      \tmop{Nicole} & 25\\
      \hline
      \tmop{Kristen} & 32 - 25 = 7\\
      \hline
    \end{array} &  & \begin{array}{p{0.425\textwidth}}
      \tmop{Plug} 25 \tmop{in} \tmop{for} $x$ \tmop{in} \tmop{the} \tmop{`Now'}
      \tmop{column}.\\
      \tmop{Nicole} \tmop{is} 25 \tmop{and} \tmop{Kristen} \tmop{is} 7.
    \end{array}
  \end{eqnarray*}
\end{example}
\pp

 A slight variation on age problems is to ask not how old the people are, but
rather ask how long until we have some relationship about their ages. In this
case we alter our table slightly. In the change column because we don't know
the time to add or subtract we will use a variable, $t$, and add or subtract
this from the `Now' column. This is shown in the next example.\pp

\begin{example}\label{Lin85}~~~ Louise is 26 years old. Her daughter is 4 years old. In how many years will
  Louise be double her daughter's age?
  \begin{eqnarray*}
    \begin{array}{|c|c|c|}
      \hline
      & \tmop{Age~} \tmop{Now} & + t\\
      \hline
      \tmop{Louise} & 26 & \mbox{\hspace{.4in}}\\
      \hline
      \tmop{Daughter} & 4 & \\
      \hline
    \end{array} &  & \begin{array}{p{0.425\textwidth}}
      \tmop{As} \tmop{we} \tmop{are} \tmop{given} \tmop{their} \tmop{ages}
      \tmop{now}, \tmop{these} \tmop{numbers} \tmop{go} \tmop{into}
      \tmop{the} \tmop{table}.\\
			\tmop{The} \tmop{change} \tmop{is} \tmop{unknown}, \tmop{so} \tmop{we} \tmop{write} $+ t$ \tmop{for} \tmop{the} \tmop{change}.
    \end{array}\\
    &  & \\
    \begin{array}{|c|c|c|}
      \hline
      & \tmop{Age~} \tmop{Now} & + t\\
      \hline
      \tmop{Louise} & 26 & 26 + t\\
      \hline
      \tmop{Daughter} & 4 & 4 + t\\
      \hline
    \end{array} &  & \begin{array}{p{0.425\textwidth}}
      \tmop{Fill} \tmop{in} \tmop{the} \tmop{change} \tmop{column} \tmop{by}
      \tmop{adding} $t$ \tmop{to} \tmop{each} \tmop{person`s} \tmop{age}.\\ \tmop{Our} \tmop{table} \tmop{is} \tmop{now} \tmop{complete}.
    \end{array}\\
    &  & \\
\end{eqnarray*}
\begin{eqnarray*}
    L = 2 D &  & \tmop{Louise} \tmop{will} \tmop{be} \tmop{double} \tmop{her}
    \tmop{daughter`s~age}\\
    (26 + t) = 2 (4 + t) &  & \tmop{Replace} \tmop{variables} \tmop{with}
    \tmop{information} \tmop{in} \tmop{change} \tmop{cells}\\
    26 + t = 8 + 2 t &  & \tmop{Distribute} \tmop{through}
    \tmop{parentheses}\\
    \tmmathbf{\underline{- t ~~~~~~- t}} &  & \tmop{Subtract} t \tmop{from}
    \tmop{both} \tmop{sides}\\
    26 = 8 + t &  & \tmop{Now} \tmop{we} \tmop{have} \tmop{an} 8 \tmop{added}
    \tmop{to} \tmop{the} t\\
    \underline{\tmmathbf{- 8 ~~- 8}}~~~ &  & \tmop{Subtract} 8 \tmop{from}
    \tmop{both} \tmop{sides}\\
    18 = t &  & \tmop{In} 18 \tmop{years} \tmop{she} \tmop{will} \tmop{be}
    \tmop{double} \tmop{her} \tmop{daughter`s} \tmop{age}
  \end{eqnarray*}
\end{example}

 Age problems have several steps to them. If we take the time to work
through each of the steps carefully, however, keeping the information organized, the problems can be solved quite nicely.\pp

 {\tmstrong{World View Note:}} The oldest man in the world was Shigechiyo Izumi from Japan who lived to be 120 years, 237 days. His exact age, however, has been disputed.