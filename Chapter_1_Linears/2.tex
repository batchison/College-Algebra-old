 {\tmstrong{Objective: Solve linear absolute value equations.}}\pp

 When solving equations with absolute value we can end up with more than one
possible answer. This is because what is in the absolute value can be either
negative or positive and we must account for both possibilities when solving
equations. This is illustrated in the following example.

\begin{example}\label{Lin34}
  
  \begin{eqnarray*}
    |x| = 7 &  & \tmop{Absolute} \tmop{value} \tmop{can} \tmop{be}
    \tmop{positive} \tmop{or} \tmop{negative}\\
    x = 7 \tmop{~or~} x = - 7 &  & \tmop{Our} \tmop{solution}
  \end{eqnarray*}
\end{example}

   Notice that we have considered two possibilities, both the positive and
  negative. Either way, the absolute value of our number will be positive $7$.\pp

 {\tmstrong{World View Note:}} The first set of rules for working with
negatives came from 7th century India. However, in 1758, almost a thousand
years later, British mathematician Francis Maseres claimed that negatives
``darken the very whole doctrines of the equations and make dark of the things
which are in their nature excessively obvious and simple.''\pp

 When we have absolute values in our problem it is important to first isolate
the absolute value, then remove the absolute value by considering both the
positive and negative solutions. Notice in the next two examples, all the
numbers outside of the absolute value are moved to the other side first before
we remove the absolute value bars and consider both positive and negative
solutions.

\begin{example}\label{Lin35}
  \begin{eqnarray*}
    5 + |x| = 8~~ &  & \tmop{Notice} \tmop{absolute} \tmop{value} \tmop{is}
    \tmop{not} \tmop{alone}\\
    \tmmathbf{\underline{- 5 ~~~~~~~~- 5}} &  & \tmop{Subtract} 5 \tmop{from}
    \tmop{both} \tmop{sides}\\
    |x| = 3~ &  & \tmop{Absolute} \tmop{value} \tmop{can} \tmop{be}
    \tmop{positive} \tmop{~or~} \tmop{negative}\\
    x = 3 \tmop{~or~} x = - 3 &  & \tmop{Our} \tmop{solution}
  \end{eqnarray*}
\end{example}

\begin{example}\label{Lin36}
  \begin{eqnarray*}
    - 4 |x| = - 20 &  & \tmop{Notice} \tmop{absolute} \tmop{value} \tmop{is}
    \tmop{not} \tmop{alone}\\
    \tmmathbf{\overline{- 4} ~~~~~ \overline{- 4}}~ &  & \tmop{Divide} \tmop{both}
    \tmop{sides} \tmop{by} - 4\\
    |x| = 5 &  & \tmop{Absolute} \tmop{value} \tmop{can} \tmop{be}
    \tmop{positive} \tmop{or} \tmop{negative}\\
    x = 5 \tmop{~or~} x = - 5 &  & \tmop{Our} \tmop{solution}
  \end{eqnarray*}
\end{example}

 Notice we never combine what is inside the absolute value with what is outside
the absolute value. This is very important as it will often change the final
result to an incorrect solution. The next example requires two steps to
isolate the absolute value. The idea is the same as a two-step equation, add
or subtract, then multiply or divide.

\begin{example}\label{Lin37}
  \begin{eqnarray*}
    5 |x| - 4 = 26 &  & \tmop{Notice} \tmop{the} \tmop{absolute} \tmop{value}
    \tmop{is} \tmop{not} \tmop{alone}\\
    \underline{\tmmathbf{+ 4 ~+ 4}} &  & \tmop{Add} 4 \tmop{to} \tmop{both}
    \tmop{sides}\\
    5 |x| = 30 &  & \tmop{Absolute} \tmop{value} \tmop{still} \tmop{not}
    \tmop{alone}\\
    \tmmathbf{\overline{5} ~~~~~~ \overline{5}}~ &  & \tmop{Divide} \tmop{both}
    \tmop{sides} \tmop{by} 5\\
    |x| = 6 &  & \tmop{Absolute} \tmop{value} \tmop{can} \tmop{be}
    \tmop{positive} \tmop{or} \tmop{negative}\\
    x = 6 \tmop{~or~} x = - 6 &  & \tmop{Our} \tmop{solution}
  \end{eqnarray*}
\end{example}

 Again we see the same process, get the absolute value alone first, then
consider the positive and negative solutions. Often the absolute value will
have more than just a variable in it. In this case we will have to solve the
resulting equations when we consider the positive and negative possibilities.
This is shown in the next example.

\begin{example}\label{Lin38}
  \begin{eqnarray*}
    |2x - 1| = 7 &  & \tmop{Absolute} \tmop{value} \tmop{can} \tmop{be}
    \tmop{positive} \tmop{or} \tmop{negative}\\
    2 x - 1 = 7 \tmop{~or~} 2 x - 1 = - 7 &  & \tmop{Two} \tmop{equations}
    \tmop{to} \tmop{solve}
  \end{eqnarray*}
   Now notice we have two equations to solve, each equation will give us a
  different solution. Both equations solve like any other two-step equation.
  \begin{eqnarray*}
    \begin{array}{l}
      2 x - 1 = 7\\
      ~~~~\tmmathbf{\underline{+ 1 ~+ 1}}\\
      2 x = 8\\
      ~\tmmathbf{\overline{2} ~~~~~ \overline{2}}\\
      x = 4
    \end{array} \tmop{~or~} \begin{array}{l}
      2 x - 1 = - 7\\
      ~~~~\underline{\tmmathbf{+ 1 ~~+ 1}}\\
      2 x = - 6\\
      \tmmathbf{\overline{2} ~~~~~~~ \overline{2}}\\
      x = - 3
    \end{array} &  & 
  \end{eqnarray*}
\end{example}  
  
 Thus, from our previous example we have two solutions, $x = 4$ or $x = - 3$.\pp

 Again, it is important to remember that the absolute value must be alone first
before we consider the positive and negative possibilities. This is
illustrated below.

\begin{example}\label{Lin39}
  \begin{eqnarray*}
    2 - 4 |2x + 3| = - 18 &  & 
  \end{eqnarray*}
  To get the absolute value alone we first need to get rid of the $2$ by
  subtracting, then divide by $- 4$. Notice we cannot combine the $2$ and $-
  4$ because they are not like terms, the $- 4$ has the absolute value
  connected to it. Also notice we do not distribute the $- 4$ into the
  absolute value. This is because the numbers outside cannot be combined with
  the numbers inside the absolute value. Thus we get the absolute value alone
  in the following way:
  \begin{eqnarray*}
    2 - 4 |2x + 3| = - 18 &  & \tmop{Notice} \tmop{absolute} \tmop{value}
    \tmop{is} \tmop{not} \tmop{alone}\\
    \underline{\tmmathbf{- 2 ~~~~~~~~~~~~~~~~~~- 2}} &  & \tmop{Subtract} 2 \tmop{from}
    \tmop{both} \tmop{sides}\\
    - 4 |2x + 3| = - 20 &  & \tmop{Absolute} \tmop{value} \tmop{still}
    \tmop{not} \tmop{alone}\\
    \tmmathbf{\overline{~~- 4~~~} ~~~~~~ \overline{- 4}} &  & \tmop{Divide} \tmop{both}
    \tmop{sides} \tmop{by} - 4\\
    |2x + 3| = 5 &  & \tmop{Absolute} \tmop{value} \tmop{can} \tmop{be}
    \tmop{positive} \tmop{or} \tmop{negative}\\
    2 x + 3 = 5 \tmop{~or~} 2 x + 3 = - 5 &  & \tmop{Two} \tmop{equations}
    \tmop{to} \tmop{solve}
  \end{eqnarray*}
   Now we just solve these two remaining equations to find our solutions.
  \begin{eqnarray*}
    \begin{array}{l}
      2 x + 3 = 5\\
      ~~~~\tmmathbf{\underline{- 3 ~~- 3}}\\
      2 x = 2\\
      \tmmathbf{ \overline{2} ~~~~~ \overline{2}}\\
      x = 1
    \end{array} \tmop{~or~} \begin{array}{l}
      2 x + 3 = - 5\\
      ~~~~\tmmathbf{\underline{- 3 ~~- 3}}\\
      2 x = - 8\\
      ~\tmmathbf{ \overline{2} ~~~~~~ \overline{2}}\\
      x = - 4
    \end{array} &  & 
  \end{eqnarray*}
  We now have our two solutions, $x = 1$ and $x = - 4$.
	\end{example}

 As we are solving absolute value equations it is important to be aware of
special cases. Remember the result of an absolute value must always be
positive. Notice what happens in the next example.

\begin{example}\label{Lin40}
  \begin{eqnarray*}
    7 + |2x - 5| = 4~ &  & \tmop{Notice} \tmop{absolute} \tmop{value} \tmop{is}
    \tmop{not} \tmop{alone}\\
    \underline{\tmmathbf{- 7 ~~~~~~~~~~~~~~- 7}} &  & \tmop{Subtract} 7 \tmop{from}
    \tmop{both} \tmop{sides}\\
    |2x - 5| = - 3~~ &  & \tmop{Result} \tmop{of} \tmop{absolute} \tmop{value}
    \tmop{is} \tmop{negative} !
  \end{eqnarray*}
\end{example}
   Notice the absolute value equals a negative number! This is impossible with
  absolute value. When this occurs we say there is {\tmstrong{no solution}} or
  $\varnothing$.\pp

 One other type of absolute value problem is when two absolute values are equal
to each other. We still will consider both the positive and negative result,
the difference here will be that we will have to distribute a negative into
the second absolute value for the negative possibility.

\begin{example}\label{Lin41}
  \begin{eqnarray*}
    |2x - 7| = |4x + 6|  &  & \tmop{Absolute} \tmop{value} \tmop{can}
    \tmop{be}\\
		& & \tmop{~~~positive} \tmop{or} \tmop{negative}\\
    2 x - 7 = 4 x + 6~~~~~&  & \tmop{Make}
    \tmop{second} \tmop{part} \tmop{of}\\
		  \tmop{or~~~} 2 x - 7 = - (4 x + 6) & &\tmop{~~~second} \tmop{equation} \tmop{negative}
  \end{eqnarray*}
  Notice the first equation is the positive possibility and has no significant
  difference other than the missing absolute value bars. The second equation
  considers the negative possibility. For this reason we have a negative in
  front of the expression which will be distributed through the equation on
  the first step of solving. So we solve both these equations as follows:
  \begin{eqnarray*}
    \begin{array}{l}
      ~~~2 x - 7 = 4 x + 6\\
      \tmmathbf{\underline{- 2 x ~~~~~- 2 x}~~~}~~~~~\\
      - 7 = 2 x + 6\\
      \tmmathbf{\underline{- 6 ~~~~~~- 6}}\\
      - 13 = 2 x\\
      ~~~\tmmathbf{\overline{2} ~~~~~ \overline{2}}\\
      -\frac{13}{2} = x\\
    			~

		\end{array} \tmop{~or~} \begin{array}{l}
      ~~2 x - 7 = - (4 x + 6)\\
      ~~2 x - 7 = - 4 x - 6\\
      \tmmathbf{\underline{+ 4 x ~~~~~~~+ 4 x}}\\
      ~~6 x - 7 = - 6\\
      ~~~~~~\tmmathbf{ \underline{+ 7 ~~+ 7}}\\
      ~~6 x = 1\\
      ~~\tmmathbf{\overline{6} ~~~~~ \overline{6}}\\
      ~~x = \frac{1}{6}
    \end{array} &  & 
  \end{eqnarray*}
   This gives us our two solutions, $x = -\frac{13}{2}$ or $x = \frac{1}{6}$.
\end{example}