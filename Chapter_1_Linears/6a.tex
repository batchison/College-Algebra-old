\subsection{Numbers and Geometry}\pp

 {\tmstrong{Objective: Solve number and geometry problems by creating and
solving a linear equation. }}\pp

 Word problems can be tricky. Often it takes a bit of practice to convert the
English sentence into a mathematical sentence. This is what we will focus on
here with some basic number problems, geometry problems, and parts problems.\pp

 A few important phrases are described below that can give us clues for how to
set up a problem.
\begin{itemizedot}
  \item {\tmstrong{A number}} (or unknown, an integer value, etc) often becomes our
  variable
  
  \item {\tmstrong{Is}} (or other forms of is: was, will be, are, etc) often
  represents equals (=)
  
  $x$ is $5$ becomes $x = 5$
  
  \item {\tmstrong{More than}} often represents addition and is usually built
  backwards, writing the second part plus the first
  
   Three more than a number becomes $x + 3$
  
  \item {\tmstrong{Less than}} often represents subtraction and is usually
  built backwards as well, writing the second part minus the first
  
   Four less than a number becomes $x - 4$ 
\end{itemizedot}
 Using these key phrases we can take a number problem and set up and solve an equation.

\pagebreak

\begin{example}\label{Lin74}~~~If 28 less than five times a certain number is 232. What is the number?
  \begin{eqnarray*}
    5 x - 28~ &  & \tmop{Subtraction} \tmop{is} \tmop{built} \tmop{backwards},
    \tmop{multiply} \tmop{the} \tmop{unknown} \tmop{by} 5\\
    5 x - 28 = 232~ &  & \tmop{``Is''} \tmop{translates} \tmop{to} \tmop{equals}\\
    \tmmathbf{\underline{+ 28 ~~+ 28}} &  & \tmop{Add} 28 \tmop{to} \tmop{both}
    \tmop{sides}\\
    5 x = 260~ &  & \tmop{The} \tmop{variable} \tmop{is} \tmop{multiplied}
    \tmop{by} 5\\
    \tmmathbf{\overline{~5~} ~~~~~ \overline{~5~}}~ &  & \tmop{Divide} \tmop{both}
    \tmop{sides} \tmop{by} 5\\
    x = 52~ &  & \tmop{The} \tmop{number} \tmop{is} 52
  \end{eqnarray*}
\end{example}

 This same idea can be extended to a more involved problem as shown in the next
example.\pp

\begin{example}\label{Lin75}~~~Fifteen more than three times a number is the same as ten less than six
  times the number. What is the number?
  \begin{eqnarray*}
    3 x + 15 &  & \tmop{First}, \tmop{addition} \tmop{is} \tmop{built}
    \tmop{backwards}\\
    6 x - 10 &  & \tmop{Then}, \tmop{subtraction} \tmop{is} \tmop{also}
    \tmop{built} \tmop{backwards}\\
    3 x + 15 = 6 x - 10 &  & \tmop{``Is''} \tmop{between} \tmop{the} \tmop{parts}
    \tmop{tells} \tmop{us} \tmop{they} \tmop{must} \tmop{be} \tmop{equal}\\
    \tmmathbf{\underline{- 3 x ~~~~~~~- 3 x}}~~~~~ &  & \tmop{Subtract} 3 x, \tmop{~so}
    \tmop{variable} \tmop{is} \tmop{all} \tmop{on} \tmop{one} \tmop{side}\\
    15 = 3 x - 10 &  & \tmop{Now} \tmop{we} \tmop{have~a} \tmop{two} -
    \tmop{step} \tmop{equation}\\
    \tmmathbf{\underline{+ 10 ~~~~~+ 10}} &  & \tmop{Add} 10 \tmop{to} \tmop{both}
    \tmop{sides}\\
    25 = 3 x &  & \tmop{The} \tmop{variable} \tmop{is} \tmop{multiplied}
    \tmop{by} 3\\
    \tmmathbf{\overline{3} ~~~~~ \overline{3}}~ &  & \tmop{Divide} \tmop{both}
    \tmop{sides} \tmop{by} 3\\
    \frac{25}{3} = x &  & \tmop{Our} \tmop{number} \tmop{is} \frac{25}{3}
  \end{eqnarray*}
\end{example}

 Another type of number problem involves consecutive integers.
{\tmstrong{Consecutive integers}} are whole numbers that come one after the other,
such as 3, 4, 5. If we are looking for several consecutive integers it is
important to first identify what they look like with variables, before we set
up the equation. This is shown in the following example.\pp

\begin{example}\label{Lin76}~~~The sum of three consecutive integers is 93. What are the integers?
  \begin{eqnarray*}
    \tmop{First} x~ &  & \tmop{Make} \tmop{the} \tmop{first} \tmop{number} x\\
    \tmop{Second} x + 1~ &  & \tmop{To} \tmop{get} \tmop{the} \tmop{next}
    \tmop{number} \tmop{we} \tmop{go} \tmop{up} \tmop{one} \tmop{or} + 1\\
    \tmop{Third} x + 2~ &  & \tmop{Add} \tmop{another} 1 (2 \tmop{total})
    \tmop{to} \tmop{get} \tmop{the} \tmop{third}
  \end{eqnarray*}
  \begin{eqnarray*}
		F + S + T = 93~ &  & \tmop{First~} (F) \tmop{plus} \tmop{Second~} (S)
    \tmop{plus}\\
		& & ~~~\tmop{Third~} (T) \tmop{equals} 93\\
    (x) + (x + 1) + (x + 2) = 93~ &  & \tmop{Replace} F, S \tmop{~and~} T \tmop{with}\\
		& & ~~~\tmop{their} \tmop{respective} \tmop{expressions}\\
  x + x + 1 + x + 2 = 93~&  & \tmop{Here} \tmop{the} \tmop{parentheses}
    \tmop{aren't} \tmop{needed}\\
    3 x + 3 = 93~ &  & \tmop{Combine} \tmop{like} \tmop{terms} x + x + x
    \tmop{and} 2 + 1\\
    \tmmathbf{\underline{- 3 ~~- 3}} &  & \tmop{Add} 3 \tmop{to} \tmop{both}
    \tmop{sides}\\
    3 x = 90~ &  & \tmop{The} \tmop{variable} \tmop{is} \tmop{multiplied}
    \tmop{by} 3\\
    \tmmathbf{\overline{3} ~~~~~ \overline{3}}~~ &  & \tmop{Divide} \tmop{both}
    \tmop{sides} \tmop{by} 3\\
    x = 30~ &  & \tmop{Our} \tmop{solution} \tmop{for} x\\
    \tmop{First~is~} 30~ &  & \tmop{Replace} x \tmop{in} \tmop{our}
    \tmop{original}~ \tmop{list} \tmop{with} 30\\
    \tmop{Second~is~} (30) + 1 = 31~ &  & \tmop{The} \tmop{numbers} \tmop{are} 30,
    31, \tmop{and} 32\\
    \tmop{Third~is~} (30) + 2 = 32~ &  & 
  \end{eqnarray*}
\end{example}

 Sometimes we will work with consecutive even or odd integers, rather than just
consecutive integers. When we had consecutive integers, we only had to add 1
to get to the next integer so we had $x$, $x + 1$, and $x + 2$ for our first,
second, and third integer respectively.\pp

Sets of even (or odd) integers, however, are
spaced apart by two. So if we want three consecutive even integers, if the
first is $x$, the next integer would be $x + 2$, then finally add two more to
get the third, $x + 4$. The same is true for consecutive odd integers, if the
first is $x$, the next will be $x + 2$, and the third would be $x + 4$. It is
important to note that we are still adding 2 and 4 even when the integers are
odd. This is because the phrase ``odd'' is referring to our $x$, not to what is
added to the integers. Consider the next two examples.

\begin{example}\label{Lin77}~~~The sum of three consecutive even integers is 246. What are the integers?
  \begin{eqnarray*}
    \tmop{First} x &  & \tmop{Make} \tmop{the} \tmop{first} x\\
    \tmop{Second} x + 2 &  & \tmop{Even} \tmop{numbers}, \tmop{so} \tmop{we}
    \tmop{add} 2 \tmop{to} \tmop{get} \tmop{the} \tmop{next}\\
    \tmop{Third} x + 4 &  & \tmop{Add} 2 \tmop{more} (4 \tmop{total})
    \tmop{to} \tmop{get} \tmop{the} \tmop{third}\\
    F + S + T = 246 &  & \tmop{Sum} \tmop{means} \tmop{add} \tmop{First} (F)
    \tmop{plus}\\
		& & ~~~\tmop{Second} (S) \tmop{plus} \tmop{Third} (T)
    \end{eqnarray*}
    \begin{eqnarray*}
	(x) + (x + 2) + (x + 4) = 246 &  & \tmop{Replace} \tmop{each} F, S,
    \tmop{and} T \tmop{with}\\
		& & ~~~\tmop{their} \tmop{respective} \tmop{expressions}\\  
  	x + x + 2 + x + 4 = 246 &  & \tmop{Here} \tmop{the} \tmop{parentheses}
    \tmop{are} \tmop{not} \tmop{needed}\\
    3 x + 6 = 246 &  & \tmop{Combine} \tmop{like} \tmop{terms} x + x + x
    \tmop{and} 2 + 4\\
    \tmmathbf{\underline{- 6 ~~- 6}} &  & \tmop{Subtract} 6 \tmop{from}
    \tmop{both} \tmop{sides}\\
    3 x = 240 &  & \tmop{The} \tmop{variable} \tmop{is} \tmop{multiplied}
    \tmop{by} 3\\
    \tmmathbf{\overline{~3~} ~~~ \overline{~3~}}~ &  & \tmop{Divide} \tmop{both}
    \tmop{sides} \tmop{by} 3\\
%  \end{eqnarray*}
%  \begin{eqnarray*}
    x = 80 &  & \tmop{Our} \tmop{solution} \tmop{for} x\\
    \tmop{First~is~} 80 &  & \tmop{Replace} x \tmop{in} \tmop{the}
    \tmop{original} \tmop{list} \tmop{with} 80\\
    \tmop{Second~is~} (80) + 2 = 82 &  & \tmop{The} \tmop{numbers} \tmop{are} 80,
    82, \tmop{and} 84\\
    \tmop{Third~is~} (80) + 4 = 84 &  & 
  \end{eqnarray*}
\end{example}

\begin{example}\label{Lin78}~~~Find three consecutive odd integers so that the sum of twice the first, the
  second and three times the third is 152.
  \begin{eqnarray*}
    \tmop{First} x~ &  & \tmop{Make} \tmop{the} \tmop{first} x\\
    \tmop{Second} x + 2~ &  & \tmop{Odd} \tmop{numbers} \tmop{so} \tmop{we}
    \tmop{add} 2~ (\tmop{same} \tmop{as} \tmop{even} !)\\
    \tmop{Third} x + 4~ &  & \tmop{Add} 2 \tmop{more}~ (4 \tmop{total})
    \tmop{to} \tmop{get} \tmop{the} \tmop{third}\\
    2 F + S + 3 T = 152~ &  & \tmop{Twice} \tmop{the} \tmop{first} \tmop{gives}
    2 F,\\
		& & ~~~\tmop{three} \tmop{times} \tmop{the} \tmop{third}
    \tmop{gives} 3 T\\
    2 (x) + (x + 2) + 3 (x + 4) = 152~ &  & \tmop{Replace~} F, S, \tmop{~and~} T
    \tmop{with}\\
		& & \tmop{their} \tmop{respective} \tmop{expressions}\\
    2 x + x + 2 + 3 x + 12 = 152~ &  & \tmop{Distribute} \tmop{through}
    \tmop{parentheses}\\
    6 x + 14 = 152~ &  & \tmop{Combine} \tmop{like} \tmop{terms} 2 x + x + 3 x
    \tmop{and} 2 + 14\\
    \tmmathbf{\underline{- 14 ~- 14}} &  & \tmop{Subtract} 14 \tmop{from}
    \tmop{both} \tmop{sides}\\
    6 x = 138~ &  & \tmop{Variable} \tmop{is} \tmop{multiplied} \tmop{by} 6\\
    \tmmathbf{\overline{~6~} ~~~~ \overline{~6~}}~ &  & \tmop{Divide} \tmop{both}
    \tmop{sides} \tmop{by} 6\\
    x = 23~ &  & \tmop{Our} \tmop{solution} \tmop{for} x\\
    \tmop{First~is~} 23~ &  & \tmop{Replace} x \tmop{with} 23 \tmop{in} \tmop{the}
    \tmop{original} \tmop{list}\\
    \tmop{Second~is~} (23) + 2 = 25~ &  & \tmop{The} \tmop{numbers} \tmop{are} 23,
    25, \tmop{and} 27\\
    \tmop{Third~is~} (23) + 4 = 27~ &  & 
  \end{eqnarray*}
\end{example}

 When we started with our first, second, and third integers for both even and
odd we had $x$, $x + 2$, and $x + 4$. The numbers added (2 and 4) do not change with successive odds
or evens.  It is our answer for $x$ that will be odd or even.\pp

 Another example of translating English sentences to mathematical sentences
comes from geometry. A well known property of triangles is that all three
angles will always add to 180 degrees. For example, the first angle may be 50 degrees,
the second 30 degrees, and the third 100 degrees. If you add these together,
$50 + 30 + 100 = 180$. We can use this property to find angles of triangles.\pp

 {\tmstrong{World View Note: }}German mathematician Bernhart Thibaut in 1809
tried to prove that the angles of a triangle add to 180 degrees without using Euclid's
parallel postulate (a point of much debate in mathematical history). He created a
proof, but it was shown to contain an error.

\begin{example}\label{Lin79}~~~The second angle of a triangle is double the first. The third angle is 40 degrees
  less than the first. Find the measure of all three angles.
  \begin{eqnarray*}
    \tmop{First~angle~is~} x~~ &  & \tmop{With} \tmop{nothing} \tmop{known} \tmop{about}
    \tmop{the} \tmop{first~angle}\\
		& & ~~~\tmop{we} \tmop{label} \tmop{it~} x\\
    \tmop{Second~angle~is~} 2 x~~ &  & \tmop{The} \tmop{second~angle} \tmop{is} \tmop{double}
    \tmop{the} \tmop{first}\\
    \tmop{Third~angle~is~} x - 40~~ &  & \tmop{The} \tmop{third~angle} \tmop{is} 40^{\circ} \tmop{less}
    \tmop{than} \tmop{the} \tmop{first}\\
    F + S + T = 180~~ &  & \tmop{All} \tmop{three} \tmop{angles} \tmop{add}
    \tmop{to} 180^{\circ}\\
    (x) + (2 x) + (x - 40) = 180~~ &  & \tmop{Replace} F, S, \tmop{and} T
    \tmop{with} \tmop{the} \tmop{labeled} \tmop{values}\\
    x + 2 x + x - 40 = 180~~ &  & \tmop{Here} \tmop{the} \tmop{parentheses}
    \tmop{are} \tmop{not} \tmop{needed}\\
    4 x - 40 = 180~~ &  & \tmop{Combine} \tmop{like} \tmop{terms}, x + 2 x + x\\
    \underline{\tmmathbf{+ 40 ~~+ 40}} &  & \tmop{Add} 40^{\circ} \tmop{to} \tmop{both}
    \tmop{sides}\\
    4 x = 220~~ &  & \tmop{The} \tmop{variable} \tmop{is} \tmop{multiplied}
    \tmop{by} 4\\
    \tmmathbf{\overline{~4~} ~~~ \overline{~4~}}~~~ &  & \tmop{Divide} \tmop{both}
    \tmop{sides} \tmop{by} 4\\
    x = 55~~ &  & \tmop{Our} \tmop{solution} \tmop{for} x\\
    \tmop{First~is~} 55~~ &  & \tmop{Replace} x \tmop{with} 55^{\circ} \tmop{in} \tmop{the}
    \tmop{original} \tmop{list} \tmop{of} \tmop{angles}\\
    \tmop{Second~is~} 2 (55) = 110~~ &  & \tmop{Our} \tmop{angles} \tmop{are} 55^{\circ},
    110^{\circ}, \tmop{and} 15^{\circ}\\
    \tmop{Third~is~} (55) - 40 = 15~~ &  & 
  \end{eqnarray*}
\end{example}

 Another geometry problem involves perimeter or the distance around an object.
For example, consider a rectangle having a length of 8 units and a width of 3 units. There
are two lengths and two widths in a rectangle (opposite sides) so we add $8 +
8 + 3 + 3 = 22$. As there are two lengths and two widths in a rectangle an
alternative to find the perimeter of a rectangle is to use the formula $P = 2
L + 2 W$. So for the rectangle of length 8 units and width 3 units the formula would give,
$P = 2 (8) + 2 (3) = 16 + 6 = 22$. With problems that we will consider here
the formula $P = 2 L + 2 W$ will be used.

\begin{example}\label{Lin80}~~~The perimeter of a rectangle is 44 units. The width is 5 units less than double the
  length. Find the dimensions of the rectangle.
  \begin{eqnarray*}
    \tmop{Length} x &  & \tmop{We} \tmop{will} \tmop{make} \tmop{the}
    \tmop{length} x\\
    \tmop{Width} 2 x - 5 &  & \tmop{Width} \tmop{is} \tmop{five} \tmop{less}
    \tmop{than} \tmop{two} \tmop{times} \tmop{the} \tmop{length}\\
    P = 2 L + 2 W &  & \tmop{The} \tmop{formula} \tmop{for} \tmop{perimeter}
    \tmop{of} \tmop{a~rectangle}\\
    (44) = 2 (x) + 2 (2 x - 5) &  & \tmop{Replace} P, L, \tmop{and} W
    \tmop{with} \tmop{labeled} \tmop{values}\\
    44 = 2 x + 4 x - 10 &  & \tmop{Distribute} \tmop{through}
    \tmop{parentheses}\\
    44 = 6 x - 10 &  & \tmop{Combine} \tmop{like} \tmop{terms} 2 x + 4 x\\
    \tmmathbf{\underline{+ 10 ~~~~~+ 10}} &  & \tmop{Add} 10 \tmop{units~to} \tmop{both}
    \tmop{sides}\\
    54 = 6 x &  & \tmop{The} \tmop{variable} \tmop{is} \tmop{multiplied}
    \tmop{by} 6\\
    \tmmathbf{\overline{6} ~~~~~ \overline{6}}~ &  & \tmop{Divide} \tmop{both}
    \tmop{sides} \tmop{by} 6\\
    9 = x &  & \tmop{Our} \tmop{solution} \tmop{for} x\\
    \tmop{Length~is~} 9 \tmop{units}&  & \tmop{Replace} x \tmop{with} 9 \tmop{in} \tmop{the}
    \tmop{original} \tmop{list} \tmop{of} \tmop{sides}\\
    \tmop{Width~is~} 2 (9) - 5 = 13 \tmop{units}&  & %\tmop{The} \tmop{dimensions} \tmop{of}
    %\tmop{the} \tmop{rectangle}\\
		%& & ~~~\tmop{are} 9 \tmop{units~by} 13 \tmop{units}.
  \end{eqnarray*}
\begin{center}
The dimensions of the rectangle are 9 units by 13 units.
\end{center}
\end{example}

 We have seen that it is important to start by clearly labeling the variables in
a short list before we begin to solve the problem. This is important in all
word problems involving variables, not just consecutive integers or geometry
problems. This is shown in the following example.

\begin{example}\label{Lin81}~~~A sofa and a love seat have a combined cost of $\$444$. The sofa costs
  double the love seat. How much do they each cost?
  \begin{eqnarray*}
    \tmop{Love} \tmop{seat~cost~is~} x &  & \tmop{With} \tmop{no} \tmop{information}
    \tmop{known~about} \tmop{the}\\
		& & ~~~\tmop{love} \tmop{seat}, \tmop{we~label~it} x\\
    \tmop{Sofa~cost~is~} 2 x &  & \tmop{Sofa} \tmop{is} \tmop{double} \tmop{the}
  	\tmop{love} \tmop{seat}, \tmop{so} \tmop{we} \tmop{multiply} \tmop{by} 2
  \end{eqnarray*}
  \begin{eqnarray*}
	  S + L = 444 &  & \tmop{Together} \tmop{they} \tmop{cost} \$444, \tmop{so}
    \tmop{we} \tmop{add}\\
    (x) + (2 x) = 444 &  & \tmop{Replace} S \tmop{and} L \tmop{with}
    \tmop{labeled} \tmop{values}\\
    &  & \tmop{Parentheses} \tmop{are} \tmop{not} \tmop{needed},\\
		3 x = 444 & & ~~~ \tmop{combine} \tmop{like} \tmop{terms} x + 2x\\
    \tmmathbf{\overline{~3~} ~~~~~ \overline{~3~}} &  & \tmop{Divide} \tmop{both}
    \tmop{sides} \tmop{by} 3\\
    x = 148 &  & \tmop{Our} \tmop{solution} \tmop{for} x\\
    \tmop{Love} \tmop{seat~cost~is~} \$148 &  & \tmop{Replace} x \tmop{with} 148
    \tmop{in} \tmop{the} \tmop{original} \tmop{list}\\
    \tmop{Sofa~cost~is~} 2 (148) = \$296 &  & \tmop{The} \tmop{love} \tmop{seat}
    \tmop{costs} \$148 \tmop{and} \tmop{the} \tmop{sofa} \tmop{costs}
    \$296
  \end{eqnarray*}
\end{example}

 Be careful on problems such as these. Many students see the phrase ``double''
and believe that means we only have to divide the $\$444$ by 2 and get
$\$222$ for one or both of the prices. As you can see this will not
work. By clearly labeling the variables in the original list we know exactly
how to set up and solve these problems.
