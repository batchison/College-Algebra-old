\subsection{Equations Containing Fractions}\par
~\par
{\tmstrong{Objective: Solve linear equations with rational coefficients by
multiplying by the least common denominator to clear the fractions. }}\par
~\par
 Often when solving linear equations we will need to work with an equation with
fraction coefficients. We can solve these problems as we have in the past.
This is demonstrated in our next example.

\begin{example}\label{Lin29}
	
  \begin{eqnarray*}
    \frac{3}{4} x - \frac{7}{2} = \frac{5}{6}~~ &  & \tmop{Focus} \tmop{on}
    \tmop{subtraction}\\
    &  & \\
    \underline{\tmmathbf{+ \frac{7}{2} ~~+ \frac{7}{2}}} &  & \tmop{Add}
    \frac{7}{2} \tmop{to} \tmop{both} \tmop{sides}
  \end{eqnarray*}
   Notice we will need to get a common denominator to add $\frac{5}{6} +
  \frac{7}{2}$. Notice we have a common denominator of $6$. So we build up the
  denominator, $\frac{7}{2} \left( \frac{3}{3} \right) = \frac{21}{6}$, and we
  can now add the fractions:
  \begin{eqnarray*}
    \frac{3}{4} x - \frac{21}{6} = \frac{5}{6}~~~ &  & \tmop{Same}
    \tmop{problem}, \tmop{with} \tmop{common} \tmop{denominator} 6\\
    &  & \\
    \tmmathbf{\underline{+ \frac{21}{6} ~~+ \frac{21}{6}}} &  & \tmop{Add}
    \frac{21}{6} \tmop{to} \tmop{both} \tmop{sides}\\
    &  & \\
    \frac{3}{4} x = \frac{26}{6}~~~ &  & \tmop{Reduce} \frac{26}{6} \tmop{to}
    \frac{13}{3}\\
    &  & \\
    \frac{3}{4} x = \frac{13}{3}~~~ &  & \tmop{Focus} \tmop{on}
    \tmop{multiplication} \tmop{by} \frac{3}{4}
  \end{eqnarray*}

   We can get rid of $\frac{3}{4}$ by dividing both sides by $\frac{3}{4}$.\pp
  Dividing by a fraction is the same as multiplying by the reciprocal, so we
  will multiply both sides by $\frac{4}{3}$.
  \begin{eqnarray*}
    \tmmathbf{\left( \frac{4}{3} \right)} \frac{3}{4} x = \frac{13}{3}
    \tmmathbf{\left( \frac{4}{3} \right)} &  & \tmop{Multiply} \tmop{by}
    \tmop{reciprocal}\\
    x = \frac{52}{9}  &  & \tmop{Our} \tmop{solution}
  \end{eqnarray*}
\end{example}

 While this process does help us arrive at the correct solution, the fractions
can make the process quite difficult. This is why we have an alternate method
for dealing with fractions - clearing fractions. Clearing fractions is nice as
it gets rid of the fractions for the majority of the problem. We can easily
clear the fractions by finding the LCD and multiplying each term by the LCD.
This is shown in the next example, the same problem as our first example, but
this time we will solve by clearing fractions.

\begin{example}\label{Lin30}
  \begin{eqnarray*}
    \frac{3}{4} x - \frac{7}{2} = \frac{5}{6} &  & \tmop{LCD} = 12,
    \tmop{multiply} \tmop{each} \tmop{term} \tmop{by} 12\\
    &  & \\
    \frac{\tmmathbf{(12)} 3}{4} x - \frac{\tmmathbf{(12)} 7}{2} =
    \frac{\tmmathbf{(12)} 5}{6} &  & \tmop{Reduce} \tmop{each} 12 \tmop{with}
    \tmop{denominators}\\
    &  & \\
    \tmmathbf{(3)} 3 x - \tmmathbf{(6)} 7 = \tmmathbf{(2)} 5 &  &
    \tmop{Multiply} \tmop{out} \tmop{each} \tmop{term}\\
    9 x - 42 = 10~~~ &  & \tmop{Focus} \tmop{on} \tmop{subtraction} \tmop{by}
    42\\
    \tmmathbf{\underline{+ 42 ~~+ 42}} &  & \tmop{Add} 42 \tmop{to} \tmop{both}
    \tmop{sides}\\
    9 x = 52~~~ &  & \tmop{Focus} \tmop{on} \tmop{multiplication} \tmop{by} 9\\
    \tmmathbf{\overline{9} ~~~~~ \overline{9}}~~~~ &  & \tmop{Divide} \tmop{both}
    \tmop{sides} \tmop{by} 9\\
    x = \frac{52}{9}~~~ &  & \tmop{Our} \tmop{solution}
  \end{eqnarray*}
\end{example}

 The next example illustrates this as well. Notice the $2$ isn't a fraction in
the original equation, but to solve it we put the $2$ over $1$ to make it a
fraction.

\begin{example}\label{Lin31}
  
  \begin{eqnarray*}
    \frac{2}{3} x - 2 = \frac{3}{2} x + \frac{1}{6} &  & \tmop{LCD} = 6,
    \tmop{multiply} \tmop{each} \tmop{term} \tmop{by} 6\\
    &  & \\
    \frac{\tmmathbf{(6)} 2}{3} x - \frac{\tmmathbf{(6)} 2}{1} =
    \frac{\tmmathbf{(6)} 3}{2} x + \frac{\tmmathbf{(6)} 1}{6} &  &
    \tmop{Reduce} 6 \tmop{with} \tmop{each} \tmop{denominator}\\
    &  & \\
    \tmmathbf{(2)} 2 x - \tmmathbf{(6)} 2 = \tmmathbf{(3)} 3 x +
    \tmmathbf{(1)} 1 &  & \tmop{Multiply} \tmop{out} \tmop{each} \tmop{term}
  \end{eqnarray*}
  \begin{eqnarray*}
    4 x - 12 = 9 x + 1 &  & \tmop{Notice} \tmop{variable} \tmop{on}
    \tmop{both} \tmop{sides}\\
    \underline{\tmmathbf{- 4 x ~~~~~~- 4 x}}~~~~  &  & \tmop{Subtract} 4 x \tmop{from}
    \tmop{both} \tmop{sides}\\
    - 12 = 5 x + 1 &  & \tmop{Focus} \tmop{on} \tmop{addition} \tmop{of} 1\\
    \tmmathbf{\underline{- 1 ~~~~~~- 1}} &  & \tmop{Subtract} 1 \tmop{from}
    \tmop{both} \tmop{sides}\\
    - 13 = 5 x &  & \tmop{Focus} \tmop{on} \tmop{multiplication} \tmop{of} 5\\
    \tmmathbf{\overline{5} ~~~~~ \overline{5}}~  &  & \tmop{Divide} \tmop{both}
    \tmop{sides} \tmop{by} 5\\
    - \frac{13}{5} = x &  & \tmop{Our} \tmop{solution}
  \end{eqnarray*}
\end{example}

 We can use this same process if there are parenthesis in the problem. We will
first distribute the coefficient in front of the parenthesis, then clear the
fractions. This is seen in the following example.

\begin{example}\label{Lin32}
  \begin{eqnarray*}
    \frac{3}{2} \left( \frac{5}{9} x + \frac{4}{27} \right) = 3 &  &
    \tmop{Distribute} \frac{3}{2} \tmop{through} \tmop{parenthesis},
    \tmop{reducing} \tmop{if} \tmop{possible}\\
    \frac{5}{6} x + \frac{2}{9} = 3 &  & \tmop{LCD} = 18, \tmop{multiply}
    \tmop{each} \tmop{term} \tmop{by} 18\\
    \frac{\tmmathbf{(18)} 5}{6} x + \frac{\tmmathbf{(18)} 2}{9} =
    \frac{\tmmathbf{(18)} 3}{9} &  & \tmop{Reduce} 18 \tmop{with} \tmop{each}
    \tmop{denominator}\\
    \tmmathbf{(3)} 5 x + \tmmathbf{(2)} 2 = \tmmathbf{(18)} 3 &  &
    \tmop{Multiply} \tmop{out} \tmop{each} \tmop{term}\\
    15 x + 4 = 54 &  & \tmop{Focus} \tmop{on} \tmop{addition} \tmop{of} 4\\
    \underline{\tmmathbf{- 4 ~- 4}} &  & \tmop{Subtract} 4 \tmop{from}
    \tmop{both} \tmop{sides}\\
    15 x = 50 &  & \tmop{Focus} \tmop{on} \tmop{multiplication} \tmop{by} 15\\
     \tmmathbf{\overline{15} ~~~~ \overline{15}} &  & \tmop{Divide} \tmop{both}
    \tmop{sides} \tmop{by} 15, \tmop{reduce} \tmop{on} \tmop{right}
    \tmop{side}\\
    x = \frac{10}{3} &  & \tmop{Our} \tmop{solution}
  \end{eqnarray*}
\end{example}

 While the problem can take many different forms, the pattern to clear the
fraction is the same, after distributing through any parentheses we multiply
each term by the LCD and reduce. This will give us a problem with no fractions
that is much easier to solve. The following example again illustrates this
process.

\begin{example}\label{Lin33}
  \begin{eqnarray*}
    \frac{3}{4} x - \frac{1}{2} = \frac{1}{3} (\frac{3}{4} x + 6) -
    \frac{7}{2} &  & \tmop{Distribute} \frac{1}{3}, \tmop{reduce} \tmop{if}
    \tmop{possible}\\
    &  & \\
    \frac{3}{4} x - \frac{1}{2} = \frac{1}{4} x + 2 - \frac{7}{2} &  &
    \tmop{LCD} = 4, \tmop{multiply} \tmop{each} \tmop{term} \tmop{by} 4\\
    &  & \\
    \frac{\tmmathbf{(4)} 3}{4} x - \frac{\tmmathbf{(4)} 1}{2} =
    \frac{\tmmathbf{(4)} 1}{4} x + \frac{\tmmathbf{(4)} 2}{1} -
    \frac{\tmmathbf{(4)} 7}{2} &  & \tmop{Reduce} 4 \tmop{with} \tmop{each}
    \tmop{denominator}\\
    &  & \\
    \tmmathbf{(1)} 3 x - \tmmathbf{(2)} 1 = \tmmathbf{(1)} 1 x +
    \tmmathbf{(4)} 2 - \tmmathbf{(2)} 7 &  & \tmop{Multiply} \tmop{out}
    \tmop{each} \tmop{term}\\
    3 x - 2 = x + 8 - 14 &  & \tmop{Combine} \tmop{like} \tmop{terms} 8 - 14\\
    %\end{eqnarray*}
		%\begin{eqnarray*}
		%
		3 x - 2 = x - 6 &  & \tmop{Notice} \tmop{variable} \tmop{on} \tmop{both}
    \tmop{sides}\\
    \tmmathbf{\underline{- x ~~~~~~- x}}~~~~  &  & \tmop{Subtract} x \tmop{from}
    \tmop{both} \tmop{sides}\\
    2 x - 2 = - 6 &  & \tmop{Focus} \tmop{on} \tmop{subtraction} \tmop{by} 2\\
    \tmmathbf{\underline{+ 2 ~~+ 2}} &  & \tmop{Add} 2 \tmop{to} \tmop{both}
    \tmop{sides}\\
    2 x = - 4 &  & \tmop{Focus} \tmop{on} \tmop{multiplication} \tmop{by} 2\\
    \tmmathbf{\overline{2} ~~~~~~~ \overline{2}} &  & \tmop{Divide} \tmop{both}
    \tmop{sides} \tmop{by} 2\\
    x = - 2 &  & \tmop{Our} \tmop{solution}
  \end{eqnarray*}
\end{example}

   {\tmstrong{World View Note:}} The Egyptians were among the first to study
  fractions and linear equations. The most famous mathematical document from
  Ancient Egypt is the Rhind Papyrus where the unknown variable was called
  ``heap''