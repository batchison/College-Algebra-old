\documentclass[11pt]{book}
%\oddsidemargin 0in
%\evensidemargin 0in
%\marginparwidth 0in

%\usepackage[top=1in
%,left=1.5in,
%bottom=1in,%right=1in,textheight=8.5in,textwidth=6in
%]{geometry}

\usepackage{amsfonts,amssymb,amsmath,amsthm,fancyhdr,supertabular,longtable,hhline}
\usepackage{colortbl}
\usepackage{docmute}
\usepackage{import, multicol,boxedminipage}
\usepackage{chapterfolder}
\usepackage[metapost,truebbox]{mfpic}
\usepackage[pdflatex]{graphicx}
\usepackage{graphics}
\usepackage{pgf, tikz}
\usepackage[matrix,arrow,curve]{xy}
\usepackage{setspace}
\usepackage{makeidx}
\usepackage{nomencl}
\usepackage[english]{babel}
\usepackage[colorlinks, hyperindex, plainpages=false, linkcolor=blue, urlcolor=blue, pdfpagelabels]{hyperref}
\usepackage[all]{hypcap}
\usepackage{cancel}
\usepackage{sectsty}
\usepackage{textcomp}
\usepackage{array}
\usepackage{vwcol}
\usepackage{cancel}
\usepackage[
type={CC},
modifier={by-nc-sa},
version={4.0},
]{doclicense}
%\usepackage{bbm}

%First appears in II.1.1
\newcommand{\tmdummy}{$\mbox{}$}
\newcommand{\tmmathbf}[1]{\ensuremath{\boldsymbol{#1}}}
\newcommand{\tmop}[1]{\ensuremath{\operatorname{#1}}}
\newcommand{\tmstrong}[1]{\textbf{#1}}
%{\theorembodyfont{\rmfamily}
%First appears in II.1.?
\newcommand{\tmem}[1]{{\em #1\/}}
%\newenvironment{enumeratenumeric}{\begin{enumerate}[1.] }{\end{enumerate}}
%First appears in II.1.6a
\newenvironment{itemizedot}{\begin{itemize} \renewcommand{\labelitemi}{$\bullet$}\renewcommand{\labelitemii}{$\bullet$}\renewcommand{\labelitemiii}{$\bullet$}\renewcommand{\labelitemiv}{$\bullet$}}{\end{itemize}}


\allsectionsfont{\mdseries \scshape}
\definecolor{ResultColor}{gray}{1.0} %SZ set to gray, 0.9
\theoremstyle{definition}  % this prevents the text in definitions, theorems, and corollaries from being italicized
\newtheorem*{defn}{Definition}%[chapter] % this item asterisked for numbering purposes (across all chapters).
%\newtheorem*{thm}{Theorem}%[chapter] % this item asterisked for numbering purposes (across all chapters).
\newtheorem{thm}{Theorem}[chapter] % this item asterisked for numbering purposes (across all chapters).
\newtheorem*{cor}%[thm]
{Corollary} % this item asterisked for numbering purposes (across all chapters).
\newtheorem{eqn}{Equation}[chapter]
\newtheorem{ex}{Example}[chapter] % this is needed for all Stitz Zeager content.
\newtheorem{example}{Example}[chapter]
\newtheorem{fig}{\sc Figure}[chapter]
\setlength{\parindent}{0in}
\setlength{\extrarowheight}{2pt}
\newcommand{\bbm}{\begin{boxedminipage}{4.80in}} %SZ set to 6.41
\newcommand{\ebm}{\end{boxedminipage}}
\newcounter{HW}
\newcounter{HWindent}

\newcommand{\comment}[1]{}
\newcommand{\pp}{\par~\par}

\begin{document}
%\subsection{Practice Problems}

\subsection*{Notation and Basic Examples}                                

%In Exercises \ref{setfunctionfirst} - \ref{setfunctionlast}, determine whether or not the relation represents $y$ as a function of $x$.  
%\begin{enumerate}
\begin{multicols}{2}
1) Function\\
3) Function\\
5) Not a function\\
7) Function\\
9) Not a function\\
11) Function\\
13) Function\\
15) Function\\
17) Not a function\\
19) Function\\
21) Function\\
23) Function\\
25) Function\\
27) Function\\
29) Not a function\\
31) Function\\
33) Function\\
35) Function\\
37) Not a function\\
39) Not a function
\end{multicols}
%\item \{$(-3, 9)$, $\;(-2, 4)$, $\;(-1, 1)$, $\;(0, 0)$, $\;(1, 1)$, $\;(2, 4)$, $\;(3, 9)\}$ \label{setfunctionfirst}
%\item  $\left\{ (-3,0), (1,6), (2, -3), (4,2), (-5,6), (4, -9), (6,2) \right\}$
%\item  $\left\{ (-3,0), (-7,6), (5,5), (6,4), (4,9), (3,0) \right\}$
%\item  $\left\{ (1,2), (4,4), (9,6), (16,8), (25,10), (36, 12), \ldots \right\}$
%\item \{($x, y) \, | \, x$ is an odd integer, and $y$ is an even integer\}
%\item \{$(x, 1) \, | \, x$ is an irrational number\}
%\item \{$(1, 0)$, $\;(2, 1)$, $\;(4, 2)$, $\;(8, 3)$, $\;(16, 4)$, $\;(32, 5), \;$ \ldots\}
%\item \{$\ldots, \; (-3, 9)$, $\;(-2, 4)$, $\;(-1, 1)$, $\;(0, 0)$, $\;(1, 1)$, $\;(2, 4)$, $\;(3, 9), \;$ \ldots\}

%\setcounter{HW}{\value{enumi}}

%\end{enumerate}

\begin{multicols}{2}
41) $f(x)=\frac{2(x+3)}{4}$\\
43) $f(x)=\sqrt{2x+3}$\\
45) $f(x)=2\sqrt{x+3}$\\
47) $f(x)=\frac{4}{\sqrt{x-13}}$\\
49) $f(x)=\sqrt{\frac{4}{x}}-13$
\end{multicols}


51) $f(x)=3-4x$
\scriptsize
\begin{multicols}{3}
\begin{itemize}
\item $f(1)=-1$
\item $f(-3)=15$
\item $f\left(\frac{3}{2} \right)=-3$
\end{itemize}
\end{multicols}

\begin{multicols}{3}
\begin{itemize}
\item  $f(4x)=3-16x$
\item $4f(x)=12-16x$
\item $f(-x)=3+4x$
\end{itemize}
\end{multicols}

\begin{multicols}{3}
\begin{itemize}
\item  $f(x-4)=19-4x$
\item $f(x) - 4=-1-4x$
\item  $f\left(x^2\right)=3-4x^2$
\end{itemize}
\end{multicols}
\normalsize

53) $f(x)=x^2-3x+2$
\scriptsize
\begin{multicols}{3}
\begin{itemize}
\item $f(1)=0$
\item $f(-3)=20$
\item $f\left(\frac{3}{2} \right)=-\frac{1}{4}$
\end{itemize}
\end{multicols}

\begin{multicols}{3}
\begin{itemize}
\item  $f(4x)=16x^2-12x+2$
\item $4f(x)=4x^2-12x+8$
\item $f(-x)=x^2+3x+2$
\end{itemize}
\end{multicols}

\begin{multicols}{3}
\begin{itemize}
\item  $f(x-4)=x^2-11x+30$
\item $f(x) - 4=x^2-3x-2$
\item  $f\left(x^2\right)=x^4-3x^2+2$
\end{itemize}
\end{multicols}
\normalsize

55) $f(x)=\dfrac{x}{x-1}$
\scriptsize
\begin{multicols}{3}
\begin{itemize}
\item $f(1)=$undefined
\item $f(-3)=\frac{3}{4}$
\item $f\left(\frac{3}{2} \right)=3$
\end{itemize}
\end{multicols}

\begin{multicols}{3}
\begin{itemize}
\item  $f(4x)=\dfrac{4x}{4x-1}$
\item $4f(x)=\dfrac{4x}{x-1}$
\item $f(-x)=\dfrac{x}{x+1}$
\end{itemize}
\end{multicols}

\begin{multicols}{3}
\begin{itemize}
\item  $f(x-4)=\dfrac{x-4}{x-5}$
\item $f(x) - 4=\dfrac{-3x+4}{x-1}$
\item  $f\left(x^2\right)=\dfrac{x^2}{x^2-1}$
\end{itemize}
\end{multicols}
\normalsize

\newpage
57) $f(x)=0$
\scriptsize
\begin{multicols}{3}
\begin{itemize}
\item $f(1)=0$
\item $f(-3)=0$
\item $f\left(\frac{3}{2} \right)=0$
\end{itemize}
\end{multicols}

\begin{multicols}{3}
\begin{itemize}
\item  $f(4x)=0$
\item $4f(x)=0$
\item $f(-x)=0$
\end{itemize}
\end{multicols}

\begin{multicols}{3}
\begin{itemize}
\item  $f(x-4)=0$
\item $f(x) - 4=-4$
\item  $f\left(x^2\right)=0$
\end{itemize}
\end{multicols}
\normalsize

%\begin{multicols}{2}
%\begin{enumerate}
%\setcounter{enumi}{\value{HW}}

%\item  $f(x) = 2x+1$ \label{funcnotationbasicfirst} 
%\item  $f(x) = 3 - 4x$

%\setcounter{HW}{\value{enumi}}
%\end{enumerate}
%\end{multicols}

%\begin{multicols}{2}
%\begin{enumerate}
%\setcounter{enumi}{\value{HW}}

%\item $f(x) = 2 - x^2$
%\item $f(x) = x^2 - 3x + 2$

%\setcounter{HW}{\value{enumi}}
%\end{enumerate}
%\end{multicols}

%\begin{multicols}{2}
%\begin{enumerate}
%\setcounter{enumi}{\value{HW}}

%\item $f(x) = \sqrt{x-1}$
%\item $f(x) = \dfrac{x}{x-1}$

%\setcounter{HW}{\value{enumi}}
%\end{enumerate}
%\end{multicols}

%\begin{multicols}{2}
%\begin{enumerate}
%\setcounter{enumi}{\value{HW}}

%\item $f(x) = 6$
%\item $f(x) = 0$ \label{funcnotationbasiclast}

%\setcounter{HW}{\value{enumi}}
%\end{enumerate}
%\end{multicols}

%\newpage

%In Exercises \ref{secondfuncnotationbasicfirst} - \ref{secondfuncnotationbasiclast}, use the given function $f$ to find and simplify the following:
59) $f(x)=5-2x$
\scriptsize
\begin{multicols}{3}
\begin{itemize}
\item  $f(2)=1$
\item  $f(-2)=9$
\item  $f(2a)=5-4a$
\end{itemize}
\end{multicols}

\begin{multicols}{3}
\begin{itemize}
\item  $2 f(a)=10-4a$
\item $f(a+2)=1-2a$
\item $f(a) + f(2)=6-2a$
\end{itemize}
\end{multicols}

\begin{multicols}{3}
\begin{itemize}
\item  $f \left( \frac{2}{a} \right)=\frac{5a-4}{a}$
\item $\frac{f(a)}{2}=\frac{5}{2}-a$
\item  $f(a + h)=5-2a-2h$
\end{itemize}
\end{multicols}
\normalsize

61) $f(x)=3x^2+3x-2$
\scriptsize
\begin{multicols}{3}
\begin{itemize}
\item  $f(2)=16$
\item  $f(-2)=4$
\item  $f(2a)=12a^2+6a-2$
\end{itemize}
\end{multicols}

\begin{multicols}{3}
\begin{itemize}
\item  $2 f(a)=6a^2+6a-4$
\item $f(a+2)=3a^2+15a+16$
\item $f(a) + f(2)=3a^2+3a+14$
\end{itemize}
\end{multicols}

\begin{multicols}{3}
\begin{itemize}
\item  $f \left( \frac{2}{a} \right)=\frac{-2a^2+6a+12}{a^2}$
\item $\frac{f(a)}{2}=\frac{3}{2}a^2+\frac{3}{2}a-1$
\item  $f(a + h)=3a^2+6ah+3h^2+3a+3h-2$
\end{itemize}
\end{multicols}
\normalsize

63) $f(x)=1$
\scriptsize
\begin{multicols}{3}
\begin{itemize}
\item  $f(2)=1$
\item  $f(-2)=1$
\item  $f(2a)=1$
\end{itemize}
\end{multicols}

\begin{multicols}{3}
\begin{itemize}
\item  $2 f(a)=2$
\item $f(a+2)=1$
\item $f(a) + f(2)=2$
\end{itemize}
\end{multicols}

\begin{multicols}{3}
\begin{itemize}
\item  $f \left( \frac{2}{a} \right)=1$
\item $\frac{f(a)}{2}=\frac{1}{2}$
\item  $f(a + h)=1$
\end{itemize}
\end{multicols}
\normalsize

65) $f(x)=\dfrac{2}{x}$
\scriptsize
\begin{multicols}{3}
\begin{itemize}
\item  $f(2)=1$
\item  $f(-2)=-1$
\item  $f(2a)=\frac{1}{a}$
\end{itemize}
\end{multicols}

\begin{multicols}{3}
\begin{itemize}
\item  $2 f(a)=\frac{4}{a}$
\item $f(a+2)=\frac{2}{a+2}$
\item $f(a) + f(2)=\frac{2+a}{a}$
\end{itemize}
\end{multicols}

\begin{multicols}{3}
\begin{itemize}
\item  $f \left( \frac{2}{a} \right)=a$
\item $\frac{f(a)}{2}=\frac{1}{a}$
\item  $f(a + h)=\frac{2}{a+h}$
\end{itemize}
\end{multicols}
\normalsize

%\begin{center}
\begin{tabular}{cll}
67) & $f(0)=3$ & $f(x)=0$ for $x=15/2$\\
69) & $f(0)=-12$ & $f(x)=0$ for $x=-3,4$\\
71) & $f(0)=1$ & $f(x)=0$ for $x=1/2$\\
73) & $f(0)=0$ & $f(x)=0$ for $x=0,4$
\end{tabular}
%\end{center}
\newpage
\subsection*{Domain and Range}

%For each of Exercises \ref{graphfunctionfirst} - \ref{graphfunctionlast} from page \pageref{graphfunctionfirst}, find the domain and range of those relations which represent $y$ as a function of $x$.  Hint: There are exactly 15 relations which represent $y$ as a function of $x$, i.e., that pass the Vertical Line Test.\\

%In Exercises \ref{finddomainfirst} - \ref{finddomainlast}, find the domain of the function.

\begin{multicols}{2}
%\begin{enumerate}
%\setcounter{enumi}{\value{HW}}
1) $(-\infty,\infty)$\\
3) $(-\infty,-1)\cup(-1,\infty)$\\
5) $(-\infty,\infty)$\\
7) $(-\infty,-6)\cup(-6,6)\cup(6,\infty)$\\
9) $(-\infty,3]$\\
11) $[-3,\infty)$\\
13) $[1/3,\infty)$\\
15) $(-\infty,\infty)$\\
17) $[1/3,6)\cup(6,\infty)$\\
19) $(-\infty,8)\cup(8,\infty)$\\
21) $(8,\infty)$\\
23) $[0,5)\cup(5,\infty)$
\end{multicols}

\newpage
\subsection*{Combining Functions}

1) $f(x)=3x+1,~~~~g(x)=4-x$ 
\scriptsize
\begin{multicols}{3}
\begin{itemize}
\item  $(f+g)(2)=9$
\item  $(f-g)(-1)=-7$
\item  $(g-f)(1)=-1$
\end{itemize}
\end{multicols}

\begin{multicols}{3}
\begin{itemize}
\item  $(fg)(\frac{1}{2})=\frac{35}{4}$
\item  $\left(\frac{f}{g}\right)(0)=\frac{1}{4}$
\item  $\left(\frac{g}{f}\right)(-2)=-\frac{6}{5}$
\end{itemize}
\end{multicols}
\normalsize

3) $f(x)=x^2-x,~~~~g(x)=12-x^2$ 
\scriptsize
\begin{multicols}{3}
\begin{itemize}
\item  $(f+g)(2)=10$
\item  $(f-g)(-1)=-9$
\item  $(g-f)(1)=11$
\end{itemize}
\end{multicols}

\begin{multicols}{3}
\begin{itemize}
\item  $(fg)(\frac{1}{2})=-\frac{47}{16}$
\item  $\left(\frac{f}{g}\right)(0)=0$
\item  $\left(\frac{g}{f}\right)(-2)=\frac{4}{3}$
\end{itemize}
\end{multicols}
\normalsize

5) $f(x)=\sqrt{x+3},~~~~g(x)=2x-1$ 
\scriptsize
\begin{multicols}{3}
\begin{itemize}
\item  $(f+g)(2)=3+\sqrt{5}$
\item  $(f-g)(-1)=3+\sqrt{2}$
\item  $(g-f)(1)=-1$
\end{itemize}
\end{multicols}

\begin{multicols}{3}
\begin{itemize}
\item  $(fg)(\frac{1}{2})=0$
\item  $\left(\frac{f}{g}\right)(0)=-\sqrt{3}$
\item  $\left(\frac{g}{f}\right)(-2)=-5$
\end{itemize}
\end{multicols}
\normalsize

7) $f(x)=2x,~~~~g(x)=\dfrac{1}{2x+1}$ 
\scriptsize
\begin{multicols}{3}
\begin{itemize}
\item  $(f+g)(2)=\frac{21}{5}$
\item  $(f-g)(-1)=-1$
\item  $(g-f)(1)=-\frac{5}{3}$
\end{itemize}
\end{multicols}

\begin{multicols}{3}
\begin{itemize}
\item  $(fg)(\frac{1}{2})=\frac{1}{2}$
\item  $\left(\frac{f}{g}\right)(0)=0$
\item  $\left(\frac{g}{f}\right)(-2)=\frac{1}{12}$
\end{itemize}
\end{multicols}
\normalsize

9) $f(x)=x^2,~~~~g(x)=\dfrac{1}{x^2}$ 
\scriptsize
\begin{multicols}{3}
\begin{itemize}
\item  $(f+g)(2)=\frac{17}{4}$
\item  $(f-g)(-1)=0$
\item  $(g-f)(1)=0$
\end{itemize}
\end{multicols}

\begin{multicols}{3}
\begin{itemize}
\item  $(fg)(\frac{1}{2})=1$
\item  $\left(\frac{f}{g}\right)(0)=$DNE
\item  $\left(\frac{g}{f}\right)(-2)=\frac{1}{16}$
\end{itemize}
\end{multicols}
\normalsize

11) $f(x)=2x+1,~~~~g(x)=x-2$ 
\scriptsize
\begin{multicols}{2}
\begin{itemize}
\item  $(f+g)(x)=3x-1,~$all reals
\item  $(f-g)(x)=x+3,~$all reals
\end{itemize}
\end{multicols}
\begin{multicols}{2}
\begin{itemize}
\item  $(fg)(x)=2x^2-3x-2,~$all reals
\item  $\left(\frac{f}{g}\right)(x)=\frac{2x+1}{x-2},~x\neq 2$
\end{itemize}
\end{multicols}
\normalsize

13) $f(x)=x^2,~~~~g(x)=3x-1$ 
\scriptsize
\begin{multicols}{2}
\begin{itemize}
\item  $(f+g)(x)=x^2+3x-1,~$all reals
\item  $(f-g)(x)=x^2-3x+1,~$all reals
\end{itemize}
\end{multicols}
\begin{multicols}{2}
\begin{itemize}
\item  $(fg)(x)=3x^3-x^2,~$all reals
\item  $\left(\frac{f}{g}\right)(x)=\frac{x^2}{3x-1},~x\neq\frac{1}{3}$
\end{itemize}
\end{multicols}
\normalsize

\newpage
15) $f(x)=x^2-4,~~~~g(x)=3x+6$ 
\scriptsize
\begin{multicols}{2}
\begin{itemize}
\item  $(f+g)(x)=x^2+3x+2,~$all reals
\item  $(f-g)(x)=x^2-3x-10,~$all reals
\end{itemize}
\end{multicols}
\begin{multicols}{2}
\begin{itemize}
\item  $(fg)(x)=3x^3+6x^2-12x-24,~$all reals
\item  $\left(\frac{f}{g}\right)(x)=\frac{x^2-4}{3x+6},~x\neq -2$
\end{itemize}
\end{multicols}
\normalsize

17) $f(x)=\dfrac{x}{2},~~~~g(x)=\dfrac{2}{x}$ 
\scriptsize
\begin{multicols}{2}
\begin{itemize}
\item  $(f+g)(x)=\frac{x^2+4}{2x},~x\neq 0$
\item  $(f-g)(x)=\frac{x^2-4}{2x},~x\neq 0$
\end{itemize}
\end{multicols}
\begin{multicols}{2}
\begin{itemize}
\item  $(fg)(x)=1,~x\neq 0$
\item  $\left(\frac{f}{g}\right)(x)=\frac{x^2}{4},~x\neq 0$
\end{itemize}
\end{multicols}
\normalsize

19) $f(x)=x,~~~~g(x)=\sqrt{x+1}$ 
\scriptsize
\begin{multicols}{2}
\begin{itemize}
\item  $(f+g)(x)=x+\sqrt{x+1},~x\geq-1$
\item  $(f-g)(x)=x-\sqrt{x+1},~x\geq-1$
\end{itemize}
\end{multicols}
\begin{multicols}{2}
\begin{itemize}
\item  $(fg)(x)=x\sqrt{x+1},~x\geq-1$
\item  $\left(\frac{f}{g}\right)(x)=\frac{x}{\sqrt{x+1}},~x>-1$
\end{itemize}
\end{multicols}
\normalsize

\begin{multicols}{2}
\begin{enumerate}
\item[21)]~~2
\item[23)]~~0
\item[25)]~~3
\item[27)]~~DNE
\item[29)]~~4
\item[31)]~~-2
\end{enumerate}
\end{multicols}

33) $f(x)=x^2,~~~~g(x)=2x+1$
\scriptsize
\begin{multicols}{3}
\begin{itemize}
\item  $(g\circ f)(0)=1$
\item  $(f\circ g)(-1)=1$
\item  $(f\circ f)(2)=16$
\end{itemize}
\end{multicols}

\begin{multicols}{3}
\begin{itemize}
\item  $(g\circ f)(-3)=19$
\item  $(f\circ g)(\frac{1}{2})=4$
\item  $(f\circ f)(-2)=16$
\end{itemize}
\end{multicols}
\normalsize

35) $f(x)=4-3x,~~~~g(x)=|x|$
\scriptsize
\begin{multicols}{3}
\begin{itemize}
\item  $(g\circ f)(0)=4$
\item  $(f\circ g)(-1)=1$
\item  $(f\circ f)(2)=10$
\end{itemize}
\end{multicols}

\begin{multicols}{3}
\begin{itemize}
\item  $(g\circ f)(-3)=13$
\item  $(f\circ g)(\frac{1}{2})=\frac{5}{2}$
\item  $(f\circ f)(-2)=-26$
\end{itemize}
\end{multicols}
\normalsize

37) $f(x)=4x+5,~~~~g(x)=\sqrt{x}$
\scriptsize
\begin{multicols}{3}
\begin{itemize}
\item  $(g\circ f)(0)=\sqrt{5}$
\item  $(f\circ g)(-1)=$DNE
\item  $(f\circ f)(2)=57$
\end{itemize}
\end{multicols}

\begin{multicols}{3}
\begin{itemize}
\item  $(g\circ f)(-3)=$DNE
\item  $(f\circ g)(\frac{1}{2})=4\sqrt{\frac{1}{2}}+5$
\item  $(f\circ f)(-2)=-7$
\end{itemize}
\end{multicols}
\normalsize

\newpage
39) $f(x)=\dfrac{3}{1-x},~~~~g(x)=\dfrac{4x}{x^2+1}$
\scriptsize
\begin{multicols}{3}
\begin{itemize}
\item  $(g\circ f)(0)=\frac{6}{5}$
\item  $(f\circ g)(-1)=1$
\item  $(f\circ f)(2)=\frac{3}{4}$
\end{itemize}
\end{multicols}

\begin{multicols}{3}
\begin{itemize}
\item  $(g\circ f)(-3)=\frac{48}{25}$
\item  $(f\circ g)(\frac{1}{2})=-5$
\item  $(f\circ f)(-2)=$DNE
\end{itemize}
\end{multicols}
\normalsize

41) $f(x)=2x+3,~~~~g(x)=x^2-9$
\scriptsize
\begin{multicols}{3}
\begin{itemize}
\item  $(g\circ f)(x)=4x^2+12x$
\item  $(f\circ g)(x)=2x^2-15$
\item  $(f\circ f)(x)=4x=9$
\end{itemize}
\end{multicols}
\normalsize

43) $f(x)=x^2-4,~~~~g(x)=|x|$
\scriptsize
\begin{multicols}{3}
\begin{itemize}
\item  $(g\circ f)(x)=|x^2-4|$
\item  $(f\circ g)(x)=x^2-4$
\item  $(f\circ f)(x)=x^4-8x^2+12$
\end{itemize}
\end{multicols}
\normalsize

45) $f(x)=|x+1|,~~~~g(x)=\sqrt{x}$
\scriptsize
\begin{multicols}{3}
\begin{itemize}
\item  $(g\circ f)(x)=\sqrt{|x+1|}$
\item  $(f\circ g)(x)=|\sqrt{x}+1|$
\item  $(f\circ f)(x)=|x+1|+1$
\end{itemize}
\end{multicols}
\normalsize

47) $f(x)=|x|,~~~~g(x)=\sqrt{4-x}$
\scriptsize
\begin{multicols}{3}
\begin{itemize}
\item  $(g\circ f)(x)=\sqrt{4-|x|}$
\item  $(f\circ g)(x)=\sqrt{4-x}$
\item  $(f\circ f)(x)=|x|$
\end{itemize}
\end{multicols}
\normalsize

49) $f(x)=3x-1,~~~~g(x)=\dfrac{1}{x+3}$
\scriptsize
\begin{multicols}{3}
\begin{itemize}
\item  $(g\circ f)(x)=\frac{1}{3x+2}$
\item  $(f\circ g)(x)=\frac{x}{x+3}$
\item  $(f\circ f)(x)=9x-4$
\end{itemize}
\end{multicols}
\normalsize

51) $f(x)=\dfrac{x}{2x+1},~~~~g(x)=\dfrac{2x+1}{x}$
\scriptsize
\begin{multicols}{3}
\begin{itemize}
\item  $(g\circ f)(x)=\frac{4x+1}{x}$
\item  $(f\circ g)(x)=\frac{2x+1}{5x+2}$
\item  $(f\circ f)(x)=\frac{x}{4x+1}$
\end{itemize}
\end{multicols}
\normalsize

\begin{multicols}{2}
\begin{enumerate}
\item[53)] $f(x)=x^3,~~~~g(x)=2x+3$
\item[55)] $f(x)=\sqrt{x},~~~~g(x)=2x-1$
\item[57)] $f(x)=\dfrac{2}{x},~~~~g(x)=5x+1$
\item[59)] $f(x)=\dfrac{x+1}{x-1},~~~~g(x)=|x|$
\item[61)] $f(x)=\dfrac{x+1}{3-2x},~~~~g(x)=2x$
\item[63)] $k\circ j\circ f\circ h\circ g$
\end{enumerate}
\end{multicols}
\begin{multicols}{4}
\begin{enumerate}
\item[65)]~~4
\item[67)]~~3
\item[69)]~~-4
\item[71)]~~0
\item[73)]~~-3
\item[75)]~~4
\item[77)]~~4
\item[79)]~~0
\end{enumerate}
\end{multicols}

\newpage
\subsection*{Inverse Functions}

\begin{multicols}{2}
\begin{enumerate}
\item[1)] $f^{-1}(x) = \dfrac{x + 2}{6}$
\item[3)] $f^{-1}(x) = 3x-10$
\item[5)] $f^{-1}(x) = \frac{1}{3}(x-5)^2+\frac{1}{3}$, $x \geq 5$
\item[7)] $f^{-1}(x) = \frac{1}{9}(x+4)^2+1$, $x \geq -4$
\item[9)] $f^{-1}(x) = \frac{1}{3} x^{5} + \frac{1}{3}$
\item[11)] $f^{-1}(x) = 5 + \sqrt{x+25}$
\item[13)] $f^{-1}(x) = 3 - \sqrt{x+4}$
\item[15)] $f^{-1}(x) = \dfrac{4x-3}{x}$
\item[17)] $f^{-1}(x) = \dfrac{4x+1}{2-3x}$
\item[19)] $f^{-1}(x) = \dfrac{-3x - 2}{x + 3}$
\end{enumerate}
\end{multicols}

\newpage
\subsection*{Transformations}

%Suppose $(2,-3)$ is on the graph of $y = f(x)$.  In Exercises \ref{transformpointfirst} - \ref{transformpointlast}, use the given point to %Theorem \ref{transformationsthm} to find a point on the graph of the given transformed function.

\begin{multicols}{2}
1) $(2,0)$\\
3) $(2,-4)$\\
5) $(2,-9)$\\
7) $(2,3)$\\
9) $(5,-2)$\\
11) $(2,13)$\\
13) $(2,-\frac{3}{2})$\\
15) $(-1,-7)$\\
17) $(1,1)$\\
\end{multicols}

Each answer below describes the resulting transformation of the graph of $f(x)=|x|$.\\
~\\
19) Shift down 2 units\\
21) Shift right 2 units\\
23) Vertical stretch (or horizontal shrink) by a factor of 2\\
25) Shift right 2 units\\
27) (22) and (23) match up; (21) and (25) match up\\
$|kx|=|k|\cdot|x|$, where $k\in\mathbb{R}$\\
~\\
Each answer below describes the resulting transformation of the graph of $f(x)=\sqrt{9-x^2}$.\\
~\\
29) Shift down 1/2 units\\
31) Shift left 4 units\\
33) Vertical shrink by a factor of 5/3\\
35) Horizontal stretch by a factor of 3/2\\
31) Shift right 3 units, vertical stretch by a factor of 4, shift up 6 units\\
~\\
39) $g(x)=-2\sqrt[3]{x+3}-1$
\begin{multicols}{2}
43) $g(x)=\sqrt{x-2}-3$\\
45) $g(x)=-\sqrt{x}-1$\\
47) $g(x)=\sqrt{-x-1}+2$\\
49) $g(x)=2\sqrt{x+3}-8$\\
51) $g(x)=\sqrt{2x-6}+1$
\end{multicols}

\newpage
\subsection*{Piecewise-Defined and Absolute Value Functions}

\subsubsection{Piecewise-Defined Functions}

%\begin{enumerate}
%\setcounter{enumi}{\value{HW}}

%\item  Let $f(x) = \left\{  \begin{array}{rcr} x + 5 & \mbox{ if } & x \leq -3 \\ \sqrt{9-x^2} & \mbox{ if } & -3 < x \leq 3 \\ -x+5 & \mbox{ if } & x > 3 \\ \end{array}        \right.$\pp
%Compute the following function values.

1) \scriptsize
$f(x) = \left\{  \begin{array}{rcr} x + 5 & \mbox{ if } & x \leq -3 \\ \sqrt{9-x^2} & \mbox{ if } & -3 < x \leq 3 \\ -x+5 & \mbox{ if } & x > 3 \\ \end{array} \right.$
\normalsize

\begin{multicols}{3}
\begin{enumerate}

\item[(a)] $f(-4)=1$
\item[(b)]  $f(-3)=2$
\item[(c)]  $f(3)=0$

\setcounter{HWindent}{\value{enumii}}
\end{enumerate}
\end{multicols}

\begin{multicols}{3}
\begin{enumerate}
\setcounter{enumii}{\value{HWindent}}

\item[(d)]  $f(3.1)=1.9$
\item[(e)]  $f(-3.01)=1.99$
\item[(f)]  $f(2)=\sqrt{5}$

\setcounter{HWindent}{\value{enumii}}
\end{enumerate}
\end{multicols}

\begin{tabular}{llll}
3) & D:~$(-\infty,\infty)$ & R:~$[1,\infty)$ & No zeros\\
5) & D:~$(-\infty,\infty)$ & R:~$[-3,3]$ & $x=3/2$\\
7) & D:~$(-\infty,\infty)$ & R:~$(-4,\infty)$ & $x=-2,0$\\
9) & D:~$(-6,-1)\cup(-1,1)\cup(1,9)$ & R:~$(-1,1)\cup(1,3)$ & $x=0$
\end{tabular}

\subsubsection{Absolute Value Functions}

%In Exercises \ref{graphabsvalexerfirst} - \ref{graphabsvalexerlast}, find the zeros of each function and the $x$- and $y$-intercepts of each graph, if any exist.  Then graph the given absolute value function and express it as a piecewise-defined function.  Use the graph to determine the domain and range of each function.

\begin{tabular}{lllll}
11) &No zeros & $y-$int at $(0,4)$ & D:~$(-\infty,\infty)$ & R:~$[4,\infty)$\\
\end{tabular}
$$f(x) = \left\{  \begin{array}{rcr} 
x+4 & \mbox{ if } & x \geq 0 \\
-x+4 & \mbox{ if } & x < 0 \\ \end{array}\right.$$
\begin{tabular}{lllll}
13) &Zero at $x=\frac{5}{2}$ & $y-$int at $(0,5)$ & D:~$(-\infty,\infty)$ & R:~$[0,\infty)$\\
\end{tabular}
$$f(x) = \left\{  \begin{array}{rcr} 
 2x-5 & \mbox{ if } & x \geq \frac{5}{2} \\
 -2x+5 & \mbox{ if } & x < \frac{5}{2} \\ \end{array}\right.$$
\begin{tabular}{lllll}
15) &Zero at $x=\frac{5}{2}$ & $y-$int at $(0,5)$ & D:~$(-\infty,\infty)$ & R:~$[0,\infty)$\\
\end{tabular}
$$f(x) = \left\{  \begin{array}{rcr} 
 2x-5 & \mbox{ if } & x \geq \frac{5}{2} \\
 -2x+5 & \mbox{ if } & x < \frac{5}{2} \\ \end{array}\right.$$
\begin{tabular}{lllll}
17) &Zeros at $x=-\frac{16}{3},~-\frac{8}{3}$ & $y-$int at $(0,8)$ & D:~$(-\infty,\infty)$ & R:~$[-4,\infty)$\\
\end{tabular}
$$f(x) = \left\{  \begin{array}{rcr} 
3x+8 & \mbox{ if } & x \geq -4 \\
-3x-16 & \mbox{ if } & x < -4 \\ \end{array}\right.$$

\end{document}