%\subsection{Practice Problems}

\subsection{Notation and Basic Examples}                                

In Exercises \ref{setfunctionfirst} - \ref{setfunctionlast}, determine whether or not the relation represents $y$ as a function of $x$.  
\begin{enumerate}

\item \{$(-3, 9)$, $\;(-2, 4)$, $\;(-1, 1)$, $\;(0, 0)$, $\;(1, 1)$, $\;(2, 4)$, $\;(3, 9)\}$ \label{setfunctionfirst}
\item  $\left\{ (-3,0), (1,6), (2, -3), (4,2), (-5,6), (4, -9), (6,2) \right\}$
\item  $\left\{ (-3,0), (-7,6), (5,5), (6,4), (4,9), (3,0) \right\}$
\item  $\left\{ (1,2), (4,4), (9,6), (16,8), (25,10), (36, 12), \ldots \right\}$
\item \{($x, y) \, | \, x$ is an odd integer, and $y$ is an even integer\}
\item \{$(x, 1) \, | \, x$ is an irrational number\}
\item \{$(1, 0)$, $\;(2, 1)$, $\;(4, 2)$, $\;(8, 3)$, $\;(16, 4)$, $\;(32, 5), \;$ \ldots\}
\item \{$\ldots, \; (-3, 9)$, $\;(-2, 4)$, $\;(-1, 1)$, $\;(0, 0)$, $\;(1, 1)$, $\;(2, 4)$, $\;(3, 9), \;$ \ldots\}

\setcounter{HW}{\value{enumi}}

\end{enumerate}

\begin{multicols}{2}

\begin{enumerate}

\setcounter{enumi}{\value{HW}}

\item $\{ (-2, y) \, | \, -3 < y < 4\}$
\item  $\{ (x,3) \, | \,  -2 \leq x < 4\}$ \label{setfunctionlast}

\setcounter{HW}{\value{enumi}}
\end{enumerate}
\end{multicols}

%\begin{multicols}{2}
%\begin{enumerate}
%\setcounter{enumi}{\value{HW}}


%\item  $\{ \left(x,x^2\right) \, | \, \text{$x$ is a real number} \}$
%\item  $\{ \left(x^2,x\right) \, | \, \text{$x$ is a real number} \}$ \label{setfunctionlast}

%\setcounter{HW}{\value{enumi}}
%\end{enumerate}
%\end{multicols}

In Exercises \ref{graphfunctionfirst} - \ref{graphfunctionlast}, determine whether or not the relation represents $y$ as a function of $x$.


\begin{multicols}{2}
\begin{enumerate}
\setcounter{enumi}{\value{HW}}


\item $~$ \vspace{-.1in} \label{graphfunctionfirst}

\begin{mfpic}[17]{-5}{2}{-2}{5}
\point[3pt]{(-4, -1), (-3, 0), (-2, 1), (-1, 2), (0, 3), (1, 4)}
\axes
\tlabel[cc](2,-0.5){\scriptsize $x$}
\tlabel[cc](0.5,4.75){\scriptsize $y$}
\xmarks{-4,-3,-2,-1,1}
\ymarks{-1,1,2,3,4}
\tlpointsep{4pt}
\axislabels {x}{{\tiny $-4 \hspace{8pt}$} -4, {\tiny $-3 \hspace{8pt}$} -3, {\tiny $-2 \hspace{8pt}$} -2, {\tiny $-1 \hspace{8pt}$} -1, {\tiny $1$} 1}
\axislabels {y}{{\tiny $-1$} -1, {\tiny $1$} 1, {\tiny $2$} 2, {\tiny $3$} 3, {\tiny $4$} 4}
\end{mfpic}

\vfill
\columnbreak

\item $~$

\begin{mfpic}[15]{-5}{2}{-2}{5}
\point[3pt]{(-4, -1), (-3, 0), (-3, 1), (-2, 1), (-1, 2), (0, 3), (1, 4)}
\axes
\tlabel[cc](2,-0.5){\scriptsize $x$}
\tlabel[cc](0.5,4.75){\scriptsize $y$}
\xmarks{-4,-3,-2,-1,1}
\ymarks{-1,1,2,3,4}
\tlpointsep{4pt}
\axislabels {x}{{\tiny $-4 \hspace{6pt}$} -4, {\tiny $-3 \hspace{6pt}$} -3, {\tiny $-2 \hspace{6pt}$} -2, {\tiny $-1 \hspace{6pt}$} -1, {\tiny $1$} 1}
\axislabels {y}{{\tiny $-1$} -1, {\tiny $1$} 1, {\tiny $2$} 2, {\tiny $3$} 3, {\tiny $4$} 4}
\end{mfpic}


\setcounter{HW}{\value{enumi}}
\end{enumerate}
\end{multicols}

\pagebreak

\begin{multicols}{2}
\begin{enumerate}
\setcounter{enumi}{\value{HW}}


\item $~$

\begin{mfpic}[15]{-3}{3}{-1}{6}
\axes
\tlabel[cc](3,-0.5){\scriptsize $x$}
\tlabel[cc](0.5,5.75){\scriptsize $y$}
\xmarks{-2,-1,1,2}
\ymarks{1,2,3,4,5}
\tlpointsep{4pt}
\axislabels {x}{{\tiny $-2 \hspace{8pt}$} -2, {\tiny $-1 \hspace{8pt}$} -1, {\tiny $1$} 1, {\tiny $2$} 2}
\axislabels {y}{{\tiny $1$} 1, {\tiny $2$} 2, {\tiny $3$} 3, {\tiny $4$} 4, {\tiny $5$} 5}
\arrow \reverse \arrow \function{-2.1, 2.1, 0.1}{x**2+1}
\end{mfpic}

\vfill
\columnbreak

\item $~$

\begin{mfpic}[15]{-4}{4}{-4}{4}
\axes
\tlabel[cc](4,-0.5){\scriptsize $x$}
\tlabel[cc](0.5,3.75){\scriptsize $y$}
\xmarks{-3,-2,-1,1,2,3}
\ymarks{-3,-2,-1,1,2,3}
\tlpointsep{4pt}
\axislabels {x}{{\tiny $-3 \hspace{8pt}$} -3, {\tiny $-2 \hspace{8pt}$} -2, {\tiny $-1 \hspace{8pt}$} -1, {\tiny $1$} 1, {\tiny $2$} 2, {\tiny $3$} 3}
\axislabels {y}{{\tiny $-3$} -3, {\tiny $-2$} -2, {\tiny $-1$} -1, {\tiny $1$} 1, {\tiny $2$} 2, {\tiny $3$} 3}
\arrow \reverse \arrow \parafcn{-2,2,0.1}{(cosh(t),sinh(t))}
\arrow \reverse \arrow \parafcn{-2,2,0.1}{(-cosh(t),sinh(t))}
\end{mfpic}


\setcounter{HW}{\value{enumi}}
\end{enumerate}
\end{multicols}



\begin{multicols}{2}
\begin{enumerate}
\setcounter{enumi}{\value{HW}}

\item $~$

\begin{mfpic}[15]{-1}{10}{-1}{4}
\axes
\tlabel[cc](10,-0.5){\scriptsize $x$}
\tlabel[cc](0.5,3.75){\scriptsize $y$}
\xmarks{1,2,3,4,5,6,7,8,9}
\ymarks{1,2,3}
\tlpointsep{4pt}
\axislabels {x}{{\tiny $1$} 1, {\tiny $2$} 2, {\tiny $3$} 3, {\tiny $4$} 4, {\tiny $5$} 5, {\tiny $6$} 6, {\tiny $7$} 7, {\tiny $8$} 8, {\tiny $9$} 9}
\axislabels {y}{{\tiny $1$} 1, {\tiny $2$} 2, {\tiny $3$} 3}
\arrow \function{2, 10, 0.1}{sqrt(x - 2)}
\point[3pt]{(2,0)}
\end{mfpic}

\vfill
\columnbreak

\item $~$

\begin{mfpic}[15]{-5}{5}{-1}{5}
\axes
\tlabel[cc](5,-0.5){\scriptsize $x$}
\tlabel[cc](0.5,4.75){\scriptsize $y$}
\xmarks{-4,-3,-2,-1,1,2,3,4}
\ymarks{1,2,3,4}
\tlpointsep{4pt}
\axislabels {x}{{\tiny $-4 \hspace{8pt}$} -4, {\tiny $-3 \hspace{8pt}$} -3, {\tiny $-2 \hspace{8pt}$} -2, {\tiny $-1 \hspace{8pt}$} -1, {\tiny $1$} 1, {\tiny $2$} 2, {\tiny $3$} 3, {\tiny $4$} 4}
\axislabels {y}{{\tiny $1$} 1, {\tiny $2$} 2, {\tiny $3$} 3, {\tiny $4$} 4}
\arrow \reverse \arrow \function{-5, 5, 0.1}{4/(x**2 + 1)}
\end{mfpic}


\setcounter{HW}{\value{enumi}}
\end{enumerate}
\end{multicols}

\begin{multicols}{2}
\begin{enumerate}
\setcounter{enumi}{\value{HW}}

\item $~$


\begin{mfpic}[17]{-4.5}{5.5}{-4}{3}
\fillcolor[gray]{.7}
\gfill \rect{(-3.97, -2.97), (4.97, 1.97)}
\dashed \polyline{(-4, -3), (-4, 2)}
\dashed \polyline{(-4, 2), (5, 2)}
\dashed \polyline{(5, 2), (5, -3)}
\dashed \polyline{(5, -3), (-4, -3)}
\axes
\tlabel[cc](5.5,-0.5){\scriptsize $x$}
\tlabel[cc](0.5,2.75){\scriptsize $y$}
\xmarks{-4,-3,-2,-1,1,2,3,4,5}
\ymarks{-3,-2,-1,1,2}
\tlpointsep{4pt}
\axislabels {x}{{\tiny $-4 \hspace{8pt}$} -4, {\tiny $-3 \hspace{8pt}$} -3, {\tiny $-2 \hspace{8pt}$} -2, {\tiny $-1 \hspace{8pt}$} -1, {\tiny $1$} 1, {\tiny $2$} 2, {\tiny $3$} 3, {\tiny $4$} 4, {\tiny $5$} 5}
\axislabels {y}{{\tiny $-3$} -3, {\tiny $-2$} -2, {\tiny $-1$} -1, {\tiny $1$} 1, {\tiny $2$} 2}
\end{mfpic}

\vfill
\columnbreak

\item $~$

\begin{mfpic}[15]{-6}{4}{-3}{5}
\function{-5,-1,0.1}{-5 - 6*x - x**2}
\function{-1,3,0.1}{x/4 - 7/4}
\point[3pt]{(-5, 0), (-1, 0)}
\gclear \circle{(-3,4), 0.1}
\circle{(-3,4), 0.1}
\gclear \circle{(-1,-2), 0.1}
\circle{(-1,-2), 0.1}
\gclear \circle{(3,-1), 0.1}
\circle{(3,-1), 0.1}
\axes
\tlabel[cc](4,-0.5){\scriptsize $x$}
\tlabel[cc](0.5,4.75){\scriptsize $y$}
\xmarks{-5 step 1 until 3}
\ymarks{-2 step 1 until 4}
\tlpointsep{4pt}
\axislabels {x}{{\tiny $-5 \hspace{8pt}$} -5, {\tiny $-4 \hspace{8pt}$} -4, {\tiny $-3 \hspace{8pt}$} -3, {\tiny $-2 \hspace{8pt}$} -2, {\tiny $-1 \hspace{8pt}$} -1, {\tiny $1$} 1, {\tiny $2$} 2, {\tiny $3$} 3}
\axislabels {y}{{\tiny $-2$} -2, {\tiny $-1$} -1, {\tiny $1$} 1, {\tiny $2$} 2, {\tiny $3$} 3, {\tiny $4$} 4}
\end{mfpic}


\setcounter{HW}{\value{enumi}}
\end{enumerate}
\end{multicols}

\newpage

\begin{multicols}{2}
\begin{enumerate}
\setcounter{enumi}{\value{HW}}

\item  $~$

\begin{mfpic}[8]{-4}{4}{-6}{10}
\point[3pt]{(-2,6), (1,-3) }
\axes
\tlabel[cc](4,-0.5){\scriptsize $x$}
\tlabel[cc](0.5,10){\scriptsize $y$}
\xmarks{-3,-2,-1,1,2,3}
\ymarks{-5,-4,-3,-2,-1,1,2,3,4,5,6,7,8,9}
\tlpointsep{4pt}
\axislabels {x}{{\tiny $-3 \hspace{6pt}$} -3,{\tiny $-2 \hspace{6pt}$} -2, {\tiny $-1 \hspace{6pt}$} -1, {\tiny $1$} 1, {\tiny $2$} 2, {\tiny $3$} 3}
\axislabels {y}{{\tiny $-5$} -5, {\tiny $-4$} -4, {\tiny $-3$} -3, {\tiny $-2$} -2, {\tiny $-1$} -1, {\tiny $1$} 1, {\tiny $2$} 2, {\tiny $3$} 3, {\tiny $4$} 4, {\tiny $5$} 5, {\tiny $6$} 6, {\tiny $7$} 7, {\tiny $8$} 8, {\tiny $9$} 9}
\arrow \function{-2,4.5,0.1}{x**2 - 2*x - 2}
\end{mfpic}

\vfill
\columnbreak

\item  $~$

\begin{mfpic}[10]{-6}{6}{-6}{6}
\axes
\tlabel[cc](6,-0.5){\scriptsize $x$}
\tlabel[cc](0.5,6){\scriptsize $y$}
\xmarks{-5,-4,-3,-2,-1,1,2,3,4,5}
\ymarks{-5,-4,-3,-2,-1,1,2,3,4,5}
\tlpointsep{4pt}
\axislabels {x}{{\tiny $-5 \hspace{6pt}$} -5,{\tiny $-4 \hspace{6pt}$} -4,{\tiny $-3 \hspace{6pt}$} -3,{\tiny $-2 \hspace{6pt}$} -2, {\tiny $-1 \hspace{6pt}$} -1, {\tiny $1$} 1, {\tiny $2$} 2, {\tiny $3$} 3, {\tiny $4$} 4, {\tiny $5$} 5}
\axislabels {y}{{\tiny $-5$} -5,{\tiny $-4$} -4,{\tiny $-3$} -3, {\tiny $-2$} -2, {\tiny $-1$} -1, {\tiny $1$} 1, {\tiny $2$} 2, {\tiny $3$} 3, {\tiny $4$} 4, {\tiny $5$} 5}
\plrfcn{0,180,5}{5*sind 3t}
\end{mfpic} 


\setcounter{HW}{\value{enumi}}
\end{enumerate}
\end{multicols}

\begin{multicols}{2}
\begin{enumerate}
\setcounter{enumi}{\value{HW}}

\item  $~$

\begin{mfpic}[10]{-6}{6}{-6}{6}
\axes
\tlabel[cc](6,-0.5){\scriptsize $x$}
\tlabel[cc](0.5,6){\scriptsize $y$}
\xmarks{-5,-4,-3,-2,-1,1,2,3,4,5}
\ymarks{-5,-4,-3,-2,-1,1,2,3,4,5}
\tlpointsep{4pt}
\axislabels {x}{{\tiny $-5 \hspace{6pt}$} -5,{\tiny $-4 \hspace{6pt}$} -4,{\tiny $-3 \hspace{6pt}$} -3,{\tiny $-2 \hspace{6pt}$} -2, {\tiny $-1 \hspace{6pt}$} -1, {\tiny $1$} 1, {\tiny $2$} 2, {\tiny $3$} 3, {\tiny $4$} 4, {\tiny $5$} 5}
\axislabels {y}{{\tiny $-5$} -5,{\tiny $-4$} -4,{\tiny $-3$} -3, {\tiny $-2$} -2, {\tiny $-1$} -1, {\tiny $1$} 1, {\tiny $2$} 2, {\tiny $3$} 3, {\tiny $4$} 4, {\tiny $5$} 5}
\function{-5,4,0.1}{0.0502*(x**3) - 0.0344*(x**2) - 0.2010*x + 2.138}
\gfill \circle{(-5,-4),0.2}
\gclear \circle{(4,4),0.2}
\circle{(4,4),0.2}
\end{mfpic} 

\vfill
\columnbreak

\item  $~$

\begin{mfpic}[10]{-2}{7}{-6}{6}
\axes
\tlabel[cc](7,-0.5){\scriptsize $x$}
\tlabel[cc](0.5,6){\scriptsize $y$}
\xmarks{-1,1,2,3,4,5,6}
\ymarks{-5,-4,-3,-2,-1,1,2,3,4,5}
\tlpointsep{4pt}
\axislabels {x}{{\tiny $-1 \hspace{6pt}$} -1, {\tiny $1$} 1, {\tiny $2$} 2, {\tiny $3$} 3, {\tiny $4$} 4, {\tiny $5$} 5, {\tiny $6$} 6}
\axislabels {y}{{\tiny $-5$} -5,{\tiny $-4$} -4,{\tiny $-3$} -3, {\tiny $-2$} -2, {\tiny $-1$} -1, {\tiny $1$} 1, {\tiny $2$} 2, {\tiny $3$} 3, {\tiny $4$} 4, {\tiny $5$} 5}
\polyline{(0,-1), (3,-4)}
\polyline{(3,1), (4,4), (6,0)}
\point[3pt]{(0,-1), (4,4), (6,0)}
\pointfillfalse
\point[3pt]{(3,-4), (3,1)}
\end{mfpic} 

\setcounter{HW}{\value{enumi}}
\end{enumerate}
\end{multicols}

\begin{multicols}{2}
\begin{enumerate}
\setcounter{enumi}{\value{HW}}

\item  $~$

\begin{mfpic}[15]{-3}{3}{-1}{5}
\axes
\tlabel[cc](3,-0.5){\scriptsize $x$}
\tlabel[cc](0.5,5){\scriptsize $y$}
\xmarks{-2,-1,1,2}
\ymarks{1,2,3,4}
\tlpointsep{4pt}
\axislabels {x}{{\tiny $-2 \hspace{6pt}$} -2, {\tiny $-1 \hspace{6pt}$} -1, {\tiny $1$} 1, {\tiny $2$} 2}
\axislabels {y}{{\tiny $1$} 1, {\tiny $2$} 2, {\tiny $3$} 3, {\tiny $4$} 4}
\arrow \reverse \arrow \function{-2.25,2.25,0.1}{4-(x**2)}
\end{mfpic} 

\vfill
\columnbreak

\item  $~$


\begin{mfpic}[15]{-3}{3}{-1}{5}
\axes
\tlabel[cc](3,-0.5){\scriptsize $x$}
\tlabel[cc](0.5,5){\scriptsize $y$}
\xmarks{-2,-1,1,2}
\ymarks{1,2,3,4}
\tlpointsep{4pt}
\axislabels {x}{{\tiny $-2 \hspace{6pt}$} -2, {\tiny $-1 \hspace{6pt}$} -1, {\tiny $1$} 1, {\tiny $2$} 2}
\axislabels {y}{{\tiny $1$} 1, {\tiny $2$} 2, {\tiny $3$} 3, {\tiny $4$} 4}
\arrow \reverse \arrow \polyline{(-2,-1), (1,4), (2,-1)}
\end{mfpic} 

\setcounter{HW}{\value{enumi}}
\end{enumerate}
\end{multicols}

\newpage

\begin{multicols}{2}
\begin{enumerate}
\setcounter{enumi}{\value{HW}}

\item  $~$

\begin{mfpic}[15]{-3}{3}{-1}{5}
\axes
\tlabel[cc](3,-0.5){\scriptsize $x$}
\tlabel[cc](0.5,5){\scriptsize $y$}
\xmarks{-2,-1,1,2}
\ymarks{1,2,3,4}
\tlpointsep{4pt}
\axislabels {x}{{\tiny $-2 \hspace{6pt}$} -2, {\tiny $-1 \hspace{6pt}$} -1, {\tiny $1$} 1, {\tiny $2$} 2}
\axislabels {y}{{\tiny $1$} 1, {\tiny $2$} 2, {\tiny $3$} 3, {\tiny $4$} 4}
\arrow \function{-2, 2, 0.1}{3-2*sqrt(x+2)}
\point[3pt]{(-2,3)}
\end{mfpic} 

\vfill
\columnbreak

\item  $~$


\begin{mfpic}[15]{-3}{3}{-1}{5}
\axes
\tlabel[cc](3,-0.5){\scriptsize $x$}
\tlabel[cc](0.5,5){\scriptsize $y$}
\xmarks{-2,-1,1,2}
\ymarks{1,2,3,4}
\tlpointsep{4pt}
\axislabels {x}{{\tiny $-2 \hspace{6pt}$} -2, {\tiny $-1 \hspace{6pt}$} -1, {\tiny $1$} 1, {\tiny $2$} 2}
\axislabels {y}{{\tiny $1$} 1, {\tiny $2$} 2, {\tiny $3$} 3, {\tiny $4$} 4}
\arrow \reverse \arrow \function{-2.25, 1.75, 0.1}{x*(x-1)*(x+2)}
\end{mfpic} 

\setcounter{HW}{\value{enumi}}
\end{enumerate}
\end{multicols}

\begin{multicols}{2}
\begin{enumerate}
\setcounter{enumi}{\value{HW}}

\item  $~$

\begin{mfpic}[15]{-3}{3}{-3}{3}
\axes
\tlabel[cc](3,-0.5){\scriptsize $x$}
\tlabel[cc](0.5,3){\scriptsize $y$}
\xmarks{-2,-1,1,2}
\ymarks{-2,-1,1,2}
\tlpointsep{4pt}
\axislabels {x}{{\tiny $-2 \hspace{6pt}$} -2, {\tiny $-1 \hspace{6pt}$} -1, {\tiny $1$} 1, {\tiny $2$} 2}
\axislabels {y}{{\tiny $1$} 1, {\tiny $2$} 2, {\tiny $-2$} -2, {\tiny $-1$} -1}
\arrow \polyline{(0,1), (-2,-2)}
\arrow \polyline{(1,2), (3,2)}
\point[3pt]{(0,1)}
\pointfillfalse
\point[3pt]{(1,2)}
\end{mfpic} 

\vfill
\columnbreak

\item  $~$


\begin{mfpic}[15]{-4}{4}{-3}{3}
\axes
\tlabel[cc](4,-0.5){\scriptsize $x$}
\tlabel[cc](0.5,3){\scriptsize $y$}
\xmarks{-3,-2,-1,1,2,3}
\ymarks{-2,-1,1,2}
\tlpointsep{4pt}
\axislabels {x}{{\tiny $-3 \hspace{6pt}$} -3,{\tiny $-2 \hspace{6pt}$} -2, {\tiny $-1 \hspace{6pt}$} -1, {\tiny $1$} 1, {\tiny $2$} 2, {\tiny $3$} 3}
\axislabels {y}{{\tiny $1$} 1, {\tiny $2$} 2, {\tiny $-2$} -2, {\tiny $-1$} -1}
\function{-3,3,0.1}{2*sin(1.05*x)}
\point[3pt]{(-3,0)}
\point[3pt]{(3,0)}
\end{mfpic} 

\setcounter{HW}{\value{enumi}}
\end{enumerate}
\end{multicols}


\begin{multicols}{2}
\begin{enumerate}
\setcounter{enumi}{\value{HW}}

\item  $~$

\begin{mfpic}[15]{-3}{3}{-3}{3}
\axes
\tlabel[cc](3,-0.5){\scriptsize $x$}
\tlabel[cc](0.5,3){\scriptsize $y$}
\xmarks{-2,-1,1,2}
\ymarks{-2,-1,1,2}
\tlpointsep{4pt}
\axislabels {x}{{\tiny $-2 \hspace{6pt}$} -2, {\tiny $-1 \hspace{6pt}$} -1, {\tiny $1$} 1, {\tiny $2$} 2}
\axislabels {y}{{\tiny $1$} 1, {\tiny $2$} 2, {\tiny $-2$} -2, {\tiny $-1$} -1}
\arrow \reverse \arrow \polyline{(2,-3), (2,3)}
\end{mfpic} 

\vfill
\columnbreak

\item  $~$ \label{graphfunctionlast}


\begin{mfpic}[15]{-3}{3}{-3}{3}
\axes
\tlabel[cc](3,-0.5){\scriptsize $x$}
\tlabel[cc](0.5,3){\scriptsize $y$}
\xmarks{-2,-1,1,2}
\ymarks{-2,-1,1,2}
\tlpointsep{4pt}
\axislabels {x}{{\tiny $-2 \hspace{6pt}$} -2, {\tiny $-1 \hspace{6pt}$} -1, {\tiny $1$} 1, {\tiny $2$} 2}
\axislabels {y}{{\tiny $1$} 1, {\tiny $2$} 2, {\tiny $-2$} -2, {\tiny $-1$} -1}
\arrow \reverse \arrow \polyline{(-3,2), (3,2)}
\end{mfpic} 

\setcounter{HW}{\value{enumi}}
\end{enumerate}
\end{multicols}

\newpage

In Exercises \ref{equfunctionfirst} - \ref{equfunctionlast}, determine whether or not the equation represents $y$ as a function of $x$.

\begin{multicols}{3}
\begin{enumerate}
\setcounter{enumi}{\value{HW}}

\item $y = x^{3} - x$ \label{equfunctionfirst}
\item $y = \sqrt{x - 2}$
\item $3x+2y=6$
%\item $x^{3}y = -4$ 
\setcounter{HW}{\value{enumi}}
\end{enumerate}
\end{multicols}

\begin{multicols}{3}
\begin{enumerate}
\setcounter{enumi}{\value{HW}}

\item $x^{2} - y^{2} = 1$
\item $y = \dfrac{x}{x^{2} - 9}$
\item $x = -6$

\setcounter{HW}{\value{enumi}}
\end{enumerate}
\end{multicols}

\begin{multicols}{3}
\begin{enumerate}
\setcounter{enumi}{\value{HW}}

\item  $x = y^2 + 4$

\item $y = x^2 + 4$
\item $x^2 + y^2 = 4$ \label{equfunctionlast}

\setcounter{HW}{\value{enumi}}
\end{enumerate}
\end{multicols}

\comment{
\begin{multicols}{3}
\begin{enumerate}
\setcounter{enumi}{\value{HW}}


\item $y = \sqrt{4-x^2}$
\item $x^2 - y^2 = 4$
\item $x^3 + y^3 = 4$


\setcounter{HW}{\value{enumi}}
\end{enumerate}
\end{multicols}


\begin{multicols}{3}
\begin{enumerate}
\setcounter{enumi}{\value{HW}}

\item $2x + 3y = 4$
\item $2xy = 4$
\item $x^2 = y^2$ 

\setcounter{HW}{\value{enumi}}
\end{enumerate}
\end{multicols}

\begin{enumerate}
\setcounter{enumi}{\value{HW}}

\item Explain why the population $P$ of Sasquatch in a given area is a function of time $t$.  What would be the range of this function?

\item Explain why the relation between your classmates and their email addresses may not be a function.  What about phone numbers and Social Security Numbers?

\setcounter{HW}{\value{enumi}}
\end{enumerate}

The process given in Example \hspace{-.1in} ~\ref{introfunctionlastexample} for determining whether an equation of a relation represents $y$ as a function of $x$ breaks down if we cannot solve the equation for $y$ in terms of $x$.  However, that does not prevent us from proving that an equation fails to represent $y$ as a function of $x$.  What we really need is two points with the same $x$-coordinate and different $y$-coordinates which both satisfy the equation so that the graph of the relation would fail the Vertical Line Test \hspace{-.1in} ~\ref{VLT}.  Discuss with your classmates how you might find such points for the relations given in Exercises \ref{notfuncequfirst} - \ref{notfuncequlast}.

\begin{multicols}{2}
\begin{enumerate}
\setcounter{enumi}{\value{HW}}

\item $x^{3} + y^{3} - 3xy = 0$ \label{notfuncequfirst}
\item $x^{4} = x^{2} + y^{2}$ 

\setcounter{HW}{\value{enumi}}
\end{enumerate}
\end{multicols}

\begin{multicols}{2}
\begin{enumerate}
\setcounter{enumi}{\value{HW}}


\item $y^{2} = x^{3} + 3x^{2}$ 
\item $(x^{2} + y^{2})^{2} = x^{3} + y^{3}$ \label{notfuncequlast}

\setcounter{HW}{\value{enumi}}
\end{enumerate}
\end{multicols}
} %End of \comment - BA

%%%%%%%%%%%%%%%%%%%%%%%%%%%%%%%%%%%%%%%%%%%%%%%%%%%%%%%%%%%%%%%%
In Exercises \ref{buildfunctionfirst} - \ref{buildfunctionlast}, find an expression for $f(x)$.

\begin{enumerate}
\setcounter{enumi}{\value{HW}}

\item $f$ is a function that takes a real number $x$ and performs the following three steps in the order given: (1) multiply by 2; (2) add 3; (3) divide by 4. \label{buildfunctionfirst}

\item $f$ is a function that takes a real number $x$ and performs the following three steps in the order given: (1) add 3; (2) multiply by 2; (3) divide by 4. 

\item $f$ is a function that takes a real number $x$ and performs the following three steps in the order given: (1) divide by 4; (2) add 3; (3) multiply by 2.

\item $f$ is a function that takes a real number $x$ and performs the following three steps in the order given: (1) multiply by 2; (2) add 3; (3) take the square root.

\item $f$ is a function that takes a real number $x$ and performs the following three steps in the order given: (1) add 3; (2) multiply by 2; (3) take the square root.

\item $f$ is a function that takes a real number $x$ and performs the following three steps in the order given: (1) add 3; (2) take the square root; (3) multiply by 2.
\item $f$ is a function that takes a real number $x$ and performs the following three steps in the order given: (1) take the square root; (2) subtract 13; (3) make the quantity the denominator of a fraction with numerator 4. 

\item  $f$ is a function that takes a real number $x$ and performs the following three steps in the order given: (1) subtract 13; (2) take the square root; (3) make the quantity the denominator of a fraction with numerator 4.  

\item  $f$ is a function that takes a real number $x$ and performs the following three steps in the order given: (1) take the square root; (2) make the quantity the denominator of a fraction with numerator 4; (3) subtract 13. 

\item  $f$ is a function that takes a real number $x$ and performs the following three steps in the order given: (1) make the quantity the denominator of a fraction with numerator 4; (2) take the square root; (3) subtract 13. \label{buildfunctionlast}

\setcounter{HW}{\value{enumi}}
\end{enumerate}

In Exercises \ref{funcnotationbasicfirst} - \ref{funcnotationbasiclast}, use the given function $f$ to find and simplify the following:

\begin{multicols}{3}
\begin{itemize}
\item $f(1)$
\item $f(-3)$
\item $f\left(\frac{3}{2} \right)$
\end{itemize}
\end{multicols}

\begin{multicols}{3}
\begin{itemize}
\item  $f(4x)$
\item $4f(x)$
\item $f(-x)$
\end{itemize}
\end{multicols}

\begin{multicols}{3}
\begin{itemize}
\item  $f(x-4)$
\item $f(x) - 4$
\item  $f\left(x^2\right)$
\end{itemize}
\end{multicols}

\begin{multicols}{2}
\begin{enumerate}
\setcounter{enumi}{\value{HW}}

\item  $f(x) = 2x+1$ \label{funcnotationbasicfirst} 
\item  $f(x) = 3 - 4x$

\setcounter{HW}{\value{enumi}}
\end{enumerate}
\end{multicols}

\begin{multicols}{2}
\begin{enumerate}
\setcounter{enumi}{\value{HW}}

\item $f(x) = 2 - x^2$
\item $f(x) = x^2 - 3x + 2$

\setcounter{HW}{\value{enumi}}
\end{enumerate}
\end{multicols}

\begin{multicols}{2}
\begin{enumerate}
\setcounter{enumi}{\value{HW}}

\item $f(x) = \sqrt{x-1}$
\item $f(x) = \dfrac{x}{x-1}$

\setcounter{HW}{\value{enumi}}
\end{enumerate}
\end{multicols}

\begin{multicols}{2}
\begin{enumerate}
\setcounter{enumi}{\value{HW}}

\item $f(x) = 6$
\item $f(x) = 0$ \label{funcnotationbasiclast}

\setcounter{HW}{\value{enumi}}
\end{enumerate}
\end{multicols}

\newpage

In Exercises \ref{secondfuncnotationbasicfirst} - \ref{secondfuncnotationbasiclast}, use the given function $f$ to find and simplify the following:

\begin{multicols}{3}
\begin{itemize}

\item  $f(2)$
\item  $f(-2)$
\item  $f(2a)$

\end{itemize}
\end{multicols}

\begin{multicols}{3}
\begin{itemize}

\item  $2 f(a)$
\item $f(a+2)$
\item $f(a) + f(2)$

\end{itemize}
\end{multicols}

\begin{multicols}{3}
\begin{itemize}

\item  $f \left( \frac{2}{a} \right)$
\item $\frac{f(a)}{2}$
\item  $f(a + h)$

\end{itemize}
\end{multicols}


\begin{multicols}{2}
\begin{enumerate}
\setcounter{enumi}{\value{HW}}

\item $f(x) = 2x-5$ \label{secondfuncnotationbasicfirst}
\item $f(x) = 5-2x$

\setcounter{HW}{\value{enumi}}
\end{enumerate}
\end{multicols}

\begin{multicols}{2}
\begin{enumerate}
\setcounter{enumi}{\value{HW}}

\item $f(x) = 2x^2 - 1$
\item $f(x) = 3x^2+3x-2$

\setcounter{HW}{\value{enumi}}
\end{enumerate}
\end{multicols}
 
\begin{multicols}{2}
\begin{enumerate}
\setcounter{enumi}{\value{HW}}

\item $f(x) = \sqrt{2x+1}$
\item $f(x) = 1$

\setcounter{HW}{\value{enumi}}
\end{enumerate}
\end{multicols}

\begin{multicols}{2}
\begin{enumerate}
\setcounter{enumi}{\value{HW}}

\item $f(x) = \dfrac{x}{2}$
\item $f(x) = \dfrac{2}{x}$ \label{secondfuncnotationbasiclast}

\setcounter{HW}{\value{enumi}}
\end{enumerate}
\end{multicols}

In Exercises \ref{findzerofuncfirst} - \ref{findzerofunclast}, use the given function $f$ to find $f(0)$ and solve $f(x) = 0$

\begin{multicols}{2}
\begin{enumerate}
\setcounter{enumi}{\value{HW}}

\item $f(x) = 2x - 1$ \label{findzerofuncfirst}
\item $f(x) = 3 - \frac{2}{5} x$

\setcounter{HW}{\value{enumi}}
\end{enumerate}
\end{multicols}

\begin{multicols}{2}
\begin{enumerate}
\setcounter{enumi}{\value{HW}}

\item $f(x) = 2x^2 - 6$
\item $f(x) = x^2 - x - 12$

\setcounter{HW}{\value{enumi}}
\end{enumerate}
\end{multicols}

\begin{multicols}{2}
\begin{enumerate}
\setcounter{enumi}{\value{HW}}

\item $f(x) = \sqrt{x+4}$
\item $f(x) = \sqrt{1-2x}$

\setcounter{HW}{\value{enumi}}
\end{enumerate}
\end{multicols}

\begin{multicols}{2}
\begin{enumerate}
\setcounter{enumi}{\value{HW}}

\item $f(x) = \dfrac{3}{4-x}$
\item $f(x) = \dfrac{3x^2-12x}{4-x^2}$ \label{findzerofunclast}

\setcounter{HW}{\value{enumi}}
\end{enumerate}
\end{multicols}

\comment{
\begin{enumerate}
\setcounter{enumi}{\value{HW}}

\item  Let $f(x) = \left\{  \begin{array}{rcr} x + 5 & \mbox{ if } & x \leq -3 \\ \sqrt{9-x^2} & \mbox{ if } & -3 < x \leq 3 \\ -x+5 & \mbox{ if } & x > 3 \\ \end{array}        \right.$ Compute the following function values.

\begin{multicols}{3}
\begin{enumerate}

\item $f(-4)$
\item  $f(-3)$
\item  $f(3)$

\setcounter{HWindent}{\value{enumii}}
\end{enumerate}
\end{multicols}

\begin{multicols}{3}
\begin{enumerate}
\setcounter{enumii}{\value{HWindent}}

\item  $f(3.001)$
\item  $f(-3.001)$
\item  $f(2)$

\setcounter{HWindent}{\value{enumii}}
\end{enumerate}
\end{multicols}
} %END of \comment - BA
\newpage

\comment{
\item Let ${\displaystyle f(x) = \left\{ \begin{array}{rcr}
x^{2} & \mbox{ if } & x \leq -1\\
\sqrt{1 - x^{2}} & \mbox{ if } & -1 < x \leq 1\\
x & \mbox{ if } & x > 1  \end{array} \right. }$  Compute the following function values.

\begin{multicols}{3}
\begin{enumerate}

\item $f(4)$
\item $f(-3)$
\item $f(1)$

\setcounter{HWindent}{\value{enumii}}
\end{enumerate}
\end{multicols}

\begin{multicols}{3}
\begin{enumerate}
\setcounter{enumii}{\value{HWindent}}

\item $f(0)$
\item $f(-1)$
\item $f(-0.999)$

\setcounter{HWindent}{\value{enumii}}
\end{enumerate}
\end{multicols}

\setcounter{HW}{\value{enumi}}
\end{enumerate}
} %End of \comment - BA

\subsection{Domain and Range}

For each of Exercises \ref{graphfunctionfirst} - \ref{graphfunctionlast} from page \pageref{graphfunctionfirst}, find the domain and range of those relations which represent $y$ as a function of $x$.  Hint: There are exactly 15 relations which represent $y$ as a function of $x$, i.e., that pass the Vertical Line Test.\\

In Exercises \ref{finddomainfirst} - \ref{finddomainlast}, find the domain of the function.

\begin{multicols}{2}
\begin{enumerate}
%\setcounter{enumi}{\value{HW}}

\item $f(x) = x^{4} - 13x^{3} + 56x^{2} - 19$ \label{finddomainfirst}
\item  $f(x) = x^2 - 4$

\setcounter{HW}{\value{enumi}}
\end{enumerate}
\end{multicols}

\begin{multicols}{2}
\begin{enumerate}
\setcounter{enumi}{\value{HW}}

\item $f(x) = \dfrac{x-2}{x+1}$
\item  $f(x) = \dfrac{3x}{x^2+x-2}$

\setcounter{HW}{\value{enumi}}
\end{enumerate}
\end{multicols}

\begin{multicols}{2}
\begin{enumerate}
\setcounter{enumi}{\value{HW}}

\item $f(x) = \dfrac{2x}{x^2+3}$
\item  $f(x) = \dfrac{2x}{x^2-3}$

\setcounter{HW}{\value{enumi}}
\end{enumerate}
\end{multicols}

\begin{multicols}{2}
\begin{enumerate}
\setcounter{enumi}{\value{HW}}

\item  $f(x) = \dfrac{x+4}{x^2 - 36}$
\item $f(x) = \dfrac{x-2}{x-2}$  

\setcounter{HW}{\value{enumi}}
\end{enumerate}
\end{multicols}

\begin{multicols}{2}
\begin{enumerate}
\setcounter{enumi}{\value{HW}}

\item  $f(x) = \sqrt{3-x}$
\item $f(x) = \sqrt{2x+5}$  

\setcounter{HW}{\value{enumi}}
\end{enumerate}
\end{multicols}

\begin{multicols}{2}
\begin{enumerate}
\setcounter{enumi}{\value{HW}}

\item  $f(x) = 9x\sqrt{x+3}$
\item $f(x) = \dfrac{\sqrt{7-x}}{x^2+1}$  

\setcounter{HW}{\value{enumi}}
\end{enumerate}
\end{multicols}

\begin{multicols}{2}
\begin{enumerate}
\setcounter{enumi}{\value{HW}}

\item  $f(x) = \sqrt{6x-2}$
\item  $f(x) = \dfrac{6}{\sqrt{6x-2}}$

\setcounter{HW}{\value{enumi}}
\end{enumerate}
\end{multicols}

\begin{multicols}{2}
\begin{enumerate}
\setcounter{enumi}{\value{HW}}

\item  $f(x) = \sqrt[3]{6x-2}$
\item  $f(x) = \dfrac{6}{4 - \sqrt{6x-2}}$

\setcounter{HW}{\value{enumi}}
\end{enumerate}
\end{multicols}

\begin{multicols}{2}
\begin{enumerate}
\setcounter{enumi}{\value{HW}}

\item  $f(x) = \dfrac{\sqrt{6x-2}}{x^2-36}$
%\item  $f(x) = \dfrac{\sqrt[3]{6x-2}}{x^2+36}$
\item $A(x) = \sqrt{x - 7} + \sqrt{9 - x}$

\setcounter{HW}{\value{enumi}}
\end{enumerate}
\end{multicols}

\begin{multicols}{2}
\begin{enumerate}
\setcounter{enumi}{\value{HW}}

\item $s(t) = \dfrac{t}{t - 8}$
\item $Q(r) = \dfrac{\sqrt{r}}{r - 8}$


\setcounter{HW}{\value{enumi}}
\end{enumerate}
\end{multicols}

\begin{multicols}{2}
\begin{enumerate}
\setcounter{enumi}{\value{HW}}

\item $p(n) = \dfrac{n}{\sqrt{n - 8}}$
%\item $A(x) = \sqrt{x - 7} + \sqrt{9 - x}$
\item $g(v) = \dfrac{1}{4 - \dfrac{1}{v^{2}}}$

\setcounter{HW}{\value{enumi}}
\end{enumerate}
\end{multicols}

%\begin{multicols}{2}
%\begin{enumerate}
%\setcounter{enumi}{\value{HW}}

%\item $\alpha(y) = \sqrt[3]{\dfrac{y}{y - 8}}$
%\item $g(v) = \dfrac{1}{4 - \dfrac{1}{v^{2}}}$

%\setcounter{HW}{\value{enumi}}
%\end{enumerate}
%\end{multicols}

\begin{multicols}{2}
\begin{enumerate}
\setcounter{enumi}{\value{HW}}

\item $T(t) = \dfrac{\sqrt{t} - 8}{5-t}$ 
\item $u(w) = \dfrac{w - 8}{5 - \sqrt{w}}$ \label{finddomainlast}

\setcounter{HW}{\value{enumi}}
\end{enumerate}
\end{multicols}

\newpage

\comment{
\begin{enumerate}
\setcounter{enumi}{\value{HW}}

\item  The area $A$ enclosed by a square, in square inches,  is a function of the length of one of its sides $x$, when measured in inches.  This relation is expressed by the formula $A(x) = x^2$ for $x > 0$.  Find $A(3)$ and solve $A(x) = 36$.  Interpret your answers to each.  Why is $x$ restricted to $x > 0$?

\item  The area $A$ enclosed by a circle, in square meters, is a function of its radius $r$, when measured in meters.  This relation is expressed by the formula $A(r) = \pi r^2$ for $r > 0$.  Find $A(2)$ and solve $A(r) = 16\pi$.  Interpret your answers to each.  Why is $r$ restricted to $r > 0$?

\item  The volume $V$ enclosed by a cube, in cubic centimeters, is a function of the length of one of its sides $x$, when measured in centimeters.  This relation is expressed by the formula $V(x) = x^3$ for $x > 0$.  Find $V(5)$ and solve $V(x) = 27$.  Interpret your answers to each.  Why is $x$ restricted to $x > 0$?

\item  The volume $V$ enclosed by a sphere, in cubic feet, is a function of the radius of the sphere $r$, when measured in feet.  This relation is expressed by the formula $V(r) =\frac{4\pi}{3} r^{3}$ for $r > 0$.  Find $V(3)$ and solve $V(r) = \frac{32\pi}{3}$.  Interpret your answers to each.  Why is $r$ restricted to $r > 0$?


\item  The height of an object dropped from the roof of an eight story building is modeled by:  $h(t) = -16t^2 + 64$, $0 \leq t \leq 2$. Here,  $h$ is the height of the object off the ground, in feet, $t$ seconds after the object is dropped.  Find $h(0)$ and solve $h(t) = 0$.  Interpret your answers to each.  Why is $t$ restricted to $0 \leq t \leq 2$?

\item  The temperature $T$ in degrees Fahrenheit $t$ hours after 6 AM is given by $T(t) = -\frac{1}{2} t^2 + 8t+3$ for $0 \leq t \leq 12$. Find and interpret $T(0)$, $T(6)$ and $T(12)$.  

\item The function $C(x) = x^2-10x+27$  models the cost, in \textit{hundreds} of dollars, to produce $x$ \textit{thousand} pens.  Find and interpret $C(0)$, $C(2)$ and $C(5)$.

\item Using data from the  \href{http://www.bts.gov/publications/national_transportation_statistics/html/table_04_23.html}{\underline{Bureau of Transportation Statistics}}, the average fuel economy $F$ in miles per gallon for passenger cars in the US can be modeled by  $F(t) = -0.0076t^2+0.45t + 16$, $0 \leq t \leq 28$, where $t$ is the number of years since $1980$. Use your calculator to find $F(0)$, $F(14)$ and $F(28)$.  Round your answers to two decimal places and interpret your answers to each.


\item The population of Sasquatch in Portage County can be modeled by the function $P(t) = \frac{150t}{t + 15}$, where $t$ represents the number of  years since 1803.  Find and interpret $P(0)$ and $P(205)$.  Discuss with your classmates what the applied domain and range of $P$ should be.

\label{Sasquatchfunc1}

\item \label{piecewiseordering} For $n$ copies of the book \textit{Me and my Sasquatch}, a print on-demand company charges $C(n)$ dollars, where $C(n)$ is determined by the formula \[{\displaystyle C(n) = \left\{ \begin{array}{rcl}  15n & \mbox{ if } & 1 \leq n \leq 25  \\
                                                            13.50n  & \mbox{ if } & 25 < n \leq 50 \\
                                                            12n & \mbox{ if } & n > 50 \\
                                     \end{array} \right. }\]
                                     
                                     
\begin{enumerate}

\item  Find and interpret $C(20)$.  % Ans:  $C(20) = 300$.  It costs $\$300$ for 20 copies of the book.

\item  \label{50vs51} How much does it cost to order 50 copies of the book?  What about 51 copies? %  Ans:  $C(50) = 675$, $\$ 675$.  $C(51) = 612$, $\$ 612$.

\item  Your answer to \ref{50vs51} should get you thinking. Suppose a bookstore estimates it will sell 50 copies of the book.  How many books can, in fact, be ordered for the same price as those 50 copies? (Round your answer to a  whole number of books.)  % Ans:  56 books.

\end{enumerate}

\item \label{piecewiseshipping} An on-line comic book retailer charges shipping costs according to the following formula \[{\displaystyle S(n) = \left\{ \begin{array}{rcl}  1.5 n + 2.5 & \mbox{ if } & 1 \leq n \leq 14  \\
                                                            0  & \mbox{ if } & n \geq 15
                                     \end{array} \right. }\]
                                     
where $n$ is the number of  comic books purchased and $S(n)$ is the shipping cost in dollars.
                                     
\begin{enumerate}

\item  What is the cost to ship 10 comic books?  %  Ans:  $S(10) = 17.5$, $\$ 17.50$.

\item  What is the significance of the formula $S(n) = 0$ for $n \geq 15$?   % Ans:  There is free shipping on orders of $15$ or more comic books. 
 
\end{enumerate}

\item  \label{piecewisemobile} The cost $C$ (in dollars) to talk $m$ minutes a month on a mobile phone plan is modeled by   \[{\displaystyle C(m) = \left\{ \begin{array}{rcl} 25 & \mbox{ if } & 0 \leq m \leq 1000 \\
                                                            25+0.1(m-1000) & \mbox{ if } & m > 1000
                                     \end{array} \right. }\]
                                     
\begin{enumerate}

\item  How much does it cost to talk $750$ minutes per month with this plan?  % Ans:  $C(750) = 25$, $\$ 25$.

\item  How much does it cost to talk $20$ hours a month with this plan?  % Ans:  $C(1200) = 45$, $\$ 45$. 

\item  Explain the terms of the plan verbally.  % Ans:  It costs $\$25$ for up to $1000$ minutes and $10$ cents per minute for each minute over $1000$ minutes.
 
\end{enumerate}


\item  \label{greatestinteger} In Section \ref{SetsofNumbers} we defined the set of \index{integer ! greatest integer function}\textbf{integers} as  $\mathbb{Z} = \{ \ldots, -3, -2, -1, 0, 1, 2, 3, \ldots\}$.\footnote{The use of the letter $\mathbb{Z}$ for the integers is ostensibly because the German word \textit{zahlen} means `to count.'}  The \index{greatest integer function}\textbf{greatest integer of \boldmath{$x$}}, denoted by $\lfloor x \rfloor$, is defined to be the largest integer $k$ with $k \leq x$.

\begin{enumerate}

\item  Find $\lfloor 0.785 \rfloor$, $\lfloor 117 \rfloor$, $\lfloor -2.001 \rfloor$, and $\lfloor \pi + 6 \rfloor$

\item  Discuss with your classmates how $\lfloor x \rfloor$ may be described as a piecewise defined function.

\smallskip

\textbf{HINT:}  There are infinitely many pieces!

\item  Is $\lfloor a + b \rfloor = \lfloor a \rfloor + \lfloor b \rfloor$ always true?  What if $a$ or $b$ is an integer?  Test some values, make a conjecture, and explain your result.

\end{enumerate}

\item We have through our examples tried to convince you that, in general, $f(a + b) \neq f(a) + f(b)$.  It has been our experience that students refuse to believe us so we'll try again with a different approach.  With the help of your classmates, find a function $f$ for which the following properties are always true.

\begin{enumerate}

\item $f(0) = f(-1 + 1) = f(-1) + f(1)$
\item $f(5) = f(2 + 3) = f(2) + f(3)$
\item $f(-6) = f(0 - 6) = f(0) - f(6)$
\item $f(a + b) = f(a) + f(b)\;$ regardless of what two numbers we give you for $a$ and  $b$.

\end{enumerate}

How many functions did you find that failed to satisfy the conditions above?  Did $f(x) = x^{2}$ work?  What about $f(x) = \sqrt{x}$ or $f(x) = 3x + 7$ or $f(x) = \dfrac{1}{x}$?  Did you find an attribute common to those functions that did succeed?  You should have, because there is only one extremely special family of functions that actually works here.  Thus we return to our previous statement, {\bf in general}, $f(a + b) \neq f(a) + f(b)$.

\end{enumerate}
} %End of \comment - BA
%%%%%%%%%%%%%%%%%%%%%%%%%%%%%%%%%%%%%%%%%%%%%%%%%%%%
\subsection{Combining Functions}

In Exercises \ref{basicarithonefirst} - \ref{basicarithonelast}, use the pair of functions $f$ and $g$ to find the following values if they exist.

\begin{multicols}{3}
\begin{itemize}

\item  $(f+g)(2)$ 
\item  $(f-g)(-1)$
\item  $(g-f)(1)$

\end{itemize}
\end{multicols}

\begin{multicols}{3}
\begin{itemize}

\item  $(fg)\left(\frac{1}{2}\right)$
\item  $\left(\frac{f}{g}\right)(0)$
\item  $\left(\frac{g}{f}\right)\left(-2\right)$

\end{itemize}
\end{multicols}

\begin{multicols}{2}
\begin{enumerate}

\item  $f(x) = 3x+1$~~~  $g(x) = 4-x$ \label{basicarithonefirst}
\item  $f(x) = x^2$ ~~~ $g(x) = -2x+1$

\setcounter{HW}{\value{enumi}}
\end{enumerate}
\end{multicols}

\begin{multicols}{2}
\begin{enumerate}
\setcounter{enumi}{\value{HW}}

\item  $f(x) = x^2 - x$ ~~~  $g(x) = 12-x^2$
\item  $f(x) = 2x^3$ ~ \mbox{$g(x) = -x^2-2x-3$}

\setcounter{HW}{\value{enumi}}
\end{enumerate}
\end{multicols}

\begin{multicols}{2}
\begin{enumerate}
\setcounter{enumi}{\value{HW}}

\item  $f(x) = \sqrt{x+3}$ ~~~  $g(x) = 2x-1$
\item  $f(x) = \sqrt{4-x}$ ~ $g(x) = \sqrt{x+2}$

\setcounter{HW}{\value{enumi}}
\end{enumerate}
\end{multicols}

\begin{multicols}{2}
\begin{enumerate}
\setcounter{enumi}{\value{HW}}

\item  $f(x) = 2x$ ~~~  $g(x) = \dfrac{1}{2x+1}$
\item  $f(x) = x^2$ ~~~ $g(x) = \dfrac{3}{2x-3}$

\setcounter{HW}{\value{enumi}}
\end{enumerate}
\end{multicols}

\begin{multicols}{2}
\begin{enumerate}
\setcounter{enumi}{\value{HW}}

\item  $f(x) = x^2$ ~~~  $g(x) = \dfrac{1}{x^2}$
\item  $f(x) = x^2+1$ ~~~ $g(x) = \dfrac{1}{x^2+1}$ \label{basicarithonelast}

\setcounter{HW}{\value{enumi}}
\end{enumerate}
\end{multicols}

In Exercises \ref{basicarithtwofirst} - \ref{basicarithtwolast}, use the pair of functions $f$ and $g$ to find the domain of the indicated function then find and simplify an expression for it.

\begin{multicols}{4}
\begin{itemize}

\item  $(f+g)(x)$
\item  $(f-g)(x)$
\item  $(fg)(x)$
\item  $\left(\frac{f}{g}\right)(x)$

\end{itemize}
\end{multicols}

\begin{multicols}{2}
\begin{enumerate}
\setcounter{enumi}{\value{HW}}

\item $f(x) = 2x+1$ ~~~ $g(x) = x-2$ \label{basicarithtwofirst}
\item $f(x) = 1-4x$ ~~~ $g(x) = 2x-1$

\setcounter{HW}{\value{enumi}}
\end{enumerate}
\end{multicols}

\begin{multicols}{2}
\begin{enumerate}
\setcounter{enumi}{\value{HW}}

\item $f(x) = x^2$ ~~~ $g(x) = 3x-1$
\item $f(x) = x^2-x$ ~~~ $g(x) = 7x$

\setcounter{HW}{\value{enumi}}
\end{enumerate}
\end{multicols}

\begin{multicols}{2}
\begin{enumerate}
\setcounter{enumi}{\value{HW}}

\item $f(x) = x^2-4$ ~~~ $g(x) = 3x+6$
\item \mbox{$f(x) = -x^2+x+6$ ~ $g(x) = x^2-9$}

\setcounter{HW}{\value{enumi}}
\end{enumerate}
\end{multicols}

\begin{multicols}{2}
\begin{enumerate}
\setcounter{enumi}{\value{HW}}

\item $f(x) = \dfrac{x}{2}$ ~~~ $g(x) = \dfrac{2}{x}$
\item $f(x) =x-1$ ~~~ $g(x) = \dfrac{1}{x-1}$

\setcounter{HW}{\value{enumi}}
\end{enumerate}
\end{multicols}

\begin{multicols}{2}
\begin{enumerate}
\setcounter{enumi}{\value{HW}}

\item $f(x) = x$ ~~~ $g(x) = \sqrt{x+1}$
\item $f(x) = g(x) = \sqrt{x-5}$ \label{basicarithtwolast}

\setcounter{HW}{\value{enumi}}
\end{enumerate}
\end{multicols}

\newpage

\comment{
In Exercises \ref{diffquotexerfirst} - \ref{diffquotexerlast}, find and simplify the difference quotient $\dfrac{f(x+h) - f(x)}{h}$ for the given function.

\begin{multicols}{2}

\begin{enumerate}
\setcounter{enumi}{\value{HW}}

\item $f(x) = 2x - 5$ \label{diffquotexerfirst}
\item $f(x) = -3x + 5$

\setcounter{HW}{\value{enumi}}
\end{enumerate}
\end{multicols}

\begin{multicols}{2}
\begin{enumerate}
\setcounter{enumi}{\value{HW}}

\item $f(x) = 6$
\item $f(x) = 3x^2 - x$

\setcounter{HW}{\value{enumi}}
\end{enumerate}
\end{multicols}

\begin{multicols}{2}
\begin{enumerate}
\setcounter{enumi}{\value{HW}}

\item $f(x) = -x^2 + 2x - 1$
\item  $f(x) = 4x^2$ 

\setcounter{HW}{\value{enumi}}
\end{enumerate}
\end{multicols}

\begin{multicols}{2}
\begin{enumerate}
\setcounter{enumi}{\value{HW}}

\item  $f(x) = x-x^2$ 
\item $f(x) = x^{3} + 1$

\setcounter{HW}{\value{enumi}}
\end{enumerate}
\end{multicols}

\begin{multicols}{2}
\begin{enumerate}
\setcounter{enumi}{\value{HW}}

\item $f(x) = mx + b\;$ where $m \neq 0$
\item $f(x) = ax^{2} + bx + c\;$ where $a \neq 0$

\setcounter{HW}{\value{enumi}}
\end{enumerate}
\end{multicols}

\begin{multicols}{2}
\begin{enumerate}
\setcounter{enumi}{\value{HW}}

\item $f(x) = \dfrac{2}{x}$
\item $f(x) = \dfrac{3}{1-x}$

\setcounter{HW}{\value{enumi}}
\end{enumerate}
\end{multicols}

\begin{multicols}{2}
\begin{enumerate}
\setcounter{enumi}{\value{HW}}

\item  $f(x) = \dfrac{1}{x^2}$
\item  $f(x) = \dfrac{2}{x+5}$

\setcounter{HW}{\value{enumi}}
\end{enumerate}
\end{multicols}

\begin{multicols}{2}
\begin{enumerate}
\setcounter{enumi}{\value{HW}}

\item $f(x) = \dfrac{1}{4x-3}$ 
\item $f(x) = \dfrac{3x}{x+1}$ 

\setcounter{HW}{\value{enumi}}
\end{enumerate}
\end{multicols}

\begin{multicols}{2}
\begin{enumerate}
\setcounter{enumi}{\value{HW}}

\item $f(x) = \dfrac{x}{x - 9}$
\item $f(x) = \dfrac{x^2}{2x+1}$

\setcounter{HW}{\value{enumi}}
\end{enumerate}
\end{multicols}

\begin{multicols}{2}
\begin{enumerate}
\setcounter{enumi}{\value{HW}}

\item  $f(x) = \sqrt{x-9}$
\item  $f(x) = \sqrt{2x+1}$

\setcounter{HW}{\value{enumi}}
\end{enumerate}
\end{multicols}

\begin{multicols}{2}
\begin{enumerate}
\setcounter{enumi}{\value{HW}}

\item  $f(x) = \sqrt{-4x+5}$
\item  $f(x) = \sqrt{4-x}$

\setcounter{HW}{\value{enumi}}
\end{enumerate}
\end{multicols}

\begin{multicols}{2}
\begin{enumerate}
\setcounter{enumi}{\value{HW}}

\item  $f(x) = \sqrt{ax+b}$, where $a \neq 0$.
\item  $f(x) = x \sqrt{x}$ 

\setcounter{HW}{\value{enumi}}
\end{enumerate}
\end{multicols}

\begin{enumerate}
\setcounter{enumi}{\value{HW}}

\item  $f(x) = \sqrt[3]{x}$.  \textbf{HINT:}  $(a-b)\left(a^2+ab+b^2\right) = a^3 - b^3$  \label{diffquotexerlast}

\setcounter{HW}{\value{enumi}}
\end{enumerate}

In Exercises \ref{econexerfirst} - \ref{econexerlast}, $C(x)$ denotes the cost to produce $x$ items and $p(x)$ denotes the price-demand function in the given economic scenario.  In each Exercise, do the following:

\begin{multicols}{2}
\begin{itemize}

\item  Find and interpret $C(0)$.
\item  Find and interpret $\overline{C}(10)$.

\end{itemize}
\end{multicols}

\begin{multicols}{2}
\begin{itemize}

\item  Find and interpret $p(5)$
\item  Find and simplify $R(x)$.

\end{itemize}
\end{multicols}

\begin{multicols}{2}
\begin{itemize}

\item  Find and simplify $P(x)$.
\item  Solve $P(x) = 0$ and interpret.

\end{itemize}
\end{multicols}

\begin{enumerate}
\setcounter{enumi}{\value{HW}}

\item  The cost, in dollars, to produce $x$ ``I'd rather be a Sasquatch'' T-Shirts is $C(x) = 2x+26$, $x \geq 0$ and the price-demand function, in dollars per shirt,  is $p(x) = 30 - 2x$, $0 \leq x \leq 15$. \label{econexerfirst}

\item  The cost, in dollars, to produce $x$ bottles of $100 \%$ All-Natural Certified Free-Trade Organic Sasquatch Tonic is $C(x) = 10x+100$, $x \geq 0$ and the price-demand function, in dollars per bottle,  is $p(x) = 35 - x$, $0 \leq x \leq 35$.


\item  The cost, in cents, to produce $x$ cups of Mountain Thunder Lemonade at Junior's Lemonade Stand  is $C(x) = 18x + 240$, $x \geq 0$ and the price-demand function, in cents per cup,  is $p(x) = 90-3x$, $0 \leq x \leq 30$.


\item  The daily cost, in dollars, to produce $x$ Sasquatch Berry Pies $C(x) = 3x + 36$, $x \geq 0$ and the price-demand function, in  dollars per pie,  is $p(x) = 12-0.5x$, $0 \leq x \leq 24$.

\item  The monthly cost, in hundreds of dollars, to produce $x$ custom built electric scooters is $C(x) = 20x + 1000$, $x \geq 0$ and the price-demand function, in  hundreds of dollars per scooter,  is $p(x) = 140-2x$, $0 \leq x \leq 70$. \label{econexerlast}

\setcounter{HW}{\value{enumi}}
\end{enumerate}
} %End of \comment - BA

In Exercises \ref{reformarithfirst} - \ref{reformarithlast}, let $f$ be the function defined by \[f = \{(-3, 4), (-2, 2), (-1, 0), (0, 1), (1, 3), (2, 4), (3, -1)\}\] and let $g$ be the function defined \[g = \{(-3, -2), (-2, 0), (-1, -4), (0, 0), (1, -3), (2, 1), (3, 2)\}\]. Compute the indicated value if it exists.

\begin{multicols}{3}
\begin{enumerate}
\setcounter{enumi}{\value{HW}}

\item $(f + g)(-3)$ \label{reformarithfirst}
\item $(f - g)(2)$
\item $(fg)(-1)$

\setcounter{HW}{\value{enumi}}
\end{enumerate}
\end{multicols}

\begin{multicols}{3}
\begin{enumerate}
\setcounter{enumi}{\value{HW}}

\item $(g + f)(1)$
\item $(g - f)(3)$
\item $(gf)(-3)$

\setcounter{HW}{\value{enumi}}
\end{enumerate}
\end{multicols}

\begin{multicols}{3}
\begin{enumerate}
\setcounter{enumi}{\value{HW}}

\item $\left(\frac{f}{g}\right)(-2)$
\item $\left(\frac{f}{g}\right)(-1)$
\item $\left(\frac{f}{g}\right)(2)$

\setcounter{HW}{\value{enumi}}
\end{enumerate}
\end{multicols}

\begin{multicols}{3}
\begin{enumerate}
\setcounter{enumi}{\value{HW}}

\item $\left(\frac{g}{f}\right)(-1)$
\item $\left(\frac{g}{f}\right)(3)$
\item $\left(\frac{g}{f}\right)(-3)$ \label{reformarithlast}

\setcounter{HW}{\value{enumi}}
\end{enumerate}
\end{multicols}

%%%%%%%%%%%%%%%%%%%%%%%%%%%%%%%%%%%%%%%%%%

In Exercises \ref{funccompeval1first} - \ref{funccompeval1last}, use the given pair of functions to find the following values if they exist.

\begin{multicols}{3}

\begin{itemize}

\item  $(g\circ f)(0)$

\item  $(f\circ g)(-1)$

\item  $(f \circ f)(2)$

\end{itemize}

\end{multicols}

\begin{multicols}{3}

\begin{itemize}

\item  $(g\circ f)(-3)$

\item  $(f\circ g)\left(\frac{1}{2}\right)$

\item  $(f \circ f)(-2)$

\end{itemize}

\end{multicols}

\begin{multicols}{2}
\begin{enumerate}
\setcounter{enumi}{\value{HW}}

\item  $f(x) = x^2$, $g(x) = 2x+1$ \label{funccompeval1first}
\item  $f(x) = 4-x$, $g(x) = 1-x^2$

\setcounter{HW}{\value{enumi}}
\end{enumerate}
\end{multicols}

\begin{multicols}{2}
\begin{enumerate}
\setcounter{enumi}{\value{HW}}

\item  $f(x) = 4-3x$, $g(x) = |x|$
\item  $f(x) = |x-1|$, $g(x) = x^2-5$

\setcounter{HW}{\value{enumi}}
\end{enumerate}
\end{multicols}

\begin{multicols}{2}
\begin{enumerate}
\setcounter{enumi}{\value{HW}}

\item  $f(x) = 4x+5$, $g(x) = \sqrt{x}$
\item  $f(x) = \sqrt{3-x}$, $g(x) = x^2+1$

\setcounter{HW}{\value{enumi}}
\end{enumerate}
\end{multicols}

%\begin{multicols}{2}
%\begin{enumerate}
%\setcounter{enumi}{\value{HW}}

%\item  $f(x) = 6-x-x^2$, $g(x) = x\sqrt{x+10}$
%\item  $f(x) = \sqrt[3]{x+1}$, $g(x) = 4x^2-x$

%\setcounter{HW}{\value{enumi}}
%\end{enumerate}
%\end{multicols}

\begin{multicols}{2}
\begin{enumerate}
\setcounter{enumi}{\value{HW}}

\item  $f(x) = \dfrac{3}{1-x}$, $g(x) = \dfrac{4x}{x^2+1}$
\item  $f(x) = \dfrac{x}{x+5}$, $g(x) = \dfrac{2}{7-x^2}$ \label{funccompeval1last}


\setcounter{HW}{\value{enumi}}
\end{enumerate}
\end{multicols}

%\begin{multicols}{2}
%\begin{enumerate}
%\setcounter{enumi}{\value{HW}}

%\item  $f(x) = \dfrac{2x}{5-x^2}$, $g(x) = \sqrt{4x+1}$
%\item  $f(x) =\sqrt{2x+5}$, $g(x) = \dfrac{10x}{x^2+1}$

%\setcounter{HW}{\value{enumi}}
%\end{enumerate}
%\end{multicols}

\newpage

In Exercises \ref{funccompexp1first} - \ref{funccompexp1last}, use the given pair of functions to find and simplify expressions for the following functions and state the domain of each using interval notation.

\begin{multicols}{3}

\begin{itemize}

\item  $(g \circ f)(x)$

\item  $(f \circ g)(x)$

\item  $(f \circ f)(x)$


\end{itemize}

\end{multicols}


\begin{multicols}{2}
\begin{enumerate}
\setcounter{enumi}{\value{HW}}

\item  $f(x) = 2x+3$, $g(x) = x^2-9$ \label{funccompexp1first}
\item  $f(x) = x^2 -x+1$, $g(x) = 3x-5$ 

\setcounter{HW}{\value{enumi}}
\end{enumerate}
\end{multicols}

\begin{multicols}{2}
\begin{enumerate}
\setcounter{enumi}{\value{HW}}

\item  $f(x) = x^2-4$, $g(x) = |x|$
\item  $f(x) = 3x-5$, $g(x) = \sqrt{x}$ 

\setcounter{HW}{\value{enumi}}
\end{enumerate}
\end{multicols}

\begin{multicols}{2}
\begin{enumerate}
\setcounter{enumi}{\value{HW}}

\item  $f(x) = |x+1|$, $g(x) = \sqrt{x}$
\item  $f(x) = 3-x^2$, $g(x) = \sqrt{x+1}$ 

\setcounter{HW}{\value{enumi}}
\end{enumerate}
\end{multicols}

\begin{multicols}{2}
\begin{enumerate}
\setcounter{enumi}{\value{HW}}

\item  $f(x) = |x|$, $g(x) = \sqrt{4-x}$
\item  \mbox{$f(x) = x^2-x-1$, $g(x) = \sqrt{x-5}$} 

\setcounter{HW}{\value{enumi}}
\end{enumerate}
\end{multicols}

\begin{multicols}{2}
\begin{enumerate}
\setcounter{enumi}{\value{HW}}

\item  $f(x) = 3x-1$, $g(x) = \dfrac{1}{x+3}$
\item  $f(x) = \dfrac{3x}{x-1}$, $g(x) =\dfrac{x}{x-3}$

\setcounter{HW}{\value{enumi}}
\end{enumerate}
\end{multicols}

\begin{multicols}{2}
\begin{enumerate}
\setcounter{enumi}{\value{HW}}

\item  $f(x) = \dfrac{x}{2x+1}$, $g(x) = \dfrac{2x+1}{x}$
\item  $f(x) =  \dfrac{2x}{x^2-4}$, $g(x) =\sqrt{1-x}$ 
 \label{funccompexp1last}

\setcounter{HW}{\value{enumi}}
\end{enumerate}
\end{multicols}

%\pagebreak

\comment{
In Exercises \ref{threefunccompfirst} - \ref{threefunccomplast}, use $f(x) = -2x$, $g(x) = \sqrt{x}$ and $h(x) = |x|$ to find and simplify expressions for the following functions and state the domain of each using interval notation.

\begin{multicols}{3}

\begin{enumerate}
\setcounter{enumi}{\value{HW}}

\item $(h\circ g \circ f)(x)$ \label{threefunccompfirst}

\item $(h\circ f \circ g)(x)$

\item $(g\circ f \circ h)(x)$

\setcounter{HW}{\value{enumi}}
\end{enumerate}
\end{multicols}

\begin{multicols}{3}
\begin{enumerate}
\setcounter{enumi}{\value{HW}}

\item $(g\circ h \circ f)(x)$ 

\item $(f\circ h \circ g)(x)$

\item $(f\circ g \circ h)(x)$ \label{threefunccomplast}

\setcounter{HW}{\value{enumi}}
\end{enumerate}
\end{multicols}
} % End of \comment - BA

In Exercises \ref{breakdowncompexfirst} - \ref{breakdownxomexlast},  write the given function as a composition of two or more non-identity functions.  (There are several correct answers, so check your answer using function composition.)

\begin{multicols}{2}
\begin{enumerate}
\setcounter{enumi}{\value{HW}}

\item  $p(x) = (2x+3)^3$ \label{breakdowncompexfirst}
\item  $P(x) = \left(x^2-x+1\right)^5$

\setcounter{HW}{\value{enumi}}
\end{enumerate}
\end{multicols}

\begin{multicols}{2}
\begin{enumerate}
\setcounter{enumi}{\value{HW}}

\item  $h(x) = \sqrt{2x-1}$
\item  $H(x) = |7-3x|$

\setcounter{HW}{\value{enumi}}
\end{enumerate}
\end{multicols}

\begin{multicols}{2}
\begin{enumerate}
\setcounter{enumi}{\value{HW}}

\item  $r(x) = \dfrac{2}{5x+1}$
\item  $R(x) = \dfrac{7}{x^2-1}$

\setcounter{HW}{\value{enumi}}
\end{enumerate}
\end{multicols}

\begin{multicols}{2}
\begin{enumerate}
\setcounter{enumi}{\value{HW}}

\item  $q(x) = \dfrac{|x|+1}{|x|-1}$
\item  $Q(x) = \dfrac{2x^3+1}{x^3-1}$

\setcounter{HW}{\value{enumi}}
\end{enumerate}
\end{multicols}

\begin{multicols}{2}
\begin{enumerate}
\setcounter{enumi}{\value{HW}}

\item  $v(x) = \dfrac{2x+1}{3-4x}$
\item  $w(x) = \dfrac{x^2}{x^4+1}$ \label{breakdownxomexlast}

\setcounter{HW}{\value{enumi}}
\end{enumerate}
\end{multicols}

\begin{enumerate}
\setcounter{enumi}{\value{HW}}

%\item Write the function $F(x) = \sqrt{\dfrac{x^{3} + 6}{x^{3} - 9}}$ as a composition of three or more non-identity functions.

\item Let $g(x) = -x, \, h(x) = x + 2, \, j(x) = 3x$ and $k(x) = x - 4$.  In what order must these functions be composed with $f(x) = \sqrt{x}$ to create $F(x) = 3\sqrt{-x + 2} - 4$?

\item What linear functions could be used to transform $f(x) = x^{3}$ into $F(x) = -\frac{1}{2}(2x - 7)^{3} + 1$?  What is the proper order of composition?

\setcounter{HW}{\value{enumi}}
\end{enumerate}

\newpage

In Exercises \ref{pointcompexfirst} - \ref{pointcompexlast}, let $f$ be the function defined by \[f = \{(-3, 4), (-2, 2), (-1, 0), (0, 1), (1, 3), (2, 4), (3, -1)\}\] and let $g$ be the function defined \[g = \{(-3, -2), (-2, 0), (-1, -4), (0, 0), (1, -3), (2, 1), (3, 2)\}\].  Find the value if it exists.

\begin{multicols}{3}
\begin{enumerate}
\setcounter{enumi}{\value{HW}}

\item $(f \circ g)(3)$ \label{pointcompexfirst}
\item $f(g(-1))$
\item $(f \circ f)(0)$

\setcounter{HW}{\value{enumi}}
\end{enumerate}
\end{multicols}

\begin{multicols}{3}
\begin{enumerate}
\setcounter{enumi}{\value{HW}}


\item $(f \circ g)(-3)$
\item $(g \circ f)(3)$
\item $g(f(-3))$


\setcounter{HW}{\value{enumi}}
\end{enumerate}
\end{multicols}

\begin{multicols}{3}
\begin{enumerate}
\setcounter{enumi}{\value{HW}}

\item $(g \circ g)(-2)$
\item $(g \circ f)(-2)$
\item $g(f(g(0)))$ \label{pointcompexlast}


\setcounter{HW}{\value{enumi}}
\end{enumerate}
\end{multicols}

\comment{
\begin{multicols}{3}
\begin{enumerate}
\setcounter{enumi}{\value{HW}}

\item $f(f(f(-1)))$
\item $f(f(f(f(f(1)))))$
\item $\underbrace{(g \circ g \circ \cdots \circ g)}_{\mbox{$n$ times}}(0)$ 


\setcounter{HW}{\value{enumi}}
\end{enumerate}
\end{multicols}
} %End of \comment - BA
%\pagebreak

In Exercises \ref{twofuncgraphcompfirst} - \ref{twofuncgraphcomplast}, use the graphs of $y=f(x)$ and $y=g(x)$ below to find the function value.

\begin{center}

\begin{tabular}{cc}

\begin{mfpic}[20]{-1}{5}{-1}{5}
\axes
\tlabel[cc](5,-0.5){\scriptsize $x$}
\tlabel[cc](0.5,5){\scriptsize $y$}
\xmarks{1,2,3,4}
\ymarks{1,2,3,4}
\tlpointsep{5pt}
\scriptsize
\axislabels {x}{{$1$} 1, {$2$} 2, {$3$} 3, {$4$} 4}
\axislabels {y}{{$1$} 1, {$2$} 2, {$3$} 3, {$4$} 4}
\polyline{(0,4), (1,2), (2,3), (3,3), (4,0)}
\point[3pt]{(0,4), (1,2), (2,3), (3,3), (4,0)}
\normalsize 
\tcaption{$y = f(x)$}
\end{mfpic}

&

\hspace{1in}

\begin{mfpic}[20]{-1}{5}{-1}{5}
\axes
\tlabel[cc](5,-0.5){\scriptsize $x$}
\tlabel[cc](0.5,5){\scriptsize $y$}
\xmarks{1,2,3,4}
\ymarks{1,2,3,4}
\tlpointsep{5pt}
\scriptsize
\axislabels {x}{{$1$} 1, {$2$} 2, {$3$} 3, {$4$} 4}
\axislabels {y}{{$1$} 1, {$2$} 2, {$3$} 3, {$4$} 4}
\polyline{(0,0), (1,3), (2,3), (3,0), (4,4)}
\point[3pt]{(0,0), (1,3), (2,3), (3,0), (4,4)}
\normalsize 
\tcaption{$y = g(x)$}
\end{mfpic}

\end{tabular}

\end{center}

\begin{multicols}{3}
\begin{enumerate}
\setcounter{enumi}{\value{HW}}

\item  $(g\circ f)(1)$ \label{twofuncgraphcompfirst}
\item  $(f \circ g)(3)$
\item  $(g\circ f)(2)$
\setcounter{HW}{\value{enumi}}
\end{enumerate}
\end{multicols}

\begin{multicols}{3}
\begin{enumerate}
\setcounter{enumi}{\value{HW}}
\item  $(f\circ g)(0)$  
\item  $(f\circ f)(1)$
\item  $(g \circ g)(1)$ \label{twofuncgraphcomplast}

\setcounter{HW}{\value{enumi}}
\end{enumerate}
\end{multicols}



\comment{
\begin{enumerate}
\setcounter{enumi}{\value{HW}}


\item The volume $V$ of a cube is a function of its side length $x$.  Let's assume that $x = t + 1$ is also a function of time $t$, where $x$ is measured in inches and $t$ is measured in minutes.  Find a formula for $V$ as a function of $t$.

\item  Suppose a local vendor charges $\$2$ per hot dog and that the number of hot dogs sold per hour $x$ is given by $x(t) = -4t^2+20t+92$, where $t$ is the number of hours since $10$ AM, $0 \leq t \leq 4$.

\begin{enumerate}

\item  Find an expression for the revenue per hour $R$ as a function of $x$.
\item  Find and simplify $\left(R \circ x\right)(t)$.  What does this represent?
\item  What is the revenue per hour at noon?

\end{enumerate}

\item Discuss with your classmates how `real-world' processes such as filling out federal income tax forms or computing your final course grade could be viewed as a use of function composition.  Find a process for which composition with itself (iteration) makes sense.

\end{enumerate}
} %End of \comment - BA

%%%%%%%%%%%%%%%%%%%%%%%%%%%%%%%%%%%%%%

\newpage
\subsection{Inverse Functions}

In Exercises \ref{inversehwfirst} - \ref{inversehwlast}, show that the given function is one-to-one and find its inverse.  Check your answers algebraically and graphically.  Verify that the range of $f$ is the domain of $f^{-1}$ and vice-versa.

\begin{multicols}{2}
\begin{enumerate}

\item $f(x) = 6x - 2$ \label{inversehwfirst}
\item $f(x) = 42-x$


\setcounter{HW}{\value{enumi}}
\end{enumerate}
\end{multicols}


\begin{multicols}{2}
\begin{enumerate}
\setcounter{enumi}{\value{HW}}

\item $f(x) = \dfrac{x-2}{3} + 4$
\item $f(x)  = 1 - \dfrac{4+3x}{5}$


\setcounter{HW}{\value{enumi}}
\end{enumerate}
\end{multicols}


\begin{multicols}{2}
\begin{enumerate}
\setcounter{enumi}{\value{HW}}

\item $f(x) = \sqrt{3x-1}+5$
\item $f(x) = 2-\sqrt{x - 5}$

\setcounter{HW}{\value{enumi}}
\end{enumerate}
\end{multicols}

\begin{multicols}{2}
\begin{enumerate}
\setcounter{enumi}{\value{HW}}

\item $f(x) = 3\sqrt{x-1}-4$
\item $f(x) = 1 - 2\sqrt{2x+5}$


\setcounter{HW}{\value{enumi}}
\end{enumerate}
\end{multicols}

\begin{multicols}{2}
\begin{enumerate}
\setcounter{enumi}{\value{HW}}

\item $f(x) = \sqrt[5]{3x-1}$
\item $f(x) = 3-\sqrt[3]{x-2}$

\setcounter{HW}{\value{enumi}}
\end{enumerate}
\end{multicols}

\begin{multicols}{2}
\begin{enumerate}
\setcounter{enumi}{\value{HW}}

\item $f(x) = x^2 - 10x$, $x \geq 5$
\item $f(x) = 3(x + 4)^{2} - 5, \; x \leq -4$

\setcounter{HW}{\value{enumi}}
\end{enumerate}
\end{multicols}


\begin{multicols}{2}
\begin{enumerate}
\setcounter{enumi}{\value{HW}}

\item $f(x) = x^2-6x+5, \; x \leq 3$
\item $f(x) = 4x^2 + 4x + 1$, $x < -1$

\setcounter{HW}{\value{enumi}}
\end{enumerate}
\end{multicols}


\begin{multicols}{2}
\begin{enumerate}
\setcounter{enumi}{\value{HW}}

\item $f(x) = \dfrac{3}{4-x}$
\item $f(x) = \dfrac{x}{1-3x}$

\setcounter{HW}{\value{enumi}}
\end{enumerate}
\end{multicols}


\begin{multicols}{2}
\begin{enumerate}
\setcounter{enumi}{\value{HW}}

\item $f(x) = \dfrac{2x-1}{3x+4}$
\item $f(x) = \dfrac{4x + 2}{3x - 6}$

\setcounter{HW}{\value{enumi}}
\end{enumerate}
\end{multicols}


\begin{multicols}{2}
\begin{enumerate}
\setcounter{enumi}{\value{HW}}

\item $f(x) = \dfrac{-3x - 2}{x + 3}$ 

\item $f(x) = \dfrac{x-2}{2x-1}$  \label{inversehwlast}

\setcounter{HW}{\value{enumi}}
\end{enumerate}
\end{multicols}

Find the inverses of the functions in Exercises \ref{genericinversefirst} - \ref{genericinverselast}.

\begin{multicols}{2}
\begin{enumerate}
\setcounter{enumi}{\value{HW}}

\item $f(x) = ax + b, \; a \neq 0$ \label{genericinversefirst}
\item $f(x) = a\sqrt{x - h} + k, \; a \neq 0, x \geq h$


\setcounter{HW}{\value{enumi}}
\end{enumerate}
\end{multicols}

\begin{multicols}{2}
\begin{enumerate}
\setcounter{enumi}{\value{HW}}
\item $f(x) = ax^{2} + bx + c$ where $a \neq 0, \, x \geq -\dfrac{b}{2a}$.

\item $f(x) = \dfrac{ax + b}{cx + d}\;$ \label{genericinverselast}

\setcounter{HW}{\value{enumi}}
\end{enumerate}
\end{multicols}

\newpage
\subsection{Transformations}

Suppose $(2,-3)$ is on the graph of $y = f(x)$.  In Exercises \ref{transformpointfirst} - \ref{transformpointlast}, use the given point to %Theorem \ref{transformationsthm} to 
find a point on the graph of the given transformed function.

\begin{multicols}{3}
\begin{enumerate}

\item $g(x) = f(x)+3$ \label{transformpointfirst}
\item $g(x) = f(x+3)$
\item $g(x) = f(x)-1$

\setcounter{HW}{\value{enumi}}
\end{enumerate}
\end{multicols}

\begin{multicols}{3}
\begin{enumerate}
\setcounter{enumi}{\value{HW}}

\item $g(x) = f(x-1)$
\item $g(x) = 3f(x)$
\item $g(x) = f(3x)$

\setcounter{HW}{\value{enumi}}
\end{enumerate}
\end{multicols}

\begin{multicols}{3}
\begin{enumerate}
\setcounter{enumi}{\value{HW}}

\item $g(x) = -f(x)$
\item $g(x) = f(-x)$
\item $g(x) = f(x-3)+1$

\setcounter{HW}{\value{enumi}}
\end{enumerate}
\end{multicols}

\begin{multicols}{3}
\begin{enumerate}
\setcounter{enumi}{\value{HW}}

\item $g(x) = 2f(x+1)$
\item $g(x) = 10 - f(x)$
\item $g(x) = 3f(2x) - 1$

\setcounter{HW}{\value{enumi}}
\end{enumerate}
\end{multicols}

\begin{multicols}{3}
\begin{enumerate}
\setcounter{enumi}{\value{HW}}

\item $g(x) = \frac{1}{2} f(4-x)$
\item \mbox{$g(x) = 5f(2x)+3$}
\item \mbox{$g(x) = 2f(1-x) -1$}

\setcounter{HW}{\value{enumi}}
\end{enumerate}
\end{multicols}

\begin{multicols}{2}
\begin{enumerate}
\setcounter{enumi}{\value{HW}}

%\item $g(x) =f\left(\dfrac{7-2x}{4}\right)$
\item $g(x) = \dfrac{f(3x) - 1}{2}$
\item $g(x) = \dfrac{4-f(3x-1)}{7}$ \label{transformpointlast}

\setcounter{HW}{\value{enumi}}
\end{enumerate}
\end{multicols}

The complete graph of $f(x)=|x|$ is given below.  In Exercises \ref{transformgraphfirst} - \ref{transformgraphlast}, use it %and Theorem \ref{transformationsthm} 
to graph the given transformed function.

\vspace{-.1in}
\begin{center}

\begin{mfpic}[15]{-5}{5}{-1}{5}
\axes
\arrow \reverse \arrow \polyline{(-4,4), (0,0), (4,4)}
\point[3pt]{(-2,2), (0,0), (2,2)}
\tlabel[cc](5,-0.25){\scriptsize $x$}
\tlabel[cc](0.25,5){\scriptsize $y$}
\tlabel[cc](-2.5,1.25){\scriptsize $(-2,2)$}
\tlabel[cc](0.75,-0.5){\scriptsize $(0,0)$}
\tlabel[cc](2.25,1.25){\scriptsize $(2,2)$}
\tcaption{The graph for Ex. \ref{transformgraphfirst} - \ref{transformgraphlast}}
\xmarks{-4,-3,-2,-1,2,3,4}
\ymarks{1,2,3,4}
\tlpointsep{5pt}
\scriptsize
\axislabels {x}{{$-4 \hspace{7pt}$} -4,{$-3 \hspace{7pt}$} -3, {$-2 \hspace{7pt}$} -2, {$-1 \hspace{7pt}$} -1, {$2$} 2,{$3$} 3,{$4$} 4}
\axislabels {y}{{$1$} 1, {$2$} 2, {$3$} 3, {$4$} 4}
\normalsize
\end{mfpic} 

\end{center}

\begin{multicols}{3}
\begin{enumerate}
\setcounter{enumi}{\value{HW}}

\item $g(x) = f(x) + 1$ \label{transformgraphfirst}
\item $g(x) = f(x) - 2$
\item $g(x) = f(x+1)$

\setcounter{HW}{\value{enumi}}
\end{enumerate}
\end{multicols}

\begin{multicols}{3}
\begin{enumerate}
\setcounter{enumi}{\value{HW}}

\item $g(x) = f(x - 2)$
\item $g(x) = 2f(x)$
\item $g(x) = f(2x)$

\setcounter{HW}{\value{enumi}}
\end{enumerate}
\end{multicols}

\begin{multicols}{3}
\begin{enumerate}
\setcounter{enumi}{\value{HW}}

\item $g(x) = 2 - f(x)$
\item $g(x) = f(2-x)$
\item $g(x) = 2-f(2-x)$ \label{transformgraphlast}

\setcounter{HW}{\value{enumi}}
\end{enumerate}
\end{multicols}


\begin{enumerate}
\setcounter{enumi}{\value{HW}}

\item Some of the answers to Exercises \ref{transformgraphfirst} - \ref{transformgraphlast} above should be the same.  Which ones match up?  What properties of the graph of $y=f(x)$ contribute to the duplication?

\setcounter{HW}{\value{enumi}}
\end{enumerate}

\newpage

\comment{
The complete graph of $y = f(x)$ is given below.  In Exercises \ref{transsecondgraphfirst} - \ref{transsecondgraphlast}, use it and Theorem \ref{transformationsthm} to graph the given transformed function.

\vspace{-.1in}
\begin{center}

\begin{mfpic}[15]{-5}{5}{-5}{5}
\axes
\polyline{(-2,0), (0,4), (2,0), (4,-2)}
\point[3pt]{(-2,0), (0,4), (2,0), (4,-2)}
\tlabel[cc](5,-0.25){\scriptsize $x$}
\tlabel[cc](0.25,5){\scriptsize $y$}
\tlabel[cc](-2.25,-1.25){\scriptsize $(-2,0)$}
\tlabel[cc](1,4){\scriptsize $(0,4)$}
\tlabel[cc](2,-1.25){\scriptsize $(2,0)$}
\tlabel[cc](4,-2.5){\scriptsize $(4,-2)$}
\tcaption{The graph for Ex. \ref{transsecondgraphfirst} - \ref{transsecondgraphlast}}
\xmarks{-4,-3,-2,-1,1,2,3,4}
\ymarks{-4,-3,-2,-1,1,2,3,4}
\tlpointsep{5pt}
\scriptsize
\axislabels {x}{{$-4 \hspace{7pt}$} -4,{$-3 \hspace{7pt}$} -3, {$-1 \hspace{7pt}$} -1,{$1$} 1,{$3$} 3,{$4$} 4}
\axislabels {y}{{$-4$} -4,{$-3$} -3,{$-2$} -2, {$-1$} -1, {$1$} 1, {$2$} 2, {$3$} 3, {$4$} 4}
\normalsize
\end{mfpic} 

\end{center}

\begin{multicols}{3}
\begin{enumerate}
\setcounter{enumi}{\value{HW}}

\item  $y = f(x) - 1$ \label{transsecondgraphfirst}
\item  $y = f(x + 1)$
\item  $y = \frac{1}{2} f(x)$

\setcounter{HW}{\value{enumi}}
\end{enumerate}
\end{multicols}

\begin{multicols}{3}
\begin{enumerate}
\setcounter{enumi}{\value{HW}}

\item  $y = f(2x)$
\item  $y = - f(x)$
\item  $y = f(-x)$

\setcounter{HW}{\value{enumi}}
\end{enumerate}
\end{multicols}

\begin{multicols}{3}
\begin{enumerate}
\setcounter{enumi}{\value{HW}}

\item  $y = f(x+1) - 1$
\item  $y = 1 - f(x)$
\item  $y = \frac{1}{2}f(x+1)-1$ \label{transsecondgraphlast}

\setcounter{HW}{\value{enumi}}
\end{enumerate}
\end{multicols}
} %End of \commment - BA

The complete graph of $f(x)=\sqrt{9-x^2}$ is given below.  In Exercises \ref{transthirdgraphfirst} - \ref{transthirdgraphlast}, use it
% and Theorem \ref{transformationsthm} 
to graph the given transformed function.

\vspace{-.1in}
\begin{center}

\begin{mfpic}[20]{-4}{4}{-1.5}{4}
\point[3pt]{(-3,0),(3,0),(0,3)}
\parafcn{0,3.14159,0.1}{(3*cos(t), 3*sin(t))}
\tlabel[cc](-3,-1){\small $\left(-3, 0 \right)$}
\tlabel[cc](0.8,3.3){\small $\left(0, 3 \right)$}
\tlabel[cc](3,-1){\small $\left(3, 0 \right)$}
\axes
\tlabel[cc](4,-0.5){\scriptsize $x$}
\tlabel[cc](0.5,4){\scriptsize $y$}
\tcaption{The graph for Ex. \ref{transthirdgraphfirst} - \ref{transthirdgraphlast}}
\xmarks{-3,-2,-1,1,2,3}
\ymarks{-1,1,2,3}
\tlpointsep{5pt}
\scriptsize
\axislabels {x}{{$-3 \hspace{7pt}$} -3, {$-2 \hspace{7pt}$} -2, {$-1 \hspace{7pt}$} -1, {$1$} 1, {$2$} 2, {$3$} 3}
\axislabels {y}{{$-1$} -1, {$1$} 1, {$2$} 2, {$3$} 3}
\normalsize
\end{mfpic}

\end{center}

\begin{multicols}{3}
\begin{enumerate}
\setcounter{enumi}{\value{HW}}

\item $g(x) = f(x) + 3$ \label{transthirdgraphfirst}
\item $h(x) = f(x) - \frac{1}{2}$
\item $j(x) = f\left(x - \frac{2}{3}\right)$

\setcounter{HW}{\value{enumi}}
\end{enumerate}
\end{multicols}

\begin{multicols}{3}
\begin{enumerate}
\setcounter{enumi}{\value{HW}}

\item $a(x) = f(x + 4)$
\item $b(x) = f(x + 1) - 1$ 
\item $c(x) = \frac{3}{5}f(x)$


\setcounter{HW}{\value{enumi}}
\end{enumerate}
\end{multicols}

\begin{multicols}{3}
\begin{enumerate}
\setcounter{enumi}{\value{HW}}


\item $d(x) = -2f(x)$
\item $k(x) = f\left(\frac{2}{3}x\right)$
\item $m(x) = -\frac{1}{4}f(3x)$

\setcounter{HW}{\value{enumi}}
\end{enumerate}
\end{multicols}

\begin{multicols}{2}
\begin{enumerate}
\setcounter{enumi}{\value{HW}}

\item $n(x) = 4f(x - 3) - 6$
\item $p(x) = 4 + f(1 - 2x)$ \label{transthirdgraphlast}
%\item $q(x) = -\frac{1}{2}f\left(\frac{x + 4}{2}\right) - 3$

\setcounter{HW}{\value{enumi}}
\end{enumerate}
\end{multicols}

\begin{enumerate}
\setcounter{enumi}{\value{HW}}

\newpage

\item The graphs of $y = f(x) = \sqrt[3]{x}$ and $y = g(x)$ are shown below. Find a formula for $g$ based on transformations of the graph of $f$.  Check your answer by confirming that the points shown on the graph of $g$ satisfy the equation $y = g(x)$.

%\[ \begin{array}{cc}

\begin{center}

\begin{mfpic}[10]{-12}{9}{-6}{6}
\point[3pt]{(0,0), (-1, -1), (1, 1), (-8, -2), (8, 2)}
\axes
\tlabel[cc](9,-0.5){\scriptsize $x$}
\tlabel[cc](0.5,6){\scriptsize $y$}
\xmarks{-11 step 1 until 8}
\ymarks{-5 step 1 until 5}
\tlpointsep{4pt}
\axislabels {x}{{\tiny $-11 \hspace{6pt}$} -11, {\tiny $-10 \hspace{6pt}$} -10, {\tiny $-9 \hspace{6pt}$} -9, {\tiny $-8 \hspace{6pt}$} -8, {\tiny $-7 \hspace{6pt}$} -7, {\tiny $-6 \hspace{6pt}$} -6, {\tiny $-5 \hspace{6pt}$} -5, {\tiny $-4 \hspace{6pt}$} -4, {\tiny $-3 \hspace{6pt}$} -3, {\tiny $-2 \hspace{6pt}$} -2, {\tiny $-1 \hspace{6pt}$} -1, {\tiny $1$} 1, {\tiny $2$} 2, {\tiny $3$} 3, {\tiny $4$} 4, {\tiny $5$} 5, {\tiny $6$} 6, {\tiny $7$} 7, {\tiny $8$} 8}
\axislabels {y}{{\tiny $-5$} -5, {\tiny $-4$} -4, {\tiny $-3$} -3, {\tiny $-2$} -2, {\tiny $-1$} -1, {\tiny $1$} 1, {\tiny $2$} 2, {\tiny $3$} 3, {\tiny $4$} 4, {\tiny $5$} 5}
\arrow \reverse \arrow \parafcn{-2.1,2.1,0.1}{(t**3,t)}
\tcaption{\scriptsize $y = \sqrt[3]{x}$}
\end{mfpic}

%&
\end{center}
\begin{center}

\begin{mfpic}[10]{-12}{9}{-6}{6}
\point[3pt]{(-11,3), (-4,1), (-3,-1), (-2,-3), (5,-5)}
\axes
\tlabel[cc](9,-0.5){\scriptsize $x$}
\tlabel[cc](0.5,6){\scriptsize $y$}
\xmarks{-11 step 1 until 8}
\ymarks{-5 step 1 until 5}
\tlpointsep{4pt}
\axislabels {x}{{\tiny $-11 \hspace{6pt}$} -11, {\tiny $-10 \hspace{6pt}$} -10, {\tiny $-9 \hspace{6pt}$} -9, {\tiny $-8 \hspace{6pt}$} -8, {\tiny $-7 \hspace{6pt}$} -7, {\tiny $-6 \hspace{6pt}$} -6, {\tiny $-5 \hspace{6pt}$} -5, {\tiny $-4 \hspace{6pt}$} -4, {\tiny $-3 \hspace{6pt}$} -3, {\tiny $-2 \hspace{6pt}$} -2, {\tiny $-1 \hspace{6pt}$} -1, {\tiny $1$} 1, {\tiny $2$} 2, {\tiny $3$} 3, {\tiny $4$} 4, {\tiny $5$} 5, {\tiny $6$} 6, {\tiny $7$} 7, {\tiny $8$} 8}
\axislabels {y}{{\tiny $-5$} -5, {\tiny $-4$} -4, {\tiny $-3$} -3, {\tiny $-2$} -2, {\tiny $-1$} -1, {\tiny $1$} 1, {\tiny $2$} 2, {\tiny $3$} 3, {\tiny $4$} 4, {\tiny $5$} 5}
\arrow \reverse \arrow \parafcn{-2.1,2.1,0.1}{((t**3 - 3),((-2*t) - 1))}
\tcaption{\scriptsize $y = g(x)$}
\end{mfpic}

%\end{array} \]

\end{center}
\newpage

\item A function $f$ is said to be $\textit{even}$ if $f(x)=f(-x)$.  The graph of an even function will be symmetric about the $y-$axis, since $f(-x)$ represents a reflection of the graph of $f$ about the $y-$axis.  Determine both algebraically (using compositions) and graphically (using transformations) whether each of the following fundamental functions is even.
\begin{enumerate}
	\item $g(x)=x^2$
	\item $k(x)=\sqrt{x}$
	\item $\ell(x)=|x|$
	\item $m(x)=x^3$
	\item $n(x)=\sqrt[3]{x}$
	\item $p(x)=\dfrac{1}{x}$
	\item $q(x)=\sqrt{9-x^2}$
\end{enumerate}
\item A function $f$ is said to be $\textit{odd}$ if $-f(x)=f(-x)$.  Since $f(-x)$ represents a reflection of the graph of $f$ about the $y-$axis and $-f(x)$ represents a reflection of the graph of $f$ about the $x-$axis, whenever these two reflections produce the same graph, the corresponding function will be odd.   In this case, the graph of an odd function is said to be \textit{symmetric about the origin}.  Determine both algebraically (using compositions) and graphically (using transformations) whether each of the following fundamental functions is odd.
\begin{enumerate}
	\item $g(x)=x^2$
	\item $k(x)=\sqrt{x}$
	\item $\ell(x)=|x|$
	\item $m(x)=x^3$
	\item $n(x)=\sqrt[3]{x}$
	\item $p(x)=\dfrac{1}{x}$
	\item $q(x)=\sqrt{9-x^2}$
\end{enumerate}
\end{enumerate}


\newpage

\comment{
The complete graph of $y = S(x)$ is given below. 

\vspace{-.1in}
\begin{center}

\begin{mfpic}[20]{-3}{3}{-4}{4}
\axes
\function{-2,2,0.1}{3*sin(1.570796327*x)}
\point[3pt]{(-2,0), (-1,-3), (0,0), (1,3), (2,0)}
\tlabel[cc](3,-0.25){\scriptsize $x$}
\tlabel[cc](0.25,4){\scriptsize $y$}
\tlabel[cc](-2,0.5){\scriptsize $(-2,0)$}
\tlabel[cc](-1,-3.5){\scriptsize $(-1,-3)$}
\tlabel[cc](0.5,0.25){\scriptsize $(0,0)$}
\tlabel[cc](1,3.5){\scriptsize $(1,3)$}
\tlabel[cc](2,-0.5){\scriptsize $(2,0)$}
\tcaption{The graph of $y=S(x)$}
\xmarks{-2,-1,1,2}
\ymarks{-3,-2,-1,1,2,3}
\tlpointsep{5pt}
\scriptsize
\axislabels {x}{{$-2 \hspace{7pt}$} -2,{$-1 \hspace{7pt}$} -1,{$1$} 1}
\axislabels {y}{{$-3$} -3,{$-2$} -2, {$-1$} -1, {$1$} 1, {$2$} 2, {$3$} 3}
\normalsize
\end{mfpic} 

\end{center}

The purpose of Exercises \ref{transformsinegraphfirst} - \ref{transformsinegraphlast} is to graph $y = \frac{1}{2}S(-x+1) + 1$ by graphing each transformation, one step at a time.

\begin{multicols}{2}
\begin{enumerate}
\setcounter{enumi}{\value{HW}}

\item $y = S_{\text{\tiny $1$}}(x) = S(x + 1)$ \label{transformsinegraphfirst}
\item  $y = S_{\text{\tiny $2$}}(x) =  S_{\text{\tiny $1$}}(-x) = S(-x + 1)$

\setcounter{HW}{\value{enumi}}
\end{enumerate}
\end{multicols}

\begin{multicols}{2}
\begin{enumerate}
\setcounter{enumi}{\value{HW}}

\item  $y = S_{\text{\tiny $3$}}(x) = \frac{1}{2}  S_{\text{\tiny $2$}}(x) =  \frac{1}{2}S(-x+1)$
\item  $y = S_{\text{\tiny $4$}}(x) = S_{\text{\tiny $3$}}(x) + 1 = \frac{1}{2}S(-x+1) + 1$ \label{transformsinegraphlast}

\setcounter{HW}{\value{enumi}}
\end{enumerate}
\end{multicols}
} % End of \comment - BA
\setcounter{HW}{\value{enumi}}

Let $f(x) = \sqrt{x}$.  Find a formula for a function $g$ whose graph is obtained from $f$ from the given sequence of transformations. 

\begin{enumerate}
\setcounter{enumi}{\value{HW}}


\item  (1) shift right 2 units; (2) shift down 3 units

\item  (1) shift down 3 units; (2) shift right 2 units

\item  (1) reflect across the $x$-axis; (2) shift up 1 unit

\item  (1) shift up 1 unit; (2) reflect across the $x$-axis

\item  (1) shift left 1 unit; (2) reflect across the $y$-axis; (3) shift up 2 units

\item  (1) reflect across the $y$-axis;  (2) shift left 1 unit;  (3) shift up 2 units

\item  (1) shift left 3 units; (2) vertical stretch by a factor of 2; (3) shift down 4 units

\item  (1) shift left 3 units; (2) shift down 4 units; (3) vertical stretch by a factor of 2

\item  (1) shift right 3 units; (2) horizontal shrink by a factor of 2; (3) shift up 1 unit

\item  (1) horizontal shrink by a factor of 2; (2) shift right 3 units; (3) shift up 1 unit


\setcounter{HW}{\value{enumi}}
\end{enumerate}

\comment{
\item For many common functions, the properties of Algebra make a horizontal scaling the same as a vertical scaling by (possibly) a different factor.  For example, we stated earlier that $\sqrt{9x} = 3\sqrt{x}$.  With the help of your classmates, find the equivalent vertical scaling produced by the horizontal scalings $y = (2x)^{3}, \, y = |5x|, \, y = \sqrt[3]{27x} \, $ and $\, y = \left(\frac{1}{2} x\right)^{2}$.  What about $y = (-2x)^{3}, \, y = |-5x|, \, y = \sqrt[3]{-27x}\, $ and $\, y = \left(-\frac{1}{2} x\right)^{2}$?

\item We mentioned earlier in the section that, in general, the order in which transformations are applied matters, yet in our first example with two transformations the order did not matter. (You could perform the shift to the left followed by the shift down or you could shift down and then left to achieve the same result.)  With the help of your classmates, determine the situations in which order does matter and those in which it does not.

\item What happens if you reflect an even function across the $y$-axis?  
\item What happens if you reflect an odd function across the $y$-axis?   
\item What happens if you reflect an even function across the $x$-axis?  
\item What happens if you reflect an odd function across the $x$-axis?  
\item How would you describe symmetry about the origin in terms of reflections?

\item As we saw in Example \ref{graphingcalctrans}, the viewing window on the graphing calculator affects how we see the transformations done to a graph.  Using two different calculators, find viewing windows so that $f(x) = x^{2}$ on the one calculator looks like $g(x) = 3x^{2}$ on the other.
} % End of \comment - BA

\newpage

\subsection{Piecewise-Defined and Absolute Value Functions}

\subsubsection{Piecewise-Defined Functions}

\begin{enumerate}
%\setcounter{enumi}{\value{HW}}

\item  Let $f(x) = \left\{  \begin{array}{rcr} x + 5 & \mbox{ if } & x \leq -3 \\ \sqrt{9-x^2} & \mbox{ if } & -3 < x \leq 3 \\ -x+5 & \mbox{ if } & x > 3 \\ \end{array}        \right.$\pp
Compute the following function values.

\begin{multicols}{3}
\begin{enumerate}

\item $f(-4)$
\item  $f(-3)$
\item  $f(3)$

\setcounter{HWindent}{\value{enumii}}
\end{enumerate}
\end{multicols}

\begin{multicols}{3}
\begin{enumerate}
\setcounter{enumii}{\value{HWindent}}

\item  $f(3.1)$
\item  $f(-3.01)$
\item  $f(2)$

\setcounter{HWindent}{\value{enumii}}
\end{enumerate}
\end{multicols}

\item Let ${\displaystyle f(x) = \left\{ \begin{array}{rcr}
x^{2} & \mbox{ if } & x \leq -1\\
\sqrt{1 - x^{2}} & \mbox{ if } & -1 < x \leq 1\\
x & \mbox{ if } & x > 1  \end{array} \right. }$\pp
Compute the following function values.

\begin{multicols}{3}
\begin{enumerate}

\item $f(4)$
\item $f(-3)$
\item $f(1)$

\setcounter{HWindent}{\value{enumii}}
\end{enumerate}
\end{multicols}

\begin{multicols}{3}
\begin{enumerate}
\setcounter{enumii}{\value{HWindent}}

\item $f(0)$
\item $f(-1)$
\item $f(-0.99)$

\setcounter{HWindent}{\value{enumii}}
\end{enumerate}
\end{multicols}

\setcounter{HW}{\value{enumi}}
\end{enumerate}

In Exercises \ref{sketchpiecefirst} - \ref{sketchpiecelast}, find all possible $x$ such that $f(x)=0$.  Then sketch the graph of the given piecewise-defined function.  Use your graph to identify the domain and range of each function.

\begin{multicols}{2}
\begin{enumerate}
\setcounter{enumi}{\value{HW}}

\item ${\displaystyle f(x) = \left\{ \begin{array}{rcl} 4-x & \mbox{ if } &  x \leq 3 \\
                                                            2 & \mbox{ if } & x > 3 
                                     \end{array} \right. }$ \label{sketchpiecefirst}

\item ${\displaystyle f(x) = \left\{ \begin{array}{rcl} x^2 & \mbox{ if } & x \leq 0 \\
                                                     2x & \mbox{ if } & x > 0
                                  \end{array} \right. }$

\setcounter{HW}{\value{enumi}}
\end{enumerate}
\end{multicols}


\begin{multicols}{2}
\begin{enumerate}
\setcounter{enumi}{\value{HW}}

\item ${\displaystyle f(x) = \left\{ \begin{array}{rcl}  -3 & \mbox{ if } & x < 0 \\
                                                        2x-3 & \mbox{ if } & \scriptsize 0 \leq x \leq 3 \\
                                                            3 & \mbox{ if } & x > 3  
                                     \end{array} \right. }$

\item ${\displaystyle f(x) = \left\{ \begin{array}{rcl} x^2 - 4 & \mbox{ if } &x \leq -2\\
                                                                  4-x^2 & \mbox{ if } & -2 < x < 2 \\
                                                         x^2-4 & \mbox{ if } & x \geq 2 
                                     \end{array} \right. }$


\setcounter{HW}{\value{enumi}}
\end{enumerate}
\end{multicols}

\newpage

\begin{multicols}{2}
\begin{enumerate}
\setcounter{enumi}{\value{HW}}

\item ${\displaystyle f(x) = \left\{ \begin{array}{rcl} -2x - 4 & \mbox{ if } &  x < 0 \\
                                                             3x & \mbox{ if } & x \geq 0 
                                     \end{array} \right. }$

%\item ${\displaystyle f(x) = \left\{ \begin{array}{rcl} \sqrt{x + 4} & \mbox{ if } & -4 \leq x < 5 \\
 %                                                       \sqrt{x - 1} & \mbox{ if } & x \geq 5
  %                                   \end{array} \right. }$
\item ${\displaystyle f(x) = \left\{ \begin{array}{rcl} x^{2} & \mbox{ if } & x \leq -2 \\
                                                        3 - x & \mbox{ if } & -2 < x < 2 \\
                                                            4 & \mbox{ if } & x \geq 2  
                                     \end{array} \right. }$

\setcounter{HW}{\value{enumi}}
\end{enumerate}
\end{multicols}

%\begin{multicols}{2}
\begin{center}
\begin{enumerate}
\setcounter{enumi}{\value{HW}}


\item ${\displaystyle f(x) = \left\{ \begin{array}{rcl} \dfrac{1}{x} & \mbox{ if } & -6 < x < -1\\
                                                                  x & \mbox{ if } & -1 < x < 1 \\
                                                           \sqrt{x} & \mbox{ if } & 1 < x < 9  
                                     \end{array} \right. }$ \label{sketchpiecelast}

\setcounter{HW}{\value{enumi}}
\end{enumerate}
%\end{multicols}
\end{center}


\subsubsection{Absolute Value Functions}

In Exercises \ref{graphabsvalexerfirst} - \ref{graphabsvalexerlast}, find the zeros of each function and the $x$- and $y$-intercepts of each graph, if any exist.  Then graph the given absolute value function and express it as a piecewise-defined function.  Use the graph to determine the domain and range of each function.

\begin{multicols}{3}
\begin{enumerate}
\setcounter{enumi}{\value{HW}}

\item $f(x) = |x + 4|$ \label{graphabsvalexerfirst}
\item $f(x) = |x| + 4$
\item $f(x) = |4x|$


\setcounter{HW}{\value{enumi}}
\end{enumerate}
\end{multicols}

\begin{multicols}{3}
\begin{enumerate}
\setcounter{enumi}{\value{HW}}

\item $f(x) = |2x -5|$ 
\item $f(x) = |-2x+5|$
\item $f(x) = 2|x-\frac{5}{2}|$


\setcounter{HW}{\value{enumi}}
\end{enumerate}
\end{multicols}

\begin{multicols}{3}
\begin{enumerate}
\setcounter{enumi}{\value{HW}}

\item $f(x) = -3|x|$ \vphantom{$\dfrac{1}{3}$}
\item $f(x) = 3|x + 4| - 4$ \vphantom{$\dfrac{1}{3}$}
\item $f(x) = \dfrac{1}{3}|2x - 1|$ \label{graphabsvalexerlast}

\setcounter{HW}{\value{enumi}}
\end{enumerate}
\end{multicols}
