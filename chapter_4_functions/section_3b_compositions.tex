\subsection{Composite Functions}

%{\tmstrong{Objective: }}\pp

%Begin Tyler Wallace:

In addition to the four basic arithmetic operations ($+,-,~\cdot~,\div$), we will now discuss a fifth operation, known as a \textit{composition} and denoted by $\circ$ (not to be confused with a product, $\cdot$). The result of a composition is called a \textit{composite function} and is defined as follows.
\[ \tmmathbf{(f \circ g) (x) = f (g (x))} \]
The notation $(f\circ g)(x)$ above should always be interpreted as ``$f$ of $g$ of $x$''.  In this situation, we consider $g$ to be the \textit{inner} function, since it is being substituted into $f$ for $x$.  Consequently, we refer to $f$ as the \textit{outer} function.\pp
Similarly, if we reversed the order of the two functions $f$ and $g$, then the resulting composite function $(g\circ f)(x)=g(f(x))$ will have inner function $f$ and outer function $g$, and should be interpreted as ``$g$ of $f$ of $x$''.  As we will see, one should never assume that the two composite functions $f\circ g$ and $g\circ f$ will be equal.\pp

The idea behind a composition, though relatively simple, can often pose a formidable challenge at first.  We will begin by evaluating a composite function at a single value.  This is accomplished by first evaluating the inner function at the specified value, and
then substituting (``plugging in'') the corresponding \textit{output} into the outer function.

\begin{example}~~~Find $(f\circ g)(3)$, where $f(x)=x^2-2x+1$ and $g(x)=x-5$.
  \begin{eqnarray*}
%    (f \circ g) (3) &  &\\
   (f \circ g) (3)=f (g (3)) &  & \text{Rewrite~} f\circ g \text{~as~inner~and~outer~functions}\\
	    &  & \\
 	g (3) = (3) - 5 = - 2~~~~~~~ &  & \text{Evaluate~inner~function~at~} x=3\\
		& & \text{Use~output~of~} -2 \text{~as~input~for~} f\\
    f (- 2) = (- 2)^2 - 2 (- 2) + 1 &  & \tmop{Evaluate~outer~function~at~} x=-2\\
    = 4 + 4 + 1~~~~~~~~~~~~ &  & \tmop{Simplify}\\
    & & \\
		(f \circ g) (3) = 9 &  & \tmop{Our} \tmop{solution}
  \end{eqnarray*}
\end{example}
  
We can also identify a composite function in terms of the variable. In the next example, we will substitute the inner function into the outer function for every instance of the variable and then simplify.  This approach is often referred to as the ``inside-out'' approach by some instructors.

\begin{example}\label{IO}~~~Find $(f \circ g) (x)$, where $f (x) = x^2 - x$ and $g (x) = x + 3$.
  \begin{eqnarray*}
    (f \circ g) (x)=f (g (x)) &  & \text{Rewrite~} f\circ g \text{~as~inner~and~outer~functions}\\
	    &  & \text{Our~inner~function~is~} g(x) = x + 3\\
    f (x + 3) &  & \text{Replace~each~} x \text{~in~} f \text{~with~} (x + 3)\\
		  &  & \text{Make~sure~to~include~parentheses!}\\
    (x + 3)^2 - (x + 3) &  & \tmop{Simplify;~expand~binomial}\\
    (x^2 + 6 x + 9) - (x + 3) &  & \tmop{Distribute~negative}\\
    x^2 + 6 x + 9 - x - 3~~ &  & \tmop{Combine~like~terms}\\
    & & \\
		(f \circ g) (x)=x^2 + 5 x + 6~~~ &  & \tmop{Our} \tmop{solution}\\
		=(x+3)(x+2) & & \text{Our~solution~in~factored~form}
  \end{eqnarray*}
\end{example}

It is important to reiterate that $(f \circ g) (x)$ usually will \textit{not} equal $(g
\circ f) (x)$ as the next example shows.  Again, we will take the ``inside-out'' approach, where the inner function is now $f$ and the outer function is $g$.

\begin{example}~~~Find $(g \circ f) (x)$, where $f (x) = x^2 - x$ and $g (x) = x + 3$.
  \begin{eqnarray*}
    (g \circ f) (x)=g (f (x)) &  & \text{Rewrite~} g\circ f \text{~as~inner~and~outer~functions}\\
	    &  & \text{Our~inner~function~is~} f(x) = x^2 - x\\
    g (x^2 - x) &  & \text{Replace~each~} x \text{~in~} f \text{~with~} (x^2 - x)\\
		& & \\
		%  &  & \text{Make~sure~to~include~parentheses!}\\
    (x^2 - x) + 3 &  & \tmop{Simplify;~remove~parentheses}\\
    & & \\
		(g \circ f) (x)=x^2 - x + 3~~ &  & \tmop{Our} \tmop{solution}
  \end{eqnarray*}
\end{example}

Notice that a simple calculation of the discriminant, $$b^2-4ac=(-1)^2-4(1)(3)=-11<0,$$ tells us that the resulting composite function is irreducible (not factorable) over the real numbers.
\newpage
Here is another example, for additional practice.
\begin{example}~~~Find $(m\circ n)(x)$, where $m(x)=5x^2-x+1$ and $n(x)=x-4$.

  \begin{eqnarray*}
    (m \circ n) (x)=m(n (x)) &  & \text{Rewrite~} m\circ n \text{~as~inner~and~outer~functions}\\
	    &  & \text{Our~inner~function~is~} n(x) = x-4\\
    g (x-4) &  & \text{Replace~each~} x \text{~in~} m \text{~with~} (x-4)\\
		  &  & \text{Make~sure~to~include~parentheses!}\\
    5(x-4)^2 - (x-4)+1 &  & \tmop{Simplify;~expand~binomial}\\
    5(x^2-8x+16) - (x-4)+1 &  & \tmop{Distribute~negative~and~the~five}\\
    5x^2-40x+80 - x + 4+1~~ &  & \tmop{Combine~like~terms}\\
    & & \\
		(m \circ n) (x)=5x^2-41x+85~~~ &  & \tmop{Our} \tmop{solution}\\
	\end{eqnarray*}
\end{example}


It is also possible to compose a function with itself, as the next example shows.
	
	\begin{example}~~~Find $(g\circ g)(x)$, where $g(x)=x^2-2x$.

  \begin{eqnarray*}
    (g \circ g) (x)=g (g (x)) &  & \text{Rewrite~} g\circ g \text{~as~inner~and~outer~functions}\\
	    &  & \text{Our~inner~function~is~} g(x) = x^2-2x\\
    g (x^2-2x) &  & \text{Replace~each~} x \text{~in~} g \text{~with~} x^2-2x\\
		  &  & \text{Make~sure~to~include~parentheses!}\\
    (x^2-2x)^2 - 2(x^2-2x) &  & \tmop{Simplify;~expand~binomial}\\
    (x^4-4x^3+4x^2) - 2(x^2 -2x) &  & \tmop{Distribute~}-2\\
    x^4-4x^3+4x^2 - 2x^2 + 4x~~ &  & \tmop{Combine~like~terms}\\
    & & \\
		(g \circ g) (x)=x^4-4x^3+2x^2+4x~~~ &  & \tmop{Our} \tmop{solution}\\
	\end{eqnarray*}
\end{example}

We close this section by demonstrating the ``outside-in'' approach to finding a composite function $f\circ g$.  The idea behind this approach is to \textit{first} rewrite the outer function $f$ by its given expression, replacing each instance of the variable with the general $g(x)$.  To see that this will yield the same result as the ``inside-out'' approach, we will revisit example \ref{IO} above.

\begin{example}~~~Find $(f \circ g) (x)$, where $f (x) = x^2 - x$ and $g (x) = x + 3$.
  \begin{eqnarray*}
    (f \circ g) (x)=f (g (x)) &  & \text{Rewrite~} f\circ g \text{~as~inner~and~outer~functions}\\
	    &  & \text{Our~outer~function~is~} f(x) = x^2 - x\\
    \left[g(x)\right]^2-\left[g(x)\right]~~~~ &  & \text{Replace~each~} x \text{~in~} f \text{~with~} g(x)\\
		 (x + 3)^2 - (x + 3) &  & \text{Replace~each~} g(x) \text{~by~} x+3\\
		& & \text{Make~sure~to~include~parentheses!}\\
    (x^2 + 6 x + 9) - (x + 3)  &  & \tmop{Simplify;~expand~binomial}\\
    x^2 + 6 x + 9 - x - 3~~ &  & \tmop{Distribute~negative}\\
    x^2 + 5 x + 6~~ &  & \tmop{Combine~like~terms}\\
    & & \\
		(f \circ g) (x)=x^2 + 5 x + 6~~~ &  & \tmop{Our} \tmop{solution}\\
		=(x+3)(x+2) & & \text{Our~solution~in~factored~form}
  \end{eqnarray*}
\end{example}


{\tmstrong{World View Note:}} The term ``function'' came from Gottfried Wilhelm
Leibniz, a German mathematician from the late 17$^{\text{th}}$ century.

	
	
	
	
	
\comment{


%Below is Tyler Wallace that is unnecessary

Often as we add, subtract, multiply, or divide functions, we do so in a way
that keeps the variable. If there is no number to plug into the equations we
will simply use each equation, in parenthesis, and simplify the expression.

\begin{example}
 \ 
\end{example}
  
  \begin{eqnarray*}
    f (x) = 2 x - 4 &  & \\
    g (x) = x^2 - x + 5 &  & \tmop{Write} \tmop{subtraction} \tmop{problem}
    \tmop{of} \tmop{functions}\\
    \tmop{Find} (f - g) (x) &  & \\
    &  & \\
    f (x) - g (x) &  & \tmop{Replace} f (x) \tmop{with} (2 x - 3) \tmop{and} g
    (x) \tmop{with} (x^2 - x + 5)\\
    (2 x - 4) - (x^2 - x + 5) &  & \tmop{Distribute} \tmop{the}
    \tmop{negative}\\
    2 x - 4 - x^2 + x - 5 &  & \tmop{Combine} \tmop{like} \tmop{terms}\\
    - x^2 + 3 x - 9 &  & \tmop{Our} \tmop{Solution}
  \end{eqnarray*}


The parenthesis are very important when we are replacing $f (x) \tmop{and} g
(x)$ with a variable. In the previous example we needed the parenthesis to
know to distribute the negative.

\begin{example}
 \ 
\end{example}
  
  \begin{eqnarray*}
    \begin{array}{l}
      f (x) = x^2 - 4 x - 5\\
      g (x) = x - 5\\
      \tmop{Find} \left( \frac{f}{g} \right) (x)
    \end{array} &  & \tmop{Write} \tmop{division} \tmop{problem} \tmop{of}
    \tmop{functions}\\
    &  & \\
    \frac{f (x)}{g (x)} &  & \tmop{Replace} f (x) \tmop{with} (x^2 - 4 x - 5)
    \tmop{and} g (x) \tmop{with} (x - 5)\\
    &  & \\
    \frac{(x^2 - 4 x - 5)}{(x - 5)} &  & \tmop{To} \tmop{simplify} \tmop{the}
    \tmop{fraction} \tmop{we} \tmop{must} \tmop{first} \tmop{factor}\\
    &  & \\
    \frac{(x - 5) (x + 1)}{(x - 5)} &  & \tmop{Divide} \tmop{out}
    \tmop{common} \tmop{factor} \tmop{of} x - 5\\
    &  & \\
    x + 1 &  & \tmop{Our} \tmop{Solution}
  \end{eqnarray*}


Just as we could substitute an expression into evaluating functions, we can
substitute an expression into the operations on functions.

\begin{example}
 \ 
\end{example}
  
  \begin{eqnarray*}
    f (x) = 2 x - 1 &  & \\
    g (x) = x + 4 &  & \tmop{Write} \tmop{as} a \tmop{sum} \tmop{of}
    \tmop{functions}\\
    \tmop{Find} (f + g) (x^2) &  & \\
    &  & \\
    f (x^2) + g (x^2) &  & \tmop{Replace} x \tmop{in} f (x) \tmop{and} g (x)
    \tmop{with} x^2\\
    {}[2 (x^2) - 1] + [(x^2) + 4] &  & \tmop{Distribute} \tmop{the} +
    \tmop{does} \tmop{not} \tmop{change} \tmop{the} \tmop{problem}\\
    2 x^2 - 1 + x^2 + 4 &  & \tmop{Combine} \tmop{like} \tmop{terms}\\
    3 x^2 + 3 &  & \tmop{Our} \tmop{Solution}
  \end{eqnarray*}


\begin{example}
 \ 
\end{example}
  
  \begin{eqnarray*}
    f (x) = 2 x - 1 &  & \\
    g (x) = x + 4 &  & \tmop{Write} \tmop{as} a \tmop{product} \tmop{of}
    \tmop{functions}\\
    \tmop{Find} (f \cdot g) (3 x) &  & \\
    &  & \\
    f (3 x) g (3 x) &  & \tmop{Replace} x \tmop{in} f (x) \tmop{and} g (x)
    \tmop{with} 3 x\\
    {}[2 (3 x) - 1] [(3 x) + 4] &  & \tmop{Multiply} \tmop{our} 2 (3 x)\\
    (6 x - 1) (3 x + 4) &  & \tmop{FOIL}\\
    18 x^2 + 24 x - 3 x - 4 &  & \tmop{Combine} \tmop{like} \tmop{terms}\\
    18 x^2 + 21 x - 4 &  & \tmop{Our} \tmop{Solution}
  \end{eqnarray*}
}

