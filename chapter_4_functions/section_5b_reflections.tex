\subsection{Reflections}

{\tmstrong{Objective: Graph functions by reflecting vertically or horizontally}}\pp

We now turn our attention to reflections. 
\comment{If we %MM edits
 reflect a point $(x,y)$ across the $x$-axis, we replace $y$ with $-y$.  If $(x,y)$ is on the graph of $f$, then $y=f(x)$, so replacing $y$ with $-y$ is the same as replacing $f(x)$ with $-f(x)$.  Hence, the graph of $y=-f(x)$ is the graph of $f$ reflected across the $x$-axis.  Similarly, the graph of $y=f(-x)$ is the graph of $f$ reflected across the $y$-axis.   Returning to the language of inputs and outputs,
}
%MM
As we know from graphing quadratics,
 multiplying the output from a function by $-1$ reflects its graph across the $x$-axis.  Alternatively, it can be verified that multiplying the input to a function by $-1$ reflects the graph across the $y$-axis.

\smallskip

\bbm

%\smallskip

\begin{thm} \label{reflections}\index{graph ! reflection about an axis}\index{reflection ! of a function graph}\textbf{Reflections}  Suppose $f$ is a function. 

\begin{itemize}

\item To graph $y=-f(x)$, reflect the graph of $y=f(x)$ across the $x$-axis by multiplying the $y$-coordinates of the points on the graph of $f$ by $-1$.

\item To graph $y=f(-x)$, reflect the graph of $y=f(x)$ across the $y$-axis by multiplying the $x$-coordinates of the points on the graph of $f$ by $-1$.

\end{itemize}

\end{thm}

\ebm

\smallskip

Applying Theorem \ref{reflections} to the graph of $y=f(x)$ given at the beginning of the section, we can graph $y=-f(x)$ by reflecting the graph of $f$ about the $x$-axis

\[ \begin{array}{ccc}

\begin{mfpic}[12]{-1}{6}{-6}{6}
\polyline{(0,1), (2,3), (4,3), (5,5)}
\point[3pt]{(0,1), (2,3), (4,3), (5,5)}
\tlabel[cc](-1,1){\scriptsize $(0,1)$}
\tlabel[cc](2,3.5){\scriptsize $(2,3)$}
\tlabel[cc](4,2.5){\scriptsize $(4,3)$}
\tlabel[cc](5,5.5){\scriptsize $(5,5)$}
\tlabel[cc](6,-0.5){\scriptsize $x$}
\tlabel[cc](0.5,6){\scriptsize $y$}
\tcaption{\scriptsize $y=f(x)$}
\axes
\xmarks{1,2,3,4,5}
\ymarks{-1,-2,-3,-4,-5,1,2,3,4,5}
\tlpointsep{4pt}
\axislabels {x}{{\tiny $1$} 1, {\tiny $2$} 2, {\tiny $3$} 3, {\tiny $4$} 4, {\tiny $5$} 5}
\axislabels {y}{{\tiny $-1$} -1,{\tiny $-2$} -2, {\tiny $-3$} -3, {\tiny $-4$} -4, {\tiny $-5$} -5, {\tiny $2$} 2, {\tiny $3$} 3, {\tiny $4$} 4, {\tiny $5$} 5}
\end{mfpic}

&

\stackrel{\stackrel{\mbox{\scriptsize reflect across $x$-axis}}{\xrightarrow{\hspace{1in}}}}{\mbox{ \scriptsize multiply each $y$-coordinate by $-1$}} 


&

\begin{mfpic}[12]{-1}{6}{-6}{6}
%%\polyline{(0,-1), (2,-3), (4,-3), (5,-5)}
%\point[3pt]{(0,-1), (2,-3), (4,-3), (5,-5)}
%\tlabel[cc](-1.25,-1){\scriptsize $(0,-1)$}
%\tlabel[cc](2,-3.5){\scriptsize $(2,-3)$}
%\tlabel[cc](4,-2.5){\scriptsize $(4,-3)$}
%\tlabel[cc](5,-5.5){\scriptsize $(5,-5)$}
%\tlabel[cc](6,-0.5){\scriptsize $x$}
%\tlabel[cc](0.5,6){\scriptsize $y$}
\tcaption{\scriptsize $y=-f(x)$}
\axes
\xmarks{1,2,3,4,5}
\ymarks{-1,-2,-3,-4,-5,1,2,3,4,5}
\tlpointsep{4pt}
\axislabels {x}{{\tiny $1$} 1, {\tiny $2$} 2, {\tiny $3$} 3, {\tiny $4$} 4, {\tiny $5$} 5}
\axislabels {y}{{\tiny $-2$} -2, {\tiny $-3$} -3, {\tiny $-4$} -4, {\tiny $-5$} -5,{\tiny $1$} 1,{\tiny $2$} 2, {\tiny $3$} 3, {\tiny $4$} 4, {\tiny $5$} 5}
\end{mfpic}

\end{array}\]

By reflecting the graph of $f$ across the $y$-axis, we obtain the graph of $y=f(-x)$.

\[ \begin{array}{ccc}

\begin{mfpic}[10]{-6}{6}{-1}{6}
\polyline{(0,1), (2,3), (4,3), (5,5)}
\point[3pt]{(0,1), (2,3), (4,3), (5,5)}
\tlabel[cc](-1,1){\scriptsize $(0,1)$}
\tlabel[cc](2,3.5){\scriptsize $(2,3)$}
\tlabel[cc](4,2.5){\scriptsize $(4,3)$}
\tlabel[cc](5,5.5){\scriptsize $(5,5)$}
\tlabel[cc](6,-0.5){\scriptsize $x$}
\tlabel[cc](0.5,6){\scriptsize $y$}
\tcaption{\scriptsize $y=f(x)$}
\axes
\xmarks{-1,-2,-3,-4,-5,1,2,3,4,5}
\ymarks{1,2,3,4,5}
\tlpointsep{4pt}
\axislabels {x}{{\tiny $-1 \hspace{7pt}$} -1, {\tiny $-2\hspace{7pt}$} -2, {\tiny $-3\hspace{7pt}$} -3, {\tiny $-4\hspace{7pt}$} -4, {\tiny $-5\hspace{7pt}$} -5, {\tiny $1$} 1, {\tiny $2$} 2, {\tiny $3$} 3, {\tiny $4$} 4, {\tiny $5$} 5}
\axislabels {y}{{\tiny $2$} 2, {\tiny $3$} 3, {\tiny $4$} 4, {\tiny $5$} 5}
\end{mfpic}

&

\stackrel{\stackrel{\mbox{\scriptsize reflect across $y$-axis}}{\xrightarrow{\hspace{1in}}}}{\stackrel{\mbox{ \scriptsize multiply each}}{\mbox{\scriptsize $x$-coordinate by $-1$}}} 

&

\begin{mfpic}[10]{-6}{6}{-1}{6}
%\polyline{(0,1), (-2,3), (-4,3), (-5,5)}
%\point[3pt]{(0,1), (-2,3), (-4,3), (-5,5)}
%\tlabel[cc](1,1){\scriptsize $(0,1)$}
%\tlabel[cc](-2,3.5){\scriptsize $(-2,3)$}
%\tlabel[cc](-4,2.5){\scriptsize $(-4,3)$}
%\tlabel[cc](-5,5.5){\scriptsize $(-5,5)$}
%\tlabel[cc](6,-0.5){\scriptsize $x$}
%\tlabel[cc](0.5,6){\scriptsize $y$}
\tcaption{\scriptsize $y=f(-x)$}
\axes
\xmarks{-1,-2,-3,-4,-5,1,2,3,4,5}
\ymarks{1,2,3,4,5}
\tlpointsep{4pt}
\axislabels {x}{{\tiny $-1 \hspace{7pt}$} -1, {\tiny $-2\hspace{7pt}$} -2, {\tiny $-3\hspace{7pt}$} -3, {\tiny $-4\hspace{7pt}$} -4, {\tiny $-5\hspace{7pt}$} -5, {\tiny $1$} 1, {\tiny $2$} 2, {\tiny $3$} 3, {\tiny $4$} 4, {\tiny $5$} 5}
\axislabels {y}{{\tiny $2$} 2, {\tiny $3$} 3, {\tiny $4$} 4, {\tiny $5$} 5}
\end{mfpic}

\end{array}\]


%KEEP THESE EXAMPLES BUT SMUSH THEM TOGETHER
With the addition of reflections, it is now more important than ever to consider the order of transformations, as the next example illustrates.

\begin{example} Let $f(x) = \sqrt{x}$.  Use the graph of $f$ to graph the following functions.  Also, state their domains and ranges.\\

%\enlargethispage{.25in}
%\vspace{-.1in}
\begin{multicols}{3}
\begin{enumerate}

\item  $g(x) = \sqrt{-x}$
\item  $j(x) = \sqrt{3-x}$
\item  $m(x) = 3 - \sqrt{x}$

\end{enumerate}
\end{multicols}

%\vspace{-.2in}
~\\
{\bf Solution.}

\begin{enumerate}

\item  Looking at a $\sqrt{-x}$ may seem unacceptable, especially after we disallowed negatives from the radicands of even roots having to make those solutions be non-real.
%MM Edit

  However, we must remember that $x$ is a variable, and as such, the quantity $-x$ isn't always negative. For example, if $x=-4$, $-x = 4$, thus $\sqrt{-x} = \sqrt{-(-4)} = 2$ is perfectly well-defined.  To find the domain analytically, we set $-x \geq 0$ which gives  $x \leq 0$, so that the domain of $g$ is $(-\infty, 0]$.  Since $g(x) = f(-x)$, Theorem \ref{reflections} tells us that the graph of $g$ is the reflection of the graph of $f$ across the $y$-axis.  We accomplish this by multiplying each $x$-coordinate on the graph of $f$ by $-1$, so that the points $(0,0)$, $(1,1)$, and $(4,2)$ move to $(0,0)$, $(-1,1)$, and $(-4,2)$, respectively.  Graphically, we see that the domain of $g$ is $(-\infty, 0]$ and the range of $g$ is the same as the range of $f$, namely $[0,\infty)$.

\[ \begin{array}{ccc}

\begin{mfpic}[11]{-5}{5}{-1}{3}
\arrow \function{0,5,0.1}{sqrt(x)}
\point[3pt]{(0,0), (1,1), (4,2)}
\tlabel[cc](-1,0.25){\tiny $(0,0)$}
\tlabel[cc](1,1.5){\tiny $(1,1)$}
\tlabel[cc](4,2.5){\tiny  $(4,2)$}
\tlabel[cc](5,-0.5){\scriptsize $x$}
\tlabel[cc](0.5,3){\scriptsize $y$}
\tcaption{\scriptsize $y=f(x)=\sqrt{x}$}
\axes
\xmarks{-1,-2,-3,-4,1,2,3,4}
\ymarks{1,2}
\tlpointsep{4pt}
\axislabels {x}{{\tiny $-1\hspace{7pt}$} -1, {\tiny $-2\hspace{7pt}$} -2, {\tiny $-3\hspace{7pt}$} -3, {\tiny $-4\hspace{7pt}$} -4,{\tiny $1$} 1, {\tiny $2$} 2, {\tiny $3$} 3, {\tiny $4$} 4}
\axislabels {y}{{\tiny $1$} 1, {\tiny $2$} 2}
\end{mfpic}

&

\stackrel{\stackrel{\mbox{\scriptsize reflect across $y$-axis}}{\xrightarrow{\hspace{1in}}}}{\stackrel{\mbox{ \scriptsize multiply each}}{\mbox{\scriptsize $x$-coordinate by $-1$}}} 

&

\begin{mfpic}[11]{-5}{5}{-1}{3}
%\arrow \function{0,-5,0.1}{sqrt(0-x)}
%\point[3pt]{(0,0), (-1,1), (-4,2)}
%\tlabel[cc](1,0.5){\tiny  $(0,0)$}
%\tlabel[cc](-2,0.5){\tiny  $(-1,1)$}
%\tlabel[cc](-5,1.5){\tiny  $(-4,2)$}
%\tlabel[cc](5,-0.5){\scriptsize $x$}
%\tlabel[cc](0.5,3){\scriptsize $y$}
\tcaption{\scriptsize $y=g(x) = f(-x)=\sqrt{-x}$}
\axes
\xmarks{-1,-2,-3,-4,1,2,3,4}
\ymarks{1,2}
\tlpointsep{4pt}
\axislabels {x}{{\tiny $-1\hspace{7pt}$} -1, {\tiny $-2\hspace{7pt}$} -2, {\tiny $-3\hspace{7pt}$} -3, {\tiny $-4\hspace{7pt}$} -4,{\tiny $1$} 1, {\tiny $2$} 2, {\tiny $3$} 3, {\tiny $4$} 4}
\axislabels {y}{{\tiny $1$} 1, {\tiny $2$} 2}
\end{mfpic}

\end{array}\]

\item  To determine the domain of  $j(x) = \sqrt{3-x}$, we solve $3-x \geq 0$ and get $x \leq 3$, or $(-\infty, 3]$.  To determine which transformations we need to apply to the graph of $f$ to obtain the graph of $j$, we rewrite $j(x) = \sqrt{-x+3} = f(-x+3)$. Comparing this formula with $f(x) = \sqrt{x}$, we see that not only are we multiplying the input $x$ by $-1$, which results in a reflection across the $y$-axis, but also we are adding $3$, which indicates a horizontal shift to the left.  Does it matter in which order we do the transformations?  If so, which order is the correct order?  Let's consider the point $(4,2)$ on the graph of $f$.  We refer to the discussion leading up to Theorem \ref{hshifts}.  We know $f(4) = 2$ and wish to find the point on $y=j(x) = f(-x+3)$ which corresponds to $(4,2)$.  We set $-x+3 = 4$ and solve.  Our first step is to subtract $3$ from both sides to get $-x=1$.  Subtracting $3$ from the $x$-coordinate $4$ is shifting the point $(4,2)$ to the left.   From $-x=1$, we then multiply (or divide) both sides by $-1$ to get $x=-1$.  Multiplying the $x$-coordinate by $-1$ corresponds to reflecting the point about the $y$-axis.  Hence, we perform the horizontal shift first, then follow it with the reflection about the $y$-axis.  Starting with $f(x) = \sqrt{x}$, we let $j_{\mbox{\tiny$1$}}(x)$ be the intermediate function which shifts the graph of $f$ $3$ units to the left, $j_{\mbox{\tiny$1$}}(x) = f(x+3)$. 

\[ \begin{array}{ccc}

\begin{mfpic}[11]{-5}{5}{-1}{3}
\arrow \function{0,5,0.1}{sqrt(x)}
\point[3pt]{(0,0), (1,1), (4,2)}
\tlabel[cc](-1,0.25){\tiny $(0,0)$}
\tlabel[cc](1,1.5){\tiny $(1,1)$}
\tlabel[cc](4,2.5){\tiny $(4,2)$}
\tlabel[cc](5,-0.5){\scriptsize $x$}
\tlabel[cc](0.5,3){\scriptsize $y$}
\tcaption{\scriptsize $y=f(x)=\sqrt{x}$}
\axes
\xmarks{1,2,3,4,-1,-2,-3,-4}
\ymarks{1,2}
\tlpointsep{4pt}
\axislabels {x}{{\tiny $-4 \hspace{7pt}$} -4,{\tiny $-3 \hspace{7pt}$} -3, {\tiny $-2 \hspace{7pt}$} -2, {\tiny $-1 \hspace{7pt}$} -1, {\tiny $1$} 1, {\tiny $2$} 2, {\tiny $3$} 3, {\tiny $4$} 4}
\axislabels {y}{{\tiny $1$} 1, {\tiny $2$} 2}
\end{mfpic}

&

\stackrel{\stackrel{\mbox{\scriptsize shift left $3$ units}}{\xrightarrow{\hspace{1in}}}}{\stackrel{\mbox{ \scriptsize subtract $3$ from each}}{\mbox{\scriptsize $x$-coordinate}}} 

&

\begin{mfpic}[11]{-5}{5}{-1}{3}
%\arrow \function{-3,2,0.1}{sqrt(x+3)}
%\point[3pt]{(-3,0), (-2,1), (1,2)}
%\tlabel[cc](-4,0.25){\tiny $(-3,0)$}
%\tlabel[cc](-2,1.5){\tiny $(-2,1)$}
%\tlabel[cc](1,2.5){\tiny $(1,2)$}
%\tlabel[cc](5,-0.5){\scriptsize $x$}
%\tlabel[cc](0.5,3){\scriptsize $y$}
\tcaption{\scriptsize $y=j_{\mbox{\tiny$1$}}(x)=f(x+3)=\sqrt{x+3}$}
\axes
\xmarks{1,2,3,4,-1,-2,-3,-4}
\ymarks{1,2}
\tlpointsep{4pt}
\axislabels {x}{{\tiny $-4 \hspace{7pt}$} -4,{\tiny $-3 \hspace{7pt}$} -3, {\tiny $-2 \hspace{7pt}$} -2, {\tiny $-1 \hspace{7pt}$} -1, {\tiny $1$} 1, {\tiny $2$} 2, {\tiny $3$} 3, {\tiny $4$} 4}
\axislabels {y}{{\tiny $1$} 1, {\tiny $2$} 2}
\end{mfpic} \\

\end{array}\]

To obtain the function $j$, we reflect the graph of $j_{\mbox{\tiny$1$}}$ about $y$-axis.   Theorem \ref{reflections} tells us we have $j(x) = j_{\mbox{\tiny$1$}}(-x)$. Putting it all together, we have $j(x) = j_{\mbox{\tiny$1$}}(-x) = f(-x+3) = \sqrt{-x+3}$, which is what we want.\footnote{If we had done the reflection first, then $j_{\mbox{\tiny$1$}}(x) = f(-x)$.  Following this by a shift left would give us $j(x) = j_{\mbox{\tiny$1$}}(x+3) = f(-(x+3)) = f(-x-3) = \sqrt{-x-3}$ which isn't what we want.  However, if we did the reflection first and followed it by a shift to the right $3$ units, we would have arrived at the function $j(x)$.  We leave it to the reader to verify the details.} From the graph, we confirm the domain of $j$ is $(-\infty, 3]$ and we get that the range is $[0, \infty)$.

\[ \begin{array}{ccc}

\begin{mfpic}[11]{-5}{5}{-1}{3}
\arrow \function{-3,2,0.1}{sqrt(x+3)}
\point[3pt]{(-3,0), (-2,1), (1,2)}
\tlabel[cc](-4,0.25){\tiny $(-3,0)$}
\tlabel[cc](-2,1.5){\tiny $(-2,1)$}
\tlabel[cc](1,2.5){\tiny $(1,2)$}
\tlabel[cc](5,-0.5){\scriptsize $x$}
\tlabel[cc](0.5,3){\scriptsize $y$}
\tcaption{\scriptsize $y=j_{\mbox{\tiny$1$}}(x)=\sqrt{x+3}$}
\axes
\xmarks{1,2,3,4,-1,-2,-3,-4}
\ymarks{1,2}
\tlpointsep{4pt}
\axislabels {x}{{\tiny $-4 \hspace{7pt}$} -4,{\tiny $-3 \hspace{7pt}$} -3, {\tiny $-2 \hspace{7pt}$} -2, {\tiny $-1 \hspace{7pt}$} -1, {\tiny $1$} 1, {\tiny $2$} 2, {\tiny $3$} 3, {\tiny $4$} 4}
\axislabels {y}{{\tiny $1$} 1, {\tiny $2$} 2}
\end{mfpic} 

&

\stackrel{\stackrel{\mbox{\scriptsize reflect across $y$-axis}}{\xrightarrow{\hspace{1in}}}}{\stackrel{\mbox{ \scriptsize multiply each}}{\mbox{\scriptsize $x$-coordinate by $-1$}}} 

&

\begin{mfpic}[11]{-5}{5}{-1}{3}
%\arrow \function{3,-2,0.1}{sqrt(3-x)}
%\point[3pt]{(3,0), (2,1), (-1,2)}
%\tlabel[cc](4,0.5){\tiny $(3,0)$}
%\tlabel[cc](2,1.5){\tiny $(2,1)$}
%\tlabel[cc](-1.5,1){\tiny $(-1,2)$}
%\tlabel[cc](5,-0.5){\scriptsize $x$}
%\tlabel[cc](0.5,3){\scriptsize $y$}
\tcaption{\scriptsize $y=j(x)=j_{\mbox{\tiny$1$}}(-x) =\sqrt{-x+3}$}
\axes
\xmarks{1,2,3,4,-1,-2,-3,-4}
\ymarks{1,2}
\tlpointsep{4pt}
\axislabels {x}{{\tiny $-4 \hspace{7pt}$} -4,{\tiny $-3 \hspace{7pt}$} -3, {\tiny $-2 \hspace{7pt}$} -2, {\tiny $-1 \hspace{7pt}$} -1, {\tiny $1$} 1, {\tiny $2$} 2, {\tiny $3$} 3, {\tiny $4$} 4}
\axislabels {y}{ {\tiny $2$} 2}
\end{mfpic} \\

\end{array}\]

\item  The domain of $m$ works out to be the domain of $f$, $[0, \infty)$.  Rewriting $m(x) = -\sqrt{x} + 3$, we see $m(x) = -f(x) + 3$.  Since we are multiplying the output of $f$ by $-1$ and then adding $3$, we once again have two transformations to deal with:  a reflection across the $x$-axis and a vertical shift.  To determine the correct order in which to apply the transformations, we imagine trying to determine the point on the graph of $m$ which corresponds to $(4,2)$ on the graph of $f$.  Since in the formula for $m(x)$, the input to $f$ is just $x$, we substitute to find  $m(4) = -f(4)+3 = -2+3=1$.  Hence, $(4,1)$ is the corresponding point on the graph of $m$. If we closely examine the arithmetic, we see that we first multiply $f(4)$ by $-1$, which corresponds to the reflection across the $x$-axis, and then we add $3$, which corresponds to the vertical shift.  If we define an intermediate function $m_{\mbox{\tiny$1$}}(x) = -f(x)$ to take care of the reflection, we get

\[ \begin{array}{ccc}

\begin{mfpic}[15]{-1}{5}{-3}{4}
\arrow \function{0,5,0.1}{sqrt(x)}
\point[3pt]{(0,0), (1,1), (4,2)}
\tlabel[cc](-1,0.25){\scriptsize $(0,0)$}
\tlabel[cc](1,1.5){\scriptsize $(1,1)$}
\tlabel[cc](4,2.5){\scriptsize $(4,2)$}
\tlabel[cc](5,-0.5){\scriptsize $x$}
\tlabel[cc](0.5,4){\scriptsize $y$}
\tcaption{\scriptsize $y=f(x)=\sqrt{x}$}
\axes
\xmarks{1,2,3,4}
\ymarks{-2,-1,1,2,3}
\tlpointsep{4pt}
\axislabels {x}{{\tiny $1$} 1, {\tiny $2$} 2, {\tiny $3$} 3, {\tiny $4$} 4}
\axislabels {y}{{\tiny $-2$} -2, {\tiny $-1$} -1,{\tiny $1$} 1, {\tiny $2$} 2, {\tiny $3$} 3}
\end{mfpic}

&

\stackrel{\stackrel{\mbox{\scriptsize reflect across $x$-axis}}{\xrightarrow{\hspace{1in}}}}{\stackrel{\mbox{ \scriptsize multiply each}}{\mbox{\scriptsize $y$-coordinate by $-1$}}} 

&

\begin{mfpic}[15]{-1}{5}{-3}{4}
%\arrow \function{0,5,0.1}{0-sqrt(x)}
%\point[3pt]{(0,0), (1,-1), (4,-2)}
%\tlabel[cc](-1,-0.5){\scriptsize $(0,0)$}
%\tlabel[cc](1,-1.75){\scriptsize $(1,-1)$}
%\tlabel[cc](4,-2.75){\scriptsize $(4,-2)$}
%\tlabel[cc](5,-0.5){\scriptsize $x$}
%\tlabel[cc](0.5,4){\scriptsize $y$}
\tcaption{\scriptsize $y=m_{\mbox{\tiny$1$}}(x)=-f(x) = -\sqrt{x}$}
\axes
\xmarks{1,2,3,4}
\ymarks{-1,-2,1,2,3}
\tlpointsep{4pt}
\axislabels {x}{{\tiny $1$} 1, {\tiny $2$} 2, {\tiny $3$} 3, {\tiny $4$} 4}
\axislabels {y}{{\tiny $-1$} -1, {\tiny $-2$} -2, {\tiny $1$} 1, {\tiny $2$} 2, {\tiny $3$} 3}
\end{mfpic} \\

\end{array} \]

To shift the graph of $m_{\mbox{\tiny$1$}}$ up $3$ units, we set $m(x) = m_{\mbox{\tiny$1$}}(x)+3$.  Since $m_{\mbox{\tiny$1$}}(x) = -f(x)$, when we put it all together, we get $m(x) = m_{\mbox{\tiny$1$}}(x)+3 = -f(x) + 3 = -\sqrt{x}+3$.   We see from the graph that the range of $m$ is $(-\infty, 3]$.

\[ \begin{array}{ccc}
 
\begin{mfpic}[15]{-1}{5}{-3}{4}
\arrow \function{0,5,0.1}{0-sqrt(x)}
%\point[3pt]{(0,0), (1,-1), (4,-2)}
%\tlabel[cc](-1,-0.5){\scriptsize $(0,0)$}
%\tlabel[cc](1,-1.75){\scriptsize $(1,-1)$}
%\tlabel[cc](4,-2.75){\scriptsize $(4,-2)$}
%\tlabel[cc](5,-0.5){\scriptsize $x$}
%\tlabel[cc](0.5,4){\scriptsize $y$}
\axes
\xmarks{1,2,3,4}
\ymarks{-1,-2,1,2,3}
\tlpointsep{4pt}
\axislabels {x}{{\tiny $1$} 1, {\tiny $2$} 2, {\tiny $3$} 3, {\tiny $4$} 4}
\axislabels {y}{{\tiny $-1$} -1, {\tiny $-2$} -2, {\tiny $1$} 1, {\tiny $2$} 2, {\tiny $3$} 3}
\end{mfpic}

&

\stackrel{\stackrel{\mbox{\scriptsize shift up $3$ units}}{\xrightarrow{\hspace{1in}}}}{\mbox{ \scriptsize add $3$ to each $y$-coordinate}} 

&

\begin{mfpic}[15]{-1}{5}{-3}{4}
%\arrow \function{0,5,0.1}{3-sqrt(x)}
%\point[3pt]{(0,3), (1,2), (4,1)}
%\tlabel[cc](-1,3){\scriptsize $(0,3)$}
%\tlabel[cc](1,2.5){\scriptsize $(1,2)$}
%%\tlabel[cc](4,1.5){\scriptsize $(4,1)$}
%\tlabel[cc](5,-0.5){\scriptsize $x$}
%\tlabel[cc](0.5,4){\scriptsize $y$}
\axes
\xmarks{1,2,3,4}
\ymarks{1,2,3,-1,-2}
\tlpointsep{4pt}
\axislabels {x}{{\tiny $1$} 1, {\tiny $2$} 2, {\tiny $3$} 3, {\tiny $4$} 4}
\axislabels {y}{{\tiny $-1$} -1, {\tiny $-2$} -2, {\tiny $1$} 1, {\tiny $2$} 2}
\end{mfpic} \\

\mbox{\scriptsize $y=m_{\mbox{\tiny$1$}}(x) = -\sqrt{x}$} & & \mbox{\scriptsize $y=m(x) = m_{\mbox{\tiny$1$}}(x) + 3 = -\sqrt{x} + 3$}

\end{array} \]

\end{enumerate}
\end{example}