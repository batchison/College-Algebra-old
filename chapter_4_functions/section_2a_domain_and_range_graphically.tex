\subsection{Identifying Domain and Range Graphically}

{\tmstrong{Objective: Identify the domain and range of a function that is described either graphically or algebraically.  Establish several fundamental functions and identify their domain and range.}}\pp

In this section, we will first discuss how one can identify the domain and range of a function using its graph.  Later, we will explore finding the domain of a function using algebraic methods.  As finding the range of a function using algebraic methods can often prove quite challenging, we will postpone this topic for a later section.  We conclude the section with an introduction to a few of the most fundamental and foundational functions in algebra.%\pp
%In every example that follows, we will use interval notation to describe our answers.
%TO DO: Ben & Mark-Do this for later editions
\begin{example}~~~Find the domain and range of the function $f$ whose graph is given below.

\begin{center}
\begin{mfpic}[20]{-3}{3}{-2}{5}
\arrow \reverse \function{-2.5,1,0.1}{4-x**2}
\gclear \circle{(1,3), 0.1}
\circle{(1,3), 0.1}
\axes
\tlabel[cc](3,-0.5){\scriptsize $x$}
\tlabel[cc](0.5,5){\scriptsize $y$}
\xmarks{-2 step 1 until 2 }
\ymarks{-1 step 1 until 4}
\tcaption{The graph of $f$}
\tlpointsep{5pt}
\scriptsize
\axislabels {x}{{$-1 \hspace{7pt}$} -1, {$1$} 1}
\axislabels {y}{{$-1$} -1, {$1$} 1, {$2$} 2, {$3$} 3, {$4$} 4}
\normalsize
\end{mfpic}
\end{center}
\end{example}

\noindent To determine the domain and range of $f$, we need to determine which $x$ and $y$-values respectively occur as coordinates of points on the given graph.\pp
%\newpage
To find the domain, it will be helpful to imagine collapsing the curve onto the $x$-axis and determining the portion of the $x$-axis that gets covered.  This is often described as \textbf{projecting} the curve onto the $x$-axis.%\pp
\newpage
Before we project, we need to pay attention to two subtle notations on the graph:  the arrowhead on the lower left corner of the graph indicates that the graph continues to curve downwards to the left forever; and the open circle at $(1,3)$ indicates that the point $(1,3)$ is \textit{not} on the graph, but all the points on the curve leading up to $(1,3)$ are on the graph.

\begin{center}

\begin{tabular}{cc}

\begin{mfpic}[16]{-4}{3}{-4}{5}
\arrow \reverse \function{-2.5,1,0.1}{4-x**2}
\gclear \circle{(1,3), 0.1}
\circle{(1,3), 0.1}
\arrow \polyline{(2,4), (2,1)}
\gclear \tlabelrect[cc](2,2){project down}
\arrow \polyline{(-3,-4), (-3,-1)}
\gclear \tlabelrect[cc](-3,-3){project up}
\axes
\tlabel[cc](3,-0.5){\scriptsize $x$}
\tlabel[cc](0.5,5){\scriptsize $y$}
\xmarks{-2 step 1 until 2 }
\ymarks{-1 step 1 until 4}
%\tcaption{The graph of $f$}
\tlpointsep{5pt}
\scriptsize
\axislabels {x}{{$-1 \hspace{7pt}$} -1, {$1$} 1}
\axislabels {y}{{$-1$} -1, {$1$} 1, {$2$} 2, {$3$} 3, {$4$} 4}
\normalsize
\end{mfpic} \hspace{.27in} &

\begin{mfpic}[16]{-4}{3}{-4}{5}
\arrow \reverse \function{-2.5,1,0.1}{4-x**2}
\gclear \circle{(1,3), 0.1}
\circle{(1,3), 0.1}
\axes
\tlabel[cc](3,-0.5){\scriptsize $x$}
\tlabel[cc](0.5,5){\scriptsize $y$}
\xmarks{-2 step 1 until 2 }
\ymarks{-1 step 1 until 4}
%\tcaption{The graph of $f$}
\tlpointsep{5pt}
\scriptsize
\axislabels {x}{{$-1 \hspace{11pt}$} -1, {$1$} 1}
\axislabels {y}{{$-1$} -1, {$1$} 1, {$2$} 2, {$3$} 3, {$4$} 4}
\normalsize
\penwd{2pt} 
\arrow \polyline{(1,0), (-4,0)}
\penwd{0.5pt} 
\gclear \circle{(1,0), 0.1}
\circle{(1,0), 0.1}
\end{mfpic} \\

\end{tabular}
\end{center}


We see from the figure that if we project the graph of $f$ to the $x$-axis, we get all real numbers less than $1$.  Using interval notation, we write the domain of $f$ as $(-\infty, 1)$.\\
%\newpage
To determine the range of $f$, we use a similar method, projecting the curve onto the $y$-axis as follows.

\begin{center}
\begin{tabular}{cc}

\begin{mfpic}[16]{-4}{3}{-4}{5}
\arrow \reverse \function{-2.5,1,0.1}{4-x**2}
\gclear \circle{(1,3), 0.1}
\circle{(1,3), 0.1}
\arrow \polyline{(3,2), (1,2)}
\tlabel[cc](2,1){project left}
\arrow \polyline{(-4,3), (-2,3)}
\tlabel[cc](-3.5,2){project right}
\axes
\tlabel[cc](3,-0.5){\scriptsize $x$}
\tlabel[cc](0.5,5){\scriptsize $y$}
\xmarks{-2 step 1 until 2 }
\ymarks{-1 step 1 until 4}
%\tcaption{The graph of $f$}
\tlpointsep{5pt}
\scriptsize
\axislabels {x}{{$-1 \hspace{7pt}$} -1, {$1$} 1}
\axislabels {y}{{$-1$} -1, {$1$} 1, {$2$} 2, {$3$} 3, {$4$} 4}
\normalsize
\end{mfpic} \hspace{.27in} &

\begin{mfpic}[16]{-4}{3}{-4}{5}
\arrow \reverse \function{-2.5,1,0.1}{4-x**2}
\gclear \circle{(1,3), 0.1}
\circle{(1,3), 0.1}
\axes
\tlabel[cc](3,-0.5){\scriptsize $x$}
\tlabel[cc](0.5,5){\scriptsize $y$}
\xmarks{-2 step 1 until 2 }
\ymarks{-1 step 1 until 4}
%\tcaption{The graph of $f$}
\tlpointsep{5pt}
\scriptsize
\axislabels {x}{{$-1 \hspace{7pt}$} -1, {$1$} 1}
\axislabels {y}{{$-1$} -1, {$1$} 1, {$2$} 2, {$3$} 3, {$4$} 4}
\normalsize
\penwd{2pt} 
\arrow \polyline{(0,4), (0,-3)}
\penwd{0.5pt} 
\gfill \circle{(0,4), 0.1}
\end{mfpic} \\

\end{tabular}

\end{center}

Note that even though there is an open circle at $(1,3)$, we still include the $y$ value of $3$ in our range, since the point $(-1,3)$ is on the graph of $f$.  We also include $y=4$ in our answer, since the point $(0,4)$ is also on our graph.  Consequently, the range of $f$ is all real numbers less than or equal to $4$, or $(-\infty, 4]$.