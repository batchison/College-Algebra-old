{\tmstrong{Objective: Solve systems of equations using the
addition/elimination method.}}\pp

When solving systems we have found that graphing is very limited when solving
equations. We then considered a second method known as substitution. This is
probably the most used idea in solving systems in various areas of algebra.
However, substitution can get ugly if we don't have a lone variable. This
leads us to our second method for solving systems of equations. This method is
known as either Elimination or Addition. We will set up the process in the
following examples, then define the five step process we can use to solve by
elimination.

\begin{example}~~~Solve the following system of equations.
      $$3 x - 4 y = 8\qquad\qquad 5 x + 4 y = - 24$$
\begin{eqnarray*}
    \begin{array}{l}
      ~~~~~~3 x - 4 y = 8\\
      \underline{\tmmathbf{+}~~~5 x + 4 y = - 24}
    \end{array} &  & \begin{array}{l}
      \text{Notice~opposite~signs~in~front~of~} y\\
			\text{Add~columns~to~eliminate~} y
    \end{array}\\ %\tmop{Notice} \tmop{opposite~signs} \tmop{in} \tmop{front}
    %\tmop{of} y. \tmop{Add} \tmop{columns} .\\
    8 x ~~~~~~~= - 16~ &  & \tmop{~Solve} \tmop{for~} x\\
    \tmmathbf{\overline{8} ~~~~~~~~~~~~~~ \overline{8}}~~  &  & \tmop{~Divide} \tmop{by~} 8\\
    x = - 2~~~ &  & \tmop{~We} \tmop{have} \tmop{our~} x!\\
    5 (- 2) + 4 y = - 24~ &  & \tmop{~Plug} \tmop{into} \tmop{either}
    \tmop{original} \tmop{equation}\\
    - 10 + 4 y = - 24~ &  &\tmop{~Simplify}\\
    \tmmathbf{\underline{+ 10 ~~~~~~~~~+ 10}} &  &  \tmop{~Add~} 10 \tmop{~to} \tmop{both} \tmop{sides}\\
    4 y = - 14~ &  & \\
    \tmmathbf{\overline{4} ~~~~~~~ \overline{4}}~~  &  & \tmop{~Divide} \tmop{by~} 4\\
    y = -\frac{7}{2}~ &  & \tmop{~Now} \tmop{we} \tmop{have} \tmop{our~} y!\\
    (x,y)=\left( - 2, -\frac{7}{2} \right) &  & \tmop{~Our} \tmop{solution}
  \end{eqnarray*}
\end{example}
  


In the previous example one variable had opposites in front of it, $- 4 y$ and
$4 y$. Adding these together eliminated the y completely. This allowed us to
solve for the $x$. This is the idea behind the addition method. However,
generally we won't have opposites in front of one of the variables. In this
case we will manipulate the equations to get the opposites we want by
multiplying one or both equations (on both sides!). This is shown in the next
example.

\begin{example}~~~Solve the following system of equations.
      $$- 6 x + 5 y = 22\qquad\qquad 2 x + 3 y = 2$$
			Notice that we can obtain ``opposite'' coefficients (one positive and one negative) in front of $x$ by multiplying both sides of the second equation by 3.
\begin{eqnarray*}
    3 (2 x + 3 y) = (2) 3 &  & \tmop{Distribute} \tmop{to} \tmop{get}
    \tmop{new} \tmop{second} \tmop{equation}\\
    6 x + 9 y = ~6~~~&  & \tmop{New} \tmop{second} \tmop{equation}\\
    \underline{- 6 x + 5 y = 22}~~ &  & \text{Add~equations~to eliminate~} x\\
    14 y = 28~~ &  &\\
    \tmmathbf{\overline{14} ~~~~ \overline{14}}~~ &  & \tmop{Divide} \tmop{both} \tmop{sides} \tmop{by~} 14 \\
    y = 2~~~~ &  & \tmop{We} \tmop{have} \tmop{our~} y!\\
    2 x + 3 (2) = 2~~~~ &  & \tmop{Plug} \tmop{into} \tmop{one} \tmop{of}
    \tmop{the} \tmop{original} \tmop{equations}\\
    2 x + 6 = 2~~~~ &  & \tmop{Simplify}\\
    \tmmathbf{\underline{- 6 ~- 6}}~~ &  & \tmop{Subtract~} 6 \tmop{~from} \tmop{both} \tmop{sides}\\
    2 x = - 4~~ &  & \\
    \tmmathbf{\overline{2} ~~~~~~~ \overline{2}}~~ &  & \tmop{Divide} \tmop{both} \tmop{sides} \tmop{by~} 2\\
    x = - 2~~ &  & \tmop{We} \tmop{also} \tmop{have} \tmop{our~} x!\\
    (x,y)=(- 2, 2) &  & \tmop{Our} \tmop{solution}
  \end{eqnarray*}
\end{example}
  


When we looked at the $x$ terms, $- 6 x$ and $2 x$ we decided to multiply the
$2 x$ by 3 to get the opposites we were looking for. What we are looking for
with our opposites is the least common multiple (LCM) of the coefficients. We
also could have solved the above problem by looking at the terms with $y$, $5
y \tmop{and} 3 y$. The LCM of 3 and 5 is 15. So we would want to multiply both
equations, the $5 y$ by 3, and the $3 y$ by$- 5$ to get opposites, $15 y$ and
$- 15 y$. This illustrates an important point: for some problems we will have to
multiply both equations by a constant (on both sides) to get the opposites we
are looking for.

\vspace{1in}
~\\

\begin{example}~~~Solve the following system of equations.
      $$3 x + 6 y = - 9\qquad\qquad 2 x + 9 y = - 26$$
Here, we can obtain opposite coefficients in front of $y$ by finding the least common multiple (LCM) of 6 and 9, which is 18.  We will therefore multiply both sides of both equations by the appropriate values to get $18y$ and $-18y$.
\begin{eqnarray*}
    3 (3 x + 6 y) = (- 9) 3 &  & \tmop{Multiply} \tmop{the} \tmop{first}
    \tmop{equation} \tmop{by} 3\\
    9 x + 18 y = - 27~~ &  & \\
    &  & \\
    - 2 (2 x + 9 y) = (- 26) (- 2) &  & \tmop{Multiply} \tmop{the}
    \tmop{second} \tmop{equation} \tmop{by} - 2\\
    - 4 x - 18 y = 52~~~~~~~~~~~ &  & \\
    &  & \\
    9 x + 18 y = - 27 &  & \tmop{Add} \tmop{two} \tmop{new} \tmop{equations}
    \tmop{together}\\
    \underline{- 4 x - 18 y = 52~~} &  & \text{~~~to eliminate~} y\\
    5 x = 25~~ &  & \\
    \tmmathbf{\overline{5} ~~~~~ \overline{5}}~~~ &  & \tmop{Divide} \tmop{both} \tmop{sides} \tmop{by~} 5\\
    x = 5~~~ &  & \tmop{We} \tmop{have} \tmop{our} \tmop{solution} \tmop{for~} x\\
    3 (5) + 6 y = - 9~ &  & \tmop{Plug} \tmop{into} \tmop{either}
    \tmop{original} \tmop{equation}\\
    15 + 6 y = - 9~ &  & \tmop{Simplify}\\
    \tmmathbf{\underline{- 15 ~~~~~~~- 15}} &  &\tmop{Subtract~} 15 \tmop{~from} \tmop{both}
    \tmop{sides}\\
    6 y = - 24 &  & \\
    \tmmathbf{\overline{6} ~~~~~~ \overline{6}}~~ &  & \tmop{Divide} \tmop{both} \tmop{sides} \tmop{by~} 6\\
    y = - 4~~ &  & \tmop{Now} \tmop{we} \tmop{have} \tmop{our} \tmop{solution}
    \tmop{for~} y\\
    (x,y)=(5, - 4) &  & \tmop{Our} \tmop{solution}
  \end{eqnarray*}
\end{example}


As we get started, it is important for each problem that all variables and
constants are aligned before we begin multiplying and adding equations. This
is illustrated in the next example which includes the five steps we will go
through to solve a problem using elimination.

\vspace{.5in}
~\\

\begin{tabular}{|l|l|}
  \hline
  ~~~~~~~~~~~~~~~~~~~~~~Problem & \begin{tabular}{l}
    $2 x - 5 y = - 13$\\
    $- 3 y + 4 = - 5 x$
  \end{tabular}\\
  \hline
  1. Line up the variables and constants. & \begin{tabular}{l}
    Rearrange the second equation\\
    $2 x - 5 y = - 13$\\
    $5 x - 3 y = - 4$\\
	\end{tabular}\\
  \hline
  2. Multiply to get opposites (use LCM). & $\begin{array}{l}
 
    \tmop{First} \tmop{Equation} : \tmop{multiply} \tmop{by} - 5\\
    - 5 (2 x - 5 y) = (- 13) (- 5)\\
    ~- 10 x + 25 y = 65\\
    \\
    \tmop{Second} \tmop{Equation} : \tmop{multiply} \tmop{by} 2\\
    2 (5 x - 3 y) = (- 4) 2\\
    ~~10 x - 6 y = - 8\\
  \end{array}$\\
  \hline
  3. Add equations to eliminate a variable.&  
	
	  $\begin{array}{l}
 ~~~~~- 10 x + 25 y = 65~~~~~~\\
 ~~~~~~\underline{~10 x~ -~ 6 y = - 8~~}\\
~~~~~~~~~~~~~~~~19 y =57\\
\end{array}$\\
\hline
  4. Solve. & $\begin{array}{l}
     ~~~~~~~~~~~~~~19y = 57\\
    ~~~~~~~~~~~~~~~\tmmathbf{\overline{19} ~~~~ \overline{19}}\\
    ~~~~~~~~~~~~~~~~~y = 3
  \end{array}$\\
  \hline
  5.~$\begin{array}{l}
	\text{Plug back into either of the}\\
	\text{given equations and solve.}
	\end{array}$
	
%	5. Plug into either of the given equations and solve. 
& \begin{tabular}{l}
    ~~~~~~$2 x - 5 (3) = - 13$\\
    ~~~~~~~~$2 x - 15 = - 13$\\
    ~~~~~~~~~~~~~$\tmmathbf{\underline{+ 15 ~+ 15}}$\\
		~~~~~~~~~~~~~~~$2 x = 2$\\
    ~~~~~~~~~~~~~~~~$\tmmathbf{\overline{2}}$ ~~ $\tmmathbf{\overline{2}}$\\
    ~~~~~~~~~~~~~~~~$x = 1$
  \end{tabular}\\
  \hline
  ~~~~~~~~~~~~~~~~~~~~~~~Solution & ~~~~~~~~~~~~$(x,y)=(1, 3)$\\
  \hline
\end{tabular}
~\pp
{\tmstrong{World View Note}}: The famous mathematical text, {\tmem{The Nine
Chapters on the Mathematical Art}}, which was printed around 179 AD in China
describes a formula very similar to Gaussian elimination which is very similar
to the addition method.\pp

Just as with graphing and substitution, it is possible to have no solution or
infinitely many solutions with elimination. If the variables all disappear from our problem, a true statement will always indicate infinitely many solutions and a false statement will always indicate no solutions.

\begin{example}~~~Solve the following system of equations.
      $$2 x - 5 y = 3\qquad\qquad - 6 x + 15 y = - 9$$
In order to obtain opposite coefficients in front of $x$, multiply the first equation by 3.
\begin{eqnarray*}
    3 (2 x - 5 y) = (3) 3 &  &\\
    6 x - 15 y = 9~~~~ &  &  \tmop{Distribute}\\
    &  & \\
    6 x - 15 y = ~9~ &  &\\
    \underline{- 6 x + 15 y = - 9} &  &  \tmop{Add} \tmop{equations} \tmop{together}\\
    0 = 0~~ &  & \tmop{True} \tmop{statement}
  \end{eqnarray*}
Since we are left with a true statement, we conclude that there are infinitely many solutions.%\pp
\end{example}
  
\begin{example}~~~Solve the following system of equations.
      $$4 x - 6 y = 8\qquad\qquad 6 x - 9 y = 15$$
Here, we will seek to obtain opposite coefficients for $x$.  This means we must find the LCM of 4 and 6, which is 12.  We will multiply both sides of both equations by the appropriate values in order to get $12x$ and $-12x$. 
\begin{eqnarray*}
    3 (4 x - 6 y) = (8) 3 &  & \tmop{Multiply} \tmop{first} \tmop{equation}
    \tmop{by~} 3\\
    12 x - 18 y = 24~~~ &  & \\
    &  & \\
    - 2 (6 x - 9 y) = (15) (- 2) &  & \tmop{Multiply} \tmop{second}
    \tmop{equation} \tmop{by~} - 2\\
    - 12 x + 18 y = - 30~~~~~~ &  & \\
    &  & \\
    12 x - 18 y = ~24~ &  &\\
    \underline{- 12 x + 18 y = - 30} &  &  \tmop{Add} \tmop{both} \tmop{new} \tmop{equations}
    \tmop{together}\\
    0 \neq - 6~~ &  & \tmop{False} \tmop{statement}
  \end{eqnarray*}
Since we are left with a false statement, we conclude that there are no solutions, or $\varnothing$.\pp
\end{example}
  
We have now covered three different methods that can be used to solve a system of
two equations with two variables: graphing, substitution, and addition/elimination. While all three can be used to solve any system, graphing works well for small integer solutions. Substitution works
best when we have a lone variable, and addition/elimination works best when the other
two methods fail. As each method has its own strengths, it is important that students become familiar with all three methods.

%\end{document}
