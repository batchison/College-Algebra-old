\subsection{Value Problems}\par

{\tmstrong{Objective: Solve value and mixture problems by setting up a system of
equations.}}\pp

One application related to solving systems of equations is known as a value problem. Value
problems are problems where each variable has a value attached to it. For example,
if our variable is the number of nickels in a person's pocket, those nickels
would have a value of five cents each. We will use a table to help us set up
and solve value problems. The basic structure of the table is shown below.
\[ \begin{array}{|c|c|c|c|}
     \hline
     & \tmop{Number} & \tmop{Value} & \tmop{Total}\\
     \hline
     \tmop{Item} 1 &  &  & \\
     \hline
     \tmop{Item} 2 &  &  & \\
     \hline
     \tmop{Total} &  &  & \\
     \hline
   \end{array} \]
The first column in the table is used for the number of things we have. Quite
often, this will contain our variables. The second column is used for the value each item has. The third column is used for the total value, which we calculate by multiplying the number by the value. For example, if we have 7 dimes, each with a value of 10 cents, the total value is $7 \cdot 10 = 70$
cents. The last row of the table is for totals. We will only use the third row
(also marked total) for the totals that are given to us. This means that this row may sometimes contain blank cells. Once the table is filled in we can easily
make equations by adding each column and setting it equal to the total at the
bottom of the column. This is shown in the following example.

%\vspace{1in}
%~\\
%\pagebreak

\begin{example}
% \
%\end{example}
  In a Luke's bank there are 11 coins that have a value of $\$1.85$. The coins are either quarters or dimes. How many of each coin does Luke have?


  \begin{eqnarray*}
    \begin{array}{|c|c|c|c|}
      \hline
      & \tmop{Number} & \tmop{Value} & \tmop{Total}\\
      \hline
      \tmop{Quarter} & q & 25 & \\
      \hline
      \tmop{Dime} & d & 10 & \\
      \hline
      \tmop{Total} &  &  & \\
      \hline
    \end{array} &  & \begin{array}{p{0.425\textwidth}}
      \tmop{Using} \tmop{the~value} \tmop{table}, \tmop{use} $q$ \tmop{for}
      \tmop{quarters}, $d$ \tmop{for} \tmop{dimes}.\pp
      \tmop{Each} \tmop{quarter's} \tmop{value} \tmop{is} 25 \tmop{cents},
      \tmop{each~dime's} \tmop{is} 10 \tmop{cents}.
    \end{array}\\
    &  & \\
    \begin{array}{|c|c|c|c|}
      \hline
      & \tmop{Number} & \tmop{Value} & \tmop{Total}\\
      \hline
      \tmop{Quarter} & q & 25 & 25 q\\
      \hline
      \tmop{Dime} & d & 10 & 10 d\\
      \hline
      \tmop{Total} &  &  & \\
      \hline
    \end{array} &  & \begin{array}{p{0.425\textwidth}}
		\text{Multiply~the~number~by} \text{the~value} \text{to~get} \text{the~totals}.
		\end{array}
  \end{eqnarray*}
We have a total of 11 coins.  This is the number total (first column).  We also know that we have \$1.85.  This is the money total (last column).
 \begin{eqnarray*}
   &  & \\
    \begin{array}{|c|c|c|c|}
      \hline
      & \tmop{Number} & \tmop{Value} & \tmop{Total}\\
      \hline
      \tmop{Quarter} & q & 25 & 25 q\\
      \hline
      \tmop{Dime} & d & 10 & 10 d\\
      \hline
      \tmop{Total} & 11 &  & 185\\
      \hline
    \end{array} &  & \begin{array}{p{0.425\textwidth}}
      \tmop{Write~the} \tmop{final} \tmop{total} \tmop{in} \tmop{cents} (185)
      \tmop{because} 25 \tmop{and} 10 \tmop{represent} \tmop{cents}.
    \end{array}%\\
		\end{eqnarray*}
		Next, we add the first and last columns to obtain our system of equations.
		$$q + d = 11\qquad\qquad 25 q + 10 d = 185$$
		We may either solve by substitution or addition/elimination.  Here, we will use addition/elimination.
	\begin{eqnarray*}
    - 10 (q + d) = (11) (- 10) &  & \tmop{Multiply} \tmop{first} \tmop{equation} \tmop{by} - 10\\
    - 10 q - 10 d = - 110~~~~~~ &  &\text{Simplify,~distribute~and~multiply} \\
    &  & \\
    - 10 q - 10 d = - 110 &  & \\
    \underline{25 q + 10 d = 185~}~ &  & \tmop{Add} \tmop{equations} \tmop{together}\\
    15 q = 75~~~ &  & \\
    \tmmathbf{\overline{15} ~~~~ \overline{15}}~~~ &  & \tmop{Divide} \tmop{both} \tmop{sides} \tmop{by~} 15.\\
    q = 5~~~~ &  & \tmop{We} \tmop{have} \tmop{our~} q\\
 \end{eqnarray*}
The number of quarters is $q=5$.  We can plug this back into one of our original equations to obtain $d$.  For simplicity, we will use the first equation.
 \begin{eqnarray*}
%    &  & \\
    (5) + d = 11 &  & \text{Solve~for~} d\\
    \tmmathbf{\underline{- 5 ~~~- 5}} &  & \tmop{Subtract~} 5\\
    d = 6~ &  & \tmop{We} \tmop{have} \tmop{our~} d
  \end{eqnarray*}
The number of dimes is $d=6$.  Luke's bank contains 6 dimes and 5 quarters.
\end{example}
 
{\tmstrong{World View Note:}} American coins are the only coins that do not
state the value of the coin. On the back of the dime it says ``one dime'' (not
10 cents). On the back of the quarter it says ``one quarter'' (not 25 cents).
On the penny it says ``one cent'' (not 1 cent). The rest of the world (Euros,
Yen, Pesos, etc.) all write the value as a number so people who don't speak the
language can easily use the coins.\pp

Ticket sales also have a value. Often different types of tickets sell for
different prices (values). These problems can be solved in much the same way.

\begin{example}
%  \
%\end{example}
  
  There were 41 tickets sold for an event. Tickets for children cost
  $\$1.50$ and tickets for adults cost $\$2.00$. Ticket sales for the event totaled $\$73.50$. How many of each type of ticket were sold?\pp
	We will use $c$ to denote the number of child tickets sold and $a$ to denote the number of adult tickets sold.
  \begin{eqnarray*}
    \begin{array}{|c|c|c|c|}
      \hline
      & \tmop{Number} & \tmop{Value} & \tmop{Total}\\
      \hline
      \tmop{Child} & c & 1.5 & \\
      \hline
      \tmop{Adult} & a & 2 & \\
      \hline
      \tmop{Total} &  &  & \\
      \hline
    \end{array} &  & \begin{array}{p{0.425\textwidth}}
      \tmop{Child} \tmop{tickets} \tmop{have} \tmop{a~value~of} \$1.50, \tmop{adult}
      \tmop{tickets~have} \tmop{a~value} \tmop{of} \$2.00
      (\tmop{we} \tmop{can} \tmop{drop} \tmop{the} \tmop{zeros} \tmop{after}
      \tmop{the} \tmop{decimal} \tmop{point}).
    \end{array}\\
    &  & \\
    \begin{array}{|c|c|c|c|}
      \hline
      & \tmop{Number} & \tmop{Value} & \tmop{Total}\\
      \hline
      \tmop{Child} & c & 1.5 & 1.5 c\\
      \hline
      \tmop{Adult} & a & 2 & 2 a\\
      \hline
      \tmop{Total} &  &  & \\
      \hline
    \end{array} &  & \begin{array}{p{0.425\textwidth}}
		\text{Multiply~the~number~by} \text{the~value} \text{to~get} \text{the~totals}.
		\end{array}\\
  \end{eqnarray*}
	~\pp
%& & \\	
We have a total of 41 tickets sold.  This is the number total (first column).  We also know that we have \$73.50.  This is the money total (last column).
 \begin{eqnarray*}
    \begin{array}{|c|c|c|c|}
      \hline
      & \tmop{Number} & \tmop{Value} & \tmop{Total}\\
      \hline
      \tmop{Child} & c & 1.5 & 1.5 c\\
      \hline
      \tmop{Adult} & a & 2 & 2 a\\
      \hline
      \tmop{Total} & 41 &  & 73.5\\
      \hline
    \end{array} &  & \begin{array}{p{0.425\textwidth}}
      \tmop{Write} \tmop{the} \tmop{ticket} \tmop{values} \tmop{in} \tmop{dollars} \tmop{as} 1.5 \tmop{and} 2.
    \end{array}%\\
		\end{eqnarray*}
		Next, we add the first and last columns to obtain our system of equations.
		$$c + a = 41\qquad\qquad 1.5 c + 2 a = 73.5$$
		We may either solve by substitution or addition/elimination.  Here, we will use substitution.
	\begin{eqnarray*}
    c + a = 41~~~~~ &  & \text{Solve~for~} a\\
    \tmmathbf{\underline{- c ~~~~~- c}}~~~~~ &  & \tmop{Subtract~} c\\
    a = 41 - c &  & \text{Equation~for~} a\\
    1.5 c + 2 \tmmathbf{(41 - c)} = 73.5~ &  & \tmop{Substitute} \tmop{into}
    \tmop{untouched} \tmop{equation}\\
    1.5 c + 82 - 2 c = 73.5~ &  & \tmop{Distribute}\\
    - 0.5 c + 82 = 73.5~ &  & \tmop{Combine} \tmop{like} \tmop{terms}\\
    \tmmathbf{\underline{- 82 ~~~- 82}}~ &  & \tmop{Subtract~} 82 \tmop{~from} \tmop{both}
    \tmop{sides}\\
    - 0.5 c = - 8.5 &  &\\
    \tmmathbf{\overline{- 0.5} ~~~ \overline{- 0.5}} &  &  \tmop{Divide} \tmop{both} \tmop{sides} \tmop{by} -
    0.5\\
    c = 17~~~ &  & \tmop{We} \tmop{have~} c%\\
 \end{eqnarray*}
The number of child tickets sold is $c=17$.  We can plug this back into our equation for $a$.
 \begin{eqnarray*}
    a = 41 - (17) &  & \tmop{Simplify}\\
    a = 24~~~~~~~~~ &  & \tmop{We} \tmop{have} \tmop{our~} a
	\end{eqnarray*}
The number of adult tickets sold is $a=24$.  We conclude that there were 17 child tickets and 24 adult tickets sold for the event.
\end{example}

Some problems will not give us the total number of items we have. Instead they
will provide a relationship between the items. Here we will have statements such
as ``There are twice as many dimes as nickels''. While it is clear that we
need to multiply one variable by 2, it may not be clear which variable gets
multiplied. Generally the equations are determined by working backwards from the actual
sentence. If there are twice as many dimes, than we multiply the other
variable (nickels) by two. So the equation would be $d = 2 n$. This type of
problem is worked out in the next example.

\begin{example}
  Bryce has a collection made up of 5-cent stamps and 8-cent
  stamps. There are three times as many 8-cent stamps as 5-cent stamps. The
  total value of all the stamps is $\$3.48$. How many of each stamp
  does Bryce have?\pp
	We will use $f$ to denote the number of 5-cent stamps and $e$ to denote the number of 8-cent stamps in the collection.
  \begin{eqnarray*}
    \begin{array}{|c|c|c|c|}
      \hline
      & \tmop{Number} & \tmop{Value} & \tmop{Total}\\
      \hline
      \tmop{Five} & f & 5 & 5 f\\
      \hline
      \tmop{Eight} & 3 f & 8 & 24 f\\
      \hline
      \tmop{Total} &  &  & 348\\
      \hline
    \end{array} &  & \begin{array}{p{0.425\textwidth}}
      \tmop{List~the} \tmop{value} \tmop{of} \tmop{each} \tmop{stamp}
      \tmop{in} \tmop{the} \tmop{value} \tmop{column}.
    \end{array}\\
    &  & \\
    \begin{array}{|c|c|c|c|}
      \hline
      & \tmop{Number} & \tmop{Value} & \tmop{Total}\\
      \hline
      \tmop{Five} & f & 5 & 5 f\\
      \hline
      \tmop{Eight} & e & 8 & 8 e\\
      \hline
      \tmop{Total} &  &  & \\
      \hline
    \end{array} &  & \begin{array}{p{0.425\textwidth}}
		\text{Multiply~the~number~by} \text{the~value} \text{to~get} \text{the~totals}.\\
		\end{array}\\
		&  & \\
   \begin{array}{|c|c|c|c|}
      \hline
      & \tmop{Number} & \tmop{Value} & \tmop{Total}\\
      \hline
      \tmop{Five} & f & 5 & 5 f\\
      \hline
      \tmop{Eight} & e & 8 & 8 e\\
      \hline
      \tmop{Total} &  &  & 348\\
      \hline
    \end{array} &  & \begin{array}{p{0.425\textwidth}}
      \tmop{The} \tmop{final} \tmop{total} \tmop{was} 338 (\tmop{written}
      \tmop{in} \tmop{cents}).\pp
      \tmop{We} \tmop{do} \tmop{not} \tmop{know} \tmop{the} \tmop{total}
      \tmop{number}, \tmop{this} \tmop{is} \tmop{left} \tmop{blank}.
    \end{array}\\
 \end{eqnarray*}
Recall that there are 3 times as many 8-cent stamps as there are 5-cent stamps.  This enables us to establish the following equation.
$$e=3f$$
Adding up the last (total) column gives us our second equation.
$$5 f + 8 e = 348$$
We will now use substitution to solve for $e$ and $f$.
  \begin{eqnarray*}
    5 f + 8 \tmmathbf{(3 f)} = 348 &  & \tmop{Substitute}
    \tmop{first} \tmop{equation} \tmop{in} \tmop{second}\\
    5 f + 24 f = 348 &  & \tmop{Simplify,~multiply}\\
    29 f = 348 &  & \tmop{Combine} \tmop{like} \tmop{terms}\\
    \tmmathbf{\overline{29} ~~~~~~ \overline{29}}  &  & \tmop{Divide} \tmop{both} \tmop{sides}
    \tmop{by~} 29.\\
    f = 12~ &  & \tmop{We} \tmop{have~} f
		\end{eqnarray*}
	There are $f=12$ 5-cent stamps.
	\begin{eqnarray*}
    e = 3 (12) &  & \tmop{Plug} \tmop{into} \tmop{first} \tmop{equation}\\
    e = 36~~~~ &  & \tmop{We} \tmop{have~} e
	\end{eqnarray*}
	There are $e=36$ 8-cent stamps.  We conclude that Bryce has twelve 5-cent stamps and thirty-six 8-cent stamps in his collection.
\end{example}

The same process for solving value problems can be applied to solving interest
problems.  For these problems we will make a slight adjustment to our table titles.\pp
\[ \begin{array}{|c|c|c|c|}
     \hline
     & \tmop{Invest} & \tmop{Rate} & \tmop{Interest}\\
     \hline
     \tmop{Account} 1 &  &  & \\
     \hline
     \tmop{Account} 2 &  &  & \\
     \hline
     \tmop{Total} &  &  & \\
     \hline
   \end{array} \]
	~\pp
Our first column represents the amount invested in each account. The second column
represents the interest rate earned (written as a decimal - move the decimal point two spaces to the left), and the last column represents the amount of interest earned. Just as
before, we multiply the investment amount by the interest rate to find the final
column, the interest earned. This is demonstrated in the following example.

%\vspace{.7in}
%~\\
%\pagebreak

\begin{example}
%  \
%\end{example}
  Nicki invests $\$4000$ in two accounts, one at 6\% interest, the other at 9\% interest for one year. At the end of the year she had earned $\$270$ in interest. How much did Nicki invest in each account?\pp
	Let $x$ represent the amount invested in the first account and $y$ represent the amount invested in the second account.
	
%\vspace{.5in}
%~\\
	
  \begin{eqnarray*}
    \begin{array}{|c|c|c|c|}
      \hline
      & \tmop{Invest} & \tmop{Rate} & \tmop{Interest}\\
      \hline
      \tmop{Account} 1 & x & 0.06 & \\
      \hline
      \tmop{Account} 2 & y & 0.09 & \\
      \hline
      \tmop{Total} &  &  & \\
      \hline
    \end{array} &  & \begin{array}{p{0.425\textwidth}}
      \tmop{Fill} \tmop{in~the} \tmop{interest} \tmop{rates} \tmop{as}
      \tmop{decimals}.
    \end{array}\\
%\end{eqnarray*}
%\begin{eqnarray*}
    &  & \\
    \begin{array}{|c|c|c|c|}
      \hline
      & \tmop{Invest} & \tmop{Rate} & \tmop{Interest}\\
      \hline
      \tmop{Account} 1 & x & 0.06 & 0.06 x\\
      \hline
      \tmop{Account} 2 & y & 0.09 & 0.09 y\\
      \hline
      \tmop{Total} &  &  & \\
      \hline
    \end{array} &  & \begin{array}{p{0.425\textwidth}}
		\text{Multiply~the~investment~by} \text{the~rate} \text{to~get} \text{the~interest}.%\\
		\end{array}\\
		&  & \\
    \begin{array}{|c|c|c|c|}
      \hline
      & \tmop{Invest} & \tmop{Rate} & \tmop{Interest}\\
      \hline
      \tmop{Account} 1 & x & 0.06 & 0.06 x\\
      \hline
      \tmop{Account} 2 & y & 0.09 & 0.09 y\\
      \hline
      \tmop{Total} & 4000 &  & 270\\
      \hline
    \end{array} &  & \begin{array}{p{0.425\textwidth}}
      \tmop{The~total} \tmop{investment} \tmop{is} \$4000.\\
      \tmop{The~total} \tmop{interest} \tmop{is} \$270.
    \end{array}
		\end{eqnarray*}
		Next, we add the first and last columns to obtain our system of equations.
		$$x + y = 4000\qquad\qquad 0.06 x + 0.09 y = 270$$
		We may either solve by substitution or addition/elimination.  Here, we will use addition/elimination.
	\begin{eqnarray*}
    - 0.06 (x + y) = (4000) (- 0.06) &  & \tmop{Multiply} \tmop{first} \tmop{equation} \tmop{by} - 0.06\\
    - 0.06 x - 0.06 y = - 240~~~~~~~~~~~ &  & \text{Simplify,~distribute~and~multiply}\\
    &  & \\
    - 0.06 x - 0.06 y = - 240 &  & \\
    \underline{~0.06 x + 0.09 y = ~270~} &  & \tmop{Add} \tmop{equations} \tmop{together}\\
    0.03 y = 30~~~~ &  & \\
    \tmmathbf{\overline{0.03} ~~~ \overline{0.03}}~~ &  & \tmop{Divide} \tmop{both} \tmop{sides} \tmop{by~} 0.03\\
    y = 1000~ &  & \tmop{We} \tmop{have~} y
	\end{eqnarray*}
	Nicki invested \$1000 at an interest rate of 9\%.  We can plug this back into one of our original equations to obtain $x$.  For simplicity, we will use the first equation.
	\begin{eqnarray*}
    x + 1000 = 4000~~~~ &  & \text{Solve~for~} x\\
    \tmmathbf{\underline{- 1000 ~~- 1000}} &  & \tmop{Subtract} 1000 \tmop{from} \tmop{both}
    \tmop{sides}\\
    x = 3000~~~~ &  & \tmop{We} \tmop{have~} x
	\end{eqnarray*}
	Nicki invested \$3000  at an interest rate of 6\%.  We conclude that Nicki invested \$3000 and \$100 at interest rates of 6\% and 9\%, respectively.
\end{example}
The same process can be used to find an unknown interest rate.
\begin{example}
  Whit invests $\$5000$ in one account and $\$8000$ in an
  account paying 4\% more in interest. He earned $\$1230$ in interest
  after one year. At what rates did Whit invest?\pp
	In the previous example, we used $x$ and $y$ to represent investment amounts.  Here, we will use $x$ and $y$ to represent interest rates for each investment.  Since we are told that the second account pays 4\% more than the first account in interest, we may immediately replace $y$ with $x+0.04$ (since $y=x+0.04$).  
  \begin{eqnarray*}
    \begin{array}{|c|c|c|c|}
      \hline
      & \tmop{Invest} & \tmop{Rate} & \tmop{Interest}\\
      \hline
      \tmop{Account} 1 & 5000 & x & \\
      \hline
      \tmop{Account} 2 & 8000 & x + 0.04 & \\
      \hline
      \tmop{Total} &  &  & \mbox{\hspace{.8in}}\\
      \hline
    \end{array} &  & \begin{array}{p{0.35\textwidth}}
      \tmop{Our} \tmop{investment} \tmop{table}. \tmop{Make} \tmop{sure} \tmop{to} \tmop{write} \tmop{all} \tmop{rates}
      \tmop{as} \tmop{decimals}!
    \end{array}\\
    &  & \\
    \begin{array}{|c|c|c|c|}
      \hline
      & \tmop{Invest} & \tmop{Rate} & \tmop{Interest}\\
      \hline
      \tmop{Account} 1 & 5000 & x & 5000 x\\
      \hline
      \tmop{Account} 2 & 8000 & x + 0.04 & 8000 x + 320\\
      \hline
      \tmop{Total} &  &  & \\
      \hline
    \end{array} &  & \begin{array}{p{0.35\textwidth}}
      \tmop{Multiply} \tmop{to} \tmop{fill} \tmop{in} \tmop{interest}
      \tmop{column}.\\
      \tmop{Make} \tmop{sure} \tmop{to} \tmop{distribute},\\
			8000 ($x$ + 0.04).
    \end{array}\\
    &  & \\
    \begin{array}{|c|c|c|c|}
      \hline
      & \tmop{Invest} & \tmop{Rate} & \tmop{Interest}\\
      \hline
      \tmop{Account} 2 & 5000 & x & 5000 x\\
      \hline
      \tmop{Account} 2 & 8000 & x + 0.04 & 8000 x + 320\\
      \hline
      \tmop{Total} &  &  & 1230\\
      \hline
    \end{array} &  & \tmop{Total} \tmop{interest} \tmop{was} \$1230.%\\
		\end{eqnarray*}
Adding up the last column gives our equation.
$$5000 x + 8000 x + 320 = 1230$$
		\begin{eqnarray*}
		5000 x + 8000 x + 320 = 1230~~ &  & \text{Solve~for~} x\\
    13000 x + 320 = 1230~~ &  & \tmop{Combine} \tmop{like} \tmop{terms}\\
    \tmmathbf{\underline{- 320 ~~- 320}} &  & \tmop{Subtract~} 320 \tmop{~from} \tmop{both}
    \tmop{sides}\\
    13000 x = ~~910~~ &  &\\
    \tmmathbf{\overline{13000} ~~~~ \overline{13000}} &  &  \tmop{Divide} \tmop{both} \tmop{sides} \tmop{by~} 13000.\\
    x = 0.07~~~ &  & \tmop{We} \tmop{have} \tmop{our~} x\\
		\end{eqnarray*}
		The first account had an interest rate of $x=7\%$.  Plugging this back into our equation for $y$ gives us $y=(0.07) + 0.04=0.11$.  The second account has an interest rate of $y=11\%$.\pp
		We conclude that Whit invested \$5000 and \$8000 at interest rates of 7\% and 11\%, respectively.
\end{example}

%\pagebreak
\comment{
{\tmstrong{Practice Problems: Solve each problem.}}

1) A collection of dimes and quarters is worth $\$15.25$. There are
103 coins in all.  How many of each is there?\par
~\par
2) A collection of half dollars and nickels is worth $\$13.40$. There
are 34 coins in  all. How many are there?\par
~\par

3) The attendance at a school concert was 578. Admission was $\$2.00$
for adults  and $\$1.50$ for children. The total
receipts were $\$985.00$. How many adults and  how many
children attended?\par
~\par

4) A purse contains $\$3.90$ made up of dimes and quarters. If there
are 21 coins in  all, how many dimes and how many quarters were there?\par
~\par

5) A boy has $\$2.25$ in nickels and dimes. If there are twice as
many dimes as nickels, how many of each kind has he?\par
~\par

6) $\$3.75$ is made up of quarters and half dollars. If the number of
quarters exceeds the number of half dollars by 3,
how many coins of each denomination are there?\par
~\par

7) A collection of 27 coins consisting of nickels and dimes amounts to
$\$2.25$. How many coins of each kind are there?\par
~\par

8) $\$3.25$ in dimes and nickels, were distributed among 45 boys. If
each received one coin, how many received dimes and how many
received nickels?\par
~\par

9) There were 429 people at a play. Admission was $\$1$ each for
adults and 75-cents each for children. The receipts were
$\$372.50$. How many children and how many adults
attended?\par
~\par

10) There were 200 tickets sold for a women's basketball game. Tickets for students were 50-cents each and for adults 75-cents
each. The total amount of  money collected was $\$132.50$.
How many of each type of ticket was sold?\par
~\par

11) There were 203 tickets sold for a volleyball game. For activity-card
holders, the price was $\$1.25$ each and for
noncard holders the price was $\$2$ each. The total
amount of money collected was $\$310$. How many of each type of
ticket was sold?\par
~\par

12) At a local ball game the hotdogs sold for $\$2.50$ each and the
hamburgers sold for $\$2.75$ each. There were 131 total
sandwiches sold for a total value of $\$342$.  How many of
each sandwich was sold?\par
~\par

13) At a recent Vikings game $\$445$ in admission tickets was taken
in. The cost of a student ticket was $\$1.50$ and the
cost of a non-student ticket was $\$2.50$. A total of
232 tickets were sold. How many students and how many non-students attended the game?\par
~\par

14) A bank contains 27 coins in dimes and quarters. The coins have a total
value  of $\$4.95$. Find the number of dimes and quarters
in the bank.\par
~\par

15) A coin purse contains 18 coins in nickels and dimes. The coins have a
total value of $\$1.15$. Find the number of nickels
and dimes in the coin purse.\par
~\par

16) A business executive bought 40 stamps for $\$9.60$. The purchase
included 25-cents stamps and 20-cents stamps. How
many of each type of stamp were bought?\par
~\par

17) A postal clerk sold some 15-cents stamps and some 25-cents
stamps. Altogether, 15 stamps were sold for a total cost
of $\$3.15$. How many of each type of stamps were sold?\par
~\par

18) A drawer contains 15-cents stamps and 18-cents stamps. The
number of 15-cents stamps is  four less than three times the
number of 18-cents stamps. The total value of all the
stamps is $\$1.29$. How many 15-cents stamps are in the
drawer?\par
~\par

19) The total value of dimes and quarters in a bank is $\$6.05$.
There are six more quarters than dimes. Find the number of
each type of coin in the bank.\par
~\par

20) A child's piggy bank contains 44 coins in quarters and dimes. The coins
have a total value of $\$8.60$. Find the number of
quarters in the bank.\par
~\par

21) A coin bank contains nickels and dimes. The number of dimes is 10 less
than twice the number of nickels. The total value of all the coins
is $\$2.75$. Find the number of each type of coin in the
bank.\par
~\par

22) A total of 26 bills are in a cash box. Some of the bills are one dollar
bills, and the rest are five dollar bills. The total amount of cash
in the box is $\$50$. Find  the number of each type of
bill in the cash box.\par
~\par

23) A bank teller cashed a check for $\$200$ using twenty dollar
bills and ten dollar  bills. In all, twelve bills were handed to
the customer. Find the number of twenty dollar bills and
the number of ten dollar bills.\par
~\par

24) A collection of stamps consists of 22-cents stamps and
40-cents stamps. The number of 22-cents stamps is
three more than four times the number of 40-cents stamps. The total value of the stamps is $\$8.34$. Find the number of
22-cents stamps in the collection.\par
~\par

25) A total of $\$27000$ is invested, part of it at 12\% and the rest
at 13\%. The total interest after one year is
$\$3385$. How much was invested at each rate?\par
~\par

26) A total of $\$50000$ is invested, part of it at 5\% and the rest
at 7.5\%. The total interest after one year is $\$3250$.
How much was invested at each rate?\par
~\par

27) A total of $\$9000$ is invested, part of it at 10\% and the rest
at 12\%. The total interest after one year is $\$1030$.
How much was invested at each rate?\par
~\par

28) A total of $\$18000$ is invested, part of it at 6\% and the rest
at 9\%. The total interest after one year is
$\$1248$. How much was invested at each rate?\par
~\par

29) An inheritance of $\$10000$ is invested in 2 ways, part at 9.5\%
and the remainder at 11\%. The combined
annual interest was $\$1038.50$. How much was
invested at each rate?\par
~\par

30) Kerry earned a total of $\$900$ last year on his investments. If
$\$7000$ was invested at a certain rate
of return and $\$9000$ was invested in a fund with a rate that was 2\% higher, find the two rates of interest.\par
~\par

31) Jason earned $\$256$ interest last year on his investments. If
$\$1600$ was invested at a certain rate of return and
$\$2400$ was invested in a fund with a rate that  was
double the rate of the first fund, find the two rates of interest.\par
~\par

32) Millicent earned $\$435$ last year in interest. If
$\$3000$ was invested at a certain  rate of return and
$\$4500$ was invested in a fund with a rate that was 2\% lower, find the two rates of interest.\par
~\par

33) A total of $\$8500$ is invested, part of it at 6\% and the rest
at 3.5\%. The total interest after one year is $\$385$.
How much was invested at each rate?\par
~\par

34) A total of $\$12000$ was invested, part of it at 9\% and the rest
at 7.5\%. The total interest after one year is
$\$1005$. How much was invested at each rate?\par
~\par

35) A total of $\$15000$ is invested, part of it at 8\% and the rest
at 11\%. The total  interest after one year is $\$1455$.
How much was invested at each rate?\par
~\par

36) A total of $\$17500$ is invested, part of it at 7.25\% and the
rest at 6.5\%. The total interest after one year is
$\$1227.50$. How much was invested at each rate?\par
~\par

37) A total of $\$6000$ is invested, part of it at 4.25\% and the
rest at 5.75\%. The total interest after one year is
$\$300$. How much was invested at each rate?\par
~\par

38) A total of $\$14000$ is invested, part of it at 5.5\% and the
rest at 9\%. The total interest after one year is $\$910$.
How much was invested at each rate?\par
~\par

39) A total of $\$11000$ is invested, part of it at 6.8\% and the
rest at 8.2\%. The total interest after one year is
$\$797$. How much was invested at each rate?\par
~\par

40) An investment portfolio earned $\$2010$ in interest last year. If
$\$3000$ was invested at a certain rate of
return and $\$24000$ was invested in a fund with a rate
that was 4\% lower, find the two rates of interest.\par
~\par

41) Samantha earned $\$1480$ in interest last year on her
investments. If $\$5000$ was invested at a certain rate of
return and $\$11000$ was invested in a fund with a rate
that was two-thirds the rate of the first fund, find the two rates of interest.\par
~\par

42) A man has $\$5.10$ in nickels, dimes, and quarters. There are
twice as many nickels as dimes and 3 more dimes than
quarters. How many coins of each kind were there?\par
~\par

43) 30 coins having a value of $\$3.30$ consists of nickels, dimes
and quarters. If there are twice as many quarters as
dimes, how many coins of each kind were there?\par
~\par

44) A bag contains nickels, dimes and quarters having a value of
$\$3.75$. If there are 40 coins in all and 3 times as
many dimes as quarters, how many coins of each kind were there?\par}

%\pagebreak
\newpage

\subsection{Mixture Problems}\par

One application related to solving systems of equations is known as a mixture problem. Mixture problems
are problems where two different solutions are mixed together, resulting in a new
final solution. We will use the following table to help us solve mixture
problems.
\begin{eqnarray*}
  \begin{array}{|c|c|c|c|}
    \hline
    & \tmop{Amount} & \tmop{Part} & \tmop{Total}\\
    \hline
    \tmop{Item} 1 &  &  & \\
    \hline
    \tmop{Item} 2 &  &  & \\
    \hline
    \tmop{Final} &  &  & \\
    \hline
  \end{array} &  & 
\end{eqnarray*}
The first column represents the amount (usually a volume or weight) of each solution that we have. The second column, labeled ``part'', represents the percentage (written as a decimal) of the particular item that is used in each solution. If we are mixing prices we will put them in this column.
Then we can multiply the amount by the part to find the total. Once we have the total, we can obtain
an equation(s) by adding the amount and/or total columns, which will enable us to solve
the problem.\pp

Basic mixture problems can have either one or two variables. We will start with a mixture problem containing one variable.

\begin{example}  
  A chemist has 70 mL of a 50\% methane solution. How much of an 80\% solution
  must he add so that the final solution is 60\% methane?
  \begin{eqnarray*}
    \begin{array}{|c|c|c|c|}
      \hline
      & \tmop{Amount} & \tmop{Part} & \tmop{Total}\\
      \hline
      \tmop{Start} & 70 & 0.5 & \\
      \hline
      \tmop{Add} & x & 0.8 & \\
      \hline
      \tmop{Final} &  &  & \\
      \hline
    \end{array} &  & %\begin{array}{p{0.425\textwidth}}
      \tmop{Set} \tmop{up} \tmop{the} \tmop{mixture} \tmop{table}.% \tmop{We}
      %\tmop{start} \tmop{with} 70 \tmop{mL,~but}
      %\tmop{do~not} \tmop{know} \tmop{how} \tmop{much} \tmop{we} \tmop{add},
      %\tmop{that} \tmop{is} $x$. \tmop{The} \tmop{part}
      %\tmop{is} \tmop{the} \tmop{percentages}, 0.5 \tmop{for} \tmop{start},
      %0.8 \tmop{for} \tmop{add}.
    %\end{array}\\
    %&  & \\
    \end{eqnarray*}
		We start the problem with 70 mL, but do not know how much of the other solution is added.  This amount is denoted by $x$ above.  The respective parts for each solution (entered as decimals) are 0.5 and 0.8.  
		\begin{eqnarray*}
		\begin{array}{|c|c|c|c|}
      \hline
      & \tmop{Amount} & \tmop{Part} & \tmop{Total}\\
      \hline
      \tmop{Start} & 70 & 0.5 & \\
      \hline
      \tmop{Add} & x & 0.8 & \\
      \hline
      \tmop{Final} & 70 + x & 0.6 & \\
      \hline
    \end{array} &  & \begin{array}{p{0.425\textwidth}}
      \tmop{Add} \tmop{the~amount} \tmop{column} \tmop{to} \tmop{get} \tmop{final}
      \tmop{amount}.\\
			\tmop{The} \tmop{part} \tmop{for} \tmop{the~final} \tmop{amount} \tmop{is} 0.6,
      \tmop{since} \tmop{we} \tmop{want} \tmop{the}
      \tmop{final} \tmop{solution} \tmop{to} \tmop{be} 60\% \tmop{methane}.
    \end{array}
\end{eqnarray*}
\begin{eqnarray*}
    %&  & \\
    \begin{array}{|c|c|c|c|}
      \hline
      & \tmop{Amount} & \tmop{Part} & \tmop{Total}\\
      \hline
      \tmop{Start} & 70 & 0.5 & 35\\
      \hline
      \tmop{Add} & x & 0.8 & 0.8 x\\
      \hline
      \tmop{Final} & 70 + x & 0.6 & 42 + 0.6 x\\
      \hline
    \end{array} &  & \begin{array}{p{0.425\textwidth}}
      \tmop{Multiply~the} \tmop{amount} \tmop{by} \tmop{the} \tmop{part} \tmop{to} \tmop{get}
      \tmop{the~total}.\pp
      \tmop{Be} \tmop{sure} \tmop{to} \tmop{distribute} \tmop{for} \tmop{the}
      \tmop{last} \tmop{row} : (70 + $x$) 0.6.
    \end{array}%\\
\end{eqnarray*}
Next we add the Total column to obtain the following equation.
\begin{center}
$35 + 0.8 x = 42 + 0.6 x$
\end{center}
\begin{eqnarray*}
    35 + 0.8 x = 42 + 0.6 x~~ & & \tmop{Solve~for~} x\\
		\tmmathbf{\underline{- 0.6 x ~~~~~~~- 0.6 x}} &  & \tmop{Subtract~} -0.6 x \tmop{~from} \tmop{both}
    \tmop{sides}\\
    35 + 0.2 x = 42~~~~~~~~~~~ &  & \\
    \tmmathbf{\underline{- 35 ~~~~~~~~~~- 35}}~~~~~~~~~ &  & \tmop{Subtract~} 35 \tmop{~from} \tmop{both}
    \tmop{sides}\\
    0.2 x = 7~~~~~~~~~~~~ &  & \\
    \tmmathbf{\overline{0.2} ~~~~ \overline{0.2}}~~~~~~~~~~ &  & \tmop{Divide} \tmop{both} \tmop{sides} \tmop{by~} 0.2\\
    x = 35~~~~~~~~~~~ &  & \tmop{We} \tmop{have} \tmop{our~} x
\end{eqnarray*}
We conclude that 35 mL of the 80\% methane solution must be added.
\end{example}
The same process can be used if the starting and final amount have a price
attached to them, rather than a percentage.
\begin{example}
  A coffee mix that sells for $\$2.50$ is to be made by mixing two
  types of coffee. The cafe plans to use 40 mL of the first type of coffee, which costs $\$3.00$.
  How much of the second type of coffee, costing $\$1.50$, should the cafe mix with the first?
  \begin{eqnarray*}
    \begin{array}{|c|c|c|c|}
      \hline
      & \tmop{Amount} & \tmop{Part} & \tmop{Total}\\
      \hline
      \tmop{Start} & 40 & 3 & \mbox{\hspace{.70in}}\\
      \hline
      \tmop{Add} & x & 1.5 & \\
      \hline
      \tmop{Final} &  &  & \\
      \hline
    \end{array} &  & %\begin{array}{p{0.425\textwidth}}
      \tmop{Set} \tmop{up} \tmop{mixture} \tmop{table}.
	\end{eqnarray*}
	We know the starting amount is 40 mL.  The amount that is added is unknown; label it as $x$.  The costs for each type of coffee, \$3 and \$1.5, are entered in the Part column.
	
	\begin{eqnarray*}
	   \begin{array}{|c|c|c|c|}
      \hline
      & \tmop{Amount} & \tmop{Part} & \tmop{Total}\\
      \hline
      \tmop{Start} & 40 & 3 & \mbox{\hspace{.70in}}\\
      \hline
      \tmop{Add} & x & 1.5 & \\
      \hline
      \tmop{Final} & 40 + x & 2.5 & \\
      \hline
    \end{array} &  & \begin{array}{p{0.425\textwidth}}
      \tmop{Add} \tmop{the} \tmop{amounts} \tmop{to} \tmop{get} \tmop{the}
      \tmop{final} \tmop{amount}.\pp
      \tmop{We} \tmop{want} \tmop{this} \tmop{final} \tmop{amount} \tmop{to}
      \tmop{sell} \tmop{for} \$2.50.
    \end{array}\\
    &  & \\
    \begin{array}{|c|c|c|c|}
      \hline
      & \tmop{Amount} & \tmop{Part} & \tmop{Total}\\
      \hline
      \tmop{Start} & 40 & 3 & 120\\
      \hline
      \tmop{Add} & x & 1.5 & 1.5 x\\
      \hline
      \tmop{Final} & 40 + x & 2.5 & 100 + 2.5 x\\
      \hline
    \end{array} &  & \begin{array}{p{0.425\textwidth}}
      \tmop{Multiply~the} \tmop{amount} \tmop{by} \tmop{the} \tmop{part} \tmop{to} \tmop{get}
      \tmop{the~total}.\\
      \tmop{Be} \tmop{sure} \tmop{to} \tmop{distribute} \tmop{on} \tmop{the}
      \tmop{last} \tmop{row} (40 + $x$) 2.5.
    \end{array}\\
\end{eqnarray*}
Next we add the Total column to obtain the following equation.
$$120 + 1.5 x = 100 + 2.5 x$$
\begin{eqnarray*}
    120 + 1.5 x = 100 + 2.5 x~ & & \tmop{Solve~for~} x\\
		\tmmathbf{\underline{- 1.5 x ~~~~~~~~- 1.5 x}} &  & \tmop{Subtract~} 1.5 x \tmop{~from} \tmop{both}
    \tmop{sides}\\
    120 = 100 + x~~~~~ &  & \\
    \tmmathbf{\underline{- 100 ~~- 100}}~~~~~~~ &  & \tmop{Subtract~} 100 \tmop{~from} \tmop{both}
    \tmop{sides}\\
    20 = x~~~~~~~~~~~~~ &  & \tmop{We} \tmop{have} \tmop{our~} x
\end{eqnarray*}
We conclude that 20 mL of the \$1.50 coffee must be added.
\end{example}

{\tmstrong{World View Note:}} Brazil is the world's largest coffee producer,
producing 2.59 million metric tons of coffee a year! That is more than three times
the amount that the second-largest producer, Vietnam, produces!\pp

The previous examples illustrate how we can construct a table in order to obtain
a related equation and solve a mixture problem. But we are also interested in \textit{systems} of
equations, with two or more unknowns. The following example is one such problem.

\begin{example}
  Farmer Reeb has two types of milk, one that is 24\% butterfat and another
  which is 18\% butterfat. How much of each should he use to end up with 42
  gallons of 20\% butterfat?\pp
	Here, we don't know either amount of milk.  Label them as $x$ and $y$, respectively.  The final amount of milk is entered in the table below as 42.
  \begin{eqnarray*}
    \begin{array}{|c|c|c|c|}
      \hline
      & \tmop{Amount} & \tmop{Part} & \tmop{Total}\\
      \hline
      \tmop{Milk} 1 & x & 0.24 & \\
      \hline
      \tmop{Milk} 2 & y & 0.18 & \\
      \hline
      \tmop{Final} & 42 & 0.2 & \\
      \hline
    \end{array} &  & \begin{array}{p{0.425\textwidth}}
			\tmop{Fill} \tmop{in} \tmop{the~Part}
      \tmop{column} \tmop{with} \tmop{the~percentage}
      \tmop{of} \tmop{each} \tmop{type} \tmop{of} \tmop{milk} \tmop{included}
      \tmop{in~the} \tmop{final} \tmop{solution}.
    \end{array}\\
    &  & \\
    \begin{array}{|c|c|c|c|}
      \hline
      & \tmop{Amount} & \tmop{Part} & \tmop{Total}\\
      \hline
      \tmop{Milk} 1 & x & 0.24 & 0.24 x\\
      \hline
      \tmop{Milk} 2 & y & 0.18 & 0.18 y\\
      \hline
      \tmop{Final} & 42 & 0.2 & 8.4\\
      \hline
    \end{array} &  & \begin{array}{p{0.425\textwidth}}
			\tmop{Multiply~the} \tmop{amount} \tmop{by} \tmop{the} \tmop{part} \tmop{to} \tmop{get}
      \tmop{the~total}.
    \end{array}\\
 \end{eqnarray*}
Adding up the Amount column provides us with our first equation.  Adding up the Total column gives our second equation.  Both are stated below.
$$x + y = 42\qquad\qquad 0.24 x + 0.18 y = 8.4$$
We will use addition/elimination to solve the new system of equations. 
\begin{eqnarray*}
    - 0.18 (x + y) = (42) (- 0.18) &  & \tmop{Multiply} \tmop{the~first} \tmop{equation} \tmop{by~} - 0.18\\
    - 0.18 x - 0.18 y = - 7.56~~~~~~~~ &  & \text{Simplify;~distribute~and~multiply}\\
    &  & \\
    - 0.18 x - 0.18 y = - 7.56 &  &\\
    \underline{0.24 x + 0.18 y = ~~8.4~~} &  &  \tmop{Add} \tmop{the} \tmop{equations}
    \tmop{together}\\
    0.06 x = 0.84~~~ &  & \\
    \tmmathbf{\overline{0.06} ~~~~ \overline{0.06}}~~~ &  & \tmop{Divide} \tmop{both} \tmop{sides} \tmop{by~} 0.06.\\
     x = 14~~~~~ &  & \tmop{We} \tmop{have} \tmop{our~} x
	\end{eqnarray*}
	We know that $x=14$ gal of the 24\% butterfat milk should be added.  Next, we can plug 14 into either of the two equations in our system to find $y$.  For simplicity, we will use the first equation.
	\begin{eqnarray*}
		(14) + y = 42~~~ &  &\tmop{Solve~for~} y\\
    \tmmathbf{\underline{- 14 ~~~~~~~- 14}} &  & \tmop{Subtract~} 14 \tmop{~from} \tmop{both}
    \tmop{sides}\\
    y = 28~~~ &  & \tmop{We} \tmop{have} \tmop{our~} y
	\end{eqnarray*}
	So $y=28$ gal of the 18\% butterfat milk should be added.  We conclude that farmer Reeb should add 14 gallons of the 24\% milk and 28 gallons of the 18\% milk in order to end up with 42 gallons of a 20\% butterfat milk.
\end{example}
  
The same process can be used to solve mixtures of prices with two unknowns.

\begin{example}  
  In a candy shop, chocolate which sells for $\$4$ a pound is mixed
  with nuts which are sold for $\$2.50$ a pound to form a
  chocolate-nut candy which sells for $\$3.50$ a pound. How much of
  each are used to make 30 pounds (lbs) of the mixture?\pp
	As we fill in the mixture table below, we will use $c$ and $n$ to represent the amounts of chocolate and nuts, respectively.
		\begin{eqnarray*}
    \begin{array}{|c|c|c|c|}
      \hline
      & \tmop{Amount} & \tmop{Part} & \tmop{Total}\\
      \hline
      \tmop{Chocolate} & c & 4 & \\
      \hline
      \tmop{Nut} & n & 2.5 & \\
      \hline
      \tmop{Final} & 30 & 3.5 & \\
      \hline
    \end{array} &  & \begin{array}{p{0.425\textwidth}}
      \tmop{We} \tmop{know} \tmop{the} \tmop{final} \tmop{amount~(30)}
      \tmop{and} \tmop{price~(\$3.5)};
      \tmop{include} \tmop{these} \tmop{in} \tmop{the} \tmop{table}.
    \end{array}\\
    &  & \\
    \begin{array}{|c|c|c|c|}
      \hline
      & \tmop{Amount} & \tmop{Part} & \tmop{Total}\\
      \hline
      \tmop{Chocolate} & c & 4 & 4 c\\
      \hline
      \tmop{Nut} & n & 2.5 & 2.5 n\\
      \hline
      \tmop{Final} & 30 & 3.5 & 105\\
      \hline
    \end{array} &  & \begin{array}{p{0.425\textwidth}}
			\tmop{Multiply~the} \tmop{amount} \tmop{by} \tmop{the} \tmop{part} \tmop{to} \tmop{get}
      \tmop{the~total}.
    \end{array}%\\
  \end{eqnarray*}
Adding up the Amount and the Total columns gives us our system of equations, shown below.
$$c+n=30\qquad\qquad 4 c + 2.5 n = 105$$
We will solve this system using substitution. 	
\begin{eqnarray*}
    c + n = 30~~~~~ &  & \text{Solve~for~} c\\
    \tmmathbf{\underline{- n ~~- n}}~~~ &  &\tmop{Subtract~} n
    \tmop{~from} \tmop{both~sides}\\
    c = 30 - n &  & \text{Equation~for~} c\\
    &  & \\
    4 \tmmathbf{(30 - n)} + 2.5 n = 105~~~ &  & \tmop{Substitute} \tmop{into}
    \tmop{untouched} \tmop{equation}\\
    120 - 4 n + 2.5 n = 105~~~ &  & \tmop{Distribute}\\
    120 - 1.5 n = 105~~~ &  & \tmop{Combine} \tmop{like} \tmop{terms}\\
    \tmmathbf{\underline{- 120 ~~~~~~~~~~- 120}} &  & \tmop{Subtract~} 120 \tmop{~from} \tmop{both}
    \tmop{sides}\\
    - 1.5 n = - 15~~ &  &\\
    \tmmathbf{\overline{- 1.5} ~~~~ \overline{- 1.5}}~ &  &  \tmop{Divide} \tmop{both} \tmop{sides} \tmop{by~} -
    1.5\\
    n = 10~~~~~ &  & \tmop{We} \tmop{have} \tmop{our~} n
		\end{eqnarray*}
		We know that $n=10$ lbs of nuts must be added.  Next, we can plug 10 into our equation for $c$.		
		\begin{eqnarray*}
    c = 30 - (10) &  & \text{Simplify}\\
    c = 20~~~~~~~~~ &  & \tmop{We} \tmop{have} \tmop{our~} c
		\end{eqnarray*}
		So $c=20$ lbs of chocolate must be added.  We conclude that 10 pounds of nuts and 20 pounds of chocolate must be mixed in order to obtain a 30-pound chocolate-nut candy mixture that sells for \$3.50 a pound.
\end{example}

With mixture problems we often are mixing with a pure solution or using water
which contains none of the chemical we are interested in. For pure solutions,
the percentage is 100\% (or 1 in the table). For water, the percentage is 0\% (or 0).
This following example solves one such problem.

\begin{example}  
  A solution of pure antifreeze is mixed with water to make a 65\% antifreeze
  solution. How much of each item should be used to make 70 L?\pp
	Here, we will use $a$ and $w$ to denote the amounts of pure antifreeze and water, respectively, that are added.
  \begin{eqnarray*}
    \begin{array}{|c|c|c|c|}
      \hline
      & \tmop{Amount} & \tmop{Part} & \tmop{Final}\\
      \hline
      \tmop{Antifreeze} & a & 1 & \\
      \hline
      \tmop{Water} & w & 0 & \\
      \hline
      \tmop{Total} & 70 & 0.65 & \\
      \hline
    \end{array} &  & \begin{array}{p{0.425\textwidth}}
      \tmop{Antifreeze}
      \tmop{is} \tmop{pure,~} 100\% \tmop{~or~} 1 \tmop{~in} \tmop{our}
      \tmop{table}, \tmop{written} \tmop{as~a}
      \tmop{decimal}.\\
			\tmop{Water} \tmop{contains} \tmop{no} \tmop{antifreeze;}
      \tmop{its}
      \tmop{percentage} \tmop{is~} 0.
    \end{array}\\
    &  & \\
    \begin{array}{|c|c|c|c|}
      \hline
      & \tmop{Amount} & \tmop{Part} & \tmop{Total}\\
      \hline
      \tmop{Antifreeze} & a & 1 & a\\
      \hline
      \tmop{Water} & w & 0 & 0\\
      \hline
      \tmop{Final} & 70 & 0.65 & 45.5\\
      \hline
    \end{array} &  & \tmop{Multiply} \tmop{across~to} \tmop{find} \tmop{totals}
		  \end{eqnarray*}
Adding up the Amount and the Total columns gives us our system of equations, shown below.
$$a + w = 70\qquad\qquad a = 45.5$$
~\pp
Since our second equation gives us $a$, we need only plug this into our first equation to find $w$.
\begin{eqnarray*}
    (45.5) + w = 70~~~~~ &  & \text{Solve for~} w\\
    \tmmathbf{\underline{- 45.5 ~~~~~~~- 45.5}} &  & \tmop{Subtract~} 45.5 \tmop{~from} \tmop{both}
    \tmop{sides}\\
    w = 24.5~~ &  & \tmop{We} \tmop{have} \tmop{our~} w
	\end{eqnarray*}	
We quickly conclude that $a=45.5$ L pure antifreeze and $w=24.5$ L of water must be used in order to produce a 70 L mixture that is 65\% antifreeze.  
\end{example}

\comment{

%\pagebreak

{\tmstrong{Practice Problems: Solve each problem.}}

1) A tank contains 8000 liters of a solution that is 40\% acid. How much water should be added to make a solution that is 30\% acid?\par
~\par

2) How much antifreeze should be added to 5 quarts of a 30\% mixture of antifreeze to make a solution that is 50\%
antifreeze?\par
~\par

3) Of 12 pounds of salt water 10\% is salt; of another mixture 3\% is salt.
How many pounds of the second should be added to the first
in order to get a mixture of 5\% salt?\par
~\par

4) How much alcohol must be added to 24 gallons of a 14\% solution of alcohol
in order to produce a 20\% solution?\par
~\par

5) How many pounds of a 4\% solution of borax must be added to 24 pounds of a 12\% solution of borax to obtain a 10\% solution of borax?\par
~\par

6) How many grams of pure acid must be added to 40 grams of a 20\% acid solution to make a solution which is 36\% acid?\par
~\par

7) A 100 LB bag of animal feed is 40\% oats. How many pounds of oats must be added to this feed to produce a mixture which is 50\% oats?\par
~\par

8) A 20 oz alloy of platinum that costs $\$220$ per ounce is mixed
with an alloy that costs $\$400$ per ounce. How many ounces of the $\$400$ alloy should be used to make an
alloy that costs $\$300$ per ounce?\par
~\par

9) How many pounds of tea that cost $\$4.20$ per pound must be mixed
with 12 lb of tea that cost $\$2.25$ per pound to make a
mixture that costs $\$3.40$ per pound?\par
~\par

10) How many liters of a solvent that costs $\$80$ per liter must be
mixed with 6 L of a solvent that costs $\$25$ per liter
to make a solvent that costs $\$36$ per liter?\par
~\par

11) How many kilograms of hard candy that cost $\$7.50$ per kilogram
must be mixed with 24 kg of jelly beans that cost
$\$3.25$ per kilogram to make a mixture
that sells for $\$4.50$ per kilogram?\par
~\par

12) How many kilograms of soil supplement that costs $\$7.00$ per
kilogram must be mixed with 20 kg of aluminum nitrate that
costs $\$3.50$ per kilogram to make a fertilizer
that costs $\$4.50$ per kilogram?\par
~\par

13) How many pounds of lima beans that cost 90-cents per pound must be
mixed with  16 lb of corn that cost 50-cents per pound to make
a mixture of vegetables that costs 65-cents per
pound?\par
~\par

14) How many liters of a blue dye that costs $\$1.60$ per liter must
be mixed with 18 L of a dye that costs $\$2.50$ per
liter to make a mixture that costs $\$1.90$ per liter?\par
~\par

15) Solution A is 50\% acid and solution B is 80\% acid. How much of each
should be used to make 100cc. of a solution that is 68\% acid?\par
~\par

16) A certain grade of milk contains 10\% butter fat and a certain grade of
cream 60\% butter fat. How many quarts of each must be taken so
as to obtain a mixture of 100 quarts that will be 45\%
butter fat?\par
~\par

17) A farmer has some cream which is 21\% butterfat and some which is 15\% butter fat. How many gallons of each must be mixed to
produce 60 gallons of cream which is 19\% butterfat?\par
~\par

18) A syrup manufacturer has some pure maple syrup and some which is 85\% maple syrup. How many liters of each should be mixed to make
150L which  is 96\% maple syrup?\par
~\par

19) A chemist wants to make 50ml of a 16\% acid solution by mixing a 13\% acid solution and an 18\% acid solution. How many milliliters of
each solution should the chemist use?\par
~\par

20) A hair dye is made by blending 7\% hydrogen peroxide solution and a 4\% hydrogen peroxide solution. How many milliliters of each
are used to make a 300 ml solution that is 5\% hydrogen
peroxide?\par
~\par

21) A paint that contains 21\% green dye is mixed with a paint that contains
15\%  green dye. How many gallons of each must be used to make 60
gal of paint that is 19\% green dye?\par
~\par

22) A candy mix sells for $\$2.20$ per kilogram. It contains
chocolates worth $\$1.80$  per kilogram and other candy
worth $\$3.00$ per kilogram. How much of each are in
15 kilograms of the mixture?\par
~\par

23) To make a weed and feed mixture, the Green Thumb Garden Shop mixes fertilizer worth $\$4.00$/lb. with a weed killer
worth $\$8.00$/lb. The mixture will cost
$\$6.00$/lb. How much of each should be used to prepare 500 lb. of the mixture?\par
~\par

24) A grocer is mixing 40 cent per lb. coffee with 60 cent per lb. coffee to
make a mixture worth 54-cents per lb. How much of each kind
of coffee should be used to make 70 lb. of the mixture?\par
~\par

25) A grocer wishes to mix sugar at 9-cents per pound with sugar at 6-cents
per pound to make 60 pounds at 7-cents per pound. What quantity
of each must he take?\par
~\par

26) A high-protein diet supplement that costs $\$6.75$ per pound is
mixed with a vitamin supplement that costs $\$3.25$
per pound. How many pounds of each should be used to make 5
lb of a mixture that costs $\$4.65$ per pound?\par
~\par

27) A goldsmith combined an alloy that costs $\$4.30$ per ounce with
an alloy that costs $\$1.80$per ounce. How many ounces
of each were used to make a mixture of 200 oz costing
$\$2.50$ per ounce?\par
~\par

28) A grocery store offers a cheese and fruit sampler that combines cheddar
cheese that costs $\$8$ per kilogram with kiwis that cost
$\$3$ per kilogram. How many kilograms of each were
used to make a 5 kg mixture that costs $\$4.50$ per kilogram?\par
~\par

29) The manager of a garden shop mixes grass seed that is 60\% rye grass with
70  lb of grass seed that is 80\% rye grass to make a mixture
that is 74\% rye grass. How much of the 60\% mixture
is used?\par
~\par

30) How many ounces of water evaporated from 50 oz of a 12\% salt solution to produce a 15\% salt solution?\par
~\par

31) A caterer made an ice cream punch by combining fruit juice that cost
$\$2.25$ per gallon with ice cream that costs
$\$3.25$ per gallon. How many gallons of each were
used to make 100 gal of punch costing $\$2.50$ per pound?\par
~\par

32) A clothing manufacturer has some pure silk thread and some thread that is 85\% silk. How many kilograms of each must be woven
together to make 75 kg of cloth that is 96\% silk?\par
~\par

33) A carpet manufacturer blends two fibers, one 20\% wool and the second 50\% wool. How many pounds of each fiber should be woven together
to produce 600 lb of a fabric that is 28\% wool?\par
~\par

34) How many pounds of coffee that is 40\% java beans must be mixed with 80 lbs of coffee that is 30\% java beans to make a coffee blend that is
32\% java beans?\par
~\par

35) The manager of a specialty food store combined almonds that cost
$\$4.50$ per pound with walnuts that cost
$\$2.50$ per pound. How many pounds of each were
used to make a 100 lb mixture that cost $\$3.24$ per pound?\par
~\par

36) A tea that is 20\% jasmine is blended with a tea that is 15\% jasmine. How many pounds of each tea are used to make 5 lb of tea
that is 18\% jasmine?\par
~\par

37) How many ounces of dried apricots must be added to 18 oz of a snack mix that contains 20\% dried apricots to make a mixture that
is 25\% dried apricots?\par
~\par

38) How many milliliters of pure chocolate must be added to 150 ml of chocolate topping that is 50\% chocolate to make a topping that is 75\%
chocolate?\par
~\par

39) How many ounces of pure bran flakes must be added to 50 oz of cereal that is 40\% bran flakes to produce a mixture that is 50\% bran
flakes?\par
~\par

40) A ground meat mixture is formed by combining meat that costs $\$2.20$ per pound with meat that costs
$\$4.20$ per pound. How many pounds of each were used to make a 50 lb mixture that costs $\$3.00$ per pound?\par
~\par

41) How many grams of pure water must be added to 50 g of pure acid to make a solution that is 40\% acid?\par
~\par

42) A lumber company combined oak wood chips that cost $\$3.10$ per pound with pine wood chips that cost $\$2.50$ per pound.
How many pounds of each were used to make an 80 lb mixture
costing $\$2.65$ per pound?\par
~\par

43) How many ounces of pure water must be added to 50 oz of a 15\% saline solution to make a saline solution that is 10\% salt?
\end{document}
}
