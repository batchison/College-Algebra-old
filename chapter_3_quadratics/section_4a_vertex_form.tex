\subsection{The Vertex Form}\pp

%\maketitle

{\tmstrong{Objective: Express a quadratic equation in vertex form.}}\pp

Recall the two forms of a quadratic equation, shown below.  In both forms, assume $a\neq 0$.%\pp
\begin{center}
\begin{tabular}{lcc}
Standard Form: & & $y=ax^2+bx+c$, where $a,b,$ and $c$ are real numbers\\
&&\\
Vertex Form: & & $y=a(x-h)^2+k,$ where $a,h,$ and $k$ are real numbers
\end{tabular}
\end{center}
Unlike the standard form, a quadratic equation written in vertex form allows for immediate recognition of the vertex ($h,k$), which will always coincide with either a maximum (if $a<0$) or a minimum (if $a>0$) on the accompanying graph, called a parabola. Additionally, using the vertex form, we can easily identify the \textit{axis of symmetry} for the parabola, which is a vertical line $x=h$ that passes through the $x$-coordinate of the vertex and ``splits'' the graph into two identical halves.\pp
When graphing parabolas, it will help to think of the axis of symmetry as a vertical line over which either half of the graph could be ``folded'', to produce the other half.  This will allow us to reflect (by symmetry) any point on the parabola to the other side of the axis of symmetry, and identify another point on the graph.  As a result, both points will have the same $y$-coordinate, and will be (horizontally) equidistant from the axis of symmetry.  By reflecting points about the axis of symmetry, we can graph not just one, but two points on the graph, for every single value of $x$ that we plug into the given equation.  %<--- BEN: May want to remove

%\newpage
\begin{example}~~~Consider $y=-2(x+1)^2+3$.\\
~\\
\label{trans1}
In this example we can see immediately that the vertex is at $(-1,3)$. It is important that we not overlook the negative value for $h$.  The axis of symmetry, passes through the $x$-coordinate for the vertex, $x=-1$.  

\newpage

%\begin{example}~~~Consider $y=-2(x+2)^2+1$.\\
\begin{vwcol}[widths={0.7,0.3},
 sep=.8cm, justify=flush,rule=0pt,indent=1em]
Now to find more points on the parabola we can plug in $x=0$. We can see that $y = -2(0+1)^2+3=1$, so $(0,1)$ is a point on our parabola.\pp
Since the point we just located sits one unit to the right of the axis of symmetry, we also know that the point $(-2,1)$, sitting one unit to the left of the axis of symmetry will also be a point on our graph.  We can always check this by plugging $x=-2$ into the equation and solving for $y$.\pp
In a similar fashion we can plug in $x=1$, a coordinate that is two units to the right of the axis of symmetry and get a $y$-coordinate of -5.  Thus an $x$-coordinate two units left of the axis, $x=-3$, will also yield a $y$-coordinate of -5.  The accompanying graph shows our parabola, with the axis of symmetry appearing as a dashed vertical line at $x=-1$.  

\begin{mfpic}[20]{-3}{1.5}{-7}{3}
\arrow \reverse \arrow \function{-3,1,.1}{((-2)*((x+1)**2))+3}
\axes
\xmarks{-3,-2,-1,1}
\ymarks{-7,-6,-5,-4,-3,-2,-1,1,2}
\tlabelsep{3pt}
\dashed \arrow \reverse \arrow[1 5pt]
\polyline{(-1,-7),(-1,2)}
\axislabels x{{$-3$}-3,{$-2$}-2,{$-1$}-1,{$1$}1,}
\axislabels y{{$-7$}-7,{$-6$}-6,{$-5$}-5,{$-4$}-4,{$-3$}-3, {$-2$}-2,{$-1$}-1,{$1$}1,{$2$}2}
\end{mfpic}
\end{vwcol}
\end{example}

We began the discussion of vertex form in the introductory section of this chapter. It follows naturally to learn how to transform a quadratic equation that is given in standard form into one written in vertex form.\pp
If $y=ax^2+bx+c$ ($a\neq0$), we can identify the $x$-coordinate for the vertex (and consequently the equation for the axis of symmetry) using the following formula.
$$h=-~\frac{b}{2a}$$ 
After identifying $h$, we can determine based upon the sign of the leading coefficient $a$ whether the vertex will be a maximum (if $a$ is negative, $a<0$) or a minimum (if $a$ is positive, $a>0$).  The equation for the vertical line $x=h$ will be our axis of symmetry.\pp

%\newpage

Finally, we know that the $y$-coordinate for our vertex must occur somewhere on the axis of symmetry.  This can easily be found by plugging $x=h$ back into the given equation for our quadratic, and simplifying to find the $y$-coordinate, which we will relabel as $k$.\pp

Once we have $h$ and $k$, we can use them, along with $a$, to write the vertex form for our quadratic, $$y=a(x-h)^2+k.$$%\pp
The following examples will clearly demonstrate this process.\\

\begin{example}~~~Identify the vertex and axis of symmetry for the parabola represented by the given quadratic equation.
\begin{eqnarray*}
y=x^2+8x-12       &  & \text{Given an equation in standard form}\\              
a=1,~~~b=8,~~~c=-12       &  & \text{Identify~} a,b, \text{~and~} c\\             
h=-~\frac{b}{2a}=-~\frac{8}{2(1)}=-4 & & \text{Identify~} h\\
x=-4 & & \text{Use~} h \text{~for~axis~of~symmetry,~a~vertical~line}\\
k= (-4)^2+8(-4)-12   &  & \text{Plug in~} h \text{~to~find~} k\\
k= 16-32-12= -28  &  &\mathrm{Simplify}\\
(-4,-28)    &  & \text{Write the vertex as an ordered pair $(h,k)$}
\end{eqnarray*}
\end{example}
%~\\
\begin{example}~~~Identify the vertex and axis of symmetry for the parabola represented by the given quadratic equation.
\begin{eqnarray*}
y=-3x^2+6x-1      &  &\text{Given an equation in standard form}\\              
a=-3,~~~b=6,~~~c=-1       &  & \text{Identify~} a,b, \text{~and~} c\\             
h=-~\frac{b}{2a}=-~\frac{6}{2(-3)}=1 & & \text{Identify~} h\\
x=1 & & \text{Use~} h \text{~for~axis~of~symmetry,~a~vertical~line}\\
k= -3(1)^2+6(1)-1   &  & \text{Plug in~} h \text{~to~find~} k\\
k= -3+6-1= 2  &  &\mathrm{Simplify}\\
(1,2)    &  & \text{Write the vertex as an ordered pair $(h,k)$}
\end{eqnarray*}
\end{example}
%~\\
\begin{example}~~~Identify the vertex and axis of symmetry for the parabola represented by the given quadratic equation.
\begin{eqnarray*}
y=-x^2-12      &  & \text{Given an equation in standard form}\\ 
a=-1,~~~b=0,~~~c=-12       &  & \text{Identify~} a,b, \text{~and~} c\\             
h=-~\frac{b}{2a}=-~\frac{0}{2(-1)}=0 & & \text{Identify~} h\\
\end{eqnarray*}
\begin{eqnarray*}
%h=-~\frac{b}{2a}=-~\frac{0}{2(-1)}=0 & & \text{Identify~} h\\
x=0 & & \text{Use~} h \text{~for~axis~of~symmetry,~a~vertical~line}\\
k= -(0)^2-12   &  & \text{Plug in~} h \text{~to~find~} k\\
k= 0-12= -12  &  &\mathrm{Simplify}\\
(0,-12)    &  & \text{Write the vertex as an ordered pair $(h,k)$}
\end{eqnarray*}
\end{example}
%~\\
There is a more algebraic (and complicated) method of transforming a quadratic equation given in standard form into one that is in vertex form, known as \textit{completing the square}.  This method will be explained in detail towards the end of the chapter.\pp
We will also see how the vertex form can be particularly useful when solving a quadratic equation, in order to identify the $x$-intercepts of the corresponding parabola.  Solving a quadratic equation using the vertex form is know as the method of \textit{extracting square roots}, and will be seen once we have had a thorough discussion of square roots, as well as complex numbers. 