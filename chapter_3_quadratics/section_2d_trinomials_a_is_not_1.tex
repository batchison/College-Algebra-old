\subsection{Trinomials with Leading Coefficient $a\neq 1$}\pp

%\maketitle

{\tmstrong{Objective: Factor trinomials using the $ac$-method when the
coefficient of $x^2$ is not one.}}\pp

When factoring trinomials we used the $ac$-method to split the middle (or linear) term and
then factor by grouping. The $ac$-method gets its name from the general
trinomial expression, $a x^2 + b x + c$, where $a, b,$ and $c$ are the
leading coefficient, linear coefficient, and constant term, respectively.\pp

{\tmstrong{World View Note:}} It was French philosopher Rene Descartes who
first used letters from the beginning of the alphabet $(a, b, c)$ to represent values we
know and letters from the end ($x, y, z$) to represent values we don't know
and must solve for.\pp

The $ac$-method is named as such because we will use the product $a \cdot c$ to help find out what two numbers we
will need for grouping later on. Previously, we always found two numbers whose product was equal to $c$,
since the leading coefficient $a$ was 1 in our expression (so $ac=1c=c$).  Now we will be working with trinomials where $a\neq1$, so we will need to identify two numbers that multiply to $ac$ and add to $b$.  Aside from this adjustment, the process will be the same as before.

\begin{example}~~~Factor the given expression.
  \begin{eqnarray*}
    3 x^2 + 11 x + 6 &  & \tmop{Multiply} \tmop{to} a c \tmop{or~} (3) (6) =
    18, \tmop{~add} \tmop{to} 11\\
    3 x^2 + 9 x + 2 x + 6 &  & \tmop{The} \tmop{numbers} \tmop{are} 9
    \tmop{and} 2, \tmop{~split} \tmop{the} \tmop{linear} \tmop{term}\\
    3 x (x + 3) + 2 (x + 3) &  & \tmop{Factor} \tmop{by} \tmop{grouping}\\
    (x + 3) (3 x + 2) &  & \tmop{Our} \tmop{solution}
  \end{eqnarray*}
\end{example}
~\\  
When $a = 1$, we were able to use a shortcut, using the numbers that split the linear term for our factors. The
previous example illustrates an important point: the shortcut does not work when $a \neq 1$.  Therefore, we must go through all the steps of grouping in order to factor the expression.

\begin{example}~~~Factor the given expression.
  \begin{eqnarray*}
    8 x^2 - 2 x - 15 &  & \tmop{Multiply} \tmop{to} a c \tmop{or~} (8) (- 15) =
    - 120, \tmop{~add} \tmop{to} - 2\\
    8 x^2 - 12 x + 10 x - 15 &  & \tmop{The} \tmop{numbers} \tmop{are} - 12
    \tmop{and} 10, \tmop{~split} \tmop{the} \tmop{linear} \tmop{term}\\
    4 x (2 x - 3) + 5 (2 x - 3) &  & \tmop{Factor} \tmop{by} \tmop{grouping}\\
    (2 x - 3) (4 x + 5) &  & \tmop{Our} \tmop{solution}
  \end{eqnarray*}
\end{example}
%~\\  
\begin{example}~~~Factor the given expression.
  \begin{eqnarray*}
    10 x^2 - 27 x + 5 &  & \tmop{Multiply} \tmop{to} a c \tmop{or~} (10) (5) =
    50, \tmop{~add} \tmop{to} - 27\\
    10 x^2 - 25 x - 2 x + 5 &  & \tmop{The} \tmop{numbers} \tmop{are} - 25
    \tmop{and} - 2, \tmop{~split} \tmop{the} \tmop{linear} \tmop{term}\\
    5 x (2 x - 5) - 1 (2 x - 5) &  & \tmop{Factor} \tmop{by} \tmop{grouping}\\
    (2 x - 5) (5 x - 1) &  & \tmop{Our} \tmop{solution}
  \end{eqnarray*}
\end{example}
~\\
The same process will work for trinomials containing two variables.\\
%~\\
\begin{example}~~~Factor the given expression.
  \begin{eqnarray*}
    4 x^2 - x y - 5 y^2 &  & \tmop{Multiply} \tmop{to} a c \tmop{or~} (4) (- 5)
    = - 20, \tmop{~add} \tmop{to} - 1\\
    4 x^2 + 4 x y - 5 x y - 5 y^2 &  & \tmop{The} \tmop{numbers} \tmop{are} 4
    \tmop{and} - 5, \tmop{~split} \tmop{the} \tmop{middle} \tmop{term}\\
    4 x (x + y) - 5 y (x + y) &  & \tmop{Factor} \tmop{by} \tmop{grouping}\\
    (x + y) (4 x - 5 y) &  & \tmop{Our} \tmop{solution}
  \end{eqnarray*}
\end{example}
~\\
As always, when factoring we will first look for a GCF before using any other
method, including the $ac$-method. Factoring out the GCF first also has the
added bonus of making the coefficients smaller, so other methods become easier.\\

\begin{example}~~~Factor the given expression.
  \begin{eqnarray*}
    18 x^3 + 33 x^2 - 30 x &  & \tmop{GCF~is}3 x, \tmop{~factor} \tmop{this}
    \tmop{out} \tmop{first}\\
    3 x (6 x^2 + 11 x - 10) &  & \tmop{Multiply} \tmop{to} a c \tmop{~or} (6)
    (- 10) = - 60, \tmop{~add} \tmop{to} 11\\
    3 x (6 x^2 + 15 x - 4 x - 10) &  & \tmop{The} \tmop{numbers} \tmop{are} 15
    \tmop{and} - 4, \tmop{~split} \tmop{the} \tmop{linear} \tmop{term}\\
    3 x [3 x (2 x + 5) - 2 (2 x + 5)] &  & \tmop{Factor} \tmop{by}
    \tmop{grouping}\\
    3 x (2 x + 5) (3 x - 2) &  & \tmop{Our} \tmop{solution}
  \end{eqnarray*}
\end{example}


As was the case with trinomials when $a = 1$, not all trinomials can be
factored easily. If there are no combinations that multiply and add correctly, then we
can say the trinomial is not easily factorable.  In such cases, the expression will require a new method of factorization, and may even be shown to be irreducible over the real numbers (the factorization will require complex numbers).  We will encounter such expressions and learn how to properly handle them before the end of this chapter.  We conclude this section with one such example.
~\\
\begin{example}~~~Factor the given expression.
  \begin{eqnarray*}
    3 x^2 + 2 x - 7 &  & \tmop{Multiply} \tmop{to} a c \tmop{or~} (3) (- 7) = -
    21, \tmop{~add} \tmop{to} 2\\
    - 3 (7) \tmop{~and~} - 7 (3) &  & \tmop{Only} \tmop{two} \tmop{ways}
    \tmop{to} \tmop{multiply} \tmop{to} - 21, \tmop{~neither} \tmop{adds} \tmop{to} 2\\
        \tmop{Not~easily~factorable} &  & \tmop{Our} \tmop{solution}
  \end{eqnarray*}
\end{example}
  
It turns out that the previous example \textit{is} factorable over the real numbers, but we will postpone this discovery until later.
