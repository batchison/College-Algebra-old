\documentclass[11pt]{book}
%\oddsidemargin 0in
%\evensidemargin 0in
%\marginparwidth 0in

%\usepackage[top=1in
%,left=1.5in,
%bottom=1in,%right=1in,textheight=8.5in,textwidth=6in
%]{geometry}

\usepackage{amsfonts,amssymb,amsmath,amsthm,fancyhdr,supertabular,longtable,hhline}
\usepackage{colortbl}
\usepackage{docmute}
\usepackage{import, multicol,boxedminipage}
\usepackage{chapterfolder}
\usepackage[metapost,truebbox]{mfpic}
\usepackage[pdflatex]{graphicx}
\usepackage{graphics}
\usepackage{pgf, tikz}
\usepackage[matrix,arrow,curve]{xy}
\usepackage{setspace}
\usepackage{makeidx}
\usepackage{nomencl}
\usepackage[english]{babel}
\usepackage[colorlinks, hyperindex, plainpages=false, linkcolor=blue, urlcolor=blue, pdfpagelabels]{hyperref}
\usepackage[all]{hypcap}
\usepackage{cancel}
\usepackage{sectsty}
\usepackage{textcomp}
\usepackage{array}
\usepackage{vwcol}
\usepackage{cancel}
\usepackage[
type={CC},
modifier={by-nc-sa},
version={4.0},
]{doclicense}
%\usepackage{bbm}

%First appears in II.1.1
\newcommand{\tmdummy}{$\mbox{}$}
\newcommand{\tmmathbf}[1]{\ensuremath{\boldsymbol{#1}}}
\newcommand{\tmop}[1]{\ensuremath{\operatorname{#1}}}
\newcommand{\tmstrong}[1]{\textbf{#1}}
%{\theorembodyfont{\rmfamily}
%First appears in II.1.?
\newcommand{\tmem}[1]{{\em #1\/}}
%\newenvironment{enumeratenumeric}{\begin{enumerate}[1.] }{\end{enumerate}}
%First appears in II.1.6a
\newenvironment{itemizedot}{\begin{itemize} \renewcommand{\labelitemi}{$\bullet$}\renewcommand{\labelitemii}{$\bullet$}\renewcommand{\labelitemiii}{$\bullet$}\renewcommand{\labelitemiv}{$\bullet$}}{\end{itemize}}


\allsectionsfont{\mdseries \scshape}
\definecolor{ResultColor}{gray}{1.0} %SZ set to gray, 0.9
\theoremstyle{definition}  % this prevents the text in definitions, theorems, and corollaries from being italicized
\newtheorem*{defn}{Definition}%[chapter] % this item asterisked for numbering purposes (across all chapters).
%\newtheorem*{thm}{Theorem}%[chapter] % this item asterisked for numbering purposes (across all chapters).
\newtheorem{thm}{Theorem}[chapter] % this item asterisked for numbering purposes (across all chapters).
\newtheorem*{cor}%[thm]
{Corollary} % this item asterisked for numbering purposes (across all chapters).
\newtheorem{eqn}{Equation}[chapter]
\newtheorem{ex}{Example}[chapter] % this is needed for all Stitz Zeager content.
\newtheorem{example}{Example}[chapter]
\newtheorem{fig}{\sc Figure}[chapter]
\setlength{\parindent}{0in}
\setlength{\extrarowheight}{2pt}
\newcommand{\bbm}{\begin{boxedminipage}{4.80in}} %SZ set to 6.41
\newcommand{\ebm}{\end{boxedminipage}}
\newcounter{HW}
\newcounter{HWindent}

\newcommand{\comment}[1]{}
\newcommand{\pp}{\par~\par}

\begin{document}
\subsection*{Introduction}

\subsubsection{Quadratic Equations}
 L=linear, Q=quadratic, N=neither, U=concave up, D=concave down
\begin{multicols}{2}
  1)~$y=x^2 + 9$, QU\\
	3)~$y=-2x+6$, L\\
	5)~$y=-3x^2-3x+9$, QD\\
  7)~$y=x^2+x+1$, QU\\
  9)~$y=-2x-17$, L%\\
%	2)~$y=5-2x+x^2$\\
%	4)~$y=5x+x^2-3x-3x^2$\\
%  6)~$y=3x^2+-x+x-3x^2+6$\\
%  8)~$y=(x-5)(2x+3)-2(x-3)$\\
%  10)~$y=(2x-4)(x-1)-2(x+3)^2+3x^2$
\end{multicols}

%~\\%\newpage

\subsubsection{An Introduction to the Vertex Form} 

%{\tmstrong{Identify the vertex and concavity (concave up or down) of each quadratic.}}
U=concave up, D=concave down%\\

\begin{multicols}{2}
  1)~$(3,4)$, U\\
  3)~$(-3,4)$, U\\
  5)~$(1,-7)$, D\\
  7)~$(0,4)$, U\\
  9)~$(0,4)$, U%\\
%  2)~$y=(x-2)^2+5$\\
%  4)~$y=-2(x-3)^2+4$\\
%  6)~$y=-(x+1)^2$\\
%  8)~$y=-\frac{1}{23}(x-8)^2+5$\\
%  10)~$y=5x^2+23$
\end{multicols}
%~\\
%\newpage

\subsection*{Factoring}

\subsubsection{Greatest Common Factor}

%{\tmstrong{Factor the common factor out of each expression.}}

\begin{multicols}{2}
  1)~ $4(1+2b^2)$\\
  3)~ $5(9x^2-5)$\\
  5)~ $7(8-5p)$\\
  7)~ $7ab(1-5a)$\\
  9)~ $-3a^2b(1-2ab)$\\
  11)~ $-5x^2(1+x+3x^2)$\\
  13)~ $10(2x^4-3x+3)$\\
  15)~ $4(7m^4+10m^3+2)$\\
  17)~ $5(6b^9+ab-3a^2)$\\
  19)~ $-8a^2b(6b-7a-7a^3)$\\
  21)~ $5x^3y^2z(4x^5z+3x^2+7y)$\\
  23)~ $10(5x^2y+y^2+7xz^2)$\\
  25)~ $5q(6pr-p+1)$\\
  27)~ $-3(6n^5-n^3+7n-1)$\\
  29)~ $-10x^{11}(4+2x-5x^2+5x^3)$\\
  31)~ $-4mn(8n^7-m^5-3n^3-4)$%\\
%~\\
%  2)~ $x - 5$\\
%  4)~ $-n - 2 n^2$\\
%  6)~ $50 x - 80 y$\\
%  8)~ $27 x^2 y^5 - 72 x^3 y^2$\\
%  10)~ $8 x^3 y^2 + 4 x^3$\\
%  12)~ $- 32 n^9 + 32 n^6 + 40 n^5$\\
%  14)~ $21 p^6 + 30 p^2 + 27$\\
%  16)~ $- 10 x^4 + 20 x^2 + 12 x$\\
%  18)~ $27 y^7 + 12 y^2 x + 9 y^2$\\
%  20)~ $30 m^6 + 15 m n^2 - 25$\\
%  22)~ $3 p + 12 q - 15 q^2 r^2$\\
%  24)~ $30 y^4 z^3 x^5 + 50 y^4 z^5 - 10 y^4 z^3 x$\\
%  26)~ $28 b + 14 b^2 + 35 b^3 + 7 b^5$\\
%  28)~ $30 a^8 + 6 a^5 + 27 a^3 + 21 a^2$\\
%  30)~ $- 24 x^6 - 4 x^4 + 12 x^3 + 4 x^2$\\
%  32)~ $- 10 y^7 + 6 y^{10} - 4 y^{10} x - 8 y^8 x$
\end{multicols}

%\newpage

\subsubsection{Grouping}

%{\tmstrong{Factor each expression completely.}}

  1)~ $(8r^2-5)(5r-1)=\left(2\sqrt{2}r-\sqrt{5}\right)\left(2\sqrt{2}r+\sqrt{5}\right)(5r-1)$\\
  3)~ $(n^2-3)(3n-2)=\left(n-\sqrt{3}\right)\left(n+\sqrt{3}\right)(3n-2)$\\
  5)~ $(3b^2-7)(5b+7)=\left(\sqrt{3}b-\sqrt{7}\right)\left(\sqrt{3}b+\sqrt{7}\right)(5b+7)$\\
  7)~ $(3x^2+2)(x+5)$\\
  9)~ $(7x^2-4)(5x-4)=\left(\sqrt{7}x-2\right)\left(\sqrt{7}x+2\right)(5x-4)$%\\
\begin{multicols}{2}
  11)~ $(7x+5)(y-7)$\\
  13)~ $(8x+3)(4y+5x)$\\
  15)~ $(8x+1)(2y-7)$\\
  17)~ $(2x+7y^2)(y-4x)$\\
  19)~ $(5x-y)(8y+7)$\\
  21)~ $(4u+3)(8v-5)$\\
  23)~ $(5x+6)(2y+5)$\\
  25)~ $(3u-7)(v-2u)$\\
  27)~ $(8y-3x)(2x+1)$%\\
 % 2)~ $35 x^3 - 10 x^2 - 56 x + 16$\\
 % 4)~ $14 v^3 + 10 v^2 - 7 v - 5$\\
 % 6)~ $6 x^3 - 48 x^2 + 5 x - 40$\\
 % 8)~ $28 p^3 + 21 p^2 + 20 p + 15$\\
 % 10)~ $7 n^3 + 21 n^2 - 5 n - 15$\\
 % 12)~ $42 r^3 - 49 r^2 + 18 r - 21$\\
 % 14)~ $15 a b - 6 a + 5 b^3 - 2 b^2$\\
 % 16)~ $3 m n - 8 m + 15 n - 40$\\
 % 18)~ $5 m n + 2 m - 25 n - 10$\\
 % 20)~ $8 x y + 56 x - y - 7$\\
 % 22)~ $4 u v + 14 u^2 + 12 v + 42 u$\\
 % 24)~ $24 x y + 25 y^2 - 20 x - 30 y^3$\\
 % 26)~ $56 a b + 14 - 49 a - 16 b$
\end{multicols}

%~\\%\newpage

\subsubsection{Trinomials with Leading Coefficient $a=1$}

%{\tmstrong{Factor each expression completely.}}
%\label{ais1}
\begin{multicols}{2}
  1)~ $(p+9)(p+8)$\\
  3)~ $(n-8)(n-1)$\\
  5)~ $(x-10)(x+1)$\\
  7)~ $(b+4)(b+8)$\\
  9)~ $(x+10)(x-7)$\\
  11)~ $(n-5)(n-3)$\\
  13)~ $(p+9)(p+6)$\\
  15)~ $(n-7)(n-8)$\\
  17)~ $(u-5v)(u-3v)$\\
  19)~ $(m+4n)(m-2n)$\\
  21)~ $(x-9y)(x-2y)$\\
  23)~ $(x+4y)(x-3y)$\\
  25)~ $(x+6y)(x-2y)$\\
  27)~ $5(a+2)(a+10)$\\
  29)~ $6(a+8)(a-4)$\\
  31)~ $6(x+2y)(x+y)$\\
  33)~ $6(x+9y)(x+7y)$%\\
%~\\
%  2)~ $x^2_{} + x - 72$\\
%  4)~ $x^2 + x - 30$\\
%  6)~ $x^2 + 13 x + 40$\\
%  8)~ $b^2 - 17 b + 70$\\
%  10)~ $x^2 + 3 x - 18$\\
%  12)~ $a^2 - 6 a - 27$\\
%  14)~ $p^2 + 7 p - 30$\\
%  16)~ $m^2 - 15 m n + 50 n^2$\\
%  18)~ $m^2 - 3 m n - 40 n^2$\\
%  20)~ $x^2 + 10 x y + 16 y^2$\\
%  22)~ $u^2 - 9 u v + 14 v^2$\\
%  24)~ $x^2 + 14 x y + 45 y^2$\\
%  26)~ $4 x^2 + 52 x + 168$\\
%  28)~ $5 n^2 - 45 n + 40$\\
%  30)~ $5 v^2 + 20 v - 25$\\
%  32)~ $5 m^2 + 30 m n - 90 n^2$\\
%  34)~ $6 m^2 - 36 m n - 162 n^2$
\end{multicols}

%\newpage

\subsubsection{Trinomials with Leading Coefficient $a\neq 1$}

%{\tmstrong{Factor each expression completely.}}
%\label{aisnot1}
\begin{multicols}{2}
  1)~ $(7x-6)(x-6)$\\
  3)~ $(7b+1)(b+2)$\\
  5)~ $(5a+7)(a-4)$\\
  7)~ $(2x-1)(x-2)$\\
  9)~ $(2x+5)(x+7)$\\
  11)~ $(2b-3)(b+1)$\\
  13)~ $(5x+3)(x+2)$\\
  15)~ $(3x-5)(x-4)$\\
  17)~ $(3x+2y)(x+5y)$\\
  19)~ $(5x-7y)(x+7y)$\\
  21)~ $3(2x+1)(x-7)$\\
  23)~ $3(7x+6)(x-5)$\\
  25)~ $3(7x-2)(x-4)$\\
  27)~ $(6x+5)(x+4)$\\
  29)~ $(4x-1)(x-4)$\\
  31)~ $(4x+y)(x+2y)$\\
  33)~ $(4m+3n)(m-3n)$\\
  35)~ $(4x+y)(x+3y)$\\
  37)~ $2(3x+5y)(2x+7y)$\\
  39)~ $4(6x-y)(x-2y)$%\\
%~\\
%  2)~ $7 n^2 - 44 n + 12$\\
%  4)~ $7 v^2 - 24 v - 16$\\
%  6)~ $5 n^2 - 7 n - 24$\\
%  8)~ $3 r^2 - 4 r - 4$\\
%  10)~ $7 x^2 + 29 x - 30$\\
%  12)~ $5 k^2 - 26 k + 24$\\
%  14)~ $3 r^2 + 16 r + 21$\\
%  16)~ $3 u^2 + 13 u v - 10 v^2$\\
%  18)~ $7 x^2 - 2 x y - 5 y^2$\\
%  20)~ $5 u^2 + 31 u v - 28 v^2$\\
%  22)~ $10 a^2 - 54 a - 36$\\
%  24)~ $21 n^2 + 45 n - 54$\\
%  26)~ $4 r^2 + r - 3$\\
%  28)~ $6 p^2 + 11 p - 7$\\
%  30)~ $4 r^2 + 3 r - 7$\\
%  32)~ $4 m^2 + 6 m n + 6 n^2$\\
%  34)~ $4 x^2 - 6 x y + 30 y^2$\\
%  36)~ $18 u^2 - 3 u v - 36 v^2$\\
%  38)~ $16 x^2 + 60 x y + 36 y^2$\\
%  40)~ $12 x^2 + 50 x y + 28 y^2$\\
\end{multicols}

%\newpage

\subsection*{Solving by Factoring}

%{\tmstrong{Set each of the following expressions equal to zero and solve for the given variable.}}\pp

%1) - 15): Expressions (1) through (15) on page \pageref{ais1}.\pp

%16) - 30): Expressions (1) through (15) on page \pageref{aisnot1}.\pp

%31) - 40): Expressions (21) through (30) on page \pageref{aisnot1}.
\begin{multicols}{2}
  1)~ $p=-9,-8$\\
	%$(p+9)(p+8)$\\
  3)~ $n=1,8$\\
	%$(n-8)(n-1)$\\
  5)~ $x=-1,10$\\
	%$(x-10)(x+1)$\\
  7)~ $b=-4,-8$\\
	%$(b+4)(b+8)$\\
  9)~ $x=-10,7$\\
	%$(x+10)(x-7)$\\
  11)~ $n=3,5$\\
	%$(n-5)(n-3)$\\
  13)~ $p=-6,-9$\\
	%$(p+9)(p+6)$\\
  15)~ $n=7,8$\\
	%$(n-7)(n-8)$\\
17)~ $n=2/7,6$\\
%$(7n-2)(n-6)$\\
19)~ $v=-4/7,4$\\
%$(7v+4)(v-4)$\\
21)~ $n=-8/5,3$\\
%$(5n+8)(n-3)$\\
23)~ $r=-2/3,2$\\
%$(3r+2)(r-2)$\\
25)~ $x=-5,6/7$\\
%$(7x-6)(x+5)$\\
27)~ $x=6/5,4$\\
%$(5x-6)(x-4)$\\
29)~ $r=-3,-7/3$\\
%$(3r+7)(r+3)$\\
  31)~ $x=-1/2,7$\\
	%$3(2x+1)(x-7)$\\
  33)~ $x=-6/7,5$\\
	%$3(7x+6)(x-5)$\\
  35)~ $x=2/7,4$\\
	%$3(7x-2)(x-4)$\\
  37)~ $x=-4,-5/6$\\
	%$(6x+5)(x+4)$\\
  39)~ $x=1/4,4$%\\
	%$(4x-1)(x-4)$%\\
\end{multicols}
%\newpage

\subsection*{Vertex Form and Graphing}

\subsubsection{The Vertex Form}

%{\tmstrong{Identify whether the quadratic is in vertex form, standard form, or both.  If it is in vertex form, then identify the vertex ($h,k$).}}
V=vertex form, S=standard form, B=both
\begin{multicols}{2}
  1)~V, $(12,5)$\\%$y=(x-12)^2+5 $\\
  3)~B, $(0,8)$\\%$y=x^2+8$\\
  5)~V, $(1,2)$\\%$y=-4(x-1)^2+2$\\
  7)~S\\%$y=x^2+3x+4$\\
  9)~B, $(0,-3)$\\%$y=x^2-3$\\
  11)~V, $(1,0)$%$y=(x-1)^2$\\
%  2)~$y=-3(x-3)^2+5 $\\
%  4)~$y=2(x-4)^2 $\\
%  6)~$y=-5(x-7)^2$\\
%  8)~$y=x^2-1$\\
%  10)~$y=(x-1)^2-3 $\\
%  12)~$y=x^2$
\end{multicols}

%~\vspace{1in}

%{\tmstrong{Each quadratic equation below has been given in standard form.  Rewrite each equation in vertex form.}}

\begin{multicols}{2}
  13)~$y=(x+1)^2-2$\\
	%$y=x^2+2x-1 $\\
  15)~$y=3(x+2)^2-13$\\
	%$y=3x^2+12x-1$\\
  17)~$y=x^2+6$\\
  19)~$y=(x+4)^2-16$\\
	%$y=x^2+8x$\\
  21)~$y=(x+2)^2-6$\\
	%$y=x^2+4x-2$\\
  23)~$y=4\left(x+\frac{5}{4}\right)^2-\frac{25}{4}$%\\
	%$y=4x^2+10x$%\\
 % 14)~$y=-3x^2-12x-5 $\\
 % 16)~$y=x^2+2x$\\
 % 18)~$y=-5x^2-40x$\\
 % 20)~$y=x^2$\\
 % 22)~$y=x^2+16x-2 $
\end{multicols}

\subsubsection{Graphing Quadratics}

%{\tmstrong{Find the vertex and intercepts of the following quadratics. Use
%this information to graph the resulting parabola.}}
\begin{center}
No.) $y$-int, ~~~ vertex, ~~~ $x$-int(s)
\end{center}
\begin{multicols}{2}
	1)~ $(0,-8), ~~~ (1,-9), ~~~ (-2,0), (4,0)$\\
	%1)$y = x^2 - 2 x - 8$\\
  3)~ $(0,10), ~~~~ (3,-8), ~~~ (1,0), ~(5,0)$\\
	%3)~ $y = 2 x^2 - 12 x + 10$\\
  5)~ $(0,-18), ~~~ (3,0), ~~~~ (3,0)$\\
	%5)~ $y = - 2 x^2 + 12 x - 18$\\
  7)~ $(0,-45), ~~~ (4,3), ~~~~ (3,0), (5,0)$\\
	%7)~ $y = - 3 x^2 + 24 x - 45$\\
  9)~ $(0,5), ~~~~~ (2,9), ~~~~~ (-1,0), (5,0)$\\
	%9)~ $y = - x^2 + 4 x_{} + 5$\\
  11)~ $(0,-5), ~~~~ (3,4), ~~~ (1,0), (5,0)$\\
	%11)~ $y = - x^2 + 6 x - 5$\\
  13)~ $(0,-24), ~~~ (4,8), ~~~ (2,0), (6,0)$\\
	%13)~ $y = - 2 x^2 + 16 x - 24$\\
  15)~~ $(0,9), ~~~~ (-2,-3), ~~~~ (-3,0), (-1,0)$\\
	%15)~ $y = 3 x^2 + 12 x + 9$\\
  17)~ $(0,75), ~~~ (4,-5), ~~~ (3,0), (5,0)$\\
	%17)~ $y = 5 x^2 - 40 x + 75$\\
  19)~ $(0,-175), ~~~ (-6,5), ~~ (-7,0), (-5,0)$%\\
	%19)~ $y = - 5 x^2 - 60 x - 175$%\\
%~\\
 % 2)~ $y = x^2 - 2 x - 3$\\
 % 4)~ $y = 2 x^2 - 12 x + 16$\\
 % 6)~ $y = - 2 x^2 + 12 x - 10$\\
 % 8)~ $y = - 3 x^2 + 12 x - 9$\\
 % 10)~ $y = - x^2 + 4 x - 3$\\
 % 12)~ $y = - 2 x^2 + 16 x - 30$\\
 % 14)~ $y = 2 x^2 + 4 x - 6$\\
 % 16)~ $y = 5 x^2 + 30 x + 45$\\
 % 18)~ $y = 5 x^2 + 20 x + 15$\\
 % 20)~ $y = - 5 x^2 + 20 x - 15$\\
\end{multicols}

%\newpage

\subsection*{Square Roots and the Imaginary Number $i$}

\subsubsection{Square Roots}

%{\tmstrong{Simplify each of the following square roots completely.}}%\pp

\begin{multicols}{2}
	1)~ $7~\sqrt[]{5}$\\
  3)~ $6$\\
  5)~ $2~\sqrt[]{3}$\\
  7)~ $6~\sqrt[]{3}$\\
  9)~ $48~\sqrt[]{2}$\\
  11)~ $-112~\sqrt[]{2}$\\
  13)~ $8~\sqrt[]{3 n}$\\
  15)~ $14v$\\
  17)~ $6x~\sqrt[]{7}$\\
  19)~ $-10k^2$\\
  21)~ $- 56x^2$\\
  23)~ $- 30~\sqrt[]{m}$\\
  25)~ $3xy~\sqrt[]{5}$\\
  27)~ $4xy~\sqrt[]{xy}$\\
  29)~ $8x^2y^2~\sqrt[]{5}$\\
  31)~ $24y~\sqrt[]{5x}$\\
  33)~ $35xy~\sqrt[]{5y}$\\
  35)~ $- 12u~\sqrt[]{5 u v}$\\
  37)~ $- 48x^2yz^2~\sqrt[]{5}$\\
  39)~ $8 j^2~\sqrt[]{5hk}$\\
  41)~ $- 12p~\sqrt[]{6m n}$%\\
%  ~\\
%  2)~ $\sqrt[]{125}$\\
%  4)~ $\sqrt[]{196}$\\
%  6)~ $\sqrt[]{72}$\\
%  8)~ $5~\sqrt[]{32}$\\
%  10)~ $7~\sqrt[]{128}$\\
%  12)~ $- 7~\sqrt[]{63}$\\
%  14)~ $\sqrt[]{343 b}$\\
%  16)~ $\sqrt[]{100 n^3}$\\
%  18)~ $\sqrt[]{200 a^3}$\\
%  20)~ $- 4~\sqrt[]{175 p^4}$\\
%  22)~ $- 2~\sqrt[]{128 n}$\\
%  24)~ $8~\sqrt[]{112 p^2}$\\
%  26)~ $\sqrt[]{72 a^3 b^4}$\\
%  28)~ $\sqrt[]{512 a^4 b^2}$\\
%  30)~ $\sqrt[]{512 m^4 n^3}$\\
%  32)~ $8~\sqrt[]{98 m n}$\\
%  34)~ $2~\sqrt[]{72 x^2 y^2}$\\
%  36)~ $- 5~\sqrt[]{72 x^3 y^4}$\\
%  38)~ $6~\sqrt[]{50 a^4 b c^2}$\\
%  40)~ $-~\sqrt[]{32 x y^2 z^3}$\\
%  42)~ $- 8~\sqrt[]{32 m^2 p^4 q}$\\
\end{multicols}

%\newpage

\subsubsection{Intro to Complex Numbers}

%{\tmstrong{Rewrite each of the following complex numbers in the form $a+bi$, where $a$ and $b$ are real numbers and $i=\sqrt{-1}$.}}%\pp

\begin{multicols}{2}
  1)~ $11+4i$\\
  %1)~ $3 - (- 8 + 4 i)$\\
  3)~ $-3+9i$\\
  %3)~ $(7 i) - (3 - 2 i)$\\
  5)~ $-3-13i$\\
  %5)~ $(- 6 i) - (3 + 7 i)$\\
  7)~ $-4-11i$\\
  %7)~ $(3 - 3 i) + (- 7 - 8 i)$\\
  9)~ $-8-2i$\\
  %9)~ $(i) - (2 + 3 i) - 6$\\
  11)~ $48$\\
  %11)~ $(6 i) (- 8 i)$\\
  13)~ $40$\\
  %13)~ $(- 5 i) (8 i)$\\
  15)~ $-49$\\
  %15)~ $(- 7 i)^2$\\
  17)~ $11+60i$\\
  %17)~ $(6 + 5 i)^2$\\
  19)~ $80-10i$\\
  %19)~ $(- 7 - 4 i) (- 8 + 6 i)$\\
  21)~ $27+38i$\\
  %21)~ $(- 4 + 5 i) (2 - 7 i)$\\
  23)~ $44+8i$\\
  %23)~ $(- 8 - 6 i) (- 4 + 2 i)$\\
  25)~ $-3+11i$\\
  %25)~ $(1 + 5 i) (2 + i)$\\
  27)~ $5+9i$\\
  %27)~ $\dfrac{- 9 + 5 i}{i}$\\
  29)~ $-\frac{3}{2}+\frac{5}{3}i$\\
  %29)~ $\dfrac{- 10 - 9 i}{6 i}$\\
  31)~ $-\frac{3}{2}+\frac{3}{4}i$\\
  %31)~ $\dfrac{- 3 - 6 i}{4 i}$\\
  33)~ $1+10i$\\
  %33)~ $\dfrac{10 - i}{- i}$\\
  35)~ $\frac{4}{101}-\frac{40}{101}i$\\
  %35)~ $\dfrac{4 i}{- 10 + i}$\\
  37)~ $\frac{56}{85}+\frac{48}{85}i$\\
  %37)~ $\dfrac{8}{7 - 6 i}$\\
  39)~ $\frac{70}{149}+\frac{49}{149}i$\\
  %39)~ $\dfrac{7}{10 - 7 i}$\\
  41)~ $-\frac{5}{37}-\frac{30}{37}i$\\
  %41)~ $\dfrac{5 i}{- 6 - i}$\\
  43)~ $9i$\\
  %43)~ $\sqrt{- 81}$\\
  45)~ $2~\sqrt{5}$\\
  %45)~ $\sqrt{- 10} \sqrt{- 2}$\\
  47)~ $\frac{1}{2}+\frac{\sqrt{3}}{2}i$\\
  %47~ $\frac{3 + \sqrt{- 27}}{6}$\\
  49)~ $2-i$\\
  %49)~ $\frac{8 - \sqrt{- 16}}{4}$\\
  51)~ $i$\\
  %51)~ $i^{73}$\\
  53)~ $1$\\
  %53)~ $i^{48}$\\
  55)~ $-1$\\
  %55)~ $i^{62}$\\
  57)~ $-1$%\\
  %57)~ $i^{154}$\\
  %2)~ $(3 i) - (7 i)$\\
  %4)~ $5 + (- 6 - 6 i)$\\
  %6)~ $(- 8 i) - (7 i) - (5 - 3 i)$\\
  %8)~ $(- 4 - i) + (1 - 5 i)$\\
  %10)~ $(5 - 4 i) + (8 - 4 i)$\\
  %12)~ $(3 i) (- 8 i)$\\
  %14)~ $(8 i) (- 4 i)$\\
  %16)~ $(- i) (7 i) (4 - 3 i)$\\
  %18)~ $(8 i) (- 2 i) (- 2 - 8 i)$\\
  %20)~ $(3 i) (- 3 i) (4 - 4 i)$\\
  %22)~ $- 8 (4 - 8 i) - 2 (- 2 - 6 i)$\\
  %24)~ $(- 6 i) (3 - 2 i) - (7 i) (4 i)$\\
  %26)~ $(- 2 + i) (3 - 5 i)$\\
  %28)~ $\dfrac{- 3 + 2 i}{- 3 i}$\\
  %30)~ $\dfrac{- 4 + 2 i}{3 i}$\\
  %32)~ $\dfrac{- 5 + 9 i}{9 i}$\\
  %34)~ $\dfrac{10}{5 i}$\\
  %36)~ $\dfrac{9 i}{1 - 5 i}$\\
  %38)~ $\dfrac{4}{4 + 6 i}$\\
  %40)~ $\dfrac{9}{- 8 - 6 i}$\\
  %42)~ $\dfrac{8 i}{6 - 7 i}$\\
%\end{multicols}
%\begin{multicols}{2}
  %44)~ $\sqrt{- 45}$\\
  %46)~ $\sqrt{- 12} \sqrt{- 2}$\\
  %48)~ $\frac{- 4 - \sqrt{- 8}}{- 4}$\\
  %50)~ $\frac{6 + \sqrt{- 32}}{4}$\\
  %52)~ $i^{251}$\\
  %54)~ $i^{68}$\\
  %56)~ $i^{181}$\\
  %58)~ $i^{51}$
\end{multicols}

%\newpage

\subsection*{Solving by Extracting Square Roots}

%{\tmstrong{Find the $x$-intercepts of each quadratic by setting $y=0$ and using the method of extracting square roots.}}

\begin{multicols}{2}
  1)~$12\pm\sqrt{5}$\\
  %1)~$y=(x-12)^2-5 $\\
  3)~$\pm 4$\\
  %3)~$y=x^2-16$\\
  5)~$1\pm\sqrt{5}$\\
  %5)~$y=-4(x-1)^2+20$\\
  7)~$-6\pm\sqrt{2}$\\
  %7)~$y=-4(x+6)^2+8$\\
  9)~$1,7$\\
  %9)~$y=(x-4)^2-9$\\
  11)~$-2\pm 4i$
  %11)$y=(x+2)^2+16$%\\
 % 2)~$y=-3(x-3)^2+30 $\\
 % 4)~$y=2(x-4)^2 -200 $\\
 % 6)~$y=-2(x-7)^2+50$\\
 % 8)~$y=x^2-4$\\
 % 10)$y=(x-1)^2-25 $\\
 % 12)$y=9(x-11)^2-81$
\end{multicols}

%\newpage

\subsection*{Completing the Square}

%{\tmstrong{Find the value that completes the square and then rewrite the given expression as a
%perfect square.}}

\begin{multicols}{2}
  1)~ $x^2 - 30 x~+$~\underline{~~225~~}~$=(x-15)^2$\\
  3)~ $m^2 - 36 m~+$~\underline{~~324~~}~$=(m-18)^2$\\
  5)~ $x^2 - 15 x~+$~\underline{~~$225/4$~~}~$=(x-15/2)^2$\\
  7)~ $y^2 - y~+$~\underline{~~$1/4$~~}~$=(y-1/2)^2$%\\
%  2)~ $a^2 - 24 a +$~\underline{\hspace{.5in}}\\
%  4)~ $x^2 - 34 x +$~\underline{\hspace{.5in}}\\
%  6)~ $r^2 - \frac{1}{9} r +$~\underline{\hspace{.5in}}\\
%  8)~ $p^2 - 17 p +$~\underline{\hspace{.5in}}
\end{multicols}

%{\tmstrong{Solve each equation by completing the square.}}

\begin{multicols}{2}
   %9)~ $$\\
  %9)~ $x^2 - 16 x + 55 = 0$\\
   11)~ $v=4\pm\sqrt{29}i$\\
	%11)~ $v^2 - 8 v + 45 = 0$\\
   %13)~ $$\\
	%13)~ $6 x^2 + 12 x + 63 = 0$\\
   15)~ $k=1\pm\frac{\sqrt{215}}{5}i$\\
	%15)~ $5 k^2 - 10 k + 48 = 0$\\
   17)~ $x=-\frac{5}{2}\pm\sqrt{86}$\\
	%17)~ $x^2 + 10 x - 57 = 4$\\
   %19)~ $$\\
	%19)~ $n^2 - 16 n + 67 = 4$\\
   21)~ $x=-1\pm\sqrt{21}i$\\
	%21)~ $2 x^2 + 4 x + 38 = - 6$\\
   %23)~ $$\\
	%23)~ $8 b^2 + 16 b - 37 = 5$\\
   25)~ $x=-5\pm 2i$\\
	%25)~ $x^2 = - 10 x - 29$\\
   27)~ $n=3,~7$\\
	%27)~ $n^2 = - 21 + 10 n$\\
   %29)~ $$\\
	%29)~ $3 k^2 + 9 = 6 k$\\
   31)~ $x=2\pm\sqrt{29}i$\\
	%31)~ $2 x^2 + 63 = 8 x$\\
   %33)~ $$\\
	%33)~ $p^2 - 8 p = - 55$\\
   35)~ $n=1,~7$\\
	%35)~ $7 n^2 - n + 7 = 7 n + 6 n^2$\\
   37)~ $b=-1\pm\frac{\sqrt{258}}{6}i$\\
	%37)~ $13 b^2 + 15 b + 44 = - 5 + 7 b^2 + 3 b$\\
   %39)~ $$\\
	%39)~ $5 x^2 + 5 x = - 31 - 5 x$\\
   41)~ $v=-\frac{5}{2}\pm\frac{\sqrt{87}}{2}i$\\
	%41)~ $v^2 + 5 v + 28 = 0$\\
   %43)~ $$\\
	%43)~ $7 x^2 - 6 x + 40 = 0$\\
   45)~ $k=\frac{7}{2}\pm\frac{\sqrt{137}}{2}i$\\
	%45)~ $k^2 - 7 k + 50 = 3$\\
   47)~ $x=-4,~\frac{12}{5}$\\
	%47)~ $5 x^2 + 8 x - 40 = 8$\\
   %49)~ $$\\
	%49)~ $m^2 = - 15 + 9 m$\\
   51)~ $r=-\frac{5}{8}\pm\frac{\sqrt{415}}{8}i$\\
	%51)~ $8 r^2 + 10 r = - 55$\\
   %53)~ $$\\
	%53)~ $5 n^2 - 8 n + 60 = - 3 n + 6 + 4 n^2$\\
   55)~ $x=-\frac{5}{2},~1$\\
	%55)~ $- 2 x^2 + 3 x - 5 = - 4 x^2$\\
  % )~ $$\\
	%~\\
 % 10)~ $n^2 - 8 n - 12 = 0$\\
 % 12)~ $b^2 + 2 b + 43 = 0$\\
 % 14)~ $3 x^2 - 6 x + 47 = 0$\\
 % 16)~ $8 a^2 + 16 a - 1 = 0$\\
 % 18)~ $p^2 - 16 p - 52 = 0$\\
 % 20)~ $m^2 - 8 m - 3 = 6$\\
 % 22)~ $6 r^2 + 12 r - 24 = - 6$\\
 % 24)~ $6 n^2 - 12 n - 14 = 4$\\
 % 26)~ $v^2 = 14 v + 36$\\
 % 28)~ $a^2 - 56 = - 10 a$\\
 % 30)~ $5 x^2 = - 26 + 10 x$\\
 % 32)~ $5 n^2 = - 10 n + 15$\\
 % 34)~ $x^2 + 8 x + 15 = 8$\\
 % 36)~ $n^2 + 4 n = 12$\\
 % 38)~ $- 3 r^2 + 12 r + 49 = - 6 r^2$\\
 % 40)~ $8 n^2 + 16 n = 64$\\
 % 42)~ $b^2 + 7 b - 33 = 0$\\
 % 44)~ $4 x^2 + 4 x + 25 = 0$\\
 % 46)~ $a^2 - 5 a + 25 = 3$\\
 % 48)~ $2 p^2 - p + 56 = - 8$\\
 % 50)~ $n^2 - n = - 41$\\
 % 52)~ $3 x^2 - 11 x = - 18$\\
 % 54)~ $4 b^2 - 15 b + 56 = 3 b^2$\\
 % 56)~ $10 v^2 - 15 v = 27 + 4 v^2 - 6 v$\\
\end{multicols}

%\newpage

\subsection*{The Quadratic Formula and the Discriminant}

%{\tmstrong{Use the discriminant in order to determine the number of real roots for each equation.  If an equation is shown to have one (or two) real root(s), set $y=0$ and use the quadratic formula to find them.}}
%\label{QF}
\begin{multicols}{2}
  %1)~$y=x^2+2x-1$\\
  1)~Two real roots,~$x=-1\pm\sqrt{2}$\\
  %3)~$y=3x^2+12x-1$\\
  3)~Two real roots,~$x=-2\pm\frac{\sqrt{39}}{3}$\\
  %5)~$y=x^2+6$\\
  5)~No real roots\\
  %7)~$y=x^2+8x$\\
  7)~Two real roots,~$x=0,~-8$\\
  %9)~$y=x^2+4x-2$\\
  9)~Two real roots,~$x=-2\pm\sqrt{6}$\\
  %11)~$y=4x^2+10x$\\
  11)~Two real roots,~$x=0,~-\frac{5}{2}$\\
%	2)~$y=-3x^2-12x-5 $\\
%  4)~$y=x^2+2x$\\
%  6)~$y=-5x^2-40x$\\
%  8)~$y=x^2$\\
%  10)~$y=x^2+16x-2 $
\end{multicols}

%{\tmstrong{Solve each equation using the quadratic formula.}}

\begin{multicols}{2}
 % 12)~ $4 a^2 + 6 = 0$\\
 % 14)~ $2 x^2 - 8 x - 2 = 0$\\
 % 16)~ $2 m^2 - 3 = 0$\\
 % 18)~ $3 r^2 - 2 r - 1 = 0$\\
 % 20)~ $4 n^2 - 36 = 0$\\
 % 22)~ $v^2 - 4 v - 5 = - 8$\\
 % 24)~ $2 a^2 + 3 a + 14 = 6$\\
 % 26)~ $3 k^2 + 3 k - 4 = 7$\\
 % 28)~ $7 x^2 + 3 x - 16 = - 2$\\
 % 30)~ $2 p^2 + 6 p - 16 = 4$\\
 % 32)~ $3 n^2 + 3 n = - 3$\\
 % 34)~ $2 x^2 = - 7 x + 49$\\
 % 36)~ $5 x^2 = 7 x + 7$\\
 % 38)~ $8 n^2 = - 3 n - 8$\\
 % 40)~ $2 x^2 + 5 x = - 3$\\
 % 42)~ $4 a^2 - 64 = 0$\\
 % 44)~ $4 p^2 + 5 p - 36 = 3 p^2$\\
 % 46)~ $- 5 n^2 - 3 n - 52 = 2 - 7 n^2$\\
 % 48)~ $7 r^2 - 12 = - 3 r$\\
 % 50)~ $2 n^2 - 9 = 4$\\
%  13)~ $3 k^2 + 2 = 0$\\
  13)~ $k=\pm\frac{\sqrt{6}}{3}i$\\
%  15)~ $6 n^2 - 1 = 0$\\
  15)~ $n=\pm\frac{\sqrt{6}}{6}$\\
%  17)~ $5 p^2 + 2 p + 6 = 0$\\
  17)~ $p=-\frac{1}{5}\pm\frac{\sqrt{29}}{5}i$\\
%  19)~ $2 x^2 - 2 x - 15 = 0$\\
%  19)~ $$\\
%  21)~ $3 b^2 + 6 = 0$\\
  21)~ $b=\pm\sqrt{2}i$\\
%  23)~ $2 x^2 + 4 x + 12 = 8$\\
%  23)~ $$\\
%  25)~ $6 n^2 - 3 n + 3 = - 4$\\
  25)~ $n=\frac{1}{4}\pm\frac{\sqrt{159}}{12}i$\\
%  27)~ $4 x^2 - 14 = - 2$\\
  27)~ $x=\pm\sqrt{3}$\\
%  29)~ $4 n^2 + 5 n = 7$\\
%  29)~ $$\\
%  31)~ $m^2 + 4 m - 48 = - 3$\\
  31)~ $m=5,~-9$\\
%  33)~ $3 b^2 - 3 = 8 b$\\
%  33)~ $$\\
%  35)~ $3 r^2 + 4 = - 6 r$\\
  35)~ $r=-1\pm\frac{\sqrt{3}}{3}i$\\
%  37)~ $6 a^2 = - 5 a + 13$\\
  37)~ $a=-\frac{5}{12}\pm\frac{\sqrt{337}}{12}$\\
%  39)~ $6 v^2 = 4 + 6 v$\\
%  39)~ $$\\
%  41)~ $x^2 = 8$\\
  41)~ $x=\pm 2\sqrt{2}$\\
%  43)~ $2 k^2 + 6 k - 16 = 2 k$\\
%  43)~ $$\\
%  45)~ $12 x^2 + x + 7 = 5 x^2 + 5 x$\\
  45)~ $x=\frac{2}{7}\pm\frac{3\sqrt{5}}{7}i$\\
%  47)~ $7 m^2 - 6 m + 6 = - m$\\
  47)~ $m=\frac{5}{14}\pm\frac{\sqrt{143}}{14}i$\\
%  49)~ $3 x^2 - 3 = x^2$\\
  49)~ $x=\pm\frac{\sqrt{6}}{2}$\\
%  51)~ $6 b^2 = b^2 + 7 - b$
%  51)~ $$\\
\end{multicols}

%\newpage

\comment{\subsection*{Applications}

\newpage

\subsection*{Quadratic Inequalities and Sign Diagrams}

{\tmstrong{Construct a sign diagram for each of the following expressions/equations.  Then using interval notation, describe the set of values for which the given expression is greater than or equal to zero.}}\pp

1) - 5): Expressions (1) through (5) on page \pageref{ais1}.\pp

6) - 10): Expressions (1) through (5) on page \pageref{aisnot1}.\pp

11) - 15): Expressions (21) through (25) on page \pageref{aisnot1}.\pp

16) - 20): Equations (1) through (5) on page \pageref{QF}.\pp}
\end{document}