\subsection{Distance, Rate and Time}\pp

 {\tmstrong{Objective: Solve distance problems by creating and solving a linear
equation. }}\pp

 An application of linear equations can be found in distance problems. When
solving distance problems we will use the relationship $r t = d$ or rate
(speed) times time equals distance. For example, if a person were to travel 30
mph (miles per hour) for 4 hours. To find the total distance we would multiply rate times time
or $(30) (4) = 120$. This person would travel a distance of 120 miles. The problems
we will be solving here will require a few more steps than described above. So to
keep the information in the problem organized we will use a table. The basic structure of the table is shown below.

\[%\begin{table}[h]
  \begin{array}{|c|c|c|c|}
    \hline
    & \mbox{Rate} & \mbox{Time} & \mbox{Distance}\\
    \hline
    \mbox{Person~1} &  &  & \\
    \hline
    \mbox{Person~2} &  &  & \\
    \hline
  \end{array} \]
	\begin{center}
	\small{Structure of Distance Problem}
	%\end{tabular}
  \end{center}

 The third column, `distance', will always be filled in by multiplying the `rate'
and `time' columns together. If we are given a total distance of both persons or
trips we will put this information below the distance column. We will now use
this table to set up and solve the following example.

\begin{example}\label{Lin86}~~~ Two joggers start from opposite ends of an 8 mile course running towards
  each other. One jogger is running at a rate of 4 mph, and the other is
  running at a rate of 6 mph. After how long will the joggers meet?
  \begin{eqnarray*}
    \begin{array}{|c|c|c|c|}
      \hline
      & \tmop{Rate} & \tmop{Time} & \tmop{Distance}\\
      \hline
      \tmop{Jogger} 1 &  &  & \\
      \hline
      \tmop{Jogger} 2 &  &  & \\
      \hline
    \end{array} &  & \begin{array}{p{0.425\textwidth}}\tmop{The} \tmop{basic} \tmop{table} \tmop{for} \tmop{the} \tmop{joggers},\\ \tmop{Jogger~1} \tmop{and} \tmop{Jogger~2}.\end{array}%\\
%    &  & \\
\end{eqnarray*}
\begin{eqnarray*}
   \begin{array}{|c|c|c|c|}
      \hline
      & \tmop{Rate} & \tmop{Time} & \tmop{Distance}\\
      \hline
      \tmop{Jogger} 1 & \tmmathbf{4} &  & \\
      \hline
      \tmop{Jogger} 2 & \tmmathbf{6} &  & \\
      \hline
    \end{array} &  & \begin{array}{p{0.425\textwidth}}
      \tmop{We} \tmop{are} \tmop{given} \tmop{the} \tmop{rates} \tmop{for}
      \tmop{each} \tmop{jogger}.\\
      \tmop{These} \tmop{are} \tmop{added} \tmop{to} \tmop{the} \tmop{table}.
    \end{array}\\
    &  & \\
    \begin{array}{|c|c|c|c|}
      \hline
      & \tmop{Rate} & \tmop{Time} & \tmop{Distance}\\
      \hline
      \tmop{Jogger} 1 & 4 & \tmmathbf{t} & \\
      \hline
      \tmop{Jogger} 2 & 6 & \tmmathbf{t} & \\
      \hline
    \end{array} &  & \begin{array}{p{0.425\textwidth}}
      \tmop{We} \tmop{only} \tmop{know} \tmop{they} \tmop{both} \tmop{start}\tmop{~and} \tmop{end} \tmop{at} \tmop{the} \tmop{same} \tmop{time}. \tmop{We} \tmop{use} \tmop{the} \tmop{variable} $t$ \tmop{for} \tmop{both} \tmop{times}.
    \end{array}\\
    &  & \\
    \begin{array}{|c|c|c|c|}
      \hline
      & \tmop{Rate} & \tmop{Time} & \tmop{Distance}\\
      \hline
      \tmop{Jogger} 1 & 4 & t & \tmmathbf{4 t}\\
      \hline
      \tmop{Jogger} 2 & 6 & t & \tmmathbf{6 t}\\
      \hline
    \end{array} &  & \begin{array}{p{0.425\textwidth}}
      \tmop{The} \tmop{distance} \tmop{column} \tmop{is} \tmop{filled}
      \tmop{in} \tmop{by} \tmop{multiplying} \tmop{rate} \tmop{by} \tmop{time}.
    \end{array}\\
    \tmmathbf{8}~~~~~~  &  & \begin{array}{p{0.425\textwidth}}
      \tmop{We} \tmop{have} \tmmathbf{\tmop{total} \tmop{distance}},\\
			8 \tmop{miles}, \tmop{under} \tmop{distance}.
    \end{array}\\
    4 t + 6 t = 8~ &  & \begin{array}{p{0.425\textwidth}}
      \tmop{Add~the~entries~in~the~distance} \tmop{column~and~set~equal~to~total~distance}.
    \end{array}\\
    10 t = 8~ &  & \begin{array}{p{0.425\textwidth}}
      \tmop{Combine} \tmop{like} \tmop{terms}, $4t + 6t$.
    \end{array}\\
    \tmmathbf{\overline{10} ~~~ \overline{10}} &  & \tmop{~Divide} \tmop{both}
    \tmop{sides} \tmop{by} 10.\\
    t = \frac{4}{5} &  & \tmop{~Our} \tmop{solution} \tmop{for} t
		\end{eqnarray*}
\begin{center}
The joggers will meet at $\frac{4}{5}$ hour, or $48$ minutes.
\end{center}
\end{example}

 As the example illustrates, once the table is filled in, the resulting equation is very easy to solve. This same process can be seen in the following example.%\pp

\begin{example}\label{Lin87}~~~ Bob and Fred start from the same point and walk in opposite directions. Bob walks 2 miles per hour faster than Fred. After 3 hours they are 30 miles
  apart. How fast did each walk?
  \begin{eqnarray*}
    \begin{array}{|c|c|c|c|}
      \hline
      & \tmop{Rate} & \tmop{Time} & \tmop{Distance}\\
      \hline
      \tmop{Bob} & \mbox{\hspace{.4in}} & \tmmathbf{3} & \\
      \hline
      \tmop{Fred} &  & \tmmathbf{3} & \\
      \hline
    \end{array} &  & \begin{array}{p{0.425\textwidth}}
      \tmop{The} \tmop{basic} \tmop{table} \tmop{with} \tmop{given}
      \tmop{times} \tmop{filled} \tmop{in}.\\
			\tmop{Both} \tmop{traveled} 3 \tmop{hours}.
    \end{array}%\\
  %  &  & \\
  \end{eqnarray*}
  \begin{eqnarray*}
    \begin{array}{|c|c|c|c|}
      \hline
      & \tmop{Rate} & \tmop{Time} & \tmop{Distance}\\
      \hline
      \tmop{Bob} & \tmmathbf{r + 2} & 3 & \\
      \hline
      \tmop{Fred} & \tmmathbf{r} & 3 & \\
      \hline
    \end{array} &  & \begin{array}{p{0.425\textwidth}}
      \tmop{Bob} \tmop{walks} 2 \tmop{mph} \tmop{faster} \tmop{than} \tmop{Fred}. \tmop{We} \tmop{know} \tmop{nothing} \tmop{about} \tmop{Fred}, \tmop{so} \tmop{use} $r$ \tmop{for} \tmop{his} \tmop{rate}.\\
      \tmop{Bob`s~rate} \tmop{is} $r~+$ 2, \tmop{showing} 2 \tmop{mph} \tmop{faster}.
    \end{array}\\
    &  & \\
    \begin{array}{|c|c|c|c|}
      \hline
      & \tmop{Rate} & \tmop{Time} & \tmop{Distance}\\
      \hline
      \tmop{Bob} & ~r + 2~ & 3 & \tmmathbf{3 r + 6}\\
      \hline
      \tmop{Fred} & r & 3 & \tmmathbf{3 r}\\
      \hline
    \end{array} &  & \begin{array}{p{0.425\textwidth}}
      \tmop{Distance} \tmop{column} \tmop{is} \tmop{filled} \tmop{in}
      \tmop{by} \tmop{multiplying} \tmop{rate} \tmop{by} \tmop{time}.\\
			\tmop{Be} \tmop{sure} \tmop{to} \tmop{distribute} \tmop{the}\\3 ($r~+$ 2) \tmop{for} \tmop{Bob}.
    \end{array}\\
    \tmmathbf{30}~~~~~  &  &  \tmmathbf{\tmop{Total} \tmop{distance}} \tmop{is} \tmop{put} \tmop{under} \tmop{distance}.\\
    3 r + 6 + 3 r = 30~ &  & \begin{array}{p{0.425\textwidth}}
      \tmop{Add~the~entries~in~the~distance} \tmop{column~and~set~equal~to~total~distance}.
    \end{array}\\
    6 r + 6 = 30~ &  & \tmop{~Combine} \tmop{like} \tmop{terms}, 3 r + 3 r.\\
    \tmmathbf{\underline{- 6 ~~- 6}} &  & \tmop{~Subtract} 6 \tmop{from}
    \tmop{both} \tmop{sides}.\\
    6 r = 24~ &  & \tmop{~The} \tmop{variable} \tmop{is} \tmop{multiplied}
    \tmop{by} 6.\\
    \tmmathbf{\overline{6} ~~~~~ \overline{6}}~~ &  & \tmop{~Divide} \tmop{both}
    \tmop{sides} \tmop{by} 6.\\
    r = 4~ &  & \tmop{~Our} \tmop{solution} \tmop{for} r\\
		& & \\
    \begin{array}{|c|c|}
      \hline
      & \tmop{Rate}\\
      \hline
      \tmop{Bob} & 4 + 2 = 6\\
      \hline
      \tmop{Fred} & 4\\
      \hline
    \end{array} &  & \begin{array}{p{0.425\textwidth}}
      \tmop{To} \tmop{answer} \tmop{the} \tmop{question} \tmop{completely} \tmop{we} \tmop{plug} 4 \tmop{in} \tmop{for} $r$ \tmop{in} \tmop{the} \tmop{table}.\\
			\tmop{Bob} \tmop{traveled} 6 \tmop{miles} \tmop{per} \tmop{hour} \tmop{and} \tmop{Fred} \tmop{traveled} 4 \tmop{mph}.
    \end{array}
  \end{eqnarray*}
\end{example}

 Some problems will require us to do a bit of work before we can just fill in
the cells. One example of this is if we are given a total time, rather than
the individual times like we had in the previous example. If we are given
total time we will write this above the time column, use $t$ for the first
person's time, and make a subtraction problem, (Total) $-~t$, for the
second person's time. This is shown in the next example.

\begin{example}\label{Lin88}~~~ Two campers left their campsite by canoe and paddled downstream at an average speed of 12 mph. They turned around and paddled back upstream at an average rate of 4 mph. The total trip took 1 hour. After how much time did the campers turn around downstream?

\pagebreak

  \begin{eqnarray*}
    \begin{array}{|c|c|c|c|}
      \hline
      & \tmop{Rate} & \tmop{Time} & \tmop{Distance}\\
      \hline
      \tmop{Down} & \tmmathbf{12} &  & \\
      \hline
      \tmop{Up} & \tmmathbf{4} &  & \\
      \hline
    \end{array} &  & \begin{array}{p{0.425\textwidth}}
      \tmop{Basic} \tmop{table} \tmop{for} \tmop{down} \tmop{and}
      \tmop{upstream}.\\
      \tmop{Given} \tmop{rates} \tmop{are} \tmop{filled} \tmop{in}.
    \end{array}\\
    &  & \\
    \tmmathbf{1}~~~~~~~~~~~~~~~~~~  &  & \tmop{Total} \tmop{time} \tmop{is} \tmop{put} \tmop{above} \tmop{`time'} \tmop{column}.\\
    \begin{array}{|c|c|c|c|}
      \hline
      & \tmop{Rate} & \tmop{Time} & \tmop{Distance}\\
      \hline
      \tmop{Down} & 12 & \tmmathbf{t} & \\
      \hline
      \tmop{Up} & 4 & \tmmathbf{1 - t} & \\
      \hline
    \end{array} &  & \begin{array}{p{0.425\textwidth}}
      \tmop{As} \tmop{we~only~} \tmop{have} \tmop{the} \tmop{total} \tmop{time}, \tmop{in} \tmop{the} \tmop{time} \tmop{down} \tmop{we} \tmop{have} $t$, \tmop{the} \tmop{time} \tmop{up} \tmop{becomes} \tmop{the}
      \tmop{difference}, \tmop{(total)} $-~t$.
    \end{array}\\
    &  & \\
    \begin{array}{|c|c|c|c|}
      \hline
      & \tmop{Rate} & \tmop{Time} & \tmop{Distance}\\
      \hline
      \tmop{Down} & 12 & t & \tmmathbf{12 t}\\
      \hline
      \tmop{Up} & 4 & 1 - t & \tmmathbf{4 - 4 t}\\
      \hline
    \end{array} &  & \begin{array}{p{0.425\textwidth}}
      \tmop{Distance} \tmop{column} \tmop{is} \tmop{found} \tmop{by}
      \tmop{multiplying} \tmop{rate} \tmop{by} \tmop{time}.\\
			\tmop{Be} \tmop{sure} \tmop{to} \tmop{distribute} 4 (1 $-~t$) \tmop{for} \tmop{upstream}. %\tmop{As} \tmop{they} \tmop{cover} \tmop{the}
%      \tmmathbf{\tmop{same}}  \tmmathbf{\tmop{distance}},\\
%      = \tmop{is} \tmop{put} \tmop{after} \tmop{the} \tmop{down}
%      \tmop{distance}
    \end{array}\\
    12 t = 4 - 4 t~ &  & \begin{array}{p{0.425\textwidth}}
		\tmop{Since} \tmop{they} \tmop{cover} \tmop{the} \tmop{same} \tmop{distance}, \tmop{set} \tmop{values} \tmop{in} \tmop{last} \tmop{column} \tmop{equal} \tmop{to} \tmop{each} \tmop{other}.
		\end{array}\\
    \tmmathbf{\underline{+ 4 t ~~~~+ 4 t}}~ &  & \begin{array}{p{0.425\textwidth}}\tmop{Add} $4t$ \tmop{to} \tmop{both} \tmop{sides} \tmop{so} \tmop{variable} \tmop{is} \tmop{only} \tmop{on} \tmop{one} \tmop{side}.
		\end{array}\\
    16 t = 4~ &  & \tmop{~Variable} \tmop{is} \tmop{multiplied} \tmop{by} 16.\\
    \tmmathbf{\overline{16} ~~~~ \overline{16}} &  & \tmop{~Divide} \tmop{both}
    \tmop{sides} \tmop{by} 16.\\
    t = \frac{1}{4}~ &  & \tmop{~Our} \tmop{solution}
  \end{eqnarray*}
\begin{center}
 The campers turned around after $\frac{1}{4}$ hour, or 15 minutes.
\end{center}
\end{example}

Another type of a distance problem is that where one person catches up with another. Here a slower person has a head start and the faster person is trying to catch up with him or her.  We want to know how long it
will take for this to happen. Our strategy for this problem will be to
use $t$ for the faster person's time, and add the amount of time for the head start to obtain the slower person's time. This is shown in the next example.

\pagebreak

\begin{example}\label{Lin89}~~~ Mike leaves his house traveling 2 miles per hour. Joy leaves 6 hours later to catch up with him, traveling 8 miles per hour. How long will it take her to catch up with him?
  \begin{eqnarray*}
    \begin{array}{|c|c|c|c|}
      \hline
      & \tmop{Rate} & \tmop{Time} & \tmop{Distance}\\
      \hline
      \tmop{Mike} & \tmmathbf{2} &  & \\
      \hline
      \tmop{Joy} & \tmmathbf{8} &  & \\
      \hline
    \end{array} &  & \begin{array}{p{0.425\textwidth}}
      \tmop{Basic} \tmop{table} \tmop{for} \tmop{Mike} \tmop{and} \tmop{Joy}.\\
      \tmop{The} \tmop{given} \tmop{rates} \tmop{are} \tmop{filled} \tmop{in}.
    \end{array}\\
    &  & \\
    \begin{array}{|c|c|c|c|}
      \hline
      & \tmop{Rate} & \tmop{Time} & \tmop{Distance}\\
      \hline
      \tmop{Mike} & 2 & \tmmathbf{t + 6} & \\
      \hline
      \tmop{Joy} & 8 & \tmmathbf{t} & \\
      \hline
    \end{array} &  & \begin{array}{p{0.425\textwidth}}
      \tmop{We~use} $t$ \tmop{to~represent} \tmop{the} \tmop{faster} \tmop{person`s} \tmop{time}.\\
      \tmop{Mike`s} \tmop{time} \tmop{is} $t~+$ 6, \tmop{showing} \tmop{his} 6
      \tmop{hour} \tmop{head} \tmop{start}.
    \end{array}\\
    &  & \\
    \begin{array}{|c|c|c|c|}
      \hline
      & \tmop{Rate} & \tmop{Time} & \tmop{Distance}\\
      \hline
      \tmop{Mike} & 2 & t + 6 & \tmmathbf{2 t + 12}\\
      \hline
      \tmop{Joy} & 8 & t & \tmmathbf{8 t}\\
      \hline
    \end{array} &  & \begin{array}{p{0.425\textwidth}}
      \tmop{Distance} \tmop{column} \tmop{is} \tmop{found} \tmop{by}
      \tmop{multiplying} \tmop{the} \tmop{rate} \tmop{by} \tmop{time}.\\
			\tmop{Be} \tmop{sure} \tmop{to} \tmop{distribute} \tmop{the}\\
			2 ($t~+$ 6) \tmop{for} \tmop{Mike}.\\
%      \tmop{As} \tmop{they} \tmop{cover} \tmop{the} \tmmathbf{\tmop{same}
 %     \tmop{distance}}, = \tmop{is} \tmop{put} \tmop{after}\\
  %\tmop{Mike}' s \tmop{distance}
    \end{array}\\
    2 t + 12 = 8 t~~~ &  &  \begin{array}{p{0.425\textwidth}}
		\tmop{Since} \tmop{they} \tmop{cover} \tmop{the} \tmop{same} \tmop{distance}, \tmop{set} \tmop{values} \tmop{in} \tmop{last} \tmop{column} \tmop{equal} \tmop{to} \tmop{each} \tmop{other}.
		\end{array}\\
    \tmmathbf{\underline{- 2 t ~~~~~~~~- 2 t}} &  & \tmop{~Subtract} 2 t \tmop{from}
    \tmop{both} \tmop{sides}.\\
    12 = 6 t~~~ &  & \tmop{~The} \tmop{variable} \tmop{is} \tmop{multiplied}
    \tmop{by} 6.\\
    \tmmathbf{\overline{6} ~~~~~ \overline{6}}~~~ &  & \tmop{~Divide} \tmop{both}
    \tmop{sides} \tmop{by} 6.\\
    2 = t~~~ &  & \tmop{~Our} \tmop{solution} \tmop{for} t
  \end{eqnarray*}
\begin{center}
Joy catches Mike after 2 hours.
\end{center}
\end{example}


 {\tmstrong{World View Note:}} The 10,000 (or 10k) race is the longest standard track
event. 10,000 meters is approximately 6.2 miles. The current (at the time of
printing) world record for this race is held by Ethiopian Kenenisa Bekele with
a time of 26 minutes, 17.53 seconds. That is a rate of 12.7 miles per hour!\pp

 As these example have shown, using a table can help keep all the given
information organized, and consequently help find the equation we
must solve. One final example illustrates this.

\pagebreak

\begin{example}\label{Lin90}~~~ On a 130 mile trip a car traveled at an average speed of 55 mph and then
  reduced its speed to 40 mph for the remainder of the trip. The trip took 2.5
  hours. How long did the car travel at 40 mph?
  \begin{eqnarray*}
    \begin{array}{|c|c|c|c|}
      \hline
      & \tmop{Rate} & \tmop{Time} & \tmop{Distance}\\
      \hline
      \tmop{Fast} & \tmmathbf{55} & \mbox{\hspace{.5in}} & \mbox{\hspace{.7in}}\\
      \hline
      \tmop{Slow} & \tmmathbf{40} &  & \\
      \hline
    \end{array} &  & \begin{array}{p{0.425\textwidth}}
      \tmop{Basic} \tmop{table} \tmop{for} \tmop{fast} \tmop{and} \tmop{slow}
      \tmop{speeds}.\\
      \tmop{The} \tmop{given} \tmop{rates} \tmop{are} \tmop{filled} \tmop{in}.
    \end{array}\\
    &  & \\
    &  & \begin{array}{p{0.425\textwidth}}
		\tmmathbf{\tmop{Total} \tmop{time}} \tmop{is} \tmop{put}
    \tmop{above} \tmop{the}\\ \tmop{`Time'} \tmop{column}.
		\end{array}\\
    \tmmathbf{2.5}~~~~~~~~~~~~~~~~~~~~~  & &\\
		\begin{array}{|c|c|c|c|}
      \hline
      & \tmop{Rate} & \tmop{Time} & \tmop{Distance}\\
      \hline
      \tmop{Fast} & 55 & \tmmathbf{t} & \mbox{\hspace{.7in}}\\
      \hline
      \tmop{Slow} & 40 & \tmmathbf{2.5 - t} & \\
      \hline
    \end{array} &  & \begin{array}{p{0.425\textwidth}}
      \tmop{Since} \tmop{total} \tmop{time} \tmop{is} \tmop{given}, \tmop{we} \tmop{assign} $t$ \tmop{for} \tmop{the} \tmop{time} \tmop{spent} \tmop{traveling} 55mph.\\
      \tmop{The} \tmop{other} \tmop{time} \tmop{is} \tmop{the}
      \tmop{difference} 2.5 $-~t$.
    \end{array}\\
    &  & \\
    2.5~~~~~~~~~~~~~~~~~~~~~ &  & \\
    \begin{array}{|c|c|c|c|}
      \hline
      & \tmop{Rate} & \tmop{Time} & \tmop{Distance}\\
      \hline
      \tmop{Fast} & 55 & t & \tmmathbf{55 t}\\
      \hline
      \tmop{Slow} & 40 & ~2.5 - t~ & \tmmathbf{100 - 40 t}\\
      \hline
    \end{array} &  & \begin{array}{p{0.425\textwidth}}
      \tmop{Distance} \tmop{column} \tmop{is} \tmop{found} \tmop{by}
      \tmop{multiplying} \tmop{rate} \tmop{by} \tmop{time}. \tmop{Be} \tmop{sure} \tmop{to} \tmop{distribute} 40 (2.5 $-~t$).
    \end{array}\\
    \tmmathbf{130}~~~~~  &  &  \begin{array}{p{0.425\textwidth}}
		\tmmathbf{\tmop{Total} \tmop{distance}} \tmop{is} \tmop{put} \tmop{under} \tmop{`Distance'} \tmop{column}.
		\end{array}\\
    55 t + 100 - 40 t = 130~~~ &  & \begin{array}{p{0.425\textwidth}}
      \tmop{Add~the~entries~in~the~distance} \tmop{column~and~set~equal} \tmop{to~total~distance}.
    \end{array}\\
    15 t + 100 = 130~~~ &  & \tmop{~Combine} \tmop{like} \tmop{terms} 55 t - 40
    t.\\
    \tmmathbf{\underline{- 100 ~~- 100}} &  & \tmop{~Subtract} 100 \tmop{from}
    \tmop{both} \tmop{sides}.\\
    15 t = 30~~~ &  & \tmop{~The} \tmop{variable} \tmop{is} \tmop{multiplied}
    \tmop{by} 30.\\
    \tmmathbf{\overline{15} ~~~~ \overline{15}}~~~ &  & \tmop{~Divide} \tmop{both}
    \tmop{sides} \tmop{by} 15.\\
    t = 2~~~ &  & \tmop{~Our} \tmop{solution} \tmop{for} t\\
    \begin{array}{|c|c|}
      \hline
      & \tmop{Time}\\
      \hline
      \tmop{Fast} & 2\\
      \hline
      \tmop{Slow} & 2.5 - 2 = 0.5\\
      \hline
    \end{array} &  & \begin{array}{p{0.425\textwidth}}
      \tmop{To} \tmop{answer} \tmop{the} \tmop{question} \tmop{we} \tmop{plug}
      2 \tmop{in} \tmop{for} $t$\\
      \tmop{The} \tmop{car} \tmop{traveled} 40 \tmop{mph} \tmop{for} 0.5
      \tmop{hours} (30 \tmop{minutes}).
    \end{array}
  \end{eqnarray*}
\end{example}