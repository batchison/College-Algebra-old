\subsection{Two-Step Equations}\pp

 {\tmstrong{Objective: Solve two-step equations by balancing and using inverse
operations.}}\pp

 After mastering the technique for solving one-step equations, we are ready to consider two-step equations. As we solve two-step
equations, the important thing to remember is that everything works backwards!
When working with one-step equations, we learned that in order to clear a
``plus five'' in the equation, we would subtract five from both sides. We
learned that to clear ``divided by seven'' we multiply by seven on both sides.
The same pattern applies to the order of operations. When solving for our
variable $x$, we use order of operations backwards as well. This means we will
add or subtract first, then multiply or divide second (then exponents, and
finally any parentheses or grouping symbols, but that's another lesson).

\begin{example}\label{Lin13}
   
  \begin{eqnarray*}
    4 x - 20 = - 8 &  & 
  \end{eqnarray*}
   We have two numbers on the same side as the $x$. We need to move the $4$ and
  the $20$ to the other side. We know to move the $4$ we need to divide, and
  to move the $20$ we will add $20$ to both sides. If order of operations
  is done backwards, we will add or subtract first. Therefore we will add $20$
  to both sides first. Once we are done with that, we will divide both sides
  by $4$. The steps are shown below.
  \begin{eqnarray*}
    4 x - 20 = - 8~~ &  & \tmop{Start} \tmop{by} \tmop{focusing} \tmop{on}
    \tmop{the} \tmop{subtract} 20\\
    \tmmathbf{\underline{+ 20 ~~+ 20}} &  & \tmop{Add} 20 \tmop{to} \tmop{both}
    \tmop{sides}\\
    4 x = 12~~ &  & \tmop{Now} \tmop{we} \tmop{focus} \tmop{on} \tmop{the} 4
    \tmop{multiplied} \tmop{by} x\\
    \tmmathbf{\overline{4} ~~~~~ \overline{4}~}~ &  & \tmop{Divide} \tmop{both}
    \tmop{sides} \tmop{by} 4\\
    x = 3~~ &  & \tmop{Our} \tmop{solution}
  \end{eqnarray*}
\end{example}

 Notice in our next example when we replace the $x$ with $3$ \ we get a true
statement.
\begin{eqnarray*}
  4 (3) - 20 = - 8 &  & \tmop{Multiply} 4 (3)\\
  12 - 20 = - 8 &  & \tmop{Subtract} 12 - 20\\
  - 8 = - 8 &  & \tmop{True} !
\end{eqnarray*}
 The same process is used to solve any two-step equation. Add or subtract
first, then multiply or divide.

\begin{example}\label{Lin14}
  
  \begin{eqnarray*}
    5 x + 7 = 7~~ &  & \tmop{Start} \tmop{by} \tmop{focusing} \tmop{on}
    \tmop{the} \tmop{plus} 7\\
    \tmmathbf{\underline{- 7 ~~- 7}} &  & \tmop{Subtract} 7 \tmop{from}
    \tmop{both} \tmop{sides}\\
    5 x = 0~~ &  & \tmop{Now} \tmop{focus} \tmop{on} \tmop{the}
    \tmop{multiplication} \tmop{by} 5\\
    \tmmathbf{\overline{5} ~~~~~ \overline{5}}~~ &  & \tmop{Divide} \tmop{both}
    \tmop{sides} \tmop{by} 5\\
    x = 0~~ &  & \tmop{Our} \tmop{solution}
  \end{eqnarray*}
\end{example}

Notice the seven subtracted out completely! Many students get stuck on this
point, do not forget that we have a number for ``nothing left'', and that
number is zero. With this in mind the process is almost identical to our first
example.\pp

 A common error students make with two-step equations is with negative signs.
Remember the sign always stays with the number. Consider the following
example.

\begin{example}\label{Lin15}
 
   \begin{eqnarray*}
    4 - 2 x = 10 &  & \tmop{Start} \tmop{by} \tmop{focusing} \tmop{on}
    \tmop{the} \tmop{positive} 4\\
    \underline{\tmmathbf{- 4 ~~~~~~- 4}} &  & \tmop{Subtract} 4 \tmop{from}
    \tmop{both} \tmop{sides}\\
    - 2 x = 6~~ &  & \tmop{Negative~} (\tmop{subtraction}) \tmop{~stays} \tmop{on}
    \tmop{the} 2 x\\
    \tmmathbf{\overline{- 2} ~~~ \overline{- 2}} &  & \tmop{Divide} \tmop{by} -
    2\\
    x = - 3 &  & \tmop{Our} \tmop{solution}
  \end{eqnarray*}
\end{example}

 The same is true even if there is no apparent coefficient in front of the variable.  The coefficient is $1$ or $-1$ in this case.
Consider the next example.

\begin{example}\label{Lin16}
 
  
  \begin{eqnarray*}
    8 - x = 2~ &  & \tmop{Start} \tmop{by} \tmop{focusing} \tmop{on} \tmop{the}
    \tmop{positive} 8\\
    \tmmathbf{\underline{- 8 ~~~~~- 8}} &  & \tmop{Subtract} 8 \tmop{from}
    \tmop{both} \tmop{sides}\\
    - x = - 6 &  & \tmop{Negative} (\tmop{subtraction}) \tmop{stays} \tmop{on}
    \tmop{the} x\\
    - 1 x = - 6 &  & \tmop{Remember}, \tmop{no} \tmop{number} \tmop{in}
    \tmop{front} \tmop{of} \tmop{variable} \tmop{means} 1\\
    \tmmathbf{\overline{- 1} ~~~~ \overline{- 1}} &  & \tmop{Divide} \tmop{both}
    \tmop{sides} \tmop{by} - 1\\
    x = 6 &  & \tmop{Our} \tmop{solution}
  \end{eqnarray*}
\end{example}

 Solving two-step equations is a very important skill to master, as we study
algebra. The first step is to add or subtract, the second is to multiply or
divide. This pattern is seen in each of the following examples.

\pagebreak

\begin{example}\label{Lin17}
~\end{example}
 
  \begin{table}[h]
    \begin{tabular}{l}
      $- 3 x + 7 = - 8$\\
      $ ~~~~~~~\tmmathbf{\underline{- 7 ~~- 7}}~$\\
      ~~~~ $- 3 x = - 15$\\
      ~~~~ $ \tmmathbf{\overline{- 3} ~~~~~ \overline{- 3}}$\\
      ~~~~~~ $x = 5$
    \end{tabular}~~~~~~~~~~~~ 
		\begin{tabular}{l}
      $- 2 + 9 x = 7~$\\
      \underline{$\tmmathbf{+ 2 ~~~~~~~~+ 2}$}\\
      ~~~~ $9 x = 9$\\
      ~~~~ $ \tmmathbf{ \overline{9} ~~~~~ \overline{9}}$\\
      ~~~~ $x = 1$
    \end{tabular}~~~~~~~~~~~~
		% \ \ \ \ \ \ \ \ \ \ \ \ \ \ \ \ \ \ \ 
		\begin{tabular}{l}
      $~~~~8 = 2 x + 10$\\
      $\tmmathbf{\underline{- 10 ~~~~~~ - 10} }$\\
      ~~~~ $- 2 = 2 x$\\
      $~~~~~~~\tmmathbf{\overline{2} ~~~~~ \overline{2}}$\\
      ~~~~ $- 1 = x$
    \end{tabular}
    
    \
    
    \
    
    \
    
    \begin{tabular}{l}
      $~~~7 - 5 x = 17$\\
      $\tmmathbf{\underline{- 7 ~~~~~~~~~- 7}}$\\
      ~~~~ $- 5 x = 10$\\
      ~~~~ $ \tmmathbf{ \overline{- 5} ~~~~~ \overline{- 5}}$\\
      ~~~~ $x = - 2$
    \end{tabular} \ \ \ \ \ \ \ \ \ \ \ \ \ \ \ \ \ \ \ \begin{tabular}{l}
      $- 5 - 3 x = - 5$\\
      $\tmmathbf{\underline{+ 5 ~~~~~~~~~+ 5}}$\\
      ~~~~ $- 3 x = 0$~\\
      ~~~~ $ \tmmathbf{ \overline{- 3} ~~~~ \overline{- 3}}$\\
      ~~~~ $x = 0$
    \end{tabular} \ \ \ \ \ \ \ \ \ \ \ \ \ \ \ \ \ \ \ \begin{tabular}{l}
      $- 3 = \frac{x}{5} - 4~$\\
      {\tmmathbf{\underline{+ 4 ~~~~~~+ 4}}}\\
			\\
      $\tmmathbf{(5)} (1) = \frac{x}{5} \tmmathbf{(5)}$\\
      \ \ \ \ $5 = x$
    \end{tabular}
    \caption{Two-Step Equation Examples}
  \end{table}

 As problems in algebra become more complex the process covered here will
remain the same. In fact, as we solve problems like those in the next example,
each one of them will have several steps to solve, but the last two steps will resemble solving a two-step equation. This is why it is very important
to master two-step equations now!

\begin{example}\label{Lin18}
 
  
  \begin{eqnarray*}
    3 x^2 + 4 - x = 6 & \mbox{\hspace{1.5in}} & \displaystyle\frac{1}{x - 8} + \displaystyle\frac{1}{x} = \displaystyle\frac{1}{3}\\
		 & & \\
		\sqrt{5 x -
    5} + 1 = x & \mbox{\hspace{1.5in}} & \log_5 (2 x - 4) = 1
  \end{eqnarray*}
\end{example}

 {\tmstrong{World View Note:}} Persian mathematician Omar Khayyam would solve
algebraic problems geometrically by intersecting graphs rather than solving
them algebraically.

%\end{document}
