\subsection{Point-Slope Form}\pp

 {\tmstrong{Objective: Give the equation of a line with a known slope and
point.}}\pp

 The slope-intercept form has the advantage of being simple to remember and
use, however, it has one major disadvantage: we must know the $y$-intercept in
order to use it! Generally we do not know the y-intercept, we only know one or
more points (that are not the $y$-intercept). In these cases we can't use the
slope intercept equation, so we will use a different, more flexible formula. If
we let the slope of an equation be $m$, and a specific point on the line be
$(x_1, y_1)$, and any other point on the line be $(x, y)$. We can use the
slope formula to make a second equation.

\begin{example}\label{Lin62}
  
  \begin{eqnarray*}
    m,~ (x_1, y_1),~ (x, y) &  & \tmop{Recall} \tmop{slope} \tmop{formula}\\
    \frac{y_2 - y_1}{x_2 - x_1} = m &  & \tmop{Plug} \tmop{in} \tmop{values}\\
    \frac{y - y_1}{x - x_1} = m &  & \tmop{Multiply} \tmop{both} \tmop{sides}
    \tmop{by} (x - x_1)\\
    y - y_1 = m (x - x_1) &  & \tmop{Our} \tmop{equation}
  \end{eqnarray*}
\end{example}

 If we know the slope, $m$ of an equation and any point on the line $(x_1,
y_1)$ we can easily plug these values into the equation above which will be
called the point-slope formula or equation.\pp

\bbm
 {\tmstrong{\[ \tmop{Point} - \tmop{slope~} \tmop{equation} : y - y_1 = m (x -
   x_1) \]}}
\ebm
\pp

\begin{example}\label{Lin63}
 ~\pp
Write the equation of the line through the point $(3, - 4)$ with a slope of
  $\displaystyle\frac{3}{5}$.
  \begin{eqnarray*}
    y - y_1 = m (x - x_1) &  & \tmop{Plug} \tmop{values} \tmop{into}
    \tmop{point} - \tmop{slope} \tmop{formula}\\
    y - (- 4) = \frac{3}{5} (x - 3) &  & \tmop{Simplify} \tmop{signs}\\
    y + 4 = \frac{3}{5} (x - 3) &  & \tmop{Our} \tmop{solution}
  \end{eqnarray*}
\end{example}

 Often, we will prefer final answers be written in slope-intercept form. If the
directions ask for the answer in slope-intercept form we will simply
distribute the slope, then solve for $y$.

\begin{example}\label{Lin64}
~\pp
   Write the equation of the line through the point $(- 6, 2)$ with a slope of
  $- \displaystyle\frac{2}{3}$ in slope-intercept form.
  \begin{eqnarray*}
    y - y_1 = m (x - x_1) &  & \tmop{Plug} \tmop{values} \tmop{into}
    \tmop{point} - \tmop{slope} \tmop{formula}\\
    y - 2 = - \frac{2}{3} \left(x - (- 6)\right) &  & \tmop{Simplify} \tmop{signs}\\
    y - 2 = - \frac{2}{3} (x + 6) &  & \tmop{Distribute~} \tmop{slope}\\
    y - 2 = - \frac{2}{3} x - 4 &  & \tmop{Solve} \tmop{for} y \tmop{by~adding~2~to~both~sides}\\
    \tmmathbf{\underline{+ 2 ~~~~~~~~ + 2}} &  & \\
    y = - \frac{2}{3} x - 2 &  & \tmop{Our} \tmop{solution}
  \end{eqnarray*}
\end{example}

 An important thing to observe about the point slope formula is that the
operation between the $x$'s and $y$'s is subtraction. This means when you
simplify the signs you will have the opposite of the numbers in the point. We
need to be very careful with signs as we use the point-slope formula.\pp

 In order to find the equation of a line we will always need to know the slope.
If we don't know the slope to begin with we will have to do some work to find
it first before we can get an equation.

\begin{example}\label{Lin65}
~\pp
   Find the equation of the line through the points $(- 2, 5) \tmop{and} (4, -
  3)$.
  \begin{eqnarray*}
    m = \frac{y_2 - y_1}{x_2 - x_1} &  & \tmop{First} \tmop{we} \tmop{must}
    \tmop{find} \tmop{the} \tmop{slope}\\
    m = \frac{- 3 - 5}{4 - (- 2)} = \frac{- 8}{6} = - \frac{4}{3} &  &
    \tmop{Plug} \tmop{values} \tmop{in} \tmop{slope} \tmop{formula} \tmop{and}
    \tmop{evaluate}\\
    y - y_1 = m (x - x_1) &  & \tmop{Use} \tmop{point} - \tmop{slope}
    \tmop{formula},\\
		& & \tmop{~~~plugging~in~slope~and~either~point}\\
    y - 5 = - \frac{4}{3} (x - (- 2)) &  & \tmop{Simplify} \tmop{signs}\\
    y - 5 = - \frac{4}{3} (x + 2) &  & \tmop{Our} \tmop{solution}
  \end{eqnarray*}
\end{example}

\begin{example}\label{Lin66}
~\pp
   Find the equation of the line through the points $(- 3, 4) \tmop{and} (- 1,
  - 2)$ in slope-intercept form.
  \begin{eqnarray*}
    m = \frac{y_2 - y_1}{x_2 - x_1} &  & \tmop{First} \tmop{we} \tmop{must}
    \tmop{find} \tmop{the} \tmop{slope}\\
    m = \frac{- 2 - 4}{- 1 - (- 3)} = \frac{- 6}{2} = - 3 &  & \tmop{Plug}
    \tmop{values} \tmop{in} \tmop{slope} \tmop{formula} \tmop{and}
    \tmop{evaluate}\\
    y - y_1 = m (x - x_1) &  & \tmop{Use} \tmop{point} - \tmop{slope} \tmop{formula},\\
		& &  \tmop{~~~plugging~in~slope~and~either~point}\\
    y - 4 = - 3 (x - (- 3)) &  & \tmop{Simplify} \tmop{signs}\\
    y - 4 = - 3 (x + 3) &  & \tmop{Distribute~} \tmop{slope}\\
    y - 4 = - 3 x - 9 &  & \tmop{Solve} \tmop{for} y \\
    \tmmathbf{\underline{+ 4 ~~~~~~~~~+ 4}} &  & \tmop{Add} 4 \tmop{to} \tmop{both} \tmop{sides}\\
    y = - 3 x - 5 &  & \tmop{Our} \tmop{solution}
  \end{eqnarray*}
\end{example}

\begin{example}\label{Lin67}
~\pp
   Find the equation of the line through the points $(6, - 2)$ and $(- 4, 1)$
  in slope-intercept form.
  \begin{eqnarray*}
    m = \frac{y_2 - y_1}{x_2 - x_1} &  & \tmop{First} \tmop{we} \tmop{must}
    \tmop{find} \tmop{the} \tmop{slope}\\
    m = \frac{1 - (- 2)}{- 4 - 6} = \frac{3}{- 10} = - \frac{3}{10} &  &
    \tmop{Plug} \tmop{values} \tmop{into} \tmop{slope} \tmop{formula}
    \tmop{and} \tmop{evaluate}\\
    y - y_1 = m (x - x_1) &  & \tmop{Use} \tmop{point} - \tmop{slope}
    \tmop{formula},\\
		& &  \tmop{~~~plugging~in~slope~and~either~point}\\
    y - (- 2) = - \frac{3}{10} (x - 6) &  & \tmop{Simplify} \tmop{signs}\\
    y + 2 = - \frac{3}{10} (x - 6) &  & \tmop{Distribute} \tmop{slope~}\\
    y + 2 = - \frac{3}{10} x + \frac{9}{5} &  & \tmop{Solve} \tmop{for} y,
    \tmop{by~subtracting} 2 \tmop{from} \tmop{both} \tmop{sides}\\
    \tmmathbf{\underline{- 2 ~~~~~~~~~- \frac{10}{5}}} &  & \tmop{Use} \frac{10}{5} \tmop{on}
    \tmop{right} \tmop{so} \tmop{we} \tmop{have} \tmop{a} \tmop{common}
    \tmop{denominator}\\
    & & \\
		y = - \frac{3}{10} x - \frac{1}{5} &  & \tmop{Our} \tmop{solution}
  \end{eqnarray*}
\end{example}

 {\tmstrong{World View Note:}} The city of K$\ddot{\text{o}}$nigsberg (now Kaliningrad, Russia)
had a river that flowed through the city breaking it into several parts. There
were 7 bridges that connected the parts of the city. In 1735 Leonhard Euler
considered the question of whether it was possible to cross each bridge
exactly once and only once. It turned out that this problem was impossible,
but the work laid the foundation of what would become graph theory.
