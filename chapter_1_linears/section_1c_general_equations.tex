\subsection{General Equations}\pp

 {\tmstrong{Objective: Solve general linear equations with variables on both
sides.}}\pp

 Often as we are solving linear equations we will need to do some work to set
them up into a form we are familiar with solving. This section will focus on
manipulating an equation we are asked to solve in such a way that we can use
our pattern for solving two-step equations to ultimately arrive at the
solution.\pp

 One such issue that needs to be addressed is parentheses. Often the
parentheses can get in the way of solving an otherwise easy problem. As you
might expect we can get rid of the unwanted parentheses by using the
distributive property. This is shown in the following example. Notice the
first step is distributing, then it is solved like any other two-step
equation.

\begin{example}\label{Lin19}
  
  \begin{eqnarray*}
    4 (2 x - 6) = 16~~ &  & \tmop{Distribute} 4 \tmop{through}
    \tmop{parentheses}\\
    8 x - 24 = 16~~ &  & \tmop{Focus} \tmop{on} \tmop{the} \tmop{subtraction}
    \tmop{first}\\
    \tmmathbf{\underline{+ 24 ~+ 24}} &  & \tmop{Add} 24 \tmop{to} \tmop{both}
    \tmop{sides}\\
    8 x = 40~~ &  & \tmop{Now} \tmop{focus} \tmop{on} \tmop{the} \tmop{multiply}
    \tmop{by} 8\\
    \tmmathbf{ \overline{8} ~~~~~~ \overline{8}}~~ &  & \tmop{Divide} \tmop{both}
    \tmop{sides} \tmop{by} 8\\
    x = 5 &  & \tmop{Our} \tmop{solution}
  \end{eqnarray*}
\end{example}

 Often after we distribute there will be some like terms on one side of the
equation. Example \ref{Lin20} shows distributing to clear the parentheses and then
combining like terms next. Notice we only combine like terms on the same side
of the equation. Once we have done this, our next example solves just like any
other two-step equation.

\begin{example}\label{Lin20}
  
  \begin{eqnarray*}
    3 (2 x - 4) + 9 = 15 &  & \tmop{Distribute} \tmop{the} 3 \tmop{through}
    \tmop{the} \tmop{parentheses}\\
    6 x - 12 + 9 = 15 &  & \tmop{Combine} \tmop{like} \tmop{terms}, - 12 + 9\\
    6 x - 3 = 15 &  & \tmop{Focus} \tmop{on} \tmop{the} \tmop{subtraction}
    \tmop{first}\\
    \tmmathbf{\underline{+ 3 ~+ 3}} &  & \tmop{Add} 3 \tmop{to} \tmop{both} \tmop{sides}\\
    6 x = 18 &  & \tmop{Now} \tmop{focus} \tmop{on} \tmop{multiply} \tmop{by}
    6\\
    \tmmathbf{\overline{6} ~~~~~~ \overline{6}} &  & \tmop{Divide} \tmop{both} \tmop{sides}
    \tmop{by} 6\\
    x = 3 &  & \tmop{Our} \tmop{solution}
  \end{eqnarray*}
\end{example}

 A second type of problem that becomes a two-step equation after a bit of work
is one where we see the variable on both sides. This is shown in the following
example.

\begin{example}\label{Lin21}
  
  \begin{eqnarray*}
    4 x - 6 = 2 x + 10 &  & 
  \end{eqnarray*}
   Notice here the {\tmem{x}} is on both the left and right sides of the
  equation. This can make it difficult to decide which side to work with. We
  fix this by moving one of the terms with $x$ to the other side, much like we
  moved a constant term. It doesn't matter which term gets moved, $4 x$ or $2
  x$, however, it would be the author's suggestion to move the smaller term
  (to avoid negative coefficients). For this reason we begin this problem by
  clearing the positive $2 x$ by subtracting $2 x$ from both sides.
  \begin{eqnarray*}
    4 x - 6 = 2 x + 10 &  & \tmop{Notice} \tmop{the} \tmop{variable} \tmop{on}
    \tmop{both} \tmop{sides}\\
    \tmmathbf{\underline{- 2 x ~~~~~- 2 x}}~~~~~~  &  & \tmop{Subtract} 2 x \tmop{from}
    \tmop{both} \tmop{sides}\\
    2 x - 6 = 10 &  & \tmop{Focus} \tmop{on} \tmop{the} \tmop{subtraction}
    \tmop{first}\\
    \tmmathbf{\underline{+ 6 ~+ 6}}  &  & \tmop{Add} 6 \tmop{to} \tmop{both}
    \tmop{sides}\\
    2 x = 16 &  & \tmop{Focus} \tmop{on} \tmop{the} \tmop{multiplication}
    \tmop{by} 2\\
    \tmmathbf{\overline{2} ~~~~~~ \overline{2} } &  & \tmop{Divide} \tmop{both}
    \tmop{sides} \tmop{by} 2\\
    x = 8 &  & \tmop{Our} \tmop{solution}
  \end{eqnarray*}
\end{example}

 The previous example shows the check on this solution. Here the solution is
plugged into the {\tmem{x}} on both the left and right sides before
simplifying.

\begin{example}\label{Lin22}

\begin{eqnarray*}
  4 (8) - 6 = 2 (8) + 10 &  & \tmop{Multiply} 4 (8) \tmop{and} 2 (8)
  \tmop{first}\\
  32 - 6 = 16 + 10 &  & \tmop{Add} \tmop{and} \tmop{Subtract}\\
  26 = 26 &  & \tmop{True} !
\end{eqnarray*}
\end{example}

 The next example illustrates the same process with negative coefficients.
Notice first the smaller term with the variable is moved to the other side,
this time by adding because the coefficient is negative.

\begin{example}\label{Lin23}
  
  \begin{eqnarray*}
    - 3 x + 9 = 6 x - 27 &  & \tmop{Notice} \tmop{the} \tmop{variable}
    \tmop{on} \tmop{both} \tmop{sides}, - 3 x \tmop{is} \tmop{smaller}\\
    \tmmathbf{\underline{+ 3 x ~~~~~~+ 3 x}}~~~~~  &  & \tmop{Add} 3 x \tmop{to}
    \tmop{both} \tmop{sides}\\
    9 = 9 x - 27 &  & \tmop{Focus} \tmop{on} \tmop{the} \tmop{subtraction}
    \tmop{by} 27\\
    \tmmathbf{\underline{+ 27 ~~~~~+ 27}} &  & \tmop{Add} 27 \tmop{to} \tmop{both}
    \tmop{sides}\\
    36 = 9 x &  & \tmop{Focus} \tmop{on} \tmop{the} \tmop{multiplication}
    \tmop{by} 9\\
    \tmmathbf{\overline{9} ~~~~~ \overline{9} }~  &  & \tmop{Divide} \tmop{both}
    \tmop{sides} \tmop{by} 9\\
    4 = x &  & \tmop{Our} \tmop{solution}
  \end{eqnarray*}
\end{example}

 Linear equations can become particularly interesting when the two processes are
combined. In the following problems we have parentheses and the variable on
both sides. Notice in each of the following examples we distribute, then
combine like terms, then move the variable to one side of the equation.

\begin{example}\label{Lin24}
  \begin{eqnarray*}
    2 (x - 5) + 3 x = x + 18~~ &  & \tmop{Distribute} \tmop{the} 2
    \tmop{through} \tmop{parentheses}\\
    2 x - 10 + 3 x = x + 18~~ &  & \tmop{Combine} \tmop{like} \tmop{terms} 2 x +
    3 x\\
    5 x - 10 = x + 18~~ &  & \tmop{Notice} \tmop{the} \tmop{variable} \tmop{is}
    \tmop{on} \tmop{both} \tmop{sides}\\
    \tmmathbf{\underline{- x ~~~~~~- x}}~~~~~~~~  &  & \tmop{Subtract} x \tmop{from}
    \tmop{both} \tmop{sides}\\
    4 x - 10 = 18~~ &  & \tmop{Focus} \tmop{on} \tmop{the} \tmop{subtraction}
    \tmop{of} 10\\
    \tmmathbf{\underline{+ 10 ~+10}}  &  & \tmop{Add} 10 \tmop{to} \tmop{both}
    \tmop{sides}\\
    4 x = 28~~ &  & \tmop{Focus} \tmop{on} \tmop{multiplication} \tmop{by} 4\\
    \tmmathbf{ \overline{4} ~~~~~ \overline{4}}~~~  &  & \tmop{Divide} \tmop{both}
    \tmop{sides} \tmop{by} 4\\
    x = 7~~ &  & \tmop{Our} \tmop{solution}
  \end{eqnarray*}
\end{example}

 Sometimes we may have to distribute more than once to clear several
parentheses. Remember to combine like terms after you distribute!

\begin{example}\label{Lin25}
  
  \begin{eqnarray*}
    3 (4 x - 5) - 4 (2 x + 1) = 5~~~ &  & \tmop{Distribute} 3 \tmop{and} - 4
    \tmop{through} \tmop{parentheses}\\
    12 x - 15 - 8 x - 4 = 5~~~ &  & \tmop{Combine} \tmop{like} \tmop{terms} 12 x
    - 8 x \tmop{and} - 15 - 4\\
    4 x - 19 = 5~~~ &  & \tmop{Focus} \tmop{on} \tmop{subtraction} \tmop{of} 19\\
    \underline{\tmmathbf{+ 19 ~~+ 19}} &  & \tmop{Add} 19 \tmop{to} \tmop{both}
    \tmop{sides}\\
    4 x = 24~~~ &  & \tmop{Focus} \tmop{on} \tmop{multiplication} \tmop{by} 4\\
    \tmmathbf{\overline{4} ~~~~~~ \overline{4}}~~~ &  & \tmop{Divide} \tmop{both}
    \tmop{sides} \tmop{by} 4\\
    x = 6~~~ &  & \tmop{Our} \tmop{solution}
  \end{eqnarray*}
\end{example}

 This leads to a 5-step process to solve any linear equation. While all five
steps aren't always needed, this can serve as a guide to solving equations.
\begin{enumerate}
  \item Distribute through any parentheses.
  
  \item Combine like terms on each side of the equation.
  
  \item Get the variables on one side by adding or subtracting
  
  \item Solve the remaining 2-step equation (add or subtract then multiply or
  divide)
  
  \item Check your answer by plugging it back in for {\tmem{x}} to find a true
  statement.  If your resulting statement is false, repeat the procedure, beginning with the first step.
\end{enumerate}
 The order of these steps is very important.\pp

 {\tmstrong{World View Note:}} The Chinese developed a method for solving
equations that involved finding each digit one at a time about 2000 years ago!\pp

 We can see each of the above five steps worked through our next example.

\begin{example}\label{Lin26}
  
  \begin{eqnarray*}
    4 (2 x - 6) + 9 = 3 (x - 7) + 8 x &  & \tmop{Distribute} 4 \tmop{and} 3
    \tmop{through} \tmop{parentheses}\\
    8 x - 24 + 9 = 3 x - 21 + 8 x &  & \tmop{Combine} \tmop{like} \tmop{terms}
    - 24 + 9 \tmop{and} 3 x + 8 x\\
    8 x - 15 = 11 x - 21 &  & \tmop{Notice} \tmop{the} \tmop{variable}
    \tmop{is} \tmop{on} \tmop{both} \tmop{sides}\\
    \tmmathbf{\underline{- 8 x ~~~~~~~- 8 x}}~~~~~~~  &  & \tmop{Subtract} 8 x \tmop{from}
    \tmop{both} \tmop{sides}\\
    - 15 = 3 x - 21 &  & \tmop{Focus} \tmop{on} \tmop{subtraction} \tmop{of}
    21\\
    \tmmathbf{\underline{+ 21 ~~~~~+ 21}}  &  & \tmop{Add} 21 \tmop{to} \tmop{both}
    \tmop{sides}\\
    6 = 3 x &  & \tmop{Focus} \tmop{on} \tmop{multiplication} \tmop{by} 3\\
    \tmmathbf{\overline{3} ~~~~ \overline{3} }~  &  & \tmop{Divide} \tmop{both}
    \tmop{sides} \tmop{by} 3\\
    2 = x &  & \tmop{Our} \tmop{solution}
  \end{eqnarray*}
  {\tmem{Check:}}
  \begin{eqnarray*}
    4 [2 (2) - 6] + 9 = 3 [(2) - 7] + 8 (2) &  & \tmop{Plug} 2 \tmop{in}
    \tmop{for} \tmop{each} x. \tmop{Multiply} \tmop{inside}
    \tmop{parentheses}\\
    4 [4 - 6] + 9 = 3 [- 5] + 8 (2)  &  & \tmop{Finish} \tmop{parentheses}
    \tmop{on} \tmop{left}, \tmop{multiply} \tmop{on} \tmop{right}\\
    4 [- 2] + 9 = - 15 + 8 (2)  &  & \tmop{Finish} \tmop{multiplication}
    \tmop{on} \tmop{both} \tmop{sides}\\
    - 8 + 9 = - 15 + 16 &  & \tmop{Add}\\
    1 = 1 &  & \tmop{True} !
  \end{eqnarray*}
\end{example}

 When we check our solution of $x = 2$ we found a true statement, $1 = 1$.
Therefore, we know our solution $x = 2$ is the correct solution for the
problem.\pp

 There are two special cases that can come up as we are solving these linear
equations. The first is illustrated in the next two examples. Notice we start
by distributing and moving the variables all to the same side.\pp

\begin{example}\label{Lin27}
  \begin{eqnarray*}
    3 (2 x - 5) = 6 x - 15 &  & \tmop{Distribute} 3 \tmop{through}
    \tmop{parentheses}\\
    6 x - 15 = 6 x - 15 &  & \tmop{Notice} \tmop{the} \tmop{variable}
    \tmop{on} \tmop{both} \tmop{sides}\\
    \tmmathbf{\underline{- 6 x ~~~~~~~- 6 x}}~~~~~~  &  & \tmop{Subtract} 6 x \tmop{from}
    \tmop{both} \tmop{sides}\\
    - 15 = - 15 &  & \tmop{Variable} \tmop{is} \tmop{gone} ! \tmop{True} !
  \end{eqnarray*}
\end{example}

   Here the variable subtracted out completely! We are left with a true
  statement, $- 15 = - 15$. If the variables subtract out completely and we
  are left with a true statement, this indicates that the equation is always
  true, no matter what $x$ is. Thus, for our solution we say {\tmstrong{all real numbers}} or $\mathbb{R}$.\pp
	It is worth mentioning that in both the previous and following examples, we are still \textit{solving} a given equation for all possible values of $x$.  When the variable is eliminated entirely, this can sometimes be confused with \textit{checking} a solution.

\begin{example}\label{Lin28}
  
  \begin{eqnarray*}
    2 (3 x - 5) - 4 x = 2 x + 7 &  & \tmop{Distribute} 2 \tmop{through}
    \tmop{parentheses}\\
    6 x - 10 - 4 x = 2 x + 7 &  & \tmop{Combine} \tmop{like} \tmop{terms} 6 x
    - 4 x\\
    2 x - 10 = 2 x + 7 &  & \tmop{Notice} \tmop{the} \tmop{variable} \tmop{is}
    \tmop{on} \tmop{both} \tmop{sides}\\
    \tmmathbf{\underline{- 2 x ~~~~~~- 2 x} }~~~~~  &  & \tmop{Subtract} 2 x \tmop{from}
    \tmop{both} \tmop{sides}\\
    - 10 \neq 7 &  & \tmop{Variable} \tmop{is} \tmop{gone} ! \tmop{False} !
  \end{eqnarray*}
\end{example}

   Again, the variable subtracted out completely! However, this time we are
  left with a false statement, this indicates that the equation is never true,
  no matter what $x$ is. Thus, for our solution we say {\tmstrong{no
  solution}} or $\varnothing$.