{\tmstrong{Objective: Solve systems of equations using substitution.}}\pp

Solving a system of equations by graphing has several limitations. First, it requires
the graph to be perfectly drawn, if the lines are not straight we may arrive
at the wrong answer. Second, graphing is not a great method to use if the
answer is really large, over 100 for example, or a decimal, since
a graph will not help us find an answer such as 3.2134. For these reasons we
will rarely use graphs to solve a given system of equations. Instead, an algebraic approach
will be used.\pp

The first algebraic approach is called substitution. We will build the
concepts of substitution through several examples, then end with a five-step
process to solve problems using this method.

\begin{example}~~~Solve the following system of equations.
$$x=5\qquad\qquad y=2x-3$$
We already know $x$ must equal 5, so we can substitute $x=5$ into the other equation.%\\ 
\begin{eqnarray*}
		y = 2 \tmmathbf{(5)} - 3 &  & \tmop{Evaluate:} \tmop{~Multiply}
    \tmop{first}\\
    y = 10 - 3~~~ &  & \tmop{Next~subtract}\\
    y = 7~~~~~~~~~~ &  & \tmop{We} \tmop{now} \tmop{also} \tmop{have} y\\
    (x,y)=(5, 7) &  & \tmop{Our} \tmop{solution}
  \end{eqnarray*}
\end{example}

When we know what one variable equals we can plug that value (or expression)
in for the variable in the other equation. It is very important that when we
substitute, the substituted value goes in parentheses. The reason for this is
shown in the next example.%\pp

\vspace{1in}
~\\
\begin{example}~~~Solve the following system of equations.
$$2 x - 3 y = 7\qquad\qquad y = 3 x - 7$$
We begin by substituting $y=3x-7$ into the other equation.
  \begin{eqnarray*}
    2 x - 3 \tmmathbf{(3 x - 7)} = 7~~~~ &  & \tmop{Solve~for} x, %\tmop{this}
    %\tmop{equation}, 
		\tmop{distributing} - 3 \tmop{first}\\
    2 x - 9 x + 21 = 7~~~~ &  & \tmop{Combine} \tmop{like} \tmop{terms} 2 x - 9x\\
    - 7 x + 21 = 7~~~~ &  &\\
    \underline{\tmmathbf{- 21 ~~- 21}} &  & \tmop{Subtract} 21 \\
    - 7 x = - 14 &  &\\
    \tmmathbf{\overline{- 7} ~~~~~ \overline{- 7}} &  & \tmop{Divide} \tmop{by} - 7 \\
    x = 2~~~~ &  & \tmop{We} \tmop{now} \tmop{have} \tmop{our} x.\\
		& & \tmop{~~~Substitute} \tmop{back} \tmop{in} \tmop{equation} \tmop{to} \tmop{find} y\\
    y = 3 \tmmathbf{(2)} - 7 &  & \tmop{Evaluate:} \tmop{~Multiply} \tmop{first}\\
    y = 6 - 7~~~~ &  & \tmop{Next~subtract}\\
    y = - 1~~~~~~~ &  & \tmop{We} \tmop{now} \tmop{also} \tmop{have} y\\
    (x,y)=(2, - 1) &  & \tmop{Our} \tmop{solution}
  \end{eqnarray*}
\end{example}
By using the entire expression $3 x - 7$ to replace $y$ in the other equation
we were able to reduce the system to a single linear equation which we can
easily solve for our first variable. However, the ``lone'' variable (a variable
with a coefficient of 1) is not always alone on one side of the equation. If
this happens we can isolate the lone variable by solving for it.

\begin{example}~~~Solve the following system of equations.
	$$3 x + 2 y = 1\qquad\qquad \tmmathbf{x} - 5 y = 6$$
%  \begin{eqnarray*}
 %   \begin{array}{l}
  %    3 x + 2 y = 1\\
   %   \tmmathbf{x} - 5 y = 6
    %\end{array} &  & \tmop{Lone} \tmop{variable} \tmop{is} x, \tmop{isolate}
    %\tmop{by} \tmop{adding} 5 y \tmop{to} \tmop{both} \tmop{sides} .\\
    The lone variable is $x$.  Isolate the lone variable by adding $5y$ to both sides.
		\begin{eqnarray*}		
		x - 5 y = 6~~~~~~~ & & \\
		\tmmathbf{\underline{+ 5 y ~~+ 5 y}}~~~  &  & \\
	  x = 6 + 5 y &  & \tmop{Substitute} \tmop{this} \tmop{into} \tmop{the}
  		\tmop{untouched} \tmop{equation}\\
    	& & \\
  		3 \tmmathbf{(6 + 5 y)} + 2 y = 1 &  & \tmop{Solve} \tmop{this}
    \tmop{equation}, \tmop{distributing} 3 \tmop{first}\\
    18 + 15 y + 2 y = 1 &  & \tmop{Combine} \tmop{like} \tmop{terms} 15 y + 2
    y\\
  \end{eqnarray*}
	\begin{eqnarray*}
			18 + 17 y = 1~~~~ &  &\\
    \tmmathbf{\underline{- 18 ~~~~~~~~~- 18}} &  & \tmop{Subtract} 18 \tmop{from} \tmop{both}
    \tmop{sides} \\
  		17 y = - 17 &  &\\
    \tmmathbf{\overline{17} ~~~~~~ \overline{17}} &  &  \tmop{Divide} \tmop{both} \tmop{sides} \tmop{by} 17\\
  	y = - 1 ~ &  & \tmop{We} \tmop{have} \tmop{our} y.\\
		& & \tmop{~~~Substitute} \tmop{back} \tmop{in} \tmop{equation} \tmop{to} \tmop{find} x\\
    x = 6 + 5 \tmmathbf{(- 1)} &  & \tmop{Evaluate:} \tmop{~Multiply} \tmop{first}\\
    x = 6 - 5~~~~~~~ &  & \tmop{Next~subtract}\\
    x = 1~~~~~~~~~~~~ &  & \tmop{We} \tmop{now} \tmop{also} \tmop{have} x\\
    (x,y)=(1, - 1) &  & \tmop{Our} \tmop{solution}
  \end{eqnarray*}
\end{example}


The process in the previous example is known as solving by substitution. This process is described and illustrated in the following table which lists the five steps to solving by substitution.\pp

\begin{tabular}{|l|c|}
  \hline
  ~~~~~~~~~~~~~~~~~~~Problem & \begin{tabular}{l}
    $4 x - 2 y = ~2~~~$\\
    $~2 x + y = - 5$
  \end{tabular}\\
  \hline
  1. Find the lone variable. & \begin{tabular}{l}
    Lone variable is $y$, in the\\
     second equation: $2 x + \tmmathbf{y} = - 5$
  \end{tabular}\\
  \hline
  2. Solve for the lone variable. & \begin{tabular}{c}
    Subtract $2x$ from both sides.\\
    $\tmmathbf{y = - 5 - 2 x}$
  \end{tabular}\\
  \hline
  3.~$\begin{array}{l}
	\text{Substitute into the untouched}\\
	\text{equation.}
	\end{array}$& $4 x - 2 \tmmathbf{(- 5 - 2 x)}
  = 2$\\
  \hline
  4. Solve. & $\begin{array}{c}
 4 x + 10 + 4 x = 2~~~~~~~\\
 8 x + 10 = 2\\
 ~~~~~~~~\tmmathbf{\underline{- 10 ~~- 10}}\\
~~~~~~~~~ 8 x = - 8\\
~~~~~~~~~ \tmmathbf{\overline{8} ~~~~~~ \overline{8}}\\
~~~~~~~~~ \tmmathbf{x = - 1}
  \end{array}$\\
  \hline
  5.~$\begin{array}{l}
	\text{Plug into lone variable}\\
	\text{equation and evaluate.}
	\end{array}$ & $\begin{array}{l}
    y = - 5 - 2 \tmmathbf{(- 1)}\\
    y = - 5 + 2\\
    \tmmathbf{y = - 3}
  \end{array}$\\
  \hline
  ~~~~~~~~~~~~~~~~~Our solution & $(x,y)=(- 1, - 3)$\\
  \hline
\end{tabular}

\vspace{.5in}
~\\

Sometimes we have several lone variables in a problem. In this case we will
have the choice on which lone variable we wish to solve for, either will give
the same final result.

\begin{example}~~~Solve the following system of equations.
    $$x + y = 5\qquad\qquad x - y = - 1$$
  \begin{eqnarray*}
\tmmathbf{x} + y = 5~~~~~ &  & \begin{array}{l}
      \text{Find the lone variable:~}x \text{~in the first or}\\
			\text{~~~second equation,} \text{or~} y \text{~in the first equation.}\\ \text{We will choose~} x \text{~in the first equation}.
    \end{array}\\
    \tmmathbf{x} + y = 5~~~~~ &  & \tmop{Solve} \tmop{for~the~lone~variable~} x\\
    \tmmathbf{\underline{- y ~~- y}}~~ &  &  \tmop{Subtract} y \tmop{from} \tmop{both} \tmop{sides}\\
    x = 5 - y &  & \tmop{Plug} \tmop{into} \tmop{the} \tmop{untouched}
    \tmop{equation}, \tmop{the} \tmop{second} \tmop{equation}\\
    \tmmathbf{(5 - y)} - y = - 1 &  & \tmop{Combine}
    \tmop{like} \tmop{terms.} \tmop{Parentheses}
    \tmop{may} \tmop{be} \tmop{removed} \\
    5 - 2 y = - 1 &  &\\
    \tmmathbf{\underline{- 5 ~~~~~~~~- 5}} &  &  \tmop{Subtract} 5 \tmop{from} \tmop{both}
    \tmop{sides}\\
    - 2 y = - 6 &  &\\
    \tmmathbf{\overline{- 2} ~~~~ \overline{- 2}} &  &  \tmop{Divide} \tmop{both} \tmop{sides} \tmop{by} - 2\\
    y = 3~~ &  & \tmop{We} \tmop{have} \tmop{our} y!\\
    x = 5 - (3) &  & \tmop{Plug} \tmop{into} \tmop{lone} \tmop{variable}
    \tmop{equation~and} \tmop{evaluate}\\
    x = 2~~~~~~~ &  & \tmop{Now} \tmop{we} \tmop{have} \tmop{our} x\\
    (x,y)=(2, 3) &  & \tmop{Our} \tmop{solution}
  \end{eqnarray*}
\end{example}

Just as with graphing it is possible to have no solution $\varnothing$
(parallel lines) or infinite solutions (same line) with the substitution
method. While we won't have a parallel line or the same line to look at and
conclude if it is one or the other, the process takes an interesting turn as
shown in the following example.

\begin{example}~~~Solve the following system of equations.
$$y + 4 = 3 x\qquad\qquad 2 y - 6 x = - 8$$
  \begin{eqnarray*}
    \tmmathbf{y} + 4 = 3 x &  & \text{Find~the~lone~variable:~} y
    \text{~in~the~first~equation}\\
  \end{eqnarray*}
  \begin{eqnarray*}
  	y + 4 = 3 x~~~~~ &  & \tmop{Solve} \tmop{for} \tmop{the} \tmop{lone}
    \tmop{variable~} y\\
    \tmmathbf{\underline{- 4 ~~- 4}}~~~~ &  & \tmop{Subtract} 4 \tmop{from} \tmop{both} \tmop{sides} \\
		y = 3 x - 4 &  & \tmop{Plug} \tmop{into} \tmop{second}
    \tmop{equation}\\
	  2 \tmmathbf{(3 x - 4)} - 6 x = - 8 &  & \tmop{Solve}, \tmop{distribute}
    \tmop{through} \tmop{parentheses}\\
    6 x - 8 - 6 x = - 8 &  & \tmop{Combine} \tmop{like} \tmop{terms} 6 x - 6x\\
    - 8 = - 8 &  & \tmop{Variables} \tmop{are} \tmop{gone} !
  \end{eqnarray*}
%\begin{center}
Since we are left with a true statement ($-8=-8$), we conclude that there are infinitely many solutions.
%\end{center}
\end{example}

Because we had a true statement, and no variables, we know that anything that
works in the first equation, will also work in the second equation. However,
we do not always end up with a true statement.

\begin{example}~~~Solve the following system of equations.
      $$6 x - 3 y = - 9\qquad\qquad - 2 x + y = 5$$
  \begin{eqnarray*}
      - 2 x + \tmmathbf{y} = 5~~~~~~~ &  & \text{Find~the~lone~variable:~} y
    \text{~in~the~second~equation}\\
    - 2 x + y = 5~~~~~~~ &  & \tmop{Solve} \tmop{for} \tmop{the} \tmop{lone}
    \tmop{variable}\\
    \tmmathbf{\underline{+ 2 x ~~~~~~+ 2 x}}~~~ &  & \tmop{Add~} 2 x \tmop{~to} \tmop{both} \tmop{sides} \\
    y = 5 + 2 x &  & \tmop{Plug} \tmop{into} \tmop{untouched}
    \tmop{equation}\\
    6 x - 3 \tmmathbf{(5 + 2 x)} = - 9 &  & \tmop{Solve}, \tmop{distribute}
    \tmop{through} \tmop{parentheses}\\
    6 x - 15 - 6 x = - 9 &  & \tmop{Combine} \tmop{like} \tmop{terms} 6 x - 6
    x\\
    - 15 \neq - 9 &  & \tmop{Variables} \tmop{are} \tmop{gone}!
  \end{eqnarray*}
%\begin{center}
Since we are left with a false statement ($-15\neq -9$) and no variables, we know that nothing will
work in both equations and we may conclude that there are no solutions, or $\varnothing$.
%\end{center}
\end{example}



{\tmstrong{World View Note:}} French mathematician Rene Descartes wrote a
book which included an appendix on geometry. It was in this book that he
suggested using letters from the end of the alphabet for unknown values. This
is why often we are solving for the variables $x, y, \tmop{and} z$.\pp

One more question needs to be considered: what if there is no lone variable?
If there is no lone variable substitution can still work, we will
just have to select one variable to solve for, and introduce fractions.

\begin{example}~~~Solve the following system of equations.
  $$5 x - 6 y = - 14\qquad\qquad - 2 x + 4 y = 12$$
  There is no lone variable, so we will solve the first equation for $x$.
	\begin{eqnarray*}
      5 \tmmathbf{x} - 6 y = - 14~~~~~~~~ &  & \tmop{Solve} \tmop{for} \tmop{our} \tmop{variable~} x\\
    \tmmathbf{\underline{+ 6 y ~~+ 6 y}}~~~~~~~ &  & \tmop{Add~} 6 y \tmop{~to} \tmop{both} \tmop{sides}\\
      5 x = - 14 + 6 y~ &  &\\
    \tmmathbf{\overline{5} ~~~~~~~ \overline{5} ~~~~~ \overline{5}}~~ &  &  \tmop{Divide} \tmop{each} \tmop{term} \tmop{by~} 5\\
    x = \frac{- 14}{5} + \frac{6 y}{5} &  & \tmop{Plug} \tmop{into}
    \tmop{untouched} \tmop{equation}\\
    - 2 \left( \frac{- 14}{5} + \frac{6 y}{5} \right) + 4 y = 12~~~~ &  &
    \tmop{Solve}, \tmop{distribute} \tmop{through} \tmop{parentheses}\\
    \frac{28}{5} - \frac{12 y}{5} + 4 y = 12~~~~ &  & \tmop{Clear}
    \tmop{fractions} \tmop{by} \tmop{multiplying} \tmop{by~} 5\\
    \frac{28 \tmmathbf{(5)}}{5} - \frac{12 y \tmmathbf{(5)}}{5} + 4 y
    \tmmathbf{(5)} = 12 \tmmathbf{(5)} &  & \tmop{Reduce} \tmop{fractions}
    \tmop{and} \tmop{multiply}\\
    28 - 12 y + 20 y = 60~~~~ &  & \tmop{Combine} \tmop{like} \tmop{terms} - 12 y
    + 20 y\\
    28 + 8 y = 60~~~~ &  &\\
    \underline{\tmmathbf{- 28 ~~~~~~~~- 28}}~ &  &  \tmop{Subtract~} 28 \tmop{~from} \tmop{both}
    \tmop{sides}\\
    8 y = 32~~~~ &  & \\
    \tmmathbf{\overline{8} ~~~~~ \overline{8}}~~~~~ &  & \tmop{Divide} \tmop{both} \tmop{sides} \tmop{by~} 8\\
    y = 4~~~~~~ &  & \tmop{We} \tmop{have} \tmop{our~} y\\
    x = -\frac{14}{5} + \frac{6 (4)}{5} &  & \tmop{Plug} \tmop{into}
    \tmop{lone} \tmop{variable} \tmop{equation}, \tmop{multiply}\\
    x = -\frac{14}{5} + \frac{24}{5}~~ &  & \tmop{Add} \tmop{fractions}\\
    x = \frac{10}{5}~~~~~~~~~~~~ &  & \tmop{Reduce} \tmop{fraction}\\
    x = 2~~~~~~~~~~~~~~ &  & \tmop{Now} \tmop{we} \tmop{have} \tmop{our~} x\\
    (x,y)=(2, 4) &  & \tmop{Our} \tmop{solution}
  \end{eqnarray*}
\end{example}

Using the fractions does make the problem a bit trickier. This is why we
have yet another method for solving systems of equations that will be discussed in
the next section.

%\end{document}
